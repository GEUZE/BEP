%TODO: calculate expectation difference in frequency

\clearpage\section{Theory}


\subsection{Piezoelectricity}
\begin{itemize}
	\item Cyrstal couples electric field to strain
	\item Can be modeled by an RLC with CP
	\item Resonance modes + example
\end{itemize}

\subsection{Crystal oscillator}


\subsection{Resonance}



\subsubsection{Q-factor}
The Q-factor characterizes a resonators' bandwidth relative to its center resonant frequency.\cite{electroniccircuits} 

\subsection{Non-linear mechanics}
XXX When an electrical potential is applied over a piezoelectric material, the material will deform from its equilibrium position when there is no electrical potential. This results in a restoring force inside the material which opposes the deforming piezoelectric force. The restoring force $F$ can be modeled as a polynomial of the form:
\begin{equation}\label{eq:restoring_polynomial}
F = \sum\limits_{i=1}^n -k_ix^i = - k_1x - k_2x^2 - \dots - k_{n}x^{n},
\end{equation}
where $k_n$ are constants which are to be determined, and $x$ is the displacement due to the deformation. For a symmetric material XXX, the restoring force should have the same magnitude when the deformation is in the positive or in the negative direction. Therefore, the restoring force must be an uneven function, so $k_i=0$ for even $i$.

For small displacements, the first term of the restoring force is dominant, and the restoring force can be approximated by Hooke's law choosing $k_i=0$ for $i\neq 1$:
\begin{equation}
F = k_1x
\end{equation}
As displacements become larger, the first term that becomes significant after $k_1$ is $k_3$. In this experiment the quartz crystals will be modeled by a system with a restoring force of the form of \autoref{eq:restoring_polynomial} with only $k_1$ and $k_3$ non-zero:
\begin{equation}\label{eq:nonlinear_restoring}
F = k_1x + k_3x^3.
\end{equation}

The differential equation for a system with \autoref{eq:nonlinear_restoring} as the restoring force is called the Duffing equation:
\begin{equation}\label{eq:duffing}
\ddot{x} + \delta \dot{x} + k_1x + k_3x^3 = g(\omega t),
\end{equation}
where $\delta$ determines the amount of damping, $k_1$ is the linear stiffness, $k_3$ is the amount of non-linearity in the restoring force and $g(\omega t)$ is a driving function with frequency $\omega$.  

\subsection{Solving the Duffing equation}
XXX some derivation of the frequency curve. 

\begin{figure}
	\centering
	\setlength\figureheight{5cm} 
	\setlength\figurewidth{0.8\textwidth}
	% This file was created by matlab2tikz.
% Minimal pgfplots version: 1.3
%
%The latest updates can be retrieved from
%  http://www.mathworks.com/matlabcentral/fileexchange/22022-matlab2tikz
%where you can also make suggestions and rate matlab2tikz.
%
\begin{tikzpicture}

\begin{axis}[%
width=0.95092\figurewidth,
height=\figureheight,
at={(0\figurewidth,0\figureheight)},
scale only axis,
every outer x axis line/.append style={black},
every x tick label/.append style={font=\color{black}},
xmin=0,
xmax=2,
xlabel={$\omega\text{/}\omega{}_\text{0}$},
every outer y axis line/.append style={black},
every y tick label/.append style={font=\color{black}},
ymin=0,
ymax=14,
ylabel={r},
axis x line*=bottom,
axis y line*=left,
legend style={legend cell align=left,align=left,fill=white}
]
\addplot [color=black,solid]
  table[row sep=crcr]{%
0.001	2.50000245000235\\
0.003001001001001	2.5000220649079\\
0.005002002002002	2.50006130053095\\
0.007003003003003	2.500120158681\\
0.009004004004004	2.5001986420727\\
0.011005005005005	2.50029675432608\\
0.013006006006006	2.50041449996704\\
0.015007007007007	2.50055188442781\\
0.017008008008008	2.50070891404758\\
0.019009009009009	2.50088559607323\\
0.02101001001001	2.50108193866012\\
0.023011011011011	2.50129795087301\\
0.025012012012012	2.50153364268712\\
0.027013013013013	2.50178902498922\\
0.029014014014014	2.50206410957888\\
0.031015015015015	2.5023589091698\\
0.033016016016016	2.50267343739123\\
0.035017017017017	2.50300770878955\\
0.037018018018018	2.50336173882988\\
0.039019019019019	2.50373554389787\\
0.04102002002002	2.50412914130149\\
0.043021021021021	2.50454254927308\\
0.045022022022022	2.50497578697137\\
0.047023023023023	2.50542887448362\\
0.049024024024024	2.505901832828\\
0.051025025025025	2.50639468395587\\
0.053026026026026	2.50690745075437\\
0.055027027027027	2.50744015704897\\
0.057028028028028	2.50799282760618\\
0.059029029029029	2.50856548813643\\
0.06103003003003	2.50915816529696\\
0.063031031031031	2.50977088669484\\
0.065032032032032	2.51040368089021\\
0.067033033033033	2.51105657739945\\
0.069034034034034	2.51172960669867\\
0.071035035035035	2.51242280022709\\
0.073036036036036	2.51313619039073\\
0.075037037037037	2.51386981056611\\
0.077038038038038	2.51462369510407\\
0.079039039039039	2.51539787933375\\
0.08104004004004	2.51619239956668\\
0.083041041041041	2.51700729310091\\
0.085042042042042	2.51784259822538\\
0.0870430430430431	2.51869835422434\\
0.089044044044044	2.51957460138185\\
0.091045045045045	2.52047138098655\\
0.093046046046046	2.52138873533634\\
0.0950470470470471	2.52232670774339\\
0.0970480480480481	2.52328534253918\\
0.0990490490490491	2.52426468507961\\
0.10105005005005	2.52526478175034\\
0.103051051051051	2.52628567997221\\
0.105052052052052	2.52732742820683\\
0.107053053053053	2.52839007596219\\
0.109054054054054	2.52947367379855\\
0.111055055055055	2.53057827333434\\
0.113056056056056	2.53170392725231\\
0.115057057057057	2.53285068930567\\
0.117058058058058	2.53401861432453\\
0.119059059059059	2.53520775822232\\
0.12106006006006	2.5364181780025\\
0.123061061061061	2.53764993176528\\
0.125062062062062	2.53890307871459\\
0.127063063063063	2.54017767916509\\
0.129064064064064	2.54147379454946\\
0.131065065065065	2.54279148742568\\
0.133066066066066	2.54413082148457\\
0.135067067067067	2.54549186155747\\
0.137068068068068	2.54687467362402\\
0.139069069069069	2.54827932482015\\
0.14107007007007	2.54970588344618\\
0.143071071071071	2.55115441897511\\
0.145072072072072	2.55262500206108\\
0.147073073073073	2.55411770454793\\
0.149074074074074	2.555632599478\\
0.151075075075075	2.55716976110103\\
0.153076076076076	2.55872926488329\\
0.155077077077077	2.56031118751683\\
0.157078078078078	2.56191560692892\\
0.159079079079079	2.56354260229165\\
0.16108008008008	2.56519225403174\\
0.163081081081081	2.56686464384051\\
0.165082082082082	2.568559854684\\
0.167083083083083	2.57027797081334\\
0.169084084084084	2.57201907777521\\
0.171085085085085	2.57378326242262\\
0.173086086086086	2.57557061292573\\
0.175087087087087	2.57738121878298\\
0.177088088088088	2.57921517083237\\
0.179089089089089	2.58107256126291\\
0.18109009009009	2.58295348362635\\
0.183091091091091	2.58485803284902\\
0.185092092092092	2.58678630524395\\
0.187093093093093	2.58873839852314\\
0.189094094094094	2.59071441181012\\
0.191095095095095	2.59271444565263\\
0.193096096096096	2.5947386020356\\
0.195097097097097	2.59678698439428\\
0.197098098098098	2.59885969762767\\
0.199099099099099	2.60095684811212\\
0.2011001001001	2.6030785437152\\
0.203101101101101	2.60522489380974\\
0.205102102102102	2.60739600928819\\
0.207103103103103	2.60959200257719\\
0.209104104104104	2.61181298765235\\
0.211105105105105	2.61405908005332\\
0.213106106106106	2.6163303968991\\
0.215107107107107	2.61862705690362\\
0.217108108108108	2.62094918039154\\
0.219109109109109	2.62329688931437\\
0.22111011011011	2.62567030726679\\
0.223111111111111	2.62806955950331\\
0.225112112112112	2.63049477295518\\
0.227113113113113	2.63294607624753\\
0.229114114114114	2.63542359971692\\
0.231115115115115	2.63792747542902\\
0.233116116116116	2.64045783719674\\
0.235117117117117	2.64301482059853\\
0.237118118118118	2.64559856299704\\
0.239119119119119	2.64820920355813\\
0.24112012012012	2.65084688327011\\
0.243121121121121	2.65351174496335\\
0.245122122122122	2.65620393333019\\
0.247123123123123	2.65892359494521\\
0.249124124124124	2.66167087828581\\
0.251125125125125	2.6644459337531\\
0.253126126126126	2.6672489136932\\
0.255127127127127	2.67007997241887\\
0.257128128128128	2.67293926623145\\
0.259129129129129	2.6758269534432\\
0.26113013013013	2.67874319440005\\
0.263131131131131	2.68168815150461\\
0.265132132132132	2.6846619892397\\
0.267133133133133	2.68766487419215\\
0.269134134134134	2.69069697507708\\
0.271135135135135	2.69375846276247\\
0.273136136136136	2.69684951029434\\
0.275137137137137	2.69997029292204\\
0.277138138138138	2.70312098812431\\
0.279139139139139	2.70630177563546\\
0.28114014014014	2.70951283747222\\
0.283141141141141	2.71275435796083\\
0.285142142142142	2.71602652376479\\
0.287143143143143	2.71932952391287\\
0.289144144144144	2.72266354982774\\
0.291145145145145	2.72602879535503\\
0.293146146146146	2.72942545679276\\
0.295147147147147	2.73285373292151\\
0.297148148148148	2.73631382503479\\
0.299149149149149	2.73980593697018\\
0.30115015015015	2.74333027514084\\
0.303151151151151	2.74688704856758\\
0.305152152152152	2.75047646891152\\
0.307153153153153	2.75409875050719\\
0.309154154154154	2.75775411039628\\
0.311155155155155	2.76144276836197\\
0.313156156156156	2.76516494696373\\
0.315157157157157	2.76892087157279\\
0.317158158158158	2.77271077040823\\
0.319159159159159	2.7765348745736\\
0.32116016016016	2.78039341809422\\
0.323161161161161	2.78428663795509\\
0.325162162162162	2.78821477413943\\
0.327163163163163	2.79217806966796\\
0.329164164164164	2.79617677063871\\
0.331165165165165	2.80021112626759\\
0.333166166166166	2.80428138892972\\
0.335167167167167	2.80838781420129\\
0.337168168168168	2.81253066090232\\
0.339169169169169	2.81671019114003\\
0.34117017017017	2.82092667035302\\
0.343171171171171	2.82518036735622\\
0.345172172172172	2.8294715543865\\
0.347173173173173	2.83380050714927\\
0.349174174174174	2.83816750486569\\
0.351175175175175	2.84257283032084\\
0.353176176176176	2.8470167699126\\
0.355177177177177	2.85149961370155\\
0.357178178178178	2.85602165546155\\
0.359179179179179	2.86058319273135\\
0.36118018018018	2.86518452686703\\
0.363181181181181	2.86982596309541\\
0.365182182182182	2.87450781056831\\
0.367183183183183	2.87923038241789\\
0.369184184184184	2.88399399581287\\
0.371185185185185	2.88879897201578\\
0.373186186186186	2.89364563644121\\
0.375187187187187	2.89853431871513\\
0.377188188188188	2.90346535273522\\
0.379189189189189	2.90843907673231\\
0.38119019019019	2.91345583333286\\
0.383191191191191	2.9185159696227\\
0.385192192192192	2.92361983721169\\
0.387193193193193	2.92876779229974\\
0.389194194194194	2.93396019574395\\
0.391195195195195	2.93919741312693\\
0.393196196196196	2.94447981482638\\
0.395197197197197	2.94980777608594\\
0.397198198198198	2.95518167708736\\
0.399199199199199	2.96060190302389\\
0.4012002002002	2.9660688441751\\
0.403201201201201	2.97158289598297\\
0.405202202202202	2.97714445912953\\
0.407203203203203	2.98275393961578\\
0.409204204204204	2.9884117488421\\
0.411205205205205	2.9941183036902\\
0.413206206206206	2.99987402660646\\
0.415207207207207	3.00567934568702\\
0.417208208208208	3.01153469476422\\
0.419209209209209	3.01744051349481\\
0.42121021021021	3.02339724744971\\
0.423211211211211	3.02940534820553\\
0.425212212212212	3.03546527343773\\
0.427213213213213	3.04157748701558\\
0.429214214214214	3.04774245909888\\
0.431215215215215	3.05396066623651\\
0.433216216216216	3.06023259146688\\
0.435217217217217	3.0665587244202\\
0.437218218218218	3.07293956142276\\
0.439219219219219	3.07937560560322\\
0.44122022022022	3.08586736700084\\
0.443221221221221	3.09241536267584\\
0.445222222222222	3.09902011682185\\
0.447223223223223	3.10568216088059\\
0.449224224224224	3.11240203365863\\
0.451225225225225	3.11918028144653\\
0.453226226226226	3.12601745814022\\
0.455227227227227	3.13291412536474\\
0.457228228228228	3.13987085260036\\
0.459229229229229	3.1468882173113\\
0.46123023023023	3.15396680507672\\
0.463231231231231	3.16110720972452\\
0.465232232232232	3.16831003346757\\
0.467233233233233	3.1755758870428\\
0.469234234234234	3.18290538985285\\
0.471235235235235	3.19029917011061\\
0.473236236236236	3.19775786498676\\
0.475237237237237	3.20528212075989\\
0.477238238238238	3.21287259297\\
0.479239239239239	3.22052994657479\\
0.48124024024024	3.22825485610916\\
0.483241241241241	3.23604800584797\\
0.485242242242242	3.24391008997198\\
0.487243243243243	3.25184181273714\\
0.489244244244244	3.25984388864739\\
0.491245245245245	3.26791704263082\\
0.493246246246246	3.27606201021953\\
0.495247247247247	3.284279537733\\
0.497248248248248	3.29257038246536\\
0.499249249249249	3.30093531287639\\
0.50125025025025	3.30937510878643\\
0.503251251251251	3.31789056157525\\
0.505252252252252	3.32648247438517\\
0.507253253253253	3.33515166232816\\
0.509254254254254	3.34389895269735\\
0.511255255255255	3.35272518518286\\
0.513256256256256	3.36163121209215\\
0.515257257257257	3.3706178985749\\
0.517258258258258	3.37968612285258\\
0.519259259259259	3.38883677645297\\
0.52126026026026	3.39807076444923\\
0.523261261261261	3.40738900570442\\
0.525262262262262	3.41679243312077\\
0.527263263263263	3.42628199389452\\
0.529264264264264	3.43585864977594\\
0.531265265265265	3.44552337733488\\
0.533266266266266	3.45527716823197\\
0.535267267267267	3.46512102949569\\
0.537268268268268	3.47505598380504\\
0.539269269269269	3.48508306977866\\
0.54127027027027	3.49520334226974\\
0.543271271271271	3.50541787266745\\
0.545272272272272	3.51572774920476\\
0.547273273273273	3.52613407727303\\
0.549274274274274	3.53663797974313\\
0.551275275275275	3.54724059729363\\
0.553276276276276	3.55794308874612\\
0.555277277277277	3.5687466314078\\
0.557278278278278	3.57965242142127\\
0.559279279279279	3.59066167412225\\
0.56128028028028	3.601775624405\\
0.563281281281281	3.61299552709543\\
0.565282282282282	3.62432265733273\\
0.567283283283283	3.63575831095934\\
0.569284284284284	3.64730380491903\\
0.571285285285285	3.6589604776643\\
0.573286286286286	3.67072968957236\\
0.575287287287287	3.68261282337035\\
0.577288288288288	3.69461128457\\
0.579289289289289	3.70672650191191\\
0.58129029029029	3.71895992781944\\
0.583291291291291	3.73131303886305\\
0.585292292292292	3.74378733623444\\
0.587293293293293	3.75638434623166\\
0.589294294294294	3.76910562075474\\
0.591295295295295	3.78195273781219\\
0.593296296296296	3.79492730203945\\
0.595297297297297	3.80803094522812\\
0.597298298298298	3.82126532686719\\
0.599299299299299	3.83463213469688\\
0.6013003003003	3.84813308527409\\
0.603301301301301	3.86176992455094\\
0.605302302302302	3.87554442846628\\
0.607303303303303	3.88945840354997\\
0.609304304304304	3.9035136875414\\
0.611305305305305	3.91771215002106\\
0.613306306306306	3.932055693057\\
0.615307307307307	3.94654625186515\\
0.617308308308308	3.96118579548476\\
0.619309309309309	3.97597632746888\\
0.62131031031031	3.99091988659031\\
0.623311311311311	4.00601854756315\\
0.625312312312312	4.02127442178083\\
0.627313313313313	4.03668965807004\\
0.629314314314314	4.05226644346232\\
0.631315315315315	4.06800700398205\\
0.633316316316316	4.08391360545233\\
0.635317317317317	4.09998855431877\\
0.637318318318318	4.11623419849152\\
0.639319319319319	4.13265292820579\\
0.64132032032032	4.14924717690192\\
0.643321321321321	4.16601942212392\\
0.645322322322322	4.1829721864388\\
0.647323323323323	4.20010803837528\\
0.649324324324324	4.21742959338345\\
0.651325325325325	4.2349395148152\\
0.653326326326326	4.25264051492606\\
0.655327327327327	4.2705353558983\\
0.657328328328328	4.28862685088665\\
0.659329329329329	4.30691786508615\\
0.66133033033033	4.32541131682245\\
0.663331331331331	4.34411017866606\\
0.665332332332332	4.36301747856886\\
0.667333333333333	4.38213630102603\\
0.669334334334334	4.40146978826041\\
0.671335335335335	4.42102114143266\\
0.673336336336336	4.44079362187527\\
0.675337337337337	4.46079055235276\\
0.677338338338338	4.48101531834559\\
0.679339339339339	4.50147136936095\\
0.68134034034034	4.52216222026922\\
0.683341341341341	4.54309145266518\\
0.685342342342342	4.56426271625738\\
0.687343343343343	4.58567973028173\\
0.689344344344344	4.60734628494284\\
0.691345345345345	4.62926624288143\\
0.693346346346346	4.6514435406672\\
0.695347347347347	4.67388219032016\\
0.697348348348348	4.69658628085562\\
0.699349349349349	4.71955997985739\\
0.70135035035035	4.74280753507511\\
0.703351351351351	4.76633327604883\\
0.705352352352352	4.79014161575683\\
0.707353353353353	4.81423705229\\
0.709354354354354	4.83862417054775\\
0.711355355355355	4.86330764396016\\
0.713356356356356	4.88829223623026\\
0.715357357357357	4.9135828030989\\
0.717358358358358	4.93918429413002\\
0.719359359359359	4.96510175451435\\
0.72136036036036	4.99134032689185\\
0.723361361361361	5.01790525319073\\
0.725362362362362	5.04480187647796\\
0.727363363363363	5.07203564282702\\
0.729364364364364	5.09961210318897\\
0.731365365365365	5.12753691527919\\
0.733366366366366	5.15581584546169\\
0.735367367367367	5.18445477063873\\
0.737368368368368	5.21345968014017\\
0.739369369369369	5.24283667760091\\
0.74137037037037	5.27259198283744\\
0.743371371371371	5.3027319337025\\
0.745372372372372	5.33326298792576\\
0.747373373373373	5.36419172492627\\
0.749374374374374	5.39552484759609\\
0.751375375375375	5.42726918404569\\
0.753376376376376	5.45943168930599\\
0.755377377377377	5.49201944697492\\
0.757378378378378	5.52503967080722\\
0.759379379379379	5.55849970623007\\
0.76138038038038	5.59240703178029\\
0.763381381381381	5.62676926044864\\
0.765382382382382	5.66159414092117\\
0.767383383383383	5.69688955870204\\
0.769384384384384	5.73266353710746\\
0.771385385385385	5.7689242381121\\
0.773386386386386	5.80567996302933\\
0.775387387387387	5.84293915301374\\
0.777388388388388	5.88071038936106\\
0.779389389389389	5.91900239358285\\
0.78139039039039	5.95782402723735\\
0.783391391391391	5.99718429149216\\
0.785392392392392	6.03709232637821\\
0.787393393393393	6.07755740973415\\
0.789394394394394	6.11858895577158\\
0.791395395395395	6.16019651326169\\
0.793396396396396	6.20238976328757\\
0.795397397397397	6.24517851652402\\
0.797398398398398	6.28857271000759\\
0.799399399399399	6.332582403346\\
0.8014004004004	6.3772177743221\\
0.803401401401401	6.42248911382897\\
0.805402402402402	6.46840682009352\\
0.807403403403404	6.51498139210963\\
0.809404404404404	6.56222342222854\\
0.811405405405405	6.61014358782399\\
0.813406406406406	6.65875264196464\\
0.815407407407407	6.70806140300604\\
0.817408408408409	6.75808074300773\\
0.819409409409409	6.80882157489464\\
0.82141041041041	6.86029483825177\\
0.823411411411412	6.91251148364897\\
0.825412412412412	6.96548245536436\\
0.827413413413413	7.01921867241333\\
0.829414414414414	7.07373100772099\\
0.831415415415415	7.1290302653115\\
0.833416416416417	7.18512715536501\\
0.835417417417417	7.24203226698078\\
0.837418418418418	7.29975603847885\\
0.839419419419419	7.35830872505479\\
0.84142042042042	7.41770036361905\\
0.843421421421422	7.47794073458756\\
0.845422422422422	7.53903932045709\\
0.847423423423423	7.60100526089576\\
0.849424424424424	7.66384730415605\\
0.851425425425425	7.72757375455474\\
0.853426426426427	7.79219241575516\\
0.855427427427428	7.85771052960111\\
0.857428428428428	7.92413471023035\\
0.859429429429429	7.99147087318689\\
0.86143043043043	8.0597241592121\\
0.863431431431432	8.12889885249231\\
0.865432432432433	8.1989982929643\\
0.867433433433433	8.27002478246736\\
0.869434434434434	8.34197948438091\\
0.871435435435435	8.41486231646764\\
0.873436436436437	8.48867183665135\\
0.875437437437438	8.5634051213824\\
0.877438438438438	8.63905763643172\\
0.879439439439439	8.71562309975367\\
0.88144044044044	8.79309333629322\\
0.883441441441442	8.87145812451081\\
0.885442442442442	8.95070503451771\\
0.887443443443444	9.03081925771576\\
0.889444444444444	9.11178342797752\\
0.891445445445445	9.19357743436221\\
0.893446446446447	9.27617822558661\\
0.895447447447447	9.35955960653918\\
0.897448448448449	9.44369202706533\\
0.899449449449449	9.52854236379384\\
0.90145045045045	9.61407369547584\\
0.903451451451452	9.70024507285272\\
0.905452452452452	9.78701128400775\\
0.907453453453454	9.87432261660523\\
0.909454454454455	9.96212461838527\\
0.911455455455455	10.0503578579392\\
0.913456456456457	10.1389576876601\\
0.915457457457457	10.2278540113336\\
0.917458458458459	10.3169710590765\\
0.91945945945946	10.4062271727636\\
0.92146046046046	10.4955346051997\\
0.923461461461462	10.5847993369698\\
0.925462462462462	10.6739209149588\\
0.927463463463464	10.7627923171802\\
0.929464464464465	10.8512998484955\\
0.931465465465466	10.9393230725991\\
0.933466466466467	11.0267347854282\\
0.935467467467467	11.1134010356037\\
0.937468468468469	11.1991811978245\\
0.93946946946947	11.2839281044723\\
0.941470470470471	11.3674882417126\\
0.943471471471471	11.4497020149887\\
0.945472472472472	11.5304040895198\\
0.947473473473474	11.6094238100348\\
0.949474474474475	11.6865857041026\\
0.951475475475476	11.7617100717345\\
0.953476476476476	11.8346136640317\\
0.955477477477477	11.9051104505958\\
0.957478478478479	11.9730124763913\\
0.95947947947948	12.0381308045863\\
0.961480480480481	12.1002765425381\\
0.963481481481481	12.1592619438712\\
0.965482482482482	12.2149015797725\\
0.967483483483484	12.2670135688865\\
0.969484484484485	12.3154208542264\\
0.971485485485486	12.359952512912\\
0.973486486486487	12.4004450836895\\
0.975487487487487	12.436743893853\\
0.977488488488489	12.4687043684429\\
0.97948948948949	12.496193301036\\
0.981490490490491	12.5190900668238\\
0.983491491491492	12.5372877579375\\
0.985492492492492	12.5506942212391\\
0.987493493493494	12.5592329800797\\
0.989494494494495	12.5628440234782\\
0.991495495495496	12.5614844469164\\
0.993496496496497	12.5551289328368\\
0.995497497497498	12.5437700610114\\
0.997498498498499	12.5274184422924\\
0.9994994994995	12.50610267228\\
1.0015005005005	12.4798691060267\\
1.0035015015015	12.4487814571021\\
1.0055025025025	12.412920228588\\
1.0075035035035	12.3723819859243\\
1.0095045045045	12.3272784856529\\
1.01150550550551	12.2777356754993\\
1.01350650650651	12.2238925826301\\
1.01550750750751	12.1659001107753\\
1.01750850850851	12.1039197647019\\
1.01950950950951	12.038122323687\\
1.02151051051051	11.9686864836474\\
1.02351151151151	11.895797487966\\
1.02551251251251	11.8196457655553\\
1.02751351351351	11.7404255938479\\
1.02951451451451	11.6583338019577\\
1.03151551551552	11.5735685280638\\
1.03351651651652	11.4863280430599\\
1.03551751751752	11.3968096492976\\
1.03751851851852	11.3052086631233\\
1.03951951951952	11.2117174854668\\
1.04152052052052	11.1165247652441\\
1.04352152152152	11.019814656318\\
1.04552252252252	10.9217661685473\\
1.04752352352352	10.8225526115214\\
1.04952452452452	10.7223411282137\\
1.05152552552553	10.6212923150799\\
1.05352652652653	10.5195599235753\\
1.05552752752753	10.4172906385155\\
1.05752852852853	10.3146239272891\\
1.05952952952953	10.2116919538198\\
1.06153053053053	10.1086195514886\\
1.06353153153153	10.0055242486462\\
1.06553253253253	9.9025163406335\\
1.06753353353353	9.79969900256065\\
1.06953453453453	9.69716843701563\\
1.07153553553554	9.59501405144045\\
1.07353653653654	9.49331866026031\\
1.07553753753754	9.39215870682885\\
1.07753853853854	9.29160450123625\\
1.07953953953954	9.19172047006689\\
1.08154054054054	9.09256541433424\\
1.08354154154154	8.99419277298434\\
1.08554254254254	8.89665088879972\\
1.08754354354354	8.79998327458902\\
1.08954454454454	8.70422887748561\\
1.09154554554555	8.60942233963095\\
1.09354654654655	8.51559425377808\\
1.09554754754755	8.42277141259471\\
1.09754854854855	8.33097705064992\\
1.09954954954955	8.24023107830567\\
1.10155055055055	8.15055030696877\\
1.10355155155155	8.06194866517555\\
1.10555255255255	7.97443740525717\\
1.10755355355355	7.88802530046443\\
1.10955455455455	7.80271883235731\\
1.11155555555556	7.71852236860193\\
1.11355655655656	7.63543833121474\\
1.11555755755756	7.55346735543724\\
1.11755855855856	7.47260843941148\\
1.11955955955956	7.39285908498153\\
1.12156056056056	7.31421542983781\\
1.12356156156156	7.23667237136495\\
1.12556256256256	7.16022368250111\\
1.12756356356356	7.08486211993805\\
1.12956456456456	7.01057952505057\\
1.13156556556557	6.93736691786163\\
1.13356656656657	6.86521458442344\\
1.13556756756757	6.79411215796336\\
1.13756856856857	6.7240486941027\\
1.13956956956957	6.65501274054201\\
1.14157057057057	6.58699240146119\\
1.14357157157157	6.51997539702447\\
1.14557257257257	6.45394911823579\\
1.14757357357357	6.38890067746403\\
1.14957457457457	6.32481695490518\\
1.15157557557558	6.26168464124965\\
1.15357657657658	6.19949027680335\\
1.15557757757758	6.13822028730202\\
1.15757857857858	6.07786101665122\\
1.15957957957958	6.01839875679322\\
1.16158058058058	5.95981977492079\\
1.16358158158158	5.90211033820168\\
1.16558258258258	5.84525673622316\\
1.16758358358358	5.78924530130031\\
1.16958458458458	5.73406242681547\\
1.17158558558559	5.6796945837353\\
1.17358658658659	5.62612833544201\\
1.17558758758759	5.57335035101027\\
1.17758858858859	5.52134741704325\\
1.17958958958959	5.47010644818857\\
1.18159059059059	5.41961449643421\\
1.18359159159159	5.36985875928021\\
1.18559259259259	5.32082658688013\\
1.18759359359359	5.27250548823312\\
1.18959459459459	5.22488313650746\\
1.1915955955956	5.1779473735684\\
1.1935965965966	5.13168621377482\\
1.1955975975976	5.08608784710694\\
1.1975985985986	5.04114064168761\\
1.1995995995996	4.99683314574626\\
1.2016006006006	4.95315408907676\\
1.2036016016016	4.91009238403322\\
1.2056026026026	4.86763712610912\\
1.2076036036036	4.8257775941357\\
1.2096046046046	4.78450325014067\\
1.21160560560561	4.74380373889051\\
1.21360660660661	4.70366888716076\\
1.21560760760761	4.66408870275055\\
1.21760860860861	4.62505337327337\\
1.21960960960961	4.58655326474966\\
1.22161061061061	4.54857892001717\\
1.22361161161161	4.51112105698329\\
1.22561261261261	4.47417056673925\\
1.22761361361361	4.43771851154921\\
1.22961461461461	4.40175612273552\\
1.23161561561562	4.36627479846553\\
1.23361661661662	4.33126610146409\\
1.23561761761762	4.29672175665573\\
1.23761861861862	4.26263364874986\\
1.23961961961962	4.2289938197801\\
1.24162062062062	4.19579446660583\\
1.24362162162162	4.16302793838325\\
1.24562262262262	4.13068673401703\\
1.24762362362362	4.09876349959391\\
1.24962462462462	4.06725102581058\\
1.25162562562563	4.03614224539654\\
1.25362662662663	4.00543023053957\\
1.25562762762763	3.97510819031847\\
1.25762862862863	3.94516946814524\\
1.25962962962963	3.91560753922216\\
1.26163063063063	3.88641600801798\\
1.26363163163163	3.85758860576195\\
1.26563263263263	3.82911918796467\\
1.26763363363363	3.80100173196093\\
1.26963463463463	3.7732303344835\\
1.27163563563564	3.74579920926446\\
1.27363663663664	3.71870268466694\\
1.27563763763764	3.69193520135284\\
1.27763863863864	3.66549130997884\\
1.27963963963964	3.6393656689318\\
1.28164064064064	3.61355304209601\\
1.28364164164164	3.58804829665858\\
1.28564264264264	3.56284640095014\\
1.28764364364364	3.53794242232309\\
1.28964464464464	3.51333152506664\\
1.29164564564565	3.48900896836001\\
1.29364664664665	3.4649701042618\\
1.29564764764765	3.44121037573873\\
1.29764864864865	3.41772531473012\\
1.29964964964965	3.39451054025102\\
1.30165065065065	3.37156175653186\\
1.30365165165165	3.34887475119541\\
1.30565265265265	3.32644539346989\\
1.30765365365365	3.30426963243971\\
1.30965465465465	3.28234349533027\\
1.31165565565566	3.26066308583036\\
1.31365665665666	3.23922458244813\\
1.31565765765766	3.21802423690316\\
1.31765865865866	3.19705837255156\\
1.31965965965966	3.17632338284558\\
1.32166066066066	3.1558157298263\\
1.32366166166166	3.13553194264927\\
1.32566266266266	3.11546861614169\\
1.32766366366366	3.09562240939224\\
1.32966466466466	3.07599004437071\\
1.33166566566567	3.0565683045799\\
1.33366666666667	3.03735403373575\\
1.33566766766767	3.01834413447813\\
1.33766866866867	2.99953556710989\\
1.33966966966967	2.9809253483647\\
1.34167067067067	2.962510550202\\
1.34367167167167	2.94428829862976\\
1.34567267267267	2.92625577255343\\
1.34767367367367	2.90841020265139\\
1.34967467467467	2.89074887027533\\
1.35167567567568	2.87326910637651\\
1.35367667667668	2.85596829045532\\
1.35567767767768	2.83884384953605\\
1.35767867867868	2.82189325716401\\
1.35967967967968	2.80511403242618\\
1.36168068068068	2.78850373899386\\
1.36368168168168	2.77205998418756\\
1.36568268268268	2.75578041806295\\
1.36768368368368	2.73966273251794\\
1.36968468468468	2.72370466042011\\
1.37168568568569	2.70790397475397\\
1.37368668668669	2.69225848778806\\
1.37568768768769	2.67676605026078\\
1.37768868868869	2.66142455058493\\
1.37968968968969	2.64623191407058\\
1.38169069069069	2.63118610216528\\
1.38369169169169	2.61628511171224\\
1.38569269269269	2.60152697422481\\
1.38769369369369	2.58690975517777\\
1.38969469469469	2.57243155331497\\
1.3916956956957	2.5580904999722\\
1.3936966966967	2.54388475841592\\
1.3956976976977	2.52981252319673\\
1.3976986986987	2.51587201951763\\
1.3996996996997	2.5020615026165\\
1.4017007007007	2.48837925716262\\
1.4037017017017	2.47482359666676\\
1.4057027027027	2.46139286290457\\
1.4077037037037	2.44808542535301\\
1.4097047047047	2.43489968063941\\
1.41170570570571	2.42183405200285\\
1.41370670670671	2.40888698876778\\
1.41570770770771	2.39605696582924\\
1.41770870870871	2.38334248314964\\
1.41970970970971	2.37074206526675\\
1.42171071071071	2.35825426081269\\
1.42371171171171	2.34587764204367\\
1.42571271271271	2.33361080437985\\
1.42771371371371	2.32145236595579\\
1.42971471471471	2.30940096718055\\
1.43171571571572	2.29745527030766\\
1.43371671671672	2.28561395901443\\
1.43571771771772	2.27387573799048\\
1.43771871871872	2.26223933253562\\
1.43971971971972	2.25070348816589\\
1.44172072072072	2.23926697022891\\
1.44372172172172	2.22792856352708\\
1.44572272272272	2.21668707194923\\
1.44772372372372	2.20554131811023\\
1.44972472472472	2.19449014299827\\
1.45172572572573	2.18353240562995\\
1.45372672672673	2.17266698271262\\
1.45572772772773	2.16189276831415\\
1.45772872872873	2.1512086735396\\
1.45972972972973	2.14061362621501\\
1.46173073073073	2.13010657057774\\
1.46373173173173	2.11968646697356\\
1.46573273273273	2.10935229156012\\
1.46773373373373	2.0991030360167\\
1.46973473473473	2.08893770726014\\
1.47173573573574	2.07885532716666\\
1.47373673673674	2.06885493229974\\
1.47573773773774	2.05893557364345\\
1.47773873873874	2.0490963163416\\
1.47973973973974	2.03933623944227\\
1.48174074074074	2.02965443564753\\
1.48374174174174	2.02005001106862\\
1.48574274274274	2.01052208498596\\
1.48774374374374	2.00106978961437\\
1.48974474474474	1.99169226987288\\
1.49174574574575	1.98238868315952\\
1.49374674674675	1.97315819913055\\
1.49574774774775	1.96399999948435\\
1.49774874874875	1.95491327774967\\
1.49974974974975	1.94589723907819\\
1.50175075075075	1.93695110004131\\
1.50375175175175	1.92807408843109\\
1.50575275275275	1.91926544306516\\
1.50775375375375	1.9105244135957\\
1.50975475475475	1.90185026032204\\
1.51175575575576	1.89324225400728\\
1.51375675675676	1.88469967569841\\
1.51575775775776	1.87622181655003\\
1.51775875875876	1.86780797765179\\
1.51975975975976	1.85945746985894\\
1.52176076076076	1.85116961362649\\
1.52376176176176	1.84294373884659\\
1.52576276276276	1.83477918468903\\
1.52776376376376	1.826675299445\\
1.52976476476476	1.81863144037386\\
1.53176576576577	1.81064697355291\\
1.53376676676677	1.80272127373008\\
1.53576776776777	1.79485372417958\\
1.53776876876877	1.78704371656022\\
1.53976976976977	1.77929065077654\\
1.54177077077077	1.77159393484272\\
1.54377177177177	1.76395298474891\\
1.54577277277277	1.75636722433031\\
1.54777377377377	1.74883608513869\\
1.54977477477477	1.74135900631637\\
1.55177577577578	1.73393543447262\\
1.55377677677678	1.72656482356236\\
1.55577777777778	1.71924663476728\\
1.55777877877878	1.71198033637906\\
1.55977977977978	1.70476540368496\\
1.56178078078078	1.69760131885542\\
1.56378178178178	1.69048757083385\\
1.56578278278278	1.68342365522846\\
1.56778378378378	1.67640907420615\\
1.56978478478478	1.66944333638834\\
1.57178578578579	1.66252595674873\\
1.57378678678679	1.655656456513\\
1.57578778778779	1.64883436306032\\
1.57778878878879	1.64205920982674\\
1.57978978978979	1.63533053621028\\
1.58179079079079	1.6286478874778\\
1.58379179179179	1.62201081467361\\
1.58579279279279	1.61541887452973\\
1.58779379379379	1.60887162937769\\
1.58979479479479	1.60236864706211\\
1.5917957957958	1.59590950085572\\
1.5937967967968	1.58949376937593\\
1.5957977977978	1.58312103650296\\
1.5977987987988	1.57679089129938\\
1.5997997997998	1.57050292793114\\
1.6018008008008	1.56425674559\\
1.6038018018018	1.55805194841731\\
1.6058028028028	1.55188814542923\\
1.6078038038038	1.5457649504432\\
1.6098048048048	1.53968198200572\\
1.61180580580581	1.53363886332145\\
1.61380680680681	1.52763522218356\\
1.61580780780781	1.52167069090524\\
1.61780880880881	1.51574490625247\\
1.61980980980981	1.50985750937796\\
1.62181081081081	1.5040081457562\\
1.62381181181181	1.49819646511973\\
1.62581281281281	1.49242212139641\\
1.62781381381381	1.48668477264784\\
1.62981481481481	1.48098408100885\\
1.63181581581582	1.47531971262799\\
1.63381681681682	1.46969133760908\\
1.63581781781782	1.46409862995374\\
1.63781881881882	1.45854126750492\\
1.63981981981982	1.45301893189138\\
1.64182082082082	1.44753130847316\\
1.64382182182182	1.44207808628784\\
1.64582282282282	1.43665895799792\\
1.64782382382382	1.43127361983892\\
1.64982482482482	1.4259217715684\\
1.65182582582583	1.42060311641586\\
1.65382682682683	1.41531736103351\\
1.65582782782783	1.41006421544775\\
1.65782882882883	1.40484339301158\\
1.65982982982983	1.39965461035774\\
1.66183083083083	1.39449758735263\\
1.66383183183183	1.38937204705105\\
1.66583283283283	1.38427771565156\\
1.66783383383383	1.37921432245276\\
1.66983483483483	1.3741815998101\\
1.67183583583584	1.36917928309354\\
1.67383683683684	1.36420711064584\\
1.67583783783784	1.3592648237415\\
1.67783883883884	1.35435216654647\\
1.67983983983984	1.34946888607841\\
1.68184084084084	1.34461473216767\\
1.68384184184184	1.33978945741882\\
1.68584284284284	1.33499281717289\\
1.68784384384384	1.33022456947017\\
1.68984484484484	1.3254844750136\\
1.69184584584585	1.32077229713276\\
1.69384684684685	1.31608780174844\\
1.69584784784785	1.31143075733777\\
1.69784884884885	1.30680093489991\\
1.69984984984985	1.30219810792226\\
1.70185085085085	1.29762205234721\\
1.70385185185185	1.29307254653949\\
1.70585285285285	1.28854937125389\\
1.70785385385385	1.28405230960362\\
1.70985485485485	1.27958114702911\\
1.71185585585586	1.27513567126726\\
1.71385685685686	1.27071567232127\\
1.71585785785786	1.26632094243083\\
1.71785885885886	1.26195127604287\\
1.71985985985986	1.25760646978267\\
1.72186086086086	1.25328632242549\\
1.72386186186186	1.24899063486862\\
1.72586286286286	1.2447192101038\\
1.72786386386386	1.24047185319018\\
1.72986486486486	1.23624837122758\\
1.73186586586587	1.23204857333021\\
1.73386686686687	1.22787227060082\\
1.73586786786787	1.22371927610517\\
1.73786886886887	1.21958940484693\\
1.73986986986987	1.21548247374298\\
1.74187087087087	1.21139830159904\\
1.74387187187187	1.20733670908569\\
1.74587287287287	1.20329751871475\\
1.74787387387387	1.19928055481599\\
1.74987487487487	1.19528564351424\\
1.75187587587588	1.19131261270679\\
1.75387687687688	1.18736129204116\\
1.75587787787788	1.18343151289316\\
1.75787887887888	1.17952310834532\\
1.75987987987988	1.17563591316565\\
1.76188088088088	1.17176976378663\\
1.76388188188188	1.16792449828458\\
1.76588288288288	1.16409995635936\\
1.76788388388388	1.16029597931425\\
1.76988488488488	1.15651241003628\\
1.77188588588589	1.15274909297669\\
1.77388688688689	1.14900587413181\\
1.77588788788789	1.14528260102411\\
1.77788888888889	1.14157912268361\\
1.77988988988989	1.13789528962949\\
1.78189089089089	1.13423095385203\\
1.78389189189189	1.13058596879474\\
1.78589289289289	1.12696018933682\\
1.78789389389389	1.12335347177579\\
1.78989489489489	1.11976567381046\\
1.7918958958959	1.11619665452407\\
1.7938968968969	1.11264627436768\\
1.7958978978979	1.10911439514385\\
1.7978988988989	1.1056008799905\\
1.7998998998999	1.10210559336499\\
1.8019009009009	1.09862840102846\\
1.8039019019019	1.0951691700304\\
1.8059029029029	1.09172776869336\\
1.8079039039039	1.08830406659799\\
1.8099049049049	1.08489793456818\\
1.81190590590591	1.08150924465647\\
1.81390690690691	1.07813787012965\\
1.81590790790791	1.07478368545456\\
1.81790890890891	1.07144656628407\\
1.81990990990991	1.0681263894433\\
1.82191091091091	1.06482303291596\\
1.82391191191191	1.06153637583094\\
1.82591291291291	1.05826629844907\\
1.82791391391391	1.05501268215002\\
1.82991491491491	1.05177540941945\\
1.83191591591592	1.04855436383627\\
1.83391691691692	1.04534943006013\\
1.83591791791792	1.04216049381902\\
1.83791891891892	1.0389874418971\\
1.83991991991992	1.03583016212264\\
1.84192092092092	1.03268854335617\\
1.84392192192192	1.02956247547876\\
1.84592292292292	1.02645184938044\\
1.84792392392392	1.02335655694887\\
1.84992492492492	1.02027649105801\\
1.85192592592593	1.01721154555709\\
1.85392692692693	1.01416161525963\\
1.85592792792793	1.01112659593263\\
1.85792892892893	1.00810638428594\\
1.85992992992993	1.00510087796175\\
1.86193093093093	1.00210997552417\\
1.86393193193193	0.99913357644901\\
1.86593293293293	0.996171581113695\\
1.86793393393393	0.993223890787267\\
1.86993493493493	0.990290407620553\\
1.87193593593594	0.987371034636452\\
1.87393693693694	0.984465675720353\\
1.87593793793794	0.981574235610665\\
1.87793893893894	0.978696619889492\\
1.87993993993994	0.975832734973407\\
1.88194094094094	0.972982488104361\\
1.88394194194194	0.970145787340704\\
1.88594294294294	0.967322541548319\\
1.88794394394394	0.964512660391879\\
1.88994494494494	0.961716054326205\\
1.89194594594595	0.958932634587748\\
1.89394694694695	0.956162313186166\\
1.89594794794795	0.953405002896022\\
1.89794894894895	0.950660617248576\\
1.89994994994995	0.947929070523692\\
1.90195095095095	0.945210277741841\\
1.90395195195195	0.942504154656209\\
1.90595295295295	0.939810617744902\\
1.90795395395395	0.937129584203252\\
1.90995495495495	0.934460971936218\\
1.91195595595596	0.931804699550883\\
1.91395695695696	0.929160686349049\\
1.91595795795796	0.926528852319921\\
1.91795895895896	0.923909118132877\\
1.91995995995996	0.921301405130343\\
1.92196096096096	0.918705635320743\\
1.92396196196196	0.916121731371545\\
1.92596296296296	0.913549616602386\\
1.92796396396396	0.910989214978288\\
1.92996496496496	0.908440451102961\\
1.93196596596597	0.905903250212175\\
1.93396696696697	0.903377538167229\\
1.93596796796797	0.900863241448493\\
1.93796896896897	0.898360287149026\\
1.93996996996997	0.895868602968279\\
1.94197097097097	0.893388117205873\\
1.94397197197197	0.89091875875545\\
1.94597297297297	0.888460457098597\\
1.94797397397397	0.886013142298859\\
1.94997497497497	0.883576744995797\\
1.95197597597598	0.881151196399148\\
1.95397697697698	0.878736428283035\\
1.95597797797798	0.876332372980253\\
1.95797897897898	0.873938963376624\\
1.95997997997998	0.871556132905424\\
1.96198098098098	0.869183815541868\\
1.96398198198198	0.866821945797668\\
1.96598298298298	0.864470458715652\\
1.96798398398398	0.862129289864449\\
1.96998498498498	0.85979837533324\\
1.97198598598599	0.857477651726563\\
1.97398698698699	0.855167056159188\\
1.97598798798799	0.852866526251052\\
1.97798898898899	0.850576000122244\\
1.97998998998999	0.848295416388064\\
1.98199099099099	0.846024714154126\\
1.98399199199199	0.843763833011532\\
1.98599299299299	0.841512713032087\\
1.98799399399399	0.839271294763586\\
1.98999499499499	0.837039519225144\\
1.991995995996	0.834817327902588\\
1.993996996997	0.832604662743897\\
1.995997997998	0.830401466154698\\
1.997998998999	0.828207680993813\\
2	0.826023250568862\\
1.997998998999	0.828207680993813\\
1.995997997998	0.830401466154698\\
1.993996996997	0.832604662743897\\
1.991995995996	0.834817327902588\\
1.98999499499499	0.837039519225144\\
1.98799399399399	0.839271294763586\\
1.98599299299299	0.841512713032087\\
1.98399199199199	0.843763833011532\\
1.98199099099099	0.846024714154126\\
1.97998998998999	0.848295416388064\\
1.97798898898899	0.850576000122244\\
1.97598798798799	0.852866526251052\\
1.97398698698699	0.855167056159189\\
1.97198598598599	0.857477651726563\\
1.96998498498498	0.85979837533324\\
1.96798398398398	0.862129289864449\\
1.96598298298298	0.864470458715652\\
1.96398198198198	0.866821945797668\\
1.96198098098098	0.869183815541868\\
1.95997997997998	0.871556132905424\\
1.95797897897898	0.873938963376624\\
1.95597797797798	0.876332372980253\\
1.95397697697698	0.878736428283035\\
1.95197597597598	0.881151196399148\\
1.94997497497497	0.883576744995797\\
1.94797397397397	0.886013142298859\\
1.94597297297297	0.888460457098598\\
1.94397197197197	0.890918758755449\\
1.94197097097097	0.893388117205873\\
1.93996996996997	0.895868602968279\\
1.93796896896897	0.898360287149026\\
1.93596796796797	0.900863241448493\\
1.93396696696697	0.903377538167229\\
1.93196596596597	0.905903250212175\\
1.92996496496496	0.908440451102961\\
1.92796396396396	0.910989214978288\\
1.92596296296296	0.913549616602386\\
1.92396196196196	0.916121731371545\\
1.92196096096096	0.918705635320744\\
1.91995995995996	0.921301405130343\\
1.91795895895896	0.923909118132877\\
1.91595795795796	0.926528852319921\\
1.91395695695696	0.929160686349049\\
1.91195595595596	0.931804699550883\\
1.90995495495495	0.934460971936218\\
1.90795395395395	0.937129584203252\\
1.90595295295295	0.939810617744902\\
1.90395195195195	0.942504154656209\\
1.90195095095095	0.945210277741841\\
1.89994994994995	0.947929070523691\\
1.89794894894895	0.950660617248576\\
1.89594794794795	0.953405002896022\\
1.89394694694695	0.956162313186166\\
1.89194594594595	0.958932634587748\\
1.88994494494494	0.961716054326206\\
1.88794394394394	0.964512660391879\\
1.88594294294294	0.967322541548319\\
1.88394194194194	0.970145787340703\\
1.88194094094094	0.972982488104361\\
1.87993993993994	0.975832734973407\\
1.87793893893894	0.978696619889493\\
1.87593793793794	0.981574235610666\\
1.87393693693694	0.984465675720353\\
1.87193593593594	0.987371034636452\\
1.86993493493493	0.990290407620553\\
1.86793393393393	0.993223890787267\\
1.86593293293293	0.996171581113695\\
1.86393193193193	0.99913357644901\\
1.86193093093093	1.00210997552417\\
1.85992992992993	1.00510087796175\\
1.85792892892893	1.00810638428594\\
1.85592792792793	1.01112659593263\\
1.85392692692693	1.01416161525963\\
1.85192592592593	1.01721154555709\\
1.84992492492492	1.02027649105802\\
1.84792392392392	1.02335655694887\\
1.84592292292292	1.02645184938044\\
1.84392192192192	1.02956247547876\\
1.84192092092092	1.03268854335617\\
1.83991991991992	1.03583016212264\\
1.83791891891892	1.0389874418971\\
1.83591791791792	1.04216049381902\\
1.83391691691692	1.04534943006013\\
1.83191591591592	1.04855436383627\\
1.82991491491491	1.05177540941945\\
1.82791391391391	1.05501268215002\\
1.82591291291291	1.05826629844907\\
1.82391191191191	1.06153637583094\\
1.82191091091091	1.06482303291596\\
1.81990990990991	1.0681263894433\\
1.81790890890891	1.07144656628407\\
1.81590790790791	1.07478368545456\\
1.81390690690691	1.07813787012965\\
1.81190590590591	1.08150924465647\\
1.8099049049049	1.08489793456818\\
1.8079039039039	1.08830406659799\\
1.8059029029029	1.09172776869336\\
1.8039019019019	1.0951691700304\\
1.8019009009009	1.09862840102846\\
1.7998998998999	1.10210559336499\\
1.7978988988989	1.1056008799905\\
1.7958978978979	1.10911439514385\\
1.7938968968969	1.11264627436768\\
1.7918958958959	1.11619665452407\\
1.78989489489489	1.11976567381046\\
1.78789389389389	1.12335347177579\\
1.78589289289289	1.12696018933682\\
1.78389189189189	1.13058596879474\\
1.78189089089089	1.13423095385203\\
1.77988988988989	1.13789528962949\\
1.77788888888889	1.14157912268361\\
1.77588788788789	1.14528260102411\\
1.77388688688689	1.14900587413181\\
1.77188588588589	1.15274909297669\\
1.76988488488488	1.15651241003628\\
1.76788388388388	1.16029597931425\\
1.76588288288288	1.16409995635936\\
1.76388188188188	1.16792449828458\\
1.76188088088088	1.17176976378663\\
1.75987987987988	1.17563591316565\\
1.75787887887888	1.17952310834533\\
1.75587787787788	1.18343151289316\\
1.75387687687688	1.18736129204116\\
1.75187587587588	1.19131261270679\\
1.74987487487487	1.19528564351424\\
1.74787387387387	1.19928055481599\\
1.74587287287287	1.20329751871475\\
1.74387187187187	1.20733670908569\\
1.74187087087087	1.21139830159904\\
1.73986986986987	1.21548247374298\\
1.73786886886887	1.21958940484693\\
1.73586786786787	1.22371927610517\\
1.73386686686687	1.22787227060082\\
1.73186586586587	1.23204857333021\\
1.72986486486486	1.23624837122758\\
1.72786386386386	1.24047185319018\\
1.72586286286286	1.2447192101038\\
1.72386186186186	1.24899063486862\\
1.72186086086086	1.25328632242549\\
1.71985985985986	1.25760646978267\\
1.71785885885886	1.26195127604287\\
1.71585785785786	1.26632094243083\\
1.71385685685686	1.27071567232127\\
1.71185585585586	1.27513567126726\\
1.70985485485485	1.27958114702911\\
1.70785385385385	1.28405230960362\\
1.70585285285285	1.28854937125389\\
1.70385185185185	1.29307254653949\\
1.70185085085085	1.29762205234721\\
1.69984984984985	1.30219810792226\\
1.69784884884885	1.30680093489991\\
1.69584784784785	1.31143075733777\\
1.69384684684685	1.31608780174844\\
1.69184584584585	1.32077229713276\\
1.68984484484484	1.3254844750136\\
1.68784384384384	1.33022456947017\\
1.68584284284284	1.33499281717289\\
1.68384184184184	1.33978945741882\\
1.68184084084084	1.34461473216767\\
1.67983983983984	1.34946888607841\\
1.67783883883884	1.35435216654647\\
1.67583783783784	1.3592648237415\\
1.67383683683684	1.36420711064584\\
1.67183583583584	1.36917928309354\\
1.66983483483483	1.3741815998101\\
1.66783383383383	1.37921432245276\\
1.66583283283283	1.38427771565156\\
1.66383183183183	1.38937204705105\\
1.66183083083083	1.39449758735263\\
1.65982982982983	1.39965461035774\\
1.65782882882883	1.40484339301158\\
1.65582782782783	1.41006421544775\\
1.65382682682683	1.41531736103351\\
1.65182582582583	1.42060311641586\\
1.64982482482482	1.4259217715684\\
1.64782382382382	1.43127361983892\\
1.64582282282282	1.43665895799792\\
1.64382182182182	1.44207808628784\\
1.64182082082082	1.44753130847315\\
1.63981981981982	1.45301893189138\\
1.63781881881882	1.45854126750492\\
1.63581781781782	1.46409862995374\\
1.63381681681682	1.46969133760908\\
1.63181581581582	1.47531971262799\\
1.62981481481481	1.48098408100885\\
1.62781381381381	1.48668477264784\\
1.62581281281281	1.49242212139641\\
1.62381181181181	1.49819646511973\\
1.62181081081081	1.5040081457562\\
1.61980980980981	1.50985750937796\\
1.61780880880881	1.51574490625247\\
1.61580780780781	1.52167069090524\\
1.61380680680681	1.52763522218356\\
1.61180580580581	1.53363886332145\\
1.6098048048048	1.53968198200572\\
1.6078038038038	1.5457649504432\\
1.6058028028028	1.55188814542923\\
1.6038018018018	1.55805194841731\\
1.6018008008008	1.56425674558999\\
1.5997997997998	1.57050292793114\\
1.5977987987988	1.57679089129938\\
1.5957977977978	1.58312103650296\\
1.5937967967968	1.58949376937594\\
1.5917957957958	1.59590950085572\\
1.58979479479479	1.60236864706211\\
1.58779379379379	1.60887162937769\\
1.58579279279279	1.61541887452973\\
1.58379179179179	1.62201081467362\\
1.58179079079079	1.6286478874778\\
1.57978978978979	1.63533053621028\\
1.57778878878879	1.64205920982674\\
1.57578778778779	1.64883436306032\\
1.57378678678679	1.655656456513\\
1.57178578578579	1.66252595674873\\
1.56978478478478	1.66944333638834\\
1.56778378378378	1.67640907420615\\
1.56578278278278	1.68342365522846\\
1.56378178178178	1.69048757083385\\
1.56178078078078	1.69760131885542\\
1.55977977977978	1.70476540368496\\
1.55777877877878	1.71198033637906\\
1.55577777777778	1.71924663476727\\
1.55377677677678	1.72656482356236\\
1.55177577577578	1.73393543447262\\
1.54977477477477	1.74135900631638\\
1.54777377377377	1.74883608513869\\
1.54577277277277	1.75636722433031\\
1.54377177177177	1.76395298474891\\
1.54177077077077	1.77159393484272\\
1.53976976976977	1.77929065077655\\
1.53776876876877	1.78704371656022\\
1.53576776776777	1.79485372417958\\
1.53376676676677	1.80272127373008\\
1.53176576576577	1.8106469735529\\
1.52976476476476	1.81863144037385\\
1.52776376376376	1.82667529944499\\
1.52576276276276	1.83477918468902\\
1.52376176176176	1.84294373884659\\
1.52176076076076	1.85116961362649\\
1.51975975975976	1.85945746985894\\
1.51775875875876	1.86780797765179\\
1.51575775775776	1.87622181655003\\
1.51375675675676	1.8846996756984\\
1.51175575575576	1.89324225400729\\
1.50975475475475	1.90185026032205\\
1.50775375375375	1.9105244135957\\
1.50575275275275	1.91926544306517\\
1.50375175175175	1.92807408843108\\
1.50175075075075	1.93695110004131\\
1.49974974974975	1.94589723907819\\
1.49774874874875	1.95491327774967\\
1.49574774774775	1.96399999948435\\
1.49374674674675	1.97315819913055\\
1.49174574574575	1.98238868315952\\
1.48974474474474	1.99169226987288\\
1.48774374374374	2.00106978961437\\
1.48574274274274	2.01052208498596\\
1.48374174174174	2.02005001106861\\
1.48174074074074	2.02965443564753\\
1.47973973973974	2.03933623944227\\
1.47773873873874	2.04909631634161\\
1.47573773773774	2.05893557364345\\
1.47373673673674	2.06885493229974\\
1.47173573573574	2.07885532716667\\
1.46973473473473	2.08893770726013\\
1.46773373373373	2.0991030360167\\
1.46573273273273	2.10935229156011\\
1.46373173173173	2.11968646697355\\
1.46173073073073	2.13010657057773\\
1.45972972972973	2.14061362621501\\
1.45772872872873	2.15120867353961\\
1.45572772772773	2.16189276831415\\
1.45372672672673	2.17266698271262\\
1.45172572572573	2.18353240562994\\
1.44972472472472	2.19449014299827\\
1.44772372372372	2.20554131811023\\
1.44572272272272	2.21668707194923\\
1.44372172172172	2.22792856352707\\
1.44172072072072	2.23926697022891\\
1.43971971971972	2.25070348816589\\
1.43771871871872	2.26223933253561\\
1.43571771771772	2.27387573799049\\
1.43371671671672	2.28561395901441\\
1.43171571571572	2.29745527030766\\
1.42971471471471	2.30940096718055\\
1.42771371371371	2.32145236595578\\
1.42571271271271	2.33361080437985\\
1.42371171171171	2.34587764204368\\
1.42171071071071	2.35825426081271\\
1.41970970970971	2.37074206526674\\
1.41770870870871	2.38334248314963\\
1.41570770770771	2.39605696582924\\
1.41370670670671	2.40888698876779\\
1.41170570570571	2.42183405200284\\
1.4097047047047	2.4348996806394\\
1.4077037037037	2.44808542535302\\
1.4057027027027	2.46139286290459\\
1.4037017017017	2.47482359666678\\
1.4017007007007	2.48837925716264\\
1.3996996996997	2.50206150261651\\
1.3976986986987	2.51587201951763\\
1.3956976976977	2.52981252319673\\
1.3936966966967	2.54388475841592\\
1.3916956956957	2.55809049997221\\
1.38969469469469	2.57243155331497\\
1.38769369369369	2.58690975517777\\
1.38569269269269	2.60152697422478\\
1.38369169169169	2.61628511171224\\
1.38169069069069	2.63118610216528\\
1.37968968968969	2.64623191407057\\
1.37768868868869	2.66142455058495\\
1.37568768768769	2.67676605026078\\
1.37368668668669	2.69225848778807\\
1.37168568568569	2.70790397475397\\
1.36968468468468	2.7237046604201\\
1.36768368368368	2.73966273251796\\
1.36568268268268	2.75578041806296\\
1.36368168168168	2.77205998418755\\
1.36168068068068	2.78850373899385\\
1.35967967967968	2.80511403242617\\
1.35767867867868	2.82189325716402\\
1.35567767767768	2.83884384953606\\
1.35367667667668	2.85596829045533\\
1.35167567567568	2.8732691063765\\
1.34967467467467	2.89074887027537\\
1.34767367367367	2.90841020265138\\
1.34567267267267	2.92625577255344\\
1.34367167167167	2.94428829862974\\
1.34167067067067	2.96251055020198\\
1.33966966966967	2.98092534836468\\
1.33766866866867	2.99953556710989\\
1.33566766766767	3.0183441344781\\
1.33366666666667	3.03735403373576\\
1.33166566566567	3.05656830457992\\
1.32966466466466	3.07599004437074\\
1.32766366366366	3.0956224093922\\
1.32566266266266	3.11546861614172\\
1.32366166166166	3.13553194264924\\
1.32166066066066	3.15581572982629\\
1.31965965965966	3.17632338284553\\
1.31765865865866	3.19705837255155\\
1.31565765765766	3.2180242369032\\
1.31365665665666	3.23922458244817\\
1.31165565565566	3.26066308583032\\
1.30965465465465	3.28234349533026\\
1.30765365365365	3.30426963243966\\
1.30565265265265	3.3264453934699\\
1.30365165165165	3.34887475119533\\
1.30165065065065	3.37156175653186\\
1.29964964964965	3.39451054025103\\
1.29764864864865	3.41772531473013\\
1.29564764764765	3.44121037573874\\
1.29364664664665	3.46497010426184\\
1.29164564564565	3.48900896836\\
1.28964464464464	3.5133315250667\\
1.28764364364364	3.53794242232308\\
1.28564264264264	3.56284640095016\\
1.28364164164164	3.5880482966586\\
1.28164064064064	3.61355304209605\\
1.27963963963964	3.63936566893183\\
1.27763863863864	3.66549130997892\\
1.27563763763764	3.69193520135283\\
1.27363663663664	3.71870268466703\\
1.27163563563564	3.74579920926432\\
1.26963463463463	3.77323033448356\\
1.26763363363363	3.80100173196089\\
1.26563263263263	3.82911918796453\\
1.26363163163163	3.8575886057619\\
1.26163063063063	3.88641600801781\\
1.25962962962963	3.91560753922214\\
1.25762862862863	3.94516946814515\\
1.25562762762763	3.97510819031854\\
1.25362662662663	4.00543023053963\\
1.25162562562563	4.0361422453965\\
1.24962462462462	4.06725102581058\\
1.24762362362362	4.09876349959393\\
1.24562262262262	4.13068673401703\\
1.24362162162162	4.16302793838335\\
1.24162062062062	4.19579446660579\\
1.23961961961962	4.22899381978014\\
1.23761861861862	4.26263364874982\\
1.23561761761762	4.29672175665571\\
1.23361661661662	4.33126610146411\\
1.23161561561562	4.36627479846551\\
1.22961461461461	4.40175612273558\\
1.22761361361361	4.43771851154943\\
1.22561261261261	4.47417056673911\\
1.22361161161161	4.51112105698328\\
1.22161061061061	4.54857892001719\\
1.21960960960961	4.58655326474971\\
1.21760860860861	4.62505337327316\\
1.21560760760761	4.66408870275021\\
1.21360660660661	4.70366888716082\\
1.21160560560561	4.74380373889069\\
1.2096046046046	4.78450325014077\\
1.2076036036036	4.82577759413594\\
1.2056026026026	4.86763712610894\\
1.2036016016016	4.91009238403321\\
1.2016006006006	4.95315408907682\\
1.1995995995996	4.9968331457463\\
1.1975985985986	5.04114064168745\\
1.1955975975976	5.08608784710681\\
1.1935965965966	5.13168621377489\\
1.1915955955956	5.1779473735686\\
1.18959459459459	5.22488313650732\\
1.18759359359359	5.27250548823291\\
1.18559259259259	5.32082658688024\\
1.18359159159159	5.36985875928035\\
1.18159059059059	5.41961449643401\\
1.17958958958959	5.47010644818814\\
1.17758858858859	5.52134741704275\\
1.17558758758759	5.57335035101011\\
1.17358658658659	5.62612833544206\\
1.17158558558559	5.67969458373503\\
1.16958458458458	5.7340624268153\\
1.16758358358358	5.78924530130023\\
1.16558258258258	5.84525673622311\\
1.16358158158158	5.90211033820159\\
1.16158058058058	5.95981977492094\\
1.15957957957958	6.01839875679402\\
1.15757857857858	6.07786101665115\\
1.15557757757758	6.13822028730218\\
1.15357657657658	6.19949027680324\\
1.15157557557558	6.26168464124962\\
1.14957457457457	6.32481695490469\\
1.14757357357357	6.38890067746318\\
1.14557257257257	6.45394911823515\\
1.14357157157157	6.5199753970242\\
1.14157057057057	6.58699240146128\\
1.13956956956957	6.65501274054208\\
1.13756856856857	6.72404869410359\\
1.13556756756757	6.79411215796215\\
1.13356656656657	6.86521458442324\\
1.13156556556557	6.93736691786048\\
1.12956456456456	7.01057952505035\\
1.12756356356356	7.08486211993794\\
1.12556256256256	7.16022368250023\\
1.12356156156156	7.2366723713648\\
1.12156056056056	7.31421542983652\\
1.11955955955956	7.39285908498089\\
1.11755855855856	7.4726084394122\\
1.11555755755756	7.55346735543766\\
1.11355655655656	7.63543833121601\\
1.11155555555556	7.71852236860083\\
1.10955455455455	7.8027188323574\\
1.10755355355355	7.88802530046612\\
1.10555255255255	7.97443740525849\\
1.10355155155155	8.061948665173\\
1.10155055055055	8.1505503069697\\
1.09954954954955	8.24023107830551\\
1.09754854854855	8.33097705064799\\
1.09554754754755	8.42277141259535\\
1.09354654654655	8.51559425377649\\
1.09154554554555	8.60942233962762\\
1.08954454454454	8.70422887748415\\
1.08754354354354	8.79998327458839\\
1.08554254254254	8.8966508887973\\
1.08354154154154	8.9941927729809\\
1.08154054054054	9.09256541433203\\
1.07953953953954	9.19172047006242\\
1.07753853853854	9.29160450123907\\
1.07553753753754	9.39215870681768\\
1.07353653653654	9.49331866025622\\
1.07153553553554	9.59501405143862\\
1.06953453453453	9.69716843700436\\
1.06753353353353	9.79969900255951\\
1.06553253253253	9.90251634063407\\
1.06353153153153	10.0055242486417\\
1.06153053053053	10.1086195514836\\
1.05952952952953	10.2116919538078\\
1.05752852852853	10.3146239272783\\
1.05552752752753	10.4172906385085\\
1.05352652652653	10.5195599235595\\
1.05152552552553	10.621292315077\\
1.04952452452452	10.7223411282188\\
1.04752352352352	10.8225526115021\\
1.04552252252252	10.9217661685377\\
1.04352152152152	11.0198146563166\\
1.04152052052052	11.1165247652423\\
1.03951951951952	11.2117174854482\\
1.03751851851852	11.3052086630958\\
1.03551751751752	11.3968096492948\\
1.03351651651652	11.4863280430342\\
1.03151551551552	11.5735685280516\\
1.02951451451451	11.6583338019228\\
1.02751351351351	11.7404255938332\\
1.02551251251251	11.8196457655334\\
1.02351151151151	11.8957974879258\\
1.02151051051051	11.9686864836139\\
1.01950950950951	12.0381223236481\\
1.01750850850851	12.1039197646672\\
1.01550750750751	12.165900110749\\
1.01350650650651	12.2238925826157\\
1.01150550550551	12.2777356754576\\
1.0095045045045	12.3272784856268\\
1.0075035035035	12.3723819858573\\
1.0055025025025	12.4129202285536\\
1.0035015015015	12.4487814570865\\
1.0015005005005	12.4798691059652\\
0.9994994994995	12.5061026722128\\
0.997498498498499	12.5274184422445\\
0.995497497497498	12.5437700609824\\
0.993496496496497	12.5551289327761\\
0.991495495495496	12.5614844468577\\
0.989494494494495	12.5628440234368\\
0.987493493493494	12.5592329800379\\
0.985492492492492	12.5506942211818\\
0.983491491491492	12.5372877579042\\
0.981490490490491	12.5190900667797\\
0.97948948948949	12.4961933009844\\
0.977488488488489	12.4687043684051\\
0.975487487487487	12.4367438938128\\
0.973486486486487	12.4004450836402\\
0.971485485485486	12.3599525128902\\
0.969484484484485	12.3154208541769\\
0.967483483483484	12.2670135688621\\
0.965482482482482	12.2149015797458\\
0.963481481481481	12.1592619438453\\
0.961480480480481	12.1002765425141\\
0.95947947947948	12.0381308045768\\
0.957478478478479	11.9730124763642\\
0.955477477477477	11.9051104505932\\
0.953476476476476	11.8346136640162\\
0.951475475475476	11.7617100717341\\
0.949474474474475	11.6865857040738\\
0.947473473473474	11.6094238100322\\
0.945472472472472	11.5304040895152\\
0.943471471471471	11.4497020149851\\
0.941470470470471	11.3674882416985\\
0.93946946946947	11.2839281077789\\
0.937468468468469	11.1991812008838\\
0.935467467467467	11.1134010383119\\
0.933466466466467	11.0267347879351\\
0.931465465465466	10.9393230748895\\
0.929464464464465	10.851299850412\\
0.927463463463464	10.7627923189702\\
0.925462462462462	10.6739209166177\\
0.923461461461462	10.584799338473\\
0.92146046046046	10.4955346065192\\
0.91945945945946	10.4062271739148\\
0.917458458458459	10.3169710600962\\
0.915457457457457	10.2278540122797\\
0.913456456456457	10.1389576885322\\
0.911455455455455	10.050357858709\\
0.909454454454455	9.96212461904538\\
0.907453453453454	9.87432261725159\\
0.905452452452452	9.78701128454914\\
0.903451451451452	9.7002450733471\\
0.90145045045045	9.61407369594522\\
0.899449449449449	9.52854236421614\\
0.897448448448449	9.44369202746785\\
0.895447447447447	9.35955960687744\\
0.893446446446447	9.27617822588321\\
0.891445445445445	9.19357743460063\\
0.889444444444444	9.11178342822472\\
0.887443443443444	9.0308192579341\\
0.885442442442442	8.95070503470292\\
0.883441441441442	8.87145812468388\\
0.88144044044044	8.79309333646744\\
0.879439439439439	8.71562309990433\\
0.877438438438438	8.63905763653406\\
0.875437437437438	8.56340512150835\\
0.873436436436437	8.48867183676721\\
0.871435435435435	8.41486231657932\\
0.869434434434434	8.34197948446361\\
0.867433433433433	8.27002478252565\\
0.865432432432433	8.19899829302238\\
0.863431431431432	8.12889885255913\\
0.86143043043043	8.05972415924797\\
0.859429429429429	7.99147087323117\\
0.857428428428428	7.92413471028138\\
0.855427427427428	7.85771052964862\\
0.853426426426427	7.79219241578764\\
0.851425425425425	7.72757375458922\\
0.849424424424424	7.66384730419312\\
0.847423423423423	7.6010052609356\\
0.845422422422422	7.5390393205046\\
0.843421421421422	7.47794073459797\\
0.84142042042042	7.41770036363327\\
0.839419419419419	7.35830872507829\\
0.837418418418418	7.29975603850201\\
0.835417417417417	7.24203226700275\\
0.833416416416417	7.18512715537593\\
0.831415415415415	7.12903026533347\\
0.829414414414414	7.07373100772639\\
0.827413413413413	7.0192186724175\\
0.825412412412412	6.96548245537451\\
0.823411411411412	6.91251148365922\\
0.82141041041041	6.86029483825509\\
0.819409409409409	6.8088215749117\\
0.817408408408409	6.7580807430206\\
0.815407407407407	6.7080614030238\\
0.813406406406406	6.65875264196652\\
0.811405405405405	6.61014358782235\\
0.809404404404404	6.56222342223927\\
0.807403403403404	6.51498139211756\\
0.805402402402402	6.46840682010333\\
0.803401401401401	6.42248911384037\\
0.8014004004004	6.37721777432373\\
0.799399399399399	6.33258240334215\\
0.797398398398398	6.28857271001158\\
0.795397397397397	6.24517851652685\\
0.793396396396396	6.20238976329108\\
0.791395395395395	6.16019651326162\\
0.789394394394394	6.11858895577865\\
0.787393393393393	6.0775574097424\\
0.785392392392392	6.03709232637418\\
0.783391391391391	5.99718429149576\\
0.78139039039039	5.95782402724382\\
0.779389389389389	5.91900239358742\\
0.777388388388388	5.88071038936047\\
0.775387387387387	5.84293915302006\\
0.773386386386386	5.80567996302684\\
0.771385385385385	5.76892423811532\\
0.769384384384384	5.73266353710923\\
0.767383383383383	5.69688955869985\\
0.765382382382382	5.6615941409178\\
0.763381381381381	5.6267692604484\\
0.76138038038038	5.59240703178264\\
0.759379379379379	5.55849970623307\\
0.757378378378378	5.52503967080421\\
0.755377377377377	5.49201944697577\\
0.753376376376376	5.45943168930634\\
0.751375375375375	5.42726918404852\\
0.749374374374374	5.39552484759797\\
0.747373373373373	5.36419172492471\\
0.745372372372372	5.33326298792841\\
0.743371371371371	5.30273193370253\\
0.74137037037037	5.27259198283695\\
0.739369369369369	5.24283667760379\\
0.737368368368368	5.21345968013944\\
0.735367367367367	5.18445477063851\\
0.733366366366366	5.15581584546309\\
0.731365365365365	5.12753691528\\
0.729364364364364	5.09961210318666\\
0.727363363363363	5.07203564282765\\
0.725362362362362	5.04480187647972\\
0.723361361361361	5.01790525318973\\
0.72136036036036	4.99134032689351\\
0.719359359359359	4.9651017545134\\
0.717358358358358	4.93918429413127\\
0.715357357357357	4.91358280310079\\
0.713356356356356	4.88829223623125\\
0.711355355355355	4.86330764396006\\
0.709354354354354	4.83862417054609\\
0.707353353353353	4.81423705229128\\
0.705352352352352	4.79014161575463\\
0.703351351351351	4.7663332760492\\
0.70135035035035	4.74280753507515\\
0.699349349349349	4.71955997985823\\
0.697348348348348	4.69658628085473\\
0.695347347347347	4.67388219031995\\
0.693346346346346	4.6514435406692\\
0.691345345345345	4.62926624288116\\
0.689344344344344	4.60734628494059\\
0.687343343343343	4.58567973028209\\
0.685342342342342	4.56426271625755\\
0.683341341341341	4.54309145266484\\
0.68134034034034	4.52216222026869\\
0.679339339339339	4.50147136935965\\
0.677338338338338	4.48101531834638\\
0.675337337337337	4.46079055235248\\
0.673336336336336	4.44079362187601\\
0.671335335335335	4.42102114143349\\
0.669334334334334	4.40146978826067\\
0.667333333333333	4.38213630102497\\
0.665332332332332	4.36301747856879\\
0.663331331331331	4.34411017866717\\
0.66133033033033	4.32541131682163\\
0.659329329329329	4.30691786508544\\
0.657328328328328	4.28862685088701\\
0.655327327327327	4.27053535589826\\
0.653326326326326	4.25264051492681\\
0.651325325325325	4.23493951481543\\
0.649324324324324	4.21742959338211\\
0.647323323323323	4.20010803837389\\
0.645322322322322	4.18297218643892\\
0.643321321321321	4.16601942212321\\
0.64132032032032	4.149247176903\\
0.639319319319319	4.13265292820573\\
0.637318318318318	4.11623419849139\\
0.635317317317317	4.09998855431878\\
0.633316316316316	4.08391360545223\\
0.631315315315315	4.06800700398255\\
0.629314314314314	4.05226644346182\\
0.627313313313313	4.03668965807064\\
0.625312312312312	4.02127442178077\\
0.623311311311311	4.00601854756266\\
0.62131031031031	3.99091988659032\\
0.619309309309309	3.97597632746848\\
0.617308308308308	3.96118579548531\\
0.615307307307307	3.94654625186552\\
0.613306306306306	3.93205569305678\\
0.611305305305305	3.91771215002147\\
0.609304304304304	3.90351368754097\\
0.607303303303303	3.88945840355012\\
0.605302302302302	3.87554442846604\\
0.603301301301301	3.86176992455061\\
0.6013003003003	3.84813308527368\\
0.599299299299299	3.83463213469714\\
0.597298298298298	3.82126532686709\\
0.595297297297297	3.80803094522838\\
0.593296296296296	3.79492730203937\\
0.591295295295295	3.78195273781208\\
0.589294294294294	3.76910562075455\\
0.587293293293293	3.75638434623188\\
0.585292292292292	3.74378733623448\\
0.583291291291291	3.73131303886302\\
0.58129029029029	3.71895992781947\\
0.579289289289289	3.70672650191141\\
0.577288288288288	3.69461128457001\\
0.575287287287287	3.68261282336996\\
0.573286286286286	3.67072968957235\\
0.571285285285285	3.6589604776638\\
0.569284284284284	3.64730380491944\\
0.567283283283283	3.6357583109594\\
0.565282282282282	3.62432265733232\\
0.563281281281281	3.61299552709496\\
0.56128028028028	3.60177562440487\\
0.559279279279279	3.59066167412176\\
0.557278278278278	3.57965242142146\\
0.555277277277277	3.56874663140763\\
0.553276276276276	3.55794308874643\\
0.551275275275275	3.54724059729375\\
0.549274274274274	3.53663797974273\\
0.547273273273273	3.5261340772732\\
0.545272272272272	3.51572774920465\\
0.543271271271271	3.5054178726672\\
0.54127027027027	3.49520334226937\\
0.539269269269269	3.48508306977859\\
0.537268268268268	3.47505598380511\\
0.535267267267267	3.46512102949577\\
0.533266266266266	3.45527716823244\\
0.531265265265265	3.44552337733468\\
0.529264264264264	3.43585864977556\\
0.527263263263263	3.4262819938945\\
0.525262262262262	3.41679243312087\\
0.523261261261261	3.40738900570409\\
0.52126026026026	3.39807076444945\\
0.519259259259259	3.3888367764531\\
0.517258258258258	3.37968612285269\\
0.515257257257257	3.37061789857519\\
0.513256256256256	3.36163121209234\\
0.511255255255255	3.35272518518288\\
0.509254254254254	3.3438989526974\\
0.507253253253253	3.33515166232794\\
0.505252252252252	3.32648247438509\\
0.503251251251251	3.31789056157522\\
0.50125025025025	3.30937510878633\\
0.499249249249249	3.30093531287609\\
0.497248248248248	3.29257038246503\\
0.495247247247247	3.28427953773317\\
0.493246246246246	3.27606201021936\\
0.491245245245245	3.26791704263085\\
0.489244244244244	3.25984388864748\\
0.487243243243243	3.25184181273724\\
0.485242242242242	3.24391008997207\\
0.483241241241241	3.23604800584813\\
0.48124024024024	3.2282548561092\\
0.479239239239239	3.22052994657492\\
0.477238238238238	3.21287259296992\\
0.475237237237237	3.20528212076\\
0.473236236236236	3.19775786498665\\
0.471235235235235	3.19029917011063\\
0.469234234234234	3.18290538985285\\
0.467233233233233	3.17557588704278\\
0.465232232232232	3.16831003346738\\
0.463231231231231	3.1611072097247\\
0.46123023023023	3.1539668050766\\
0.459229229229229	3.14688821731117\\
0.457228228228228	3.13987085260041\\
0.455227227227227	3.13291412536451\\
0.453226226226226	3.12601745814023\\
0.451225225225225	3.11918028144642\\
0.449224224224224	3.11240203365872\\
0.447223223223223	3.10568216088064\\
0.445222222222222	3.09902011682182\\
0.443221221221221	3.09241536267607\\
0.44122022022022	3.08586736700073\\
0.439219219219219	3.07937560560325\\
0.437218218218218	3.07293956142292\\
0.435217217217217	3.06655872442021\\
0.433216216216216	3.06023259146689\\
0.431215215215215	3.05396066623666\\
0.429214214214214	3.04774245909882\\
0.427213213213213	3.04157748701562\\
0.425212212212212	3.03546527343773\\
0.423211211211211	3.02940534820556\\
0.42121021021021	3.02339724744963\\
0.419209209209209	3.01744051349467\\
0.417208208208208	3.0115346947643\\
0.415207207207207	3.00567934568717\\
0.413206206206206	2.99987402660657\\
0.411205205205205	2.99411830369017\\
0.409204204204204	2.98841174884205\\
0.407203203203203	2.98275393961582\\
0.405202202202202	2.97714445912962\\
0.403201201201201	2.97158289598308\\
0.4012002002002	2.96606884417507\\
0.399199199199199	2.96060190302395\\
0.397198198198198	2.95518167708722\\
0.395197197197197	2.94980777608587\\
0.393196196196196	2.94447981482632\\
0.391195195195195	2.93919741312691\\
0.389194194194194	2.93396019574399\\
0.387193193193193	2.92876779229976\\
0.385192192192192	2.92361983721179\\
0.383191191191191	2.91851596962258\\
0.38119019019019	2.91345583333301\\
0.379189189189189	2.90843907673221\\
0.377188188188188	2.9034653527353\\
0.375187187187187	2.89853431871511\\
0.373186186186186	2.89364563644117\\
0.371185185185185	2.88879897201583\\
0.369184184184184	2.8839939958129\\
0.367183183183183	2.87923038241783\\
0.365182182182182	2.87450781056829\\
0.363181181181181	2.86982596309533\\
0.36118018018018	2.86518452686711\\
0.359179179179179	2.86058319273133\\
0.357178178178178	2.8560216554615\\
0.355177177177177	2.85149961370145\\
0.353176176176176	2.84701676991262\\
0.351175175175175	2.84257283032081\\
0.349174174174174	2.83816750486577\\
0.347173173173173	2.83380050714929\\
0.345172172172172	2.82947155438659\\
0.343171171171171	2.82518036735632\\
0.34117017017017	2.82092667035316\\
0.339169169169169	2.81671019114003\\
0.337168168168168	2.81253066090231\\
0.335167167167167	2.80838781420132\\
0.333166166166166	2.80428138892979\\
0.331165165165165	2.80021112626757\\
0.329164164164164	2.79617677063869\\
0.327163163163163	2.79217806966803\\
0.325162162162162	2.78821477413939\\
0.323161161161161	2.78428663795511\\
0.32116016016016	2.78039341809415\\
0.319159159159159	2.77653487457361\\
0.317158158158158	2.77271077040826\\
0.315157157157157	2.76892087157284\\
0.313156156156156	2.76516494696376\\
0.311155155155155	2.76144276836191\\
0.309154154154154	2.75775411039629\\
0.307153153153153	2.7540987505072\\
0.305152152152152	2.75047646891154\\
0.303151151151151	2.7468870485675\\
0.30115015015015	2.74333027514082\\
0.299149149149149	2.73980593697022\\
0.297148148148148	2.73631382503468\\
0.295147147147147	2.73285373292158\\
0.293146146146146	2.72942545679268\\
0.291145145145145	2.72602879535505\\
0.289144144144144	2.72266354982773\\
0.287143143143143	2.71932952391278\\
0.285142142142142	2.71602652376482\\
0.283141141141141	2.71275435796084\\
0.28114014014014	2.70951283747224\\
0.279139139139139	2.7063017756354\\
0.277138138138138	2.70312098812436\\
0.275137137137137	2.69997029292199\\
0.273136136136136	2.69684951029435\\
0.271135135135135	2.69375846276243\\
0.269134134134134	2.6906969750771\\
0.267133133133133	2.68766487419221\\
0.265132132132132	2.68466198923975\\
0.263131131131131	2.68168815150464\\
0.26113013013013	2.67874319440005\\
0.259129129129129	2.67582695344321\\
0.257128128128128	2.67293926623146\\
0.255127127127127	2.6700799724189\\
0.253126126126126	2.66724891369316\\
0.251125125125125	2.66444593375311\\
0.249124124124124	2.66167087828579\\
0.247123123123123	2.6589235949452\\
0.245122122122122	2.65620393333022\\
0.243121121121121	2.65351174496333\\
0.24112012012012	2.65084688327016\\
0.239119119119119	2.64820920355814\\
0.237118118118118	2.64559856299705\\
0.235117117117117	2.6430148205985\\
0.233116116116116	2.64045783719675\\
0.231115115115115	2.63792747542901\\
0.229114114114114	2.63542359971692\\
0.227113113113113	2.63294607624753\\
0.225112112112112	2.63049477295524\\
0.223111111111111	2.62806955950331\\
0.22111011011011	2.6256703072668\\
0.219109109109109	2.62329688931435\\
0.217108108108108	2.62094918039151\\
0.215107107107107	2.61862705690359\\
0.213106106106106	2.61633039689912\\
0.211105105105105	2.61405908005334\\
0.209104104104104	2.61181298765237\\
0.207103103103103	2.60959200257716\\
0.205102102102102	2.6073960092882\\
0.203101101101101	2.60522489380976\\
0.2011001001001	2.60307854371519\\
0.199099099099099	2.60095684811212\\
0.197098098098098	2.5988596976277\\
0.195097097097097	2.59678698439425\\
0.193096096096096	2.5947386020356\\
0.191095095095095	2.59271444565264\\
0.189094094094094	2.5907144118101\\
0.187093093093093	2.58873839852312\\
0.185092092092092	2.58678630524395\\
0.183091091091091	2.58485803284904\\
0.18109009009009	2.58295348362636\\
0.179089089089089	2.58107256126293\\
0.177088088088088	2.57921517083236\\
0.175087087087087	2.57738121878298\\
0.173086086086086	2.57557061292574\\
0.171085085085085	2.57378326242262\\
0.169084084084084	2.57201907777523\\
0.167083083083083	2.57027797081332\\
0.165082082082082	2.56855985468402\\
0.163081081081081	2.56686464384049\\
0.16108008008008	2.56519225403174\\
0.159079079079079	2.56354260229163\\
0.157078078078078	2.56191560692892\\
0.155077077077077	2.56031118751684\\
0.153076076076076	2.55872926488331\\
0.151075075075075	2.55716976110103\\
0.149074074074074	2.55563259947799\\
0.147073073073073	2.55411770454792\\
0.145072072072072	2.55262500206107\\
0.143071071071071	2.55115441897512\\
0.14107007007007	2.54970588344618\\
0.139069069069069	2.54827932482014\\
0.137068068068068	2.54687467362403\\
0.135067067067067	2.54549186155748\\
0.133066066066066	2.54413082148457\\
0.131065065065065	2.54279148742568\\
0.129064064064064	2.54147379454947\\
0.127063063063063	2.54017767916508\\
0.125062062062062	2.5389030787146\\
0.123061061061061	2.53764993176529\\
0.12106006006006	2.5364181780025\\
0.119059059059059	2.53520775822232\\
0.117058058058058	2.53401861432452\\
0.115057057057057	2.53285068930568\\
0.113056056056056	2.53170392725231\\
0.111055055055055	2.53057827333434\\
0.109054054054054	2.52947367379855\\
0.107053053053053	2.5283900759622\\
0.105052052052052	2.52732742820683\\
0.103051051051051	2.52628567997221\\
0.10105005005005	2.52526478175033\\
0.0990490490490491	2.5242646850796\\
0.0970480480480481	2.52328534253918\\
0.0950470470470471	2.52232670774339\\
0.093046046046046	2.52138873533633\\
0.091045045045045	2.52047138098654\\
0.089044044044044	2.51957460138186\\
0.0870430430430431	2.51869835422434\\
0.085042042042042	2.51784259822539\\
0.083041041041041	2.51700729310091\\
0.08104004004004	2.51619239956668\\
0.079039039039039	2.51539787933375\\
0.077038038038038	2.51462369510406\\
0.075037037037037	2.51386981056611\\
0.073036036036036	2.51313619039073\\
0.071035035035035	2.51242280022709\\
0.069034034034034	2.51172960669866\\
0.067033033033033	2.51105657739946\\
0.065032032032032	2.51040368089021\\
0.063031031031031	2.50977088669484\\
0.06103003003003	2.50915816529694\\
0.059029029029029	2.50856548813644\\
0.057028028028028	2.50799282760619\\
0.055027027027027	2.50744015704897\\
0.053026026026026	2.50690745075437\\
0.051025025025025	2.50639468395587\\
0.049024024024024	2.505901832828\\
0.047023023023023	2.50542887448362\\
0.045022022022022	2.50497578697137\\
0.043021021021021	2.50454254927309\\
0.04102002002002	2.50412914130149\\
0.039019019019019	2.50373554389787\\
0.037018018018018	2.50336173882988\\
0.035017017017017	2.50300770878955\\
0.033016016016016	2.50267343739123\\
0.031015015015015	2.5023589091698\\
0.029014014014014	2.50206410957888\\
0.027013013013013	2.50178902498922\\
0.025012012012012	2.50153364268713\\
0.023011011011011	2.50129795087301\\
0.02101001001001	2.50108193866012\\
0.019009009009009	2.50088559607324\\
0.017008008008008	2.50070891404759\\
0.015007007007007	2.50055188442781\\
0.013006006006006	2.50041449996704\\
0.011005005005005	2.50029675432608\\
0.009004004004004	2.5001986420727\\
0.007003003003003	2.500120158681\\
0.005002002002002	2.50006130053095\\
0.003001001001001	2.5000220649079\\
0.001	2.50000245000235\\
0.003001001001001	2.5000220649079\\
0.005002002002002	2.50006130053095\\
0.007003003003003	2.500120158681\\
0.009004004004004	2.5001986420727\\
0.011005005005005	2.50029675432608\\
0.013006006006006	2.50041449996703\\
0.015007007007007	2.50055188442781\\
0.017008008008008	2.50070891404758\\
0.019009009009009	2.50088559607324\\
0.02101001001001	2.50108193866012\\
0.023011011011011	2.50129795087301\\
0.025012012012012	2.50153364268712\\
0.027013013013013	2.50178902498922\\
0.029014014014014	2.50206410957888\\
0.031015015015015	2.5023589091698\\
0.033016016016016	2.50267343739123\\
0.035017017017017	2.50300770878955\\
0.037018018018018	2.50336173882988\\
0.039019019019019	2.50373554389787\\
0.04102002002002	2.50412914130149\\
0.043021021021021	2.50454254927309\\
0.045022022022022	2.50497578697137\\
0.047023023023023	2.50542887448362\\
0.049024024024024	2.505901832828\\
0.051025025025025	2.50639468395587\\
0.053026026026026	2.50690745075437\\
0.055027027027027	2.50744015704897\\
0.057028028028028	2.50799282760618\\
0.059029029029029	2.50856548813644\\
0.06103003003003	2.50915816529695\\
0.063031031031031	2.50977088669484\\
0.065032032032032	2.5104036808902\\
0.067033033033033	2.51105657739945\\
0.069034034034034	2.51172960669867\\
0.071035035035035	2.51242280022709\\
0.073036036036036	2.51313619039073\\
0.075037037037037	2.51386981056611\\
0.077038038038038	2.51462369510406\\
0.079039039039039	2.51539787933375\\
0.08104004004004	2.51619239956668\\
0.083041041041041	2.51700729310091\\
0.085042042042042	2.51784259822539\\
0.0870430430430431	2.51869835422434\\
0.089044044044044	2.51957460138185\\
0.091045045045045	2.52047138098654\\
0.093046046046046	2.52138873533633\\
0.0950470470470471	2.5223267077434\\
0.0970480480480481	2.52328534253918\\
0.0990490490490491	2.52426468507961\\
0.10105005005005	2.52526478175034\\
0.103051051051051	2.52628567997222\\
0.105052052052052	2.52732742820683\\
0.107053053053053	2.52839007596219\\
0.109054054054054	2.52947367379855\\
0.111055055055055	2.53057827333434\\
0.113056056056056	2.53170392725231\\
0.115057057057057	2.53285068930567\\
0.117058058058058	2.53401861432453\\
0.119059059059059	2.53520775822232\\
0.12106006006006	2.5364181780025\\
0.123061061061061	2.53764993176528\\
0.125062062062062	2.53890307871459\\
0.127063063063063	2.5401776791651\\
0.129064064064064	2.54147379454946\\
0.131065065065065	2.54279148742568\\
0.133066066066066	2.54413082148457\\
0.135067067067067	2.54549186155747\\
0.137068068068068	2.54687467362402\\
0.139069069069069	2.54827932482015\\
0.14107007007007	2.54970588344617\\
0.143071071071071	2.55115441897511\\
0.145072072072072	2.55262500206108\\
0.147073073073073	2.55411770454793\\
0.149074074074074	2.555632599478\\
0.151075075075075	2.55716976110103\\
0.153076076076076	2.55872926488329\\
0.155077077077077	2.56031118751683\\
0.157078078078078	2.56191560692892\\
0.159079079079079	2.56354260229164\\
0.16108008008008	2.56519225403174\\
0.163081081081081	2.56686464384051\\
0.165082082082082	2.568559854684\\
0.167083083083083	2.57027797081334\\
0.169084084084084	2.57201907777521\\
0.171085085085085	2.57378326242262\\
0.173086086086086	2.57557061292573\\
0.175087087087087	2.57738121878299\\
0.177088088088088	2.57921517083237\\
0.179089089089089	2.58107256126291\\
0.18109009009009	2.58295348362635\\
0.183091091091091	2.58485803284902\\
0.185092092092092	2.58678630524395\\
0.187093093093093	2.58873839852314\\
0.189094094094094	2.59071441181012\\
0.191095095095095	2.59271444565263\\
0.193096096096096	2.59473860203559\\
0.195097097097097	2.59678698439428\\
0.197098098098098	2.59885969762767\\
0.199099099099099	2.60095684811213\\
0.2011001001001	2.6030785437152\\
0.203101101101101	2.60522489380974\\
0.205102102102102	2.60739600928819\\
0.207103103103103	2.6095920025772\\
0.209104104104104	2.61181298765235\\
0.211105105105105	2.61405908005331\\
0.213106106106106	2.61633039689909\\
0.215107107107107	2.61862705690362\\
0.217108108108108	2.62094918039154\\
0.219109109109109	2.62329688931437\\
0.22111011011011	2.62567030726679\\
0.223111111111111	2.62806955950331\\
0.225112112112112	2.63049477295517\\
0.227113113113113	2.63294607624752\\
0.229114114114114	2.6354235997169\\
0.231115115115115	2.63792747542902\\
0.233116116116116	2.64045783719674\\
0.235117117117117	2.64301482059853\\
0.237118118118118	2.64559856299704\\
0.239119119119119	2.64820920355813\\
0.24112012012012	2.65084688327012\\
0.243121121121121	2.65351174496334\\
0.245122122122122	2.65620393333019\\
0.247123123123123	2.65892359494522\\
0.249124124124124	2.6616708782858\\
0.251125125125125	2.6644459337531\\
0.253126126126126	2.66724891369321\\
0.255127127127127	2.67007997241887\\
0.257128128128128	2.67293926623144\\
0.259129129129129	2.67582695344321\\
0.26113013013013	2.67874319440003\\
0.263131131131131	2.68168815150461\\
0.265132132132132	2.6846619892397\\
0.267133133133133	2.68766487419216\\
0.269134134134134	2.69069697507708\\
0.271135135135135	2.69375846276247\\
0.273136136136136	2.69684951029434\\
0.275137137137137	2.69997029292204\\
0.277138138138138	2.70312098812431\\
0.279139139139139	2.70630177563546\\
0.28114014014014	2.70951283747223\\
0.283141141141141	2.71275435796082\\
0.285142142142142	2.71602652376478\\
0.287143143143143	2.71932952391287\\
0.289144144144144	2.72266354982774\\
0.291145145145145	2.72602879535502\\
0.293146146146146	2.72942545679279\\
0.295147147147147	2.73285373292152\\
0.297148148148148	2.73631382503477\\
0.299149149149149	2.73980593697015\\
0.30115015015015	2.74333027514084\\
0.303151151151151	2.74688704856758\\
0.305152152152152	2.75047646891152\\
0.307153153153153	2.75409875050718\\
0.309154154154154	2.75775411039629\\
0.311155155155155	2.76144276836196\\
0.313156156156156	2.76516494696371\\
0.315157157157157	2.76892087157278\\
0.317158158158158	2.77271077040824\\
0.319159159159159	2.7765348745736\\
0.32116016016016	2.78039341809422\\
0.323161161161161	2.78428663795506\\
0.325162162162162	2.78821477413942\\
0.327163163163163	2.79217806966798\\
0.329164164164164	2.7961767706387\\
0.331165165165165	2.80021112626756\\
0.333166166166166	2.80428138892972\\
0.335167167167167	2.80838781420131\\
0.337168168168168	2.8125306609023\\
0.339169169169169	2.81671019114005\\
0.34117017017017	2.82092667035303\\
0.343171171171171	2.82518036735619\\
0.345172172172172	2.82947155438651\\
0.347173173173173	2.83380050714927\\
0.349174174174174	2.83816750486572\\
0.351175175175175	2.84257283032081\\
0.353176176176176	2.8470167699126\\
0.355177177177177	2.85149961370157\\
0.357178178178178	2.85602165546152\\
0.359179179179179	2.86058319273139\\
0.36118018018018	2.86518452686704\\
0.363181181181181	2.86982596309543\\
0.365182182182182	2.87450781056829\\
0.367183183183183	2.87923038241789\\
0.369184184184184	2.88399399581284\\
0.371185185185185	2.88879897201578\\
0.373186186186186	2.89364563644121\\
0.375187187187187	2.89853431871513\\
0.377188188188188	2.90346535273519\\
0.379189189189189	2.90843907673231\\
0.38119019019019	2.91345583333285\\
0.383191191191191	2.91851596962268\\
0.385192192192192	2.92361983721169\\
0.387193193193193	2.92876779229976\\
0.389194194194194	2.93396019574398\\
0.391195195195195	2.93919741312694\\
0.393196196196196	2.94447981482632\\
0.395197197197197	2.94980777608594\\
0.397198198198198	2.95518167708735\\
0.399199199199199	2.96060190302391\\
0.4012002002002	2.96606884417509\\
0.403201201201201	2.97158289598297\\
0.405202202202202	2.9771444591296\\
0.407203203203203	2.9827539396158\\
0.409204204204204	2.98841174884212\\
0.411205205205205	2.99411830369024\\
0.413206206206206	2.99987402660646\\
0.415207207207207	3.00567934568701\\
0.417208208208208	3.01153469476422\\
0.419209209209209	3.01744051349485\\
0.42121021021021	3.02339724744975\\
0.423211211211211	3.02940534820552\\
0.425212212212212	3.03546527343776\\
0.427213213213213	3.04157748701555\\
0.429214214214214	3.04774245909888\\
0.431215215215215	3.05396066623651\\
0.433216216216216	3.06023259146684\\
0.435217217217217	3.06655872442015\\
0.437218218218218	3.07293956142278\\
0.439219219219219	3.07937560560321\\
0.44122022022022	3.0858673670008\\
0.443221221221221	3.09241536267586\\
0.445222222222222	3.09902011682181\\
0.447223223223223	3.10568216088062\\
0.449224224224224	3.11240203365863\\
0.451225225225225	3.11918028144651\\
0.453226226226226	3.12601745814021\\
0.455227227227227	3.1329141253647\\
0.457228228228228	3.13987085260032\\
0.459229229229229	3.14688821731136\\
0.46123023023023	3.15396680507669\\
0.463231231231231	3.16110720972451\\
0.465232232232232	3.16831003346754\\
0.467233233233233	3.17557588704274\\
0.469234234234234	3.1829053898528\\
0.471235235235235	3.19029917011052\\
0.473236236236236	3.19775786498676\\
0.475237237237237	3.20528212075981\\
0.477238238238238	3.21287259297001\\
0.479239239239239	3.22052994657484\\
0.48124024024024	3.22825485610915\\
0.483241241241241	3.23604800584798\\
0.485242242242242	3.24391008997195\\
0.487243243243243	3.25184181273722\\
0.489244244244244	3.25984388864737\\
0.491245245245245	3.26791704263081\\
0.493246246246246	3.27606201021947\\
0.495247247247247	3.28427953773299\\
0.497248248248248	3.29257038246536\\
0.499249249249249	3.30093531287635\\
0.50125025025025	3.30937510878641\\
0.503251251251251	3.31789056157524\\
0.505252252252252	3.32648247438514\\
0.507253253253253	3.33515166232822\\
0.509254254254254	3.34389895269738\\
0.511255255255255	3.35272518518292\\
0.513256256256256	3.36163121209218\\
0.515257257257257	3.37061789857488\\
0.517258258258258	3.3796861228526\\
0.519259259259259	3.38883677645295\\
0.52126026026026	3.39807076444923\\
0.523261261261261	3.40738900570437\\
0.525262262262262	3.41679243312074\\
0.527263263263263	3.4262819938945\\
0.529264264264264	3.43585864977602\\
0.531265265265265	3.44552337733495\\
0.533266266266266	3.4552771682319\\
0.535267267267267	3.46512102949561\\
0.537268268268268	3.47505598380497\\
0.539269269269269	3.48508306977873\\
0.54127027027027	3.49520334226977\\
0.543271271271271	3.50541787266742\\
0.545272272272272	3.51572774920473\\
0.547273273273273	3.52613407727305\\
0.549274274274274	3.53663797974314\\
0.551275275275275	3.54724059729365\\
0.553276276276276	3.55794308874609\\
0.555277277277277	3.56874663140769\\
0.557278278278278	3.57965242142123\\
0.559279279279279	3.59066167412237\\
0.56128028028028	3.60177562440506\\
0.563281281281281	3.6129955270954\\
0.565282282282282	3.62432265733286\\
0.567283283283283	3.63575831095935\\
0.569284284284284	3.64730380491914\\
0.571285285285285	3.65896047766413\\
0.573286286286286	3.67072968957243\\
0.575287287287287	3.6826128233702\\
0.577288288288288	3.69461128456993\\
0.579289289289289	3.7067265019119\\
0.58129029029029	3.71895992781946\\
0.583291291291291	3.73131303886318\\
0.585292292292292	3.74378733623431\\
0.587293293293293	3.75638434623162\\
0.589294294294294	3.76910562075459\\
0.591295295295295	3.78195273781199\\
0.593296296296296	3.79492730203923\\
0.595297297297297	3.80803094522826\\
0.597298298298298	3.82126532686709\\
0.599299299299299	3.83463213469682\\
0.6013003003003	3.8481330852741\\
0.603301301301301	3.8617699245511\\
0.605302302302302	3.87554442846635\\
0.607303303303303	3.88945840354991\\
0.609304304304304	3.90351368754151\\
0.611305305305305	3.91771215002106\\
0.613306306306306	3.93205569305679\\
0.615307307307307	3.94654625186493\\
0.617308308308308	3.96118579548484\\
0.619309309309309	3.97597632746871\\
0.62131031031031	3.99091988659012\\
0.623311311311311	4.00601854756343\\
0.625312312312312	4.02127442178065\\
0.627313313313313	4.03668965807\\
0.629314314314314	4.05226644346233\\
0.631315315315315	4.06800700398188\\
0.633316316316316	4.08391360545235\\
0.635317317317317	4.09998855431864\\
0.637318318318318	4.11623419849164\\
0.639319319319319	4.13265292820577\\
0.64132032032032	4.14924717690182\\
0.643321321321321	4.16601942212386\\
0.645322322322322	4.18297218643894\\
0.647323323323323	4.2001080383752\\
0.649324324324324	4.21742959338297\\
0.651325325325325	4.23493951481546\\
0.653326326326326	4.25264051492614\\
0.655327327327327	4.270535355898\\
0.657328328328328	4.28862685088622\\
0.659329329329329	4.30691786508625\\
0.66133033033033	4.32541131682238\\
0.663331331331331	4.34411017866622\\
0.665332332332332	4.36301747856913\\
0.667333333333333	4.38213630102632\\
0.669334334334334	4.40146978826024\\
0.671335335335335	4.42102114143265\\
0.673336336336336	4.4407936218757\\
0.675337337337337	4.46079055235268\\
0.677338338338338	4.48101531834594\\
0.679339339339339	4.50147136936085\\
0.68134034034034	4.52216222026881\\
0.683341341341341	4.54309145266479\\
0.685342342342342	4.56426271625742\\
0.687343343343343	4.58567973028168\\
0.689344344344344	4.60734628494233\\
0.691345345345345	4.62926624288172\\
0.693346346346346	4.65144354066727\\
0.695347347347347	4.67388219032053\\
0.697348348348348	4.69658628085558\\
0.699349349349349	4.71955997985805\\
0.70135035035035	4.74280753507477\\
0.703351351351351	4.7663332760484\\
0.705352352352352	4.79014161575703\\
0.707353353353353	4.81423705228976\\
0.709354354354354	4.83862417054778\\
0.711355355355355	4.86330764395962\\
0.713356356356356	4.88829223623066\\
0.715357357357357	4.9135828030983\\
0.717358358358358	4.93918429413029\\
0.719359359359359	4.96510175451397\\
0.72136036036036	4.99134032689199\\
0.723361361361361	5.01790525319087\\
0.725362362362362	5.04480187647781\\
0.727363363363363	5.07203564282716\\
0.729364364364364	5.09961210318846\\
0.731365365365365	5.12753691527879\\
0.733366366366366	5.15581584546153\\
0.735367367367367	5.18445477063918\\
0.737368368368368	5.21345968014007\\
0.739369369369369	5.24283667760072\\
0.74137037037037	5.27259198283781\\
0.743371371371371	5.30273193370272\\
0.745372372372372	5.33326298792644\\
0.747373373373373	5.36419172492586\\
0.749374374374374	5.39552484759642\\
0.751375375375375	5.42726918404576\\
0.753376376376376	5.45943168930488\\
0.755377377377377	5.49201944697406\\
0.757378378378378	5.52503967080725\\
0.759379379379379	5.55849970623032\\
0.76138038038038	5.5924070317804\\
0.763381381381381	5.62676926044767\\
0.765382382382382	5.6615941409209\\
0.767383383383383	5.69688955870229\\
0.769384384384384	5.73266353710705\\
0.771385385385385	5.76892423811207\\
0.773386386386386	5.80567996302871\\
0.775387387387387	5.84293915301359\\
0.777388388388388	5.88071038936225\\
0.779389389389389	5.91900239358357\\
0.78139039039039	5.9578240272368\\
0.783391391391391	5.99718429149162\\
0.785392392392392	6.03709232637824\\
0.787393393393393	6.07755740973493\\
0.789394394394394	6.11858895577177\\
0.791395395395395	6.16019651326116\\
0.793396396396396	6.20238976328727\\
0.795397397397397	6.24517851652401\\
0.797398398398398	6.28857271000854\\
0.799399399399399	6.33258240334516\\
0.8014004004004	6.37721777432125\\
0.803401401401401	6.42248911382977\\
0.805402402402402	6.46840682009404\\
0.807403403403404	6.51498139210998\\
0.809404404404404	6.56222342222847\\
0.811405405405405	6.61014358782324\\
0.813406406406406	6.65875264196433\\
0.815407407407407	6.70806140300622\\
0.817408408408409	6.75808074300816\\
0.819409409409409	6.80882157489353\\
0.82141041041041	6.8602948382534\\
0.823411411411412	6.9125114836491\\
0.825412412412412	6.96548245536464\\
0.827413413413413	7.01921867241392\\
0.829414414414414	7.07373100772045\\
0.831415415415415	7.129030265312\\
0.833416416416417	7.18512715536393\\
0.835417417417417	7.24203226697959\\
0.837418418418418	7.29975603847857\\
0.839419419419419	7.35830872505671\\
0.84142042042042	7.41770036361655\\
0.843421421421422	7.47794073458759\\
0.845422422422422	7.53903932045749\\
0.847423423423423	7.60100526089604\\
0.849424424424424	7.66384730415572\\
0.851425425425425	7.72757375455679\\
0.853426426426427	7.79219241575574\\
0.855427427427428	7.85771052960109\\
0.857428428428428	7.92413471023011\\
0.859429429429429	7.99147087318377\\
0.86143043043043	8.05972415921379\\
0.863431431431432	8.12889885249411\\
0.865432432432433	8.19899829296598\\
0.867433433433433	8.27002478246586\\
0.869434434434434	8.34197948438112\\
0.871435435435435	8.41486231646995\\
0.873436436436437	8.48867183664939\\
0.875437437437438	8.56340512137906\\
0.877438438438438	8.6390576364332\\
0.879439439439439	8.71562309975417\\
0.88144044044044	8.79309333629382\\
0.883441441441442	8.87145812450961\\
0.885442442442442	8.95070503451327\\
0.887443443443444	9.03081925771718\\
0.889444444444444	9.111783427979\\
0.891445445445445	9.19357743435769\\
0.893446446446447	9.27617822559195\\
0.895447447447447	9.35955960654179\\
0.897448448448449	9.44369202706884\\
0.899449449449449	9.52854236379153\\
0.90145045045045	9.61407369547152\\
0.903451451451452	9.70024507285012\\
0.905452452452452	9.78701128401174\\
0.907453453453454	9.87432261660506\\
0.909454454454455	9.96212461838166\\
0.911455455455455	10.0503578579447\\
0.913456456456457	10.1389576876681\\
0.915457457457457	10.2278540113345\\
0.917458458458459	10.31697105908\\
0.91945945945946	10.406227172762\\
0.92146046046046	10.4955346051976\\
0.923461461461462	10.5847993369641\\
0.925462462462462	10.6739209149586\\
0.927463463463464	10.7627923171774\\
0.929464464464465	10.8512998484952\\
0.931465465465466	10.9393230726029\\
0.933466466466467	11.0267347854254\\
0.935467467467467	11.1134010356147\\
0.937468468468469	11.19918119782\\
0.93946946946947	11.2839281044681\\
0.941470470470471	11.3674882416946\\
0.943471471471471	11.4497020149863\\
0.945472472472472	11.5304040895159\\
0.947473473473474	11.6094238100336\\
0.949474474474475	11.6865857040738\\
0.951475475475476	11.7617100717443\\
0.953476476476476	11.8346136640146\\
0.955477477477477	11.9051104505959\\
0.957478478478479	11.973012476367\\
0.95947947947948	12.0381308045843\\
0.961480480480481	12.100276542513\\
0.963481481481481	12.159261943854\\
0.965482482482482	12.2149015797592\\
0.967483483483484	12.267013568869\\
0.969484484484485	12.3154208541826\\
0.971485485485486	12.3599525128883\\
0.973486486486487	12.4004450836441\\
0.975487487487487	12.4367438938295\\
0.977488488488489	12.4687043684114\\
0.97948948948949	12.4961933009936\\
0.981490490490491	12.5190900667924\\
0.983491491491492	12.5372877579213\\
0.985492492492492	12.5506942211889\\
0.987493493493494	12.5592329800461\\
0.989494494494495	12.5628440234629\\
0.991495495495496	12.561484446895\\
0.993496496496497	12.5551289327876\\
0.995497497497498	12.5437700610021\\
0.997498498498499	12.527418442277\\
0.9994994994995	12.5061026722595\\
1.0015005005005	12.47986910603\\
1.0035015015015	12.4487814571419\\
1.0055025025025	12.4129202285872\\
1.0075035035035	12.3723819859194\\
1.0095045045045	12.3272784857047\\
1.01150550550551	12.2777356755765\\
1.01350650650651	12.2238925827458\\
1.01550750750751	12.1659001108718\\
1.01750850850851	12.1039197648201\\
1.01950950950951	12.0381223238403\\
1.02151051051051	11.968686483861\\
1.02351151151151	11.8957974881884\\
1.02551251251251	11.8196457658296\\
1.02751351351351	11.740425594175\\
1.02951451451451	11.6583338023067\\
1.03151551551552	11.5735685284915\\
1.03351651651652	11.48632804347\\
1.03551751751752	11.3968096497896\\
1.03751851851852	11.3052086637275\\
1.03951951951952	11.2117174861544\\
1.04152052052052	11.1165247660401\\
1.04352152152152	11.0198146572085\\
1.04552252252252	10.9217661695573\\
1.04752352352352	10.8225526125999\\
1.04952452452452	10.7223411294461\\
1.05152552552553	10.6212923165398\\
1.05352652652653	10.5195599251105\\
1.05552752752753	10.417290640182\\
1.05752852852853	10.3146239291446\\
1.05952952952953	10.2116919557822\\
1.06153053053053	10.1086195537559\\
1.06353153153153	10.0055242510148\\
1.06553253253253	9.9025163432239\\
1.06753353353353	9.79969900539987\\
1.06953453453453	9.69716844002673\\
1.07153553553554	9.59501405471707\\
1.07353653653654	9.49331866025621\\
1.07553753753754	9.39215870681768\\
1.07753853853854	9.29160450123907\\
1.07953953953954	9.19172047006242\\
1.08154054054054	9.09256541433203\\
1.08354154154154	8.9941927729809\\
1.08554254254254	8.8966508887973\\
1.08754354354354	8.79998327458839\\
1.08954454454454	8.70422887748415\\
1.09154554554555	8.60942233962762\\
1.09354654654655	8.51559425377649\\
1.09554754754755	8.42277141259536\\
1.09754854854855	8.33097705064799\\
1.09954954954955	8.24023107830551\\
1.10155055055055	8.1505503069697\\
1.10355155155155	8.061948665173\\
1.10555255255255	7.97443740525849\\
1.10755355355355	7.88802530046612\\
1.10955455455455	7.8027188323574\\
1.11155555555556	7.71852236860083\\
1.11355655655656	7.63543833121601\\
1.11555755755756	7.55346735543766\\
1.11755855855856	7.4726084394122\\
1.11955955955956	7.39285908498089\\
1.12156056056056	7.31421542983652\\
1.12356156156156	7.2366723713648\\
1.12556256256256	7.16022368250023\\
1.12756356356356	7.08486211993794\\
1.12956456456456	7.01057952505035\\
1.13156556556557	6.93736691786048\\
1.13356656656657	6.86521458442324\\
1.13556756756757	6.79411215796215\\
1.13756856856857	6.72404869410359\\
1.13956956956957	6.65501274054208\\
1.14157057057057	6.58699240146128\\
1.14357157157157	6.5199753970242\\
1.14557257257257	6.45394911823515\\
1.14757357357357	6.38890067746318\\
1.14957457457457	6.32481695490469\\
1.15157557557558	6.26168464124962\\
1.15357657657658	6.19949027680324\\
1.15557757757758	6.13822028730218\\
1.15757857857858	6.07786101665115\\
1.15957957957958	6.01839875679402\\
1.16158058058058	5.95981977492094\\
1.16358158158158	5.90211033820159\\
1.16558258258258	5.84525673622311\\
1.16758358358358	5.78924530130023\\
1.16958458458458	5.7340624268153\\
1.17158558558559	5.67969458373503\\
1.17358658658659	5.62612833544206\\
1.17558758758759	5.57335035101011\\
1.17758858858859	5.52134741704275\\
1.17958958958959	5.47010644818814\\
1.18159059059059	5.41961449643401\\
1.18359159159159	5.36985875928035\\
1.18559259259259	5.32082658688024\\
1.18759359359359	5.27250548823291\\
1.18959459459459	5.22488313650732\\
1.1915955955956	5.1779473735686\\
1.1935965965966	5.13168621377489\\
1.1955975975976	5.08608784710681\\
1.1975985985986	5.04114064168745\\
1.1995995995996	4.99683314574631\\
1.2016006006006	4.95315408907682\\
1.2036016016016	4.91009238403321\\
1.2056026026026	4.86763712610894\\
1.2076036036036	4.82577759413594\\
1.2096046046046	4.78450325014077\\
1.21160560560561	4.74380373889069\\
1.21360660660661	4.70366888716082\\
1.21560760760761	4.66408870275021\\
1.21760860860861	4.62505337327316\\
1.21960960960961	4.58655326474971\\
1.22161061061061	4.54857892001719\\
1.22361161161161	4.51112105698328\\
1.22561261261261	4.47417056673911\\
1.22761361361361	4.43771851154943\\
1.22961461461461	4.40175612273558\\
1.23161561561562	4.36627479846552\\
1.23361661661662	4.33126610146411\\
1.23561761761762	4.29672175665571\\
1.23761861861862	4.26263364874982\\
1.23961961961962	4.22899381978014\\
1.24162062062062	4.19579446660579\\
1.24362162162162	4.16302793838335\\
1.24562262262262	4.13068673401703\\
1.24762362362362	4.09876349959392\\
1.24962462462462	4.06725102581058\\
1.25162562562563	4.0361422453965\\
1.25362662662663	4.00543023053963\\
1.25562762762763	3.97510819031853\\
1.25762862862863	3.94516946814515\\
1.25962962962963	3.91560753922213\\
1.26163063063063	3.88641600801781\\
1.26363163163163	3.8575886057619\\
1.26563263263263	3.82911918796453\\
1.26763363363363	3.80100173196089\\
1.26963463463463	3.77323033448356\\
1.27163563563564	3.74579920926432\\
1.27363663663664	3.71870268466703\\
1.27563763763764	3.69193520135283\\
1.27763863863864	3.66549130997891\\
1.27963963963964	3.63936566893183\\
1.28164064064064	3.61355304209605\\
1.28364164164164	3.5880482966586\\
1.28564264264264	3.56284640095016\\
1.28764364364364	3.53794242232308\\
1.28964464464464	3.5133315250667\\
1.29164564564565	3.48900896836\\
1.29364664664665	3.46497010426184\\
1.29564764764765	3.44121037573874\\
1.29764864864865	3.41772531473013\\
1.29964964964965	3.39451054025103\\
1.30165065065065	3.37156175653186\\
1.30365165165165	3.34887475119533\\
1.30565265265265	3.3264453934699\\
1.30765365365365	3.30426963243966\\
1.30965465465465	3.28234349533026\\
1.31165565565566	3.26066308583032\\
1.31365665665666	3.23922458244817\\
1.31565765765766	3.2180242369032\\
1.31765865865866	3.19705837255155\\
1.31965965965966	3.17632338284553\\
1.32166066066066	3.15581572982629\\
1.32366166166166	3.13553194264924\\
1.32566266266266	3.11546861614172\\
1.32766366366366	3.0956224093922\\
1.32966466466466	3.07599004437074\\
1.33166566566567	3.05656830457992\\
1.33366666666667	3.03735403373575\\
1.33566766766767	3.0183441344781\\
1.33766866866867	2.99953556710989\\
1.33966966966967	2.98092534836468\\
1.34167067067067	2.96251055020198\\
1.34367167167167	2.94428829862974\\
1.34567267267267	2.92625577255344\\
1.34767367367367	2.90841020265138\\
1.34967467467467	2.89074887027537\\
1.35167567567568	2.8732691063765\\
1.35367667667668	2.85596829045533\\
1.35567767767768	2.83884384953606\\
1.35767867867868	2.82189325716402\\
1.35967967967968	2.80511403242617\\
1.36168068068068	2.78850373899385\\
1.36368168168168	2.77205998418755\\
1.36568268268268	2.75578041806296\\
1.36768368368368	2.73966273251796\\
1.36968468468468	2.7237046604201\\
1.37168568568569	2.70790397475397\\
1.37368668668669	2.69225848778807\\
1.37568768768769	2.67676605026078\\
1.37768868868869	2.66142455058495\\
1.37968968968969	2.64623191407056\\
1.38169069069069	2.63118610216528\\
1.38369169169169	2.61628511171224\\
1.38569269269269	2.60152697422478\\
1.38769369369369	2.58690975517777\\
1.38969469469469	2.57243155331497\\
1.3916956956957	2.55809049997221\\
1.3936966966967	2.54388475841593\\
1.3956976976977	2.52981252319673\\
1.3976986986987	2.51587201951763\\
1.3996996996997	2.50206150261651\\
1.4017007007007	2.48837925716264\\
1.4037017017017	2.47482359666678\\
1.4057027027027	2.46139286290459\\
1.4077037037037	2.44808542535302\\
1.4097047047047	2.4348996806394\\
1.41170570570571	2.42183405200284\\
1.41370670670671	2.40888698876779\\
1.41570770770771	2.39605696582924\\
1.41770870870871	2.38334248314963\\
1.41970970970971	2.37074206526674\\
1.42171071071071	2.35825426081271\\
1.42371171171171	2.34587764204367\\
1.42571271271271	2.33361080437985\\
1.42771371371371	2.32145236595578\\
1.42971471471471	2.30940096718055\\
1.43171571571572	2.29745527030766\\
1.43371671671672	2.28561395901441\\
1.43571771771772	2.27387573799049\\
1.43771871871872	2.26223933253561\\
1.43971971971972	2.25070348816589\\
1.44172072072072	2.2392669702289\\
1.44372172172172	2.22792856352707\\
1.44572272272272	2.21668707194923\\
1.44772372372372	2.20554131811023\\
1.44972472472472	2.19449014299827\\
1.45172572572573	2.18353240562994\\
1.45372672672673	2.17266698271262\\
1.45572772772773	2.16189276831415\\
1.45772872872873	2.15120867353961\\
1.45972972972973	2.14061362621501\\
1.46173073073073	2.13010657057773\\
1.46373173173173	2.11968646697355\\
1.46573273273273	2.10935229156011\\
1.46773373373373	2.0991030360167\\
1.46973473473473	2.08893770726013\\
1.47173573573574	2.07885532716667\\
1.47373673673674	2.06885493229974\\
1.47573773773774	2.05893557364345\\
1.47773873873874	2.04909631634161\\
1.47973973973974	2.03933623944227\\
1.48174074074074	2.02965443564753\\
1.48374174174174	2.02005001106861\\
1.48574274274274	2.01052208498596\\
1.48774374374374	2.00106978961437\\
1.48974474474474	1.99169226987288\\
1.49174574574575	1.98238868315952\\
1.49374674674675	1.97315819913055\\
1.49574774774775	1.96399999948435\\
1.49774874874875	1.95491327774967\\
1.49974974974975	1.94589723907819\\
1.50175075075075	1.93695110004131\\
1.50375175175175	1.92807408843108\\
1.50575275275275	1.91926544306517\\
1.50775375375375	1.9105244135957\\
1.50975475475475	1.90185026032205\\
1.51175575575576	1.89324225400729\\
1.51375675675676	1.8846996756984\\
1.51575775775776	1.87622181655003\\
1.51775875875876	1.86780797765179\\
1.51975975975976	1.85945746985894\\
1.52176076076076	1.85116961362649\\
1.52376176176176	1.84294373884659\\
1.52576276276276	1.83477918468902\\
1.52776376376376	1.82667529944499\\
1.52976476476476	1.81863144037386\\
1.53176576576577	1.8106469735529\\
1.53376676676677	1.80272127373008\\
1.53576776776777	1.79485372417958\\
1.53776876876877	1.78704371656022\\
1.53976976976977	1.77929065077655\\
1.54177077077077	1.77159393484272\\
1.54377177177177	1.76395298474891\\
1.54577277277277	1.75636722433031\\
1.54777377377377	1.74883608513869\\
1.54977477477477	1.74135900631638\\
1.55177577577578	1.73393543447262\\
1.55377677677678	1.72656482356236\\
1.55577777777778	1.71924663476727\\
1.55777877877878	1.71198033637906\\
1.55977977977978	1.70476540368496\\
1.56178078078078	1.69760131885542\\
1.56378178178178	1.69048757083385\\
1.56578278278278	1.68342365522846\\
1.56778378378378	1.67640907420615\\
1.56978478478478	1.66944333638834\\
1.57178578578579	1.66252595674873\\
1.57378678678679	1.655656456513\\
1.57578778778779	1.64883436306032\\
1.57778878878879	1.64205920982674\\
1.57978978978979	1.63533053621028\\
1.58179079079079	1.6286478874778\\
1.58379179179179	1.62201081467362\\
1.58579279279279	1.61541887452973\\
1.58779379379379	1.60887162937769\\
1.58979479479479	1.60236864706211\\
1.5917957957958	1.59590950085572\\
1.5937967967968	1.58949376937593\\
1.5957977977978	1.58312103650296\\
1.5977987987988	1.57679089129938\\
1.5997997997998	1.57050292793114\\
1.6018008008008	1.56425674558999\\
1.6038018018018	1.55805194841731\\
1.6058028028028	1.55188814542923\\
1.6078038038038	1.5457649504432\\
1.6098048048048	1.53968198200572\\
1.61180580580581	1.53363886332145\\
1.61380680680681	1.52763522218356\\
1.61580780780781	1.52167069090524\\
1.61780880880881	1.51574490625247\\
1.61980980980981	1.50985750937796\\
1.62181081081081	1.5040081457562\\
1.62381181181181	1.49819646511973\\
1.62581281281281	1.49242212139641\\
1.62781381381381	1.48668477264784\\
1.62981481481481	1.48098408100885\\
1.63181581581582	1.47531971262799\\
1.63381681681682	1.46969133760908\\
1.63581781781782	1.46409862995374\\
1.63781881881882	1.45854126750492\\
1.63981981981982	1.45301893189138\\
1.64182082082082	1.44753130847315\\
1.64382182182182	1.44207808628784\\
1.64582282282282	1.43665895799792\\
1.64782382382382	1.43127361983892\\
1.64982482482482	1.4259217715684\\
1.65182582582583	1.42060311641587\\
1.65382682682683	1.41531736103351\\
1.65582782782783	1.41006421544775\\
1.65782882882883	1.40484339301158\\
1.65982982982983	1.39965461035774\\
1.66183083083083	1.39449758735263\\
1.66383183183183	1.38937204705105\\
1.66583283283283	1.38427771565156\\
1.66783383383383	1.37921432245276\\
1.66983483483483	1.3741815998101\\
1.67183583583584	1.36917928309354\\
1.67383683683684	1.36420711064584\\
1.67583783783784	1.3592648237415\\
1.67783883883884	1.35435216654647\\
1.67983983983984	1.34946888607841\\
1.68184084084084	1.34461473216767\\
1.68384184184184	1.33978945741882\\
1.68584284284284	1.33499281717289\\
1.68784384384384	1.33022456947017\\
1.68984484484484	1.3254844750136\\
1.69184584584585	1.32077229713276\\
1.69384684684685	1.31608780174844\\
1.69584784784785	1.31143075733777\\
1.69784884884885	1.30680093489991\\
1.69984984984985	1.30219810792226\\
1.70185085085085	1.29762205234721\\
1.70385185185185	1.29307254653949\\
1.70585285285285	1.28854937125389\\
1.70785385385385	1.28405230960362\\
1.70985485485485	1.27958114702911\\
1.71185585585586	1.27513567126726\\
1.71385685685686	1.27071567232127\\
1.71585785785786	1.26632094243083\\
1.71785885885886	1.26195127604287\\
1.71985985985986	1.25760646978267\\
1.72186086086086	1.25328632242549\\
1.72386186186186	1.24899063486862\\
1.72586286286286	1.2447192101038\\
1.72786386386386	1.24047185319018\\
1.72986486486486	1.23624837122758\\
1.73186586586587	1.23204857333021\\
1.73386686686687	1.22787227060082\\
1.73586786786787	1.22371927610517\\
1.73786886886887	1.21958940484693\\
1.73986986986987	1.21548247374298\\
1.74187087087087	1.21139830159904\\
1.74387187187187	1.20733670908569\\
1.74587287287287	1.20329751871475\\
1.74787387387387	1.19928055481599\\
1.74987487487487	1.19528564351424\\
1.75187587587588	1.19131261270679\\
1.75387687687688	1.18736129204116\\
1.75587787787788	1.18343151289316\\
1.75787887887888	1.17952310834533\\
1.75987987987988	1.17563591316565\\
1.76188088088088	1.17176976378663\\
1.76388188188188	1.16792449828458\\
1.76588288288288	1.16409995635935\\
1.76788388388388	1.16029597931425\\
1.76988488488488	1.15651241003628\\
1.77188588588589	1.15274909297669\\
1.77388688688689	1.14900587413181\\
1.77588788788789	1.14528260102411\\
1.77788888888889	1.14157912268361\\
1.77988988988989	1.13789528962949\\
1.78189089089089	1.13423095385203\\
1.78389189189189	1.13058596879474\\
1.78589289289289	1.12696018933682\\
1.78789389389389	1.12335347177579\\
1.78989489489489	1.11976567381046\\
1.7918958958959	1.11619665452407\\
1.7938968968969	1.11264627436768\\
1.7958978978979	1.10911439514385\\
1.7978988988989	1.1056008799905\\
1.7998998998999	1.10210559336499\\
1.8019009009009	1.09862840102846\\
1.8039019019019	1.0951691700304\\
1.8059029029029	1.09172776869336\\
1.8079039039039	1.08830406659799\\
1.8099049049049	1.08489793456818\\
1.81190590590591	1.08150924465647\\
1.81390690690691	1.07813787012965\\
1.81590790790791	1.07478368545456\\
1.81790890890891	1.07144656628407\\
1.81990990990991	1.0681263894433\\
1.82191091091091	1.06482303291596\\
1.82391191191191	1.06153637583094\\
1.82591291291291	1.05826629844907\\
1.82791391391391	1.05501268215002\\
1.82991491491491	1.05177540941945\\
1.83191591591592	1.04855436383627\\
1.83391691691692	1.04534943006013\\
1.83591791791792	1.04216049381902\\
1.83791891891892	1.0389874418971\\
1.83991991991992	1.03583016212264\\
1.84192092092092	1.03268854335617\\
1.84392192192192	1.02956247547876\\
1.84592292292292	1.02645184938044\\
1.84792392392392	1.02335655694887\\
1.84992492492492	1.02027649105802\\
1.85192592592593	1.01721154555709\\
1.85392692692693	1.01416161525963\\
1.85592792792793	1.01112659593263\\
1.85792892892893	1.00810638428594\\
1.85992992992993	1.00510087796175\\
1.86193093093093	1.00210997552417\\
1.86393193193193	0.99913357644901\\
1.86593293293293	0.996171581113695\\
1.86793393393393	0.993223890787267\\
1.86993493493493	0.990290407620553\\
1.87193593593594	0.987371034636452\\
1.87393693693694	0.984465675720353\\
1.87593793793794	0.981574235610665\\
1.87793893893894	0.978696619889492\\
1.87993993993994	0.975832734973407\\
1.88194094094094	0.972982488104361\\
1.88394194194194	0.970145787340703\\
1.88594294294294	0.967322541548319\\
1.88794394394394	0.964512660391879\\
1.88994494494494	0.961716054326206\\
1.89194594594595	0.958932634587748\\
1.89394694694695	0.956162313186166\\
1.89594794794795	0.953405002896022\\
1.89794894894895	0.950660617248576\\
1.89994994994995	0.947929070523691\\
1.90195095095095	0.94521027774184\\
1.90395195195195	0.942504154656209\\
1.90595295295295	0.939810617744902\\
1.90795395395395	0.937129584203252\\
1.90995495495495	0.934460971936218\\
1.91195595595596	0.931804699550883\\
1.91395695695696	0.92916068634905\\
1.91595795795796	0.926528852319921\\
1.91795895895896	0.923909118132877\\
1.91995995995996	0.921301405130343\\
1.92196096096096	0.918705635320743\\
1.92396196196196	0.916121731371545\\
1.92596296296296	0.913549616602386\\
1.92796396396396	0.910989214978288\\
1.92996496496496	0.908440451102961\\
1.93196596596597	0.905903250212175\\
1.93396696696697	0.903377538167229\\
1.93596796796797	0.900863241448493\\
1.93796896896897	0.898360287149026\\
1.93996996996997	0.895868602968279\\
1.94197097097097	0.893388117205873\\
1.94397197197197	0.890918758755449\\
1.94597297297297	0.888460457098598\\
1.94797397397397	0.886013142298859\\
1.94997497497497	0.883576744995797\\
1.95197597597598	0.881151196399148\\
1.95397697697698	0.878736428283035\\
1.95597797797798	0.876332372980252\\
1.95797897897898	0.873938963376624\\
1.95997997997998	0.871556132905424\\
1.96198098098098	0.869183815541868\\
1.96398198198198	0.866821945797668\\
1.96598298298298	0.864470458715652\\
1.96798398398398	0.862129289864449\\
1.96998498498498	0.85979837533324\\
1.97198598598599	0.857477651726563\\
1.97398698698699	0.855167056159188\\
1.97598798798799	0.852866526251052\\
1.97798898898899	0.850576000122244\\
1.97998998998999	0.848295416388064\\
1.98199099099099	0.846024714154126\\
1.98399199199199	0.843763833011532\\
1.98599299299299	0.841512713032086\\
1.98799399399399	0.839271294763586\\
1.98999499499499	0.837039519225144\\
1.991995995996	0.834817327902588\\
1.993996996997	0.832604662743897\\
1.995997997998	0.830401466154698\\
1.997998998999	0.828207680993813\\
2	0.826023250568862\\
};
\addlegendentry{$\alpha\text{ = 0}$};

\addplot [color=blue,solid]
  table[row sep=crcr]{%
0.001	2.46625082659764\\
0.003001001001001	2.46626941868887\\
0.005002002002002	2.46630660831371\\
0.007003003003003	2.4663623970542\\
0.009004004004004	2.46643678728374\\
0.011005005005005	2.46652978216725\\
0.013006006006006	2.46664138566155\\
0.015007007007007	2.46677160251568\\
0.017008008008008	2.46692043827141\\
0.019009009009009	2.46708789926377\\
0.02101001001001	2.46727399262166\\
0.023011011011011	2.46747872626856\\
0.025012012012012	2.46770210892331\\
0.027013013013013	2.46794415010094\\
0.029014014014014	2.46820486011363\\
0.031015015015015	2.46848425007169\\
0.033016016016016	2.46878233188471\\
0.035017017017017	2.46909911826263\\
0.037018018018018	2.46943462271708\\
0.039019019019019	2.46978885956265\\
0.04102002002002	2.47016184391831\\
0.043021021021021	2.47055359170891\\
0.045022022022022	2.47096411966672\\
0.047023023023023	2.47139344533309\\
0.049024024024024	2.47184158706015\\
0.051025025025025	2.47230856401266\\
0.053026026026026	2.47279439616986\\
0.055027027027027	2.47329910432745\\
0.057028028028028	2.47382271009961\\
0.059029029029029	2.47436523592119\\
0.06103003003003	2.47492670504987\\
0.063031031031031	2.47550714156849\\
0.065032032032032	2.47610657038738\\
0.067033033033033	2.47672501724686\\
0.069034034034034	2.47736250871981\\
0.071035035035035	2.47801907221422\\
0.073036036036036	2.47869473597598\\
0.075037037037037	2.47938952909163\\
0.077038038038038	2.48010348149127\\
0.079039039039039	2.48083662395153\\
0.08104004004004	2.48158898809857\\
0.083041041041041	2.48236060641132\\
0.085042042042042	2.48315151222465\\
0.0870430430430431	2.48396173973268\\
0.089044044044044	2.48479132399221\\
0.091045045045045	2.48564030092622\\
0.093046046046046	2.48650870732742\\
0.0950470470470471	2.48739658086198\\
0.0970480480480481	2.48830396007324\\
0.0990490490490491	2.4892308843856\\
0.10105005005005	2.49017739410847\\
0.103051051051051	2.4911435304403\\
0.105052052052052	2.49212933547272\\
0.107053053053053	2.49313485219475\\
0.109054054054054	2.49416012449717\\
0.111055055055055	2.49520519717689\\
0.113056056056056	2.49627011594151\\
0.115057057057057	2.49735492741388\\
0.117058058058058	2.49845967913686\\
0.119059059059059	2.49958441957809\\
0.12106006006006	2.50072919813494\\
0.123061061061061	2.50189406513947\\
0.125062062062062	2.50307907186357\\
0.127063063063063	2.50428427052416\\
0.129064064064064	2.50550971428849\\
0.131065065065065	2.50675545727957\\
0.133066066066066	2.50802155458173\\
0.135067067067067	2.50930806224613\\
0.137068068068068	2.51061503729664\\
0.139069069069069	2.51194253773554\\
0.14107007007007	2.51329062254959\\
0.143071071071071	2.51465935171597\\
0.145072072072072	2.51604878620853\\
0.147073073073073	2.51745898800406\\
0.149074074074074	2.5188900200886\\
0.151075075075075	2.52034194646405\\
0.153076076076076	2.52181483215471\\
0.155077077077077	2.52330874321406\\
0.157078078078078	2.52482374673155\\
0.159079079079079	2.52635991083964\\
0.16108008008008	2.52791730472083\\
0.163081081081081	2.52949599861487\\
0.165082082082082	2.53109606382609\\
0.167083083083083	2.53271757273091\\
0.169084084084084	2.53436059878527\\
0.171085085085085	2.5360252165325\\
0.173086086086086	2.537711501611\\
0.175087087087087	2.53941953076226\\
0.177088088088088	2.54114938183894\\
0.179089089089089	2.54290113381304\\
0.18109009009009	2.5446748667843\\
0.183091091091091	2.5464706619886\\
0.185092092092092	2.54828860180666\\
0.187093093093093	2.55012876977269\\
0.189094094094094	2.55199125058334\\
0.191095095095095	2.55387613010669\\
0.193096096096096	2.55578349539143\\
0.195097097097097	2.55771343467612\\
0.197098098098098	2.55966603739871\\
0.199099099099099	2.56164139420605\\
0.2011001001001	2.56363959696371\\
0.203101101101101	2.56566073876577\\
0.205102102102102	2.56770491394497\\
0.207103103103103	2.5697722180828\\
0.209104104104104	2.57186274801988\\
0.211105105105105	2.57397660186648\\
0.213106106106106	2.57611387901315\\
0.215107107107107	2.57827468014152\\
0.217108108108108	2.5804591072353\\
0.219109109109109	2.58266726359141\\
0.22111011011011	2.58489925383125\\
0.223111111111111	2.5871551839122\\
0.225112112112112	2.58943516113929\\
0.227113113113113	2.5917392941769\\
0.229114114114114	2.59406769306083\\
0.231115115115115	2.5964204692104\\
0.233116116116116	2.59879773544083\\
0.235117117117117	2.60119960597569\\
0.237118118118118	2.60362619645961\\
0.239119119119119	2.60607762397115\\
0.24112012012012	2.60855400703586\\
0.243121121121121	2.61105546563952\\
0.245122122122122	2.61358212124155\\
0.247123123123123	2.61613409678866\\
0.249124124124124	2.61871151672865\\
0.251125125125125	2.62131450702442\\
0.253126126126126	2.62394319516819\\
0.255127127127127	2.62659771019592\\
0.257128128128128	2.62927818270189\\
0.259129129129129	2.63198474485358\\
0.26113013013013	2.63471753040662\\
0.263131131131131	2.63747667472011\\
0.265132132132132	2.64026231477204\\
0.267133133133133	2.64307458917494\\
0.269134134134134	2.64591363819181\\
0.271135135135135	2.64877960375221\\
0.273136136136136	2.65167262946858\\
0.275137137137137	2.65459286065282\\
0.277138138138138	2.65754044433311\\
0.279139139139139	2.66051552927085\\
0.28114014014014	2.66351826597806\\
0.283141141141141	2.6665488067347\\
0.285142142142142	2.66960730560654\\
0.287143143143143	2.67269391846312\\
0.289144144144144	2.67580880299597\\
0.291145145145145	2.67895211873705\\
0.293146146146146	2.68212402707756\\
0.295147147147147	2.68532469128685\\
0.297148148148148	2.68855427653178\\
0.299149149149149	2.69181294989611\\
0.30115015015015	2.69510088040032\\
0.303151151151151	2.69841823902178\\
0.305152152152152	2.70176519871484\\
0.307153153153153	2.7051419344316\\
0.309154154154154	2.7085486231428\\
0.311155155155155	2.71198544385881\\
0.313156156156156	2.71545257765126\\
0.315157157157157	2.71895020767467\\
0.317158158158158	2.72247851918851\\
0.319159159159159	2.72603769957949\\
0.32116016016016	2.72962793838428\\
0.323161161161161	2.7332494273123\\
0.325162162162162	2.73690236026907\\
0.327163163163163	2.74058693337972\\
0.329164164164164	2.74430334501282\\
0.331165165165165	2.74805179580459\\
0.333166166166166	2.75183248868343\\
0.335167167167167	2.7556456288947\\
0.337168168168168	2.7594914240259\\
0.339169169169169	2.76337008403218\\
0.34117017017017	2.76728182126214\\
0.343171171171171	2.77122685048405\\
0.345172172172172	2.77520538891231\\
0.347173173173173	2.77921765623437\\
0.349174174174174	2.78326387463795\\
0.351175175175175	2.78734426883857\\
0.353176176176176	2.79145906610759\\
0.355177177177177	2.79560849630043\\
0.357178178178178	2.79979279188531\\
0.359179179179179	2.80401218797227\\
0.36118018018018	2.80826692234264\\
0.363181181181181	2.81255723547882\\
0.365182182182182	2.81688337059454\\
0.367183183183183	2.82124557366539\\
0.369184184184184	2.82564409345993\\
0.371185185185185	2.8300791815709\\
0.373186186186186	2.83455109244721\\
0.375187187187187	2.83906008342606\\
0.377188188188188	2.84360641476552\\
0.379189189189189	2.8481903496777\\
0.38119019019019	2.85281215436209\\
0.383191191191191	2.85747209803949\\
0.385192192192192	2.86217045298638\\
0.387193193193193	2.86690749456957\\
0.389194194194194	2.87168350128149\\
0.391195195195195	2.8764987547758\\
0.393196196196196	2.88135353990348\\
0.395197197197197	2.88624814474933\\
0.397198198198198	2.89118286066909\\
0.399199199199199	2.8961579823267\\
0.4012002002002	2.90117380773241\\
0.403201201201201	2.90623063828109\\
0.405202202202202	2.91132877879105\\
0.407203203203203	2.91646853754359\\
0.409204204204204	2.92165022632257\\
0.411205205205205	2.92687416045484\\
0.413206206206206	2.93214065885114\\
0.415207207207207	2.93745004404726\\
0.417208208208208	2.94280264224587\\
0.419209209209209	2.94819878335892\\
0.42121021021021	2.95363880105033\\
0.423211211211211	2.95912303277937\\
0.425212212212212	2.96465181984446\\
0.427213213213213	2.9702255074276\\
0.429214214214214	2.97584444463908\\
0.431215215215215	2.98150898456311\\
0.433216216216216	2.98721948430351\\
0.435217217217217	2.99297630503032\\
0.437218218218218	2.9987798120267\\
0.439219219219219	3.00463037473652\\
0.44122022022022	3.01052836681239\\
0.443221221221221	3.01647416616421\\
0.445222222222222	3.02246815500844\\
0.447223223223223	3.02851071991771\\
0.449224224224224	3.03460225187104\\
0.451225225225225	3.04074314630477\\
0.453226226226226	3.04693380316367\\
0.455227227227227	3.05317462695319\\
0.457228228228228	3.05946602679147\\
0.459229229229229	3.06580841646276\\
0.46123023023023	3.07220221447072\\
0.463231231231231	3.07864784409271\\
0.465232232232232	3.08514573343431\\
0.467233233233233	3.09169631548471\\
0.469234234234234	3.09830002817235\\
0.471235235235235	3.1049573144215\\
0.473236236236236	3.11166862220875\\
0.475237237237237	3.118434404621\\
0.477238238238238	3.12525511991298\\
0.479239239239239	3.13213123156615\\
0.48124024024024	3.13906320834754\\
0.483241241241241	3.14605152436957\\
0.485242242242242	3.15309665915022\\
0.487243243243243	3.16019909767355\\
0.489244244244244	3.16735933045112\\
0.491245245245245	3.17457785358375\\
0.493246246246246	3.18185516882381\\
0.495247247247247	3.18919178363784\\
0.497248248248248	3.19658821127006\\
0.499249249249249	3.20404497080605\\
0.50125025025025	3.21156258723719\\
0.503251251251251	3.21914159152521\\
0.505252252252252	3.22678252066745\\
0.507253253253253	3.23448591776277\\
0.509254254254254	3.24225233207752\\
0.511255255255255	3.25008231911208\\
0.513256256256256	3.25797644066796\\
0.515257257257257	3.26593526491524\\
0.517258258258258	3.27395936646024\\
0.519259259259259	3.28204932641366\\
0.52126026026026	3.29020573245941\\
0.523261261261261	3.29842917892316\\
0.525262262262262	3.30672026684168\\
0.527263263263263	3.31507960403237\\
0.529264264264264	3.32350780516299\\
0.531265265265265	3.33200549182165\\
0.533266266266266	3.34057329258716\\
0.535267267267267	3.34921184309963\\
0.537268268268268	3.35792178613075\\
0.539269269269269	3.36670377165492\\
0.54127027027027	3.37555845692028\\
0.543271271271271	3.38448650651912\\
0.545272272272272	3.39348859245976\\
0.547273273273273	3.40256539423711\\
0.549274274274274	3.41171759890383\\
0.551275275275275	3.42094590114158\\
0.553276276276276	3.43025100333161\\
0.555277277277277	3.43963361562588\\
0.557278278278278	3.44909445601736\\
0.559279279279279	3.45863425041072\\
0.56128028028028	3.46825373269218\\
0.563281281281281	3.47795364479932\\
0.565282282282282	3.48773473679065\\
0.567283283283283	3.4975977669143\\
0.569284284284284	3.50754350167663\\
0.571285285285285	3.51757271590993\\
0.573286286286286	3.52768619283996\\
0.575287287287287	3.53788472415232\\
0.577288288288288	3.54816911005801\\
0.579289289289289	3.5585401593593\\
0.58129029029029	3.5689986895123\\
0.583291291291291	3.57954552669178\\
0.585292292292292	3.59018150585141\\
0.587293293293293	3.60090747078623\\
0.589294294294294	3.61172427419089\\
0.591295295295295	3.62263277771858\\
0.593296296296296	3.63363385203804\\
0.595297297297297	3.64472837688788\\
0.597298298298298	3.65591724113164\\
0.599299299299299	3.66720134280799\\
0.6013003003003	3.6785815891824\\
0.603301301301301	3.69005889679373\\
0.605302302302302	3.7016341915012\\
0.607303303303303	3.71330840852771\\
0.609304304304304	3.7250824925008\\
0.611305305305305	3.73695739749231\\
0.613306306306306	3.74893408705375\\
0.615307307307307	3.76101353425047\\
0.617308308308308	3.77319672169157\\
0.619309309309309	3.78548464155753\\
0.62131031031031	3.79787829562483\\
0.623311311311311	3.81037869528584\\
0.625312312312312	3.82298686156674\\
0.627313313313313	3.83570382514035\\
0.629314314314314	3.84853062633641\\
0.631315315315315	3.86146831514584\\
0.633316316316316	3.874517951223\\
0.635317317317317	3.88768060388065\\
0.637318318318318	3.90095735208346\\
0.639319319319319	3.91434928443314\\
0.64132032032032	3.92785749915133\\
0.643321321321321	3.94148310405492\\
0.645322322322322	3.95522721652748\\
0.647323323323323	3.96909096348282\\
0.649324324324324	3.98307548132518\\
0.651325325325325	3.99718191590031\\
0.653326326326326	4.01141142244089\\
0.655327327327327	4.02576516550644\\
0.657328328328328	4.040244318914\\
0.659329329329329	4.05485006566233\\
0.66133033033033	4.06958359784825\\
0.663331331331331	4.08444611657571\\
0.665332332332332	4.0994388318545\\
0.667333333333333	4.11456296249247\\
0.669334334334334	4.12981973597876\\
0.671335335335335	4.14521038835509\\
0.673336336336336	4.16073616408062\\
0.675337337337337	4.17639831588544\\
0.677338338338338	4.19219810461423\\
0.679339339339339	4.20813679905766\\
0.68134034034034	4.22421567577637\\
0.683341341341341	4.24043601890923\\
0.685342342342342	4.25679911997498\\
0.687343343343343	4.27330627765585\\
0.689344344344344	4.28995879757408\\
0.691345345345345	4.30675799205282\\
0.693346346346346	4.32370517986532\\
0.695347347347347	4.34080168596889\\
0.697348348348348	4.35804884122551\\
0.699349349349349	4.37544798211109\\
0.70135035035035	4.39300045040375\\
0.703351351351351	4.41070759286453\\
0.705352352352352	4.42857076089491\\
0.707353353353353	4.44659131018601\\
0.709354354354354	4.46477060034521\\
0.711355355355355	4.48310999451031\\
0.713356356356356	4.50161085894405\\
0.715357357357357	4.52027456261367\\
0.717358358358358	4.53910247675054\\
0.719359359359359	4.55809597439004\\
0.72136036036036	4.57725642990061\\
0.723361361361361	4.59658521848366\\
0.725362362362362	4.61608371566168\\
0.727363363363363	4.63575329674246\\
0.729364364364364	4.6555953362667\\
0.731365365365365	4.67561120743307\\
0.733366366366366	4.69580228150325\\
0.735367367367367	4.71616992718755\\
0.737368368368368	4.73671551000678\\
0.739369369369369	4.7574403916336\\
0.74137037037037	4.77834592921611\\
0.743371371371371	4.79943347467127\\
0.745372372372372	4.82070437396766\\
0.747373373373373	4.84215996637529\\
0.749374374374374	4.86380158369699\\
0.751375375375375	4.88563054948327\\
0.753376376376376	4.90764817821278\\
0.755377377377377	4.92985577446268\\
0.757378378378378	4.95225463204581\\
0.759379379379379	4.97484603313221\\
0.76138038038038	4.99763124734772\\
0.763381381381381	5.02061153084441\\
0.765382382382382	5.04378812535526\\
0.767383383383383	5.06716225722534\\
0.769384384384384	5.09073513641435\\
0.771385385385385	5.11450795549013\\
0.773386386386386	5.13848188859028\\
0.775387387387387	5.16265809036429\\
0.777388388388388	5.18703769490245\\
0.779389389389389	5.21162181463377\\
0.78139039039039	5.23641153921456\\
0.783391391391391	5.26140793439143\\
0.785392392392392	5.28661204084868\\
0.787393393393393	5.31202487303631\\
0.789394394394394	5.33764741798348\\
0.791395395395395	5.36348063409655\\
0.793396396396396	5.38952544993785\\
0.795397397397397	5.41578276299088\\
0.797398398398398	5.44225343841951\\
0.799399399399399	5.46893830780343\\
0.8014004004004	5.49583816787515\\
0.803401401401401	5.52295377923688\\
0.805402402402402	5.55028586481798\\
0.807403403403404	5.5778351095837\\
0.809404404404404	5.60560215775151\\
0.811405405405405	5.63358761246092\\
0.813406406406406	5.66179203422449\\
0.815407407407407	5.69021593961802\\
0.817408408408409	5.71885979997107\\
0.819409409409409	5.74772404005915\\
0.82141041041041	5.77680903679951\\
0.823411411411412	5.8061151179521\\
0.825412412412412	5.83564256082752\\
0.827413413413413	5.86539159100388\\
0.829414414414414	5.89536238105415\\
0.831415415415415	5.92555504928613\\
0.833416416416417	5.95596965849671\\
0.835417417417417	5.9866062147425\\
0.837418418418418	6.01746466612856\\
0.839419419419419	6.04854490161727\\
0.84142042042042	6.07984674985918\\
0.843421421421422	6.11136997804777\\
0.845422422422422	6.14311429079994\\
0.847423423423423	6.17507932906421\\
0.849424424424424	6.20726466905826\\
0.851425425425425	6.2396698212378\\
0.853426426426427	6.27229422929832\\
0.855427427427428	6.30513726921161\\
0.857428428428428	6.33819824829838\\
0.859429429429429	6.37147640433884\\
0.86143043043043	6.40497090472245\\
0.863431431431432	6.43868084563837\\
0.865432432432433	6.47260525130783\\
0.867433433433433	6.50674307325956\\
0.869434434434434	6.54109318964939\\
0.871435435435435	6.57565440462497\\
0.873436436436437	6.61042544773633\\
0.875437437437438	6.64540497339306\\
0.877438438438438	6.68059156036846\\
0.879439439439439	6.71598371135131\\
0.88144044044044	6.7515798525451\\
0.883441441441442	6.78737833331511\\
0.885442442442442	6.82337742588297\\
0.887443443443444	6.85957532506845\\
0.889444444444444	6.89597014807807\\
0.891445445445445	6.93255993433969\\
0.893446446446447	6.9693426453823\\
0.895447447447447	7.00631616475985\\
0.897448448448449	7.04347829801798\\
0.899449449449449	7.08082677270184\\
0.90145045045045	7.11835923840369\\
0.903451451451452	7.156073266848\\
0.905452452452452	7.19396635201211\\
0.907453453453454	7.23203591028005\\
0.909454454454455	7.27027928062686\\
0.911455455455455	7.30869372483072\\
0.913456456456457	7.34727642770978\\
0.915457457457457	7.38602449738043\\
0.917458458458459	7.42493496553366\\
0.91945945945946	7.46400478772557\\
0.92146046046046	7.50323084367839\\
0.923461461461462	7.54260993758759\\
0.925462462462462	7.58213879843088\\
0.927463463463464	7.62181408027435\\
0.929464464464465	7.66163236257104\\
0.931465465465466	7.70159015044674\\
0.933466466466467	7.74168387496774\\
0.935467467467467	7.78190989338508\\
0.937468468468469	7.82226448934931\\
0.93946946946947	7.86274387308991\\
0.941470470470471	7.90334418155291\\
0.943471471471471	7.94406147849033\\
0.945472472472472	7.98489175449446\\
0.947473473473474	8.0258309269699\\
0.949474474474475	8.06687484003604\\
0.951475475475476	8.10801926435224\\
0.953476476476476	8.14925989685755\\
0.955477477477477	8.19059236041687\\
0.957478478478479	8.23201220336454\\
0.95947947947948	8.27351489893631\\
0.961480480480481	8.31509584458031\\
0.963481481481481	8.35675036113674\\
0.965482482482482	8.39847369187604\\
0.967483483483484	8.44026100138434\\
0.969484484484485	8.48210737428473\\
0.971485485485486	8.52400781378208\\
0.973486486486487	8.56595724001845\\
0.975487487487487	8.60795048822555\\
0.977488488488489	8.64998230665964\\
0.97948948948949	8.69204735430366\\
0.981490490490491	8.73414019831993\\
0.983491491491492	8.77625531123633\\
0.985492492492492	8.81838706784687\\
0.987493493493494	8.86052974180697\\
0.989494494494495	8.90267750190173\\
0.991495495495496	8.94482440796419\\
0.993496496496497	8.98696440641869\\
0.995497497497498	9.02909132542221\\
0.997498498498499	9.07119886957491\\
0.9994994994995	9.11328061416786\\
1.0015005005005	9.15532999893406\\
1.0035015015015	9.19734032126514\\
1.0055025025025	9.23930472885313\\
1.0075035035035	9.28121621171281\\
1.0095045045045	9.32306759353611\\
1.01150550550551	9.36485152232495\\
1.01350650650651	9.40656046024449\\
1.01550750750751	9.44818667263183\\
1.01750850850851	9.48972221608976\\
1.01950950950951	9.53115892558714\\
1.02151051051051	9.57248840047936\\
1.02351151151151	9.61370198935345\\
1.02551251251251	9.65479077359147\\
1.02751351351351	9.69574554953443\\
1.02951451451451	9.73655680911565\\
1.03151551551552	9.77721471881714\\
1.03351651651652	9.81770909678615\\
1.03551751751752	9.85802938792889\\
1.03751851851852	9.89816463677711\\
1.03951951951952	9.93810345789757\\
1.04152052052052	9.97783400358593\\
1.04352152152152	10.0173439285539\\
1.04552252252252	10.0566203512799\\
1.04752352352352	10.0956498116511\\
1.04952452452452	10.1344182244742\\
1.05152552552553	10.1729108283726\\
1.05352652652653	10.211112129524\\
1.05552752752753	10.249005839611\\
1.05752852852853	10.2865748072666\\
1.05952952952953	10.3238009421914\\
1.06153053053053	10.3606651309933\\
1.06353153153153	10.3971471436519\\
1.06553253253253	10.4332255293384\\
1.06753353353353	10.4688775001132\\
1.06953453453453	10.5040788007812\\
1.07153553553554	10.5388035628882\\
1.07353653653654	10.5730241404957\\
1.07553753753754	10.6067109249454\\
1.07753853853854	10.6398321353176\\
1.07953953953954	10.6723535806656\\
1.08154054054054	10.7042383893554\\
1.08354154154154	10.7354466999108\\
1.08554254254254	10.7659353066245\\
1.08754354354354	10.7956572517796\\
1.08954454454454	10.8245613545631\\
1.09154554554555	10.8525916645397\\
1.09354654654655	10.879686824767\\
1.09554754754755	10.905779326082\\
1.09754854854855	10.9307946295528\\
1.09954954954955	10.954650128236\\
1.10155055055055	10.9772539117703\\
1.10355155155155	10.9985032873619\\
1.10555255255255	11.0182829975165\\
1.10755355355355	11.0364630572411\\
1.10955455455455	11.0528961096341\\
1.11155555555556	11.0674141663014\\
1.11355655655656	11.0798245541807\\
1.11555755755756	11.0899048276322\\
1.11755855855856	11.097396315759\\
1.11955955955956	11.1019958471041\\
1.12156056056056	11.1033450072488\\
1.12356156156156	11.1010160078656\\
1.12556256256256	11.0944928275206\\
1.12756356356356	11.0831456413648\\
1.12956456456456	11.0661955492478\\
1.13156556556557	11.0426650038907\\
1.13356656656657	11.0113067316063\\
1.13556756756757	10.9704996531671\\
1.13756856856857	10.9180932738632\\
1.13956956956957	10.8511707835445\\
1.14157057057057	10.7656851732026\\
1.14357157157157	10.6559098575457\\
1.14557257257257	10.5136864889124\\
1.14757357357357	10.327782503369\\
1.14957457457457	10.0850958593999\\
1.15157557557558	9.77882345875767\\
1.15357657657658	9.42554882065737\\
1.15557757757758	9.0656778694884\\
1.15757857857858	8.73221158711388\\
1.15957957957958	8.43557085497145\\
1.16158058058058	8.1735210230886\\
1.16358158158158	7.94055790256468\\
1.16558258258258	7.7313917185937\\
1.16758358358358	7.5417122004921\\
1.16958458458458	7.36815611345953\\
1.17158558558559	7.20811460761062\\
1.17358658658659	7.05954884592045\\
1.17558758758759	6.92084596231456\\
1.17758858858859	6.79071284026373\\
1.17958958958959	6.66809891832726\\
1.18159059059059	6.55213998286067\\
1.18359159159159	6.4421168790879\\
1.18559259259259	6.3374248253924\\
1.18759359359359	6.23755031890939\\
1.18959459459459	6.14205353374042\\
1.1915955955956	6.05055474075267\\
1.1935965965966	5.96272370801525\\
1.1955975975976	5.87827133704842\\
1.1975985985986	5.79694299569842\\
1.1995995995996	5.71851315273175\\
1.2016006006006	5.64278102160445\\
1.2036016016016	5.56956699430478\\
1.2056026026026	5.49870969945054\\
1.2076036036036	5.43006355789841\\
1.2096046046046	5.36349673807824\\
1.21160560560561	5.29888943493493\\
1.21360660660661	5.23613241273234\\
1.21560760760761	5.17512576445187\\
1.21760860860861	5.1157778501124\\
1.21960960960961	5.05800438377301\\
1.22161061061061	5.00172764478568\\
1.22361161161161	4.94687579343215\\
1.22561261261261	4.89338227469617\\
1.22761361361361	4.8411852968056\\
1.22961461461461	4.79022737349186\\
1.23161561561562	4.74045492078008\\
1.23361661661662	4.69181790063792\\
1.23561761761762	4.64426950519489\\
1.23761861861862	4.59776587519577\\
1.23961961961962	4.55226585042019\\
1.24162062062062	4.50773074501229\\
1.24362162162162	4.46412414680942\\
1.24562262262262	4.42141173683734\\
1.24762362362362	4.37956112669426\\
1.24962462462462	4.33854171171849\\
1.25162562562563	4.29832453811364\\
1.25362662662663	4.25888218246368\\
1.25562762762763	4.22018864224355\\
1.25762862862863	4.18221923615175\\
1.25962962962963	4.14495051319858\\
1.26163063063063	4.10836016964034\\
1.26363163163163	4.07242697294575\\
1.26563263263263	4.03713069209655\\
1.26763363363363	4.00245203357229\\
1.26963463463463	3.96837258248524\\
1.27163563563564	3.93487474835669\\
1.27363663663664	3.90194171510167\\
1.27563763763764	3.86955739483405\\
1.27763863863864	3.83770638513413\\
1.27963963963964	3.80637392947916\\
1.28164064064064	3.7755458805492\\
1.28364164164164	3.74520866615653\\
1.28564264264264	3.71534925757905\\
1.28764364364364	3.68595514009218\\
1.28964464464464	3.65701428551346\\
1.29164564564565	3.62851512659909\\
1.29364664664665	3.60044653313983\\
1.29564764764765	3.57279778961975\\
1.29764864864865	3.54555857431963\\
1.29964964964965	3.51871893974665\\
1.30165065065065	3.49226929429497\\
1.30365165165165	3.46620038503957\\
1.30565265265265	3.44050328158163\\
1.30765365365365	3.41516936086702\\
1.30965465465465	3.39019029290917\\
1.31165565565566	3.36555802734796\\
1.31365665665666	3.34126478078913\\
1.31565765765766	3.31730302486857\\
1.31765865865866	3.2936654749896\\
1.31965965965966	3.27034507969038\\
1.32166066066066	3.24733501059762\\
1.32366166166166	3.22462865292686\\
1.32566266266266	3.20221959649483\\
1.32766366366366	3.18010162721033\\
1.32966466466466	3.1582687190115\\
1.33166566566567	3.13671502622294\\
1.33366666666667	3.11543487630602\\
1.33566766766767	3.09442276297684\\
1.33766866866867	3.0736733396704\\
1.33966966966967	3.05318141333008\\
1.34167067067067	3.03294193850135\\
1.34367167167167	3.01295001171363\\
1.34567267267267	2.9932008661325\\
1.34767367367367	2.97368986646554\\
1.34967467467467	2.95441250410983\\
1.35167567567568	2.93536439252419\\
1.35367667667668	2.91654126281546\\
1.35567767767768	2.89793895952688\\
1.35767867867868	2.8795534366166\\
1.35967967967968	2.86138075361627\\
1.36168068068068	2.84341707195979\\
1.36368168168168	2.82565865147377\\
1.36568268268268	2.80810184702014\\
1.36768368368368	2.79074310528388\\
1.36968468468468	2.77357896169684\\
1.37168568568569	2.75660603749293\\
1.37368668668669	2.73982103688562\\
1.37568768768769	2.72322074436288\\
1.37768868868869	2.70680202209343\\
1.37968968968969	2.69056180743851\\
1.38169069069069	2.67449711056424\\
1.38369169169169	2.65860501214953\\
1.38569269269269	2.64288266118493\\
1.38769369369369	2.62732727285797\\
1.38969469469469	2.61193612652114\\
1.3916956956957	2.59670656373828\\
1.3936966966967	2.58163598640583\\
1.3956976976977	2.56672185494566\\
1.3976986986987	2.5519616865659\\
1.3996996996997	2.5373530535868\\
1.4017007007007	2.52289358182874\\
1.4037017017017	2.50858094905952\\
1.4057027027027	2.49441288349813\\
1.4077037037037	2.48038716237302\\
1.4097047047047	2.4665016105318\\
1.41170570570571	2.45275409910074\\
1.41370670670671	2.43914254419145\\
1.41570770770771	2.42566490565311\\
1.41770870870871	2.41231918586811\\
1.41970970970971	2.39910342858913\\
1.42171071071071	2.38601571781642\\
1.42371171171171	2.37305417671293\\
1.42571271271271	2.36021696655646\\
1.42771371371371	2.3475022857266\\
1.42971471471471	2.33490836872578\\
1.43171571571572	2.32243348523231\\
1.43371671671672	2.31007593918474\\
1.43571771771772	2.29783406789605\\
1.43771871871872	2.28570624119635\\
1.43971971971972	2.27369086060333\\
1.44172072072072	2.261786358519\\
1.44372172172172	2.24999119745203\\
1.44572272272272	2.23830386926447\\
1.44772372372372	2.22672289444209\\
1.44972472472472	2.21524682138738\\
1.45172572572573	2.20387422573425\\
1.45372672672673	2.19260370968389\\
1.45572772772773	2.18143390136072\\
1.45772872872873	2.17036345418795\\
1.45972972972973	2.15939104628171\\
1.46173073073073	2.1485153798635\\
1.46373173173173	2.13773518068989\\
1.46573273273273	2.12704919749913\\
1.46773373373373	2.11645620147396\\
1.46973473473473	2.10595498571997\\
1.47173573573574	2.09554436475924\\
1.47373673673674	2.08522317403832\\
1.47573773773774	2.07499026945045\\
1.47773873873874	2.06484452687127\\
1.47973973973974	2.05478484170761\\
1.48174074074074	2.04481012845906\\
1.48374174174174	2.03491932029164\\
1.48574274274274	2.0251113686234\\
1.48774374374374	2.01538524272139\\
1.48974474474474	2.00573992930969\\
1.49174574574575	1.99617443218816\\
1.49374674674675	1.98668777186148\\
1.49574774774775	1.97727898517817\\
1.49774874874875	1.96794712497939\\
1.49974974974975	1.95869125975696\\
1.50175075075075	1.94951047332055\\
1.50375175175175	1.94040386447359\\
1.50575275275275	1.93137054669765\\
1.50775375375375	1.92240964784517\\
1.50975475475475	1.91352030983993\\
1.51175575575576	1.90470168838553\\
1.51375675675676	1.8959529526811\\
1.51575775775776	1.88727328514436\\
1.51775875875876	1.8786618811418\\
1.51975975975976	1.87011794872547\\
1.52176076076076	1.86164070837657\\
1.52376176176176	1.8532293927553\\
1.52576276276276	1.84488324645707\\
1.52776376376376	1.83660152577459\\
1.52976476476476	1.82838349846595\\
1.53176576576577	1.82022844352827\\
1.53376676676677	1.8121356509769\\
1.53576776776777	1.80410442163003\\
1.53776876876877	1.79613406689838\\
1.53976976976977	1.78822390858004\\
1.54177077077077	1.78037327866013\\
1.54377177177177	1.77258151911531\\
1.54577277277277	1.76484798172281\\
1.54777377377377	1.75717202787402\\
1.54977477477477	1.74955302839246\\
1.55177577577578	1.74199036335592\\
1.55377677677678	1.7344834219228\\
1.55577777777778	1.72703160216241\\
1.55777877877878	1.71963431088924\\
1.55977977977978	1.71229096350089\\
1.56178078078078	1.70500098381988\\
1.56378178178178	1.69776380393883\\
1.56578278278278	1.69057886406934\\
1.56778378378378	1.68344561239413\\
1.56978478478478	1.67636350492249\\
1.57178578578579	1.66933200534906\\
1.57378678678679	1.66235058491561\\
1.57578778778779	1.65541872227592\\
1.57778878878879	1.64853590336369\\
1.57978978978979	1.64170162126322\\
1.58179079079079	1.63491537608299\\
1.58379179179179	1.62817667483196\\
1.58579279279279	1.62148503129857\\
1.58779379379379	1.61483996593226\\
1.58979479479479	1.60824100572759\\
1.5917957957958	1.60168768411087\\
1.5937967967968	1.59517954082908\\
1.5957977977978	1.58871612184122\\
1.5977987987988	1.58229697921196\\
1.5997997997998	1.57592167100746\\
1.6018008008008	1.5695897611934\\
1.6038018018018	1.56330081953511\\
1.6058028028028	1.55705442149984\\
1.6078038038038	1.5508501481609\\
1.6098048048048	1.54468758610394\\
1.61180580580581	1.538566327335\\
1.61380680680681	1.53248596919053\\
1.61580780780781	1.5264461142492\\
1.61780880880881	1.52044637024551\\
1.61980980980981	1.51448634998508\\
1.62181081081081	1.50856567126173\\
1.62381181181181	1.50268395677616\\
1.62581281281281	1.49684083405623\\
1.62781381381381	1.49103593537891\\
1.62981481481481	1.48526889769366\\
1.63181581581582	1.47953936254736\\
1.63381681681682	1.47384697601079\\
1.63581781781782	1.46819138860639\\
1.63781881881882	1.46257225523757\\
1.63981981981982	1.45698923511932\\
1.64182082082082	1.45144199171013\\
1.64382182182182	1.44593019264532\\
1.64582282282282	1.44045350967154\\
1.64782382382382	1.43501161858259\\
1.64982482482482	1.42960419915644\\
1.65182582582583	1.42423093509343\\
1.65382682682683	1.41889151395565\\
1.65582782782783	1.41358562710747\\
1.65782882882883	1.40831296965719\\
1.65982982982983	1.40307324039972\\
1.66183083083083	1.39786614176041\\
1.66383183183183	1.39269137973989\\
1.66583283283283	1.38754866385988\\
1.66783383383383	1.38243770711011\\
1.66983483483483	1.37735822589608\\
1.67183583583584	1.37230993998789\\
1.67383683683684	1.36729257246993\\
1.67583783783784	1.36230584969153\\
1.67783883883884	1.35734950121845\\
1.67983983983984	1.35242325978532\\
1.68184084084084	1.34752686124888\\
1.68384184184184	1.34266004454205\\
1.68584284284284	1.33782255162887\\
1.68784384384384	1.33301412746019\\
1.68984484484484	1.32823451993021\\
1.69184584584585	1.3234834798337\\
1.69384684684685	1.31876076082405\\
1.69584784784785	1.31406611937205\\
1.69784884884885	1.30939931472531\\
1.69984984984985	1.30476010886852\\
1.70185085085085	1.30014826648428\\
1.70385185185185	1.2955635549147\\
1.70585285285285	1.29100574412359\\
1.70785385385385	1.28647460665934\\
1.70985485485485	1.28196991761848\\
1.71185585585586	1.27749145460974\\
1.71385685685686	1.27303899771885\\
1.71585785785786	1.2686123294739\\
1.71785885885886	1.26421123481122\\
1.71985985985986	1.25983550104195\\
1.72186086086086	1.25548491781909\\
1.72386186186186	1.25115927710513\\
1.72586286286286	1.24685837314021\\
1.72786386386386	1.24258200241086\\
1.72986486486486	1.23832996361919\\
1.73186586586587	1.23410205765262\\
1.73386686686687	1.22989808755416\\
1.73586786786787	1.22571785849308\\
1.73786886886887	1.22156117773615\\
1.73986986986987	1.21742785461931\\
1.74187087087087	1.21331770051982\\
1.74387187187187	1.20923052882883\\
1.74587287287287	1.20516615492447\\
1.74787387387387	1.20112439614527\\
1.74987487487487	1.19710507176411\\
1.75187587587588	1.1931080029625\\
1.75387687687688	1.18913301280539\\
1.75587787787788	1.18517992621624\\
1.75787887887888	1.18124856995259\\
1.75987987987988	1.17733877258199\\
1.76188088088088	1.17345036445832\\
1.76388188188188	1.16958317769844\\
1.76588288288288	1.16573704615927\\
1.76788388388388	1.16191180541522\\
1.76988488488488	1.15810729273591\\
1.77188588588589	1.15432334706433\\
1.77388688688689	1.15055980899526\\
1.77588788788789	1.14681652075412\\
1.77788888888889	1.14309332617601\\
1.77988988988989	1.13939007068523\\
1.78189089089089	1.135706601275\\
1.78389189189189	1.13204276648753\\
1.78589289289289	1.12839841639443\\
1.78789389389389	1.12477340257735\\
1.78989489489489	1.12116757810898\\
1.7918958958959	1.11758079753431\\
1.7938968968969	1.11401291685215\\
1.7958978978979	1.11046379349699\\
1.7978988988989	1.10693328632108\\
1.7998998998999	1.10342125557682\\
1.8019009009009	1.09992756289934\\
1.8039019019019	1.09645207128948\\
1.8059029029029	1.09299464509686\\
1.8079039039039	1.08955515000332\\
1.8099049049049	1.08613345300656\\
1.81190590590591	1.08272942240401\\
1.81390690690691	1.07934292777699\\
1.81590790790791	1.07597383997505\\
1.81790890890891	1.07262203110055\\
1.81990990990991	1.0692873744935\\
1.82191091091091	1.06596974471657\\
1.82391191191191	1.06266901754038\\
1.82591291291291	1.05938506992893\\
1.82791391391391	1.05611778002531\\
1.82991491491491	1.05286702713758\\
1.83191591591592	1.04963269172487\\
1.83391691691692	1.04641465538363\\
1.83591791791792	1.04321280083417\\
1.83791891891892	1.04002701190732\\
1.83991991991992	1.03685717353131\\
1.84192092092092	1.03370317171879\\
1.84392192192192	1.03056489355413\\
1.84592292292292	1.02744222718081\\
1.84792392392392	1.02433506178903\\
1.84992492492492	1.02124328760346\\
1.85192592592593	1.01816679587124\\
1.85392692692693	1.01510547885006\\
1.85592792792793	1.01205922979643\\
1.85792892892893	1.00902794295417\\
1.85992992992993	1.00601151354299\\
1.86193093093093	1.00300983774725\\
1.86393193193193	1.0000228127049\\
1.86593293293293	0.997050336496537\\
1.86793393393393	0.994092308134657\\
1.86993493493493	0.991148627552996\\
1.87193593593594	0.988219195596068\\
1.87393693693694	0.985303914008826\\
1.87593793793794	0.98240268542646\\
1.87793893893894	0.979515413364348\\
1.87993993993994	0.976642002208128\\
1.88194094094094	0.973782357203919\\
1.88394194194194	0.970936384448661\\
1.88594294294294	0.968103990880596\\
1.88794394394394	0.965285084269871\\
1.88994494494494	0.962479573209271\\
1.89194594594595	0.959687367105072\\
1.89394694694695	0.95690837616802\\
1.89594794794795	0.954142511404429\\
1.89794894894895	0.951389684607397\\
1.89994994994995	0.948649808348139\\
1.90195095095095	0.945922795967432\\
1.90395195195195	0.943208561567184\\
1.90595295295295	0.940507020002097\\
1.90795395395395	0.937818086871453\\
1.90995495495495	0.935141678511002\\
1.91195595595596	0.932477711984959\\
1.91395695695696	0.929826105078103\\
1.91595795795796	0.927186776287981\\
1.91795895895896	0.924559644817213\\
1.91995995995996	0.921944630565892\\
1.92196096096096	0.919341654124094\\
1.92396196196196	0.916750636764471\\
1.92596296296296	0.914171500434952\\
1.92796396396396	0.911604167751524\\
1.92996496496496	0.90904856199112\\
1.93196596596597	0.90650460708459\\
1.93396696696697	0.903972227609762\\
1.93596796796797	0.901451348784593\\
1.93796896896897	0.898941896460406\\
1.93996996996997	0.896443797115217\\
1.94197097097097	0.893956977847137\\
1.94397197197197	0.891481366367869\\
1.94597297297297	0.889016890996281\\
1.94797397397397	0.886563480652056\\
1.94997497497497	0.884121064849428\\
1.95197597597598	0.881689573690997\\
1.95397697697698	0.879268937861615\\
1.95597797797798	0.876859088622353\\
1.95797897897898	0.874459957804542\\
1.95997997997998	0.872071477803891\\
1.96198098098098	0.869693581574668\\
1.96398198198198	0.867326202623965\\
1.96598298298298	0.864969275006026\\
1.96798398398398	0.862622733316652\\
1.96998498498498	0.860286512687662\\
1.97198598598599	0.857960548781435\\
1.97398698698699	0.855644777785516\\
1.97598798798799	0.853339136407278\\
1.97798898898899	0.851043561868663\\
1.97998998998999	0.84875799190098\\
1.98199099099099	0.846482364739761\\
1.98399199199199	0.844216619119692\\
1.98599299299299	0.841960694269592\\
1.98799399399399	0.839714529907464\\
1.98999499499499	0.837478066235595\\
1.991995995996	0.835251243935723\\
1.993996996997	0.833034004164258\\
1.995997997998	0.830826288547562\\
1.997998998999	0.828628039177282\\
2	0.826439198605741\\
1.997998998999	0.828628039177282\\
1.995997997998	0.830826288547562\\
1.993996996997	0.833034004164258\\
1.991995995996	0.835251243935723\\
1.98999499499499	0.837478066235595\\
1.98799399399399	0.839714529907464\\
1.98599299299299	0.841960694269592\\
1.98399199199199	0.844216619119692\\
1.98199099099099	0.846482364739761\\
1.97998998998999	0.84875799190098\\
1.97798898898899	0.851043561868664\\
1.97598798798799	0.853339136407278\\
1.97398698698699	0.855644777785516\\
1.97198598598599	0.857960548781435\\
1.96998498498498	0.860286512687662\\
1.96798398398398	0.862622733316652\\
1.96598298298298	0.864969275006027\\
1.96398198198198	0.867326202623965\\
1.96198098098098	0.869693581574668\\
1.95997997997998	0.872071477803891\\
1.95797897897898	0.874459957804543\\
1.95597797797798	0.876859088622353\\
1.95397697697698	0.879268937861615\\
1.95197597597598	0.881689573690997\\
1.94997497497497	0.884121064849428\\
1.94797397397397	0.886563480652056\\
1.94597297297297	0.889016890996282\\
1.94397197197197	0.89148136636787\\
1.94197097097097	0.893956977847136\\
1.93996996996997	0.896443797115216\\
1.93796896896897	0.898941896460406\\
1.93596796796797	0.901451348784593\\
1.93396696696697	0.903972227609762\\
1.93196596596597	0.90650460708459\\
1.92996496496496	0.90904856199112\\
1.92796396396396	0.911604167751524\\
1.92596296296296	0.914171500434952\\
1.92396196196196	0.916750636764472\\
1.92196096096096	0.919341654124094\\
1.91995995995996	0.921944630565892\\
1.91795895895896	0.924559644817213\\
1.91595795795796	0.927186776287982\\
1.91395695695696	0.929826105078103\\
1.91195595595596	0.932477711984959\\
1.90995495495495	0.935141678511002\\
1.90795395395395	0.937818086871453\\
1.90595295295295	0.940507020002097\\
1.90395195195195	0.943208561567184\\
1.90195095095095	0.945922795967432\\
1.89994994994995	0.948649808348138\\
1.89794894894895	0.951389684607397\\
1.89594794794795	0.954142511404429\\
1.89394694694695	0.95690837616802\\
1.89194594594595	0.959687367105072\\
1.88994494494494	0.962479573209272\\
1.88794394394394	0.965285084269871\\
1.88594294294294	0.968103990880596\\
1.88394194194194	0.970936384448661\\
1.88194094094094	0.973782357203919\\
1.87993993993994	0.976642002208128\\
1.87793893893894	0.979515413364348\\
1.87593793793794	0.98240268542646\\
1.87393693693694	0.985303914008826\\
1.87193593593594	0.988219195596068\\
1.86993493493493	0.991148627552996\\
1.86793393393393	0.994092308134657\\
1.86593293293293	0.997050336496537\\
1.86393193193193	1.0000228127049\\
1.86193093093093	1.00300983774725\\
1.85992992992993	1.00601151354299\\
1.85792892892893	1.00902794295417\\
1.85592792792793	1.01205922979643\\
1.85392692692693	1.01510547885006\\
1.85192592592593	1.01816679587124\\
1.84992492492492	1.02124328760346\\
1.84792392392392	1.02433506178903\\
1.84592292292292	1.02744222718081\\
1.84392192192192	1.03056489355413\\
1.84192092092092	1.03370317171879\\
1.83991991991992	1.03685717353131\\
1.83791891891892	1.04002701190732\\
1.83591791791792	1.04321280083417\\
1.83391691691692	1.04641465538363\\
1.83191591591592	1.04963269172487\\
1.82991491491491	1.05286702713758\\
1.82791391391391	1.05611778002531\\
1.82591291291291	1.05938506992893\\
1.82391191191191	1.06266901754038\\
1.82191091091091	1.06596974471657\\
1.81990990990991	1.0692873744935\\
1.81790890890891	1.07262203110055\\
1.81590790790791	1.07597383997505\\
1.81390690690691	1.07934292777699\\
1.81190590590591	1.08272942240401\\
1.8099049049049	1.08613345300656\\
1.8079039039039	1.08955515000332\\
1.8059029029029	1.09299464509686\\
1.8039019019019	1.09645207128948\\
1.8019009009009	1.09992756289934\\
1.7998998998999	1.10342125557682\\
1.7978988988989	1.10693328632108\\
1.7958978978979	1.11046379349699\\
1.7938968968969	1.11401291685215\\
1.7918958958959	1.11758079753431\\
1.78989489489489	1.12116757810898\\
1.78789389389389	1.12477340257735\\
1.78589289289289	1.12839841639443\\
1.78389189189189	1.13204276648753\\
1.78189089089089	1.135706601275\\
1.77988988988989	1.13939007068523\\
1.77788888888889	1.14309332617601\\
1.77588788788789	1.14681652075412\\
1.77388688688689	1.15055980899526\\
1.77188588588589	1.15432334706433\\
1.76988488488488	1.15810729273591\\
1.76788388388388	1.16191180541522\\
1.76588288288288	1.16573704615927\\
1.76388188188188	1.16958317769844\\
1.76188088088088	1.17345036445832\\
1.75987987987988	1.17733877258199\\
1.75787887887888	1.18124856995259\\
1.75587787787788	1.18517992621624\\
1.75387687687688	1.18913301280539\\
1.75187587587588	1.1931080029625\\
1.74987487487487	1.19710507176411\\
1.74787387387387	1.20112439614527\\
1.74587287287287	1.20516615492447\\
1.74387187187187	1.20923052882883\\
1.74187087087087	1.21331770051982\\
1.73986986986987	1.21742785461931\\
1.73786886886887	1.22156117773615\\
1.73586786786787	1.22571785849308\\
1.73386686686687	1.22989808755416\\
1.73186586586587	1.23410205765262\\
1.72986486486486	1.23832996361919\\
1.72786386386386	1.24258200241086\\
1.72586286286286	1.24685837314021\\
1.72386186186186	1.25115927710513\\
1.72186086086086	1.25548491781909\\
1.71985985985986	1.25983550104195\\
1.71785885885886	1.26421123481122\\
1.71585785785786	1.2686123294739\\
1.71385685685686	1.27303899771885\\
1.71185585585586	1.27749145460974\\
1.70985485485485	1.28196991761848\\
1.70785385385385	1.28647460665934\\
1.70585285285285	1.29100574412359\\
1.70385185185185	1.2955635549147\\
1.70185085085085	1.30014826648428\\
1.69984984984985	1.30476010886852\\
1.69784884884885	1.30939931472531\\
1.69584784784785	1.31406611937205\\
1.69384684684685	1.31876076082405\\
1.69184584584585	1.3234834798337\\
1.68984484484484	1.32823451993021\\
1.68784384384384	1.33301412746019\\
1.68584284284284	1.33782255162887\\
1.68384184184184	1.34266004454205\\
1.68184084084084	1.34752686124888\\
1.67983983983984	1.35242325978532\\
1.67783883883884	1.35734950121845\\
1.67583783783784	1.36230584969153\\
1.67383683683684	1.36729257246993\\
1.67183583583584	1.37230993998789\\
1.66983483483483	1.37735822589608\\
1.66783383383383	1.38243770711011\\
1.66583283283283	1.38754866385988\\
1.66383183183183	1.39269137973988\\
1.66183083083083	1.39786614176041\\
1.65982982982983	1.40307324039972\\
1.65782882882883	1.40831296965719\\
1.65582782782783	1.41358562710747\\
1.65382682682683	1.41889151395565\\
1.65182582582583	1.42423093509343\\
1.64982482482482	1.42960419915644\\
1.64782382382382	1.43501161858259\\
1.64582282282282	1.44045350967154\\
1.64382182182182	1.44593019264532\\
1.64182082082082	1.45144199171013\\
1.63981981981982	1.45698923511932\\
1.63781881881882	1.46257225523757\\
1.63581781781782	1.46819138860639\\
1.63381681681682	1.47384697601079\\
1.63181581581582	1.47953936254736\\
1.62981481481481	1.48526889769366\\
1.62781381381381	1.49103593537892\\
1.62581281281281	1.49684083405623\\
1.62381181181181	1.50268395677616\\
1.62181081081081	1.50856567126173\\
1.61980980980981	1.51448634998508\\
1.61780880880881	1.52044637024551\\
1.61580780780781	1.5264461142492\\
1.61380680680681	1.53248596919053\\
1.61180580580581	1.53856632733499\\
1.6098048048048	1.54468758610394\\
1.6078038038038	1.5508501481609\\
1.6058028028028	1.55705442149984\\
1.6038018018018	1.56330081953511\\
1.6018008008008	1.56958976119339\\
1.5997997997998	1.57592167100746\\
1.5977987987988	1.58229697921196\\
1.5957977977978	1.58871612184122\\
1.5937967967968	1.59517954082908\\
1.5917957957958	1.60168768411087\\
1.58979479479479	1.60824100572759\\
1.58779379379379	1.61483996593225\\
1.58579279279279	1.62148503129857\\
1.58379179179179	1.62817667483196\\
1.58179079079079	1.63491537608298\\
1.57978978978979	1.64170162126322\\
1.57778878878879	1.64853590336369\\
1.57578778778779	1.65541872227592\\
1.57378678678679	1.6623505849156\\
1.57178578578579	1.66933200534906\\
1.56978478478478	1.67636350492249\\
1.56778378378378	1.68344561239412\\
1.56578278278278	1.69057886406934\\
1.56378178178178	1.69776380393883\\
1.56178078078078	1.70500098381987\\
1.55977977977978	1.71229096350089\\
1.55777877877878	1.71963431088923\\
1.55577777777778	1.72703160216241\\
1.55377677677678	1.7344834219228\\
1.55177577577578	1.74199036335592\\
1.54977477477477	1.74955302839246\\
1.54777377377377	1.75717202787402\\
1.54577277277277	1.7648479817228\\
1.54377177177177	1.7725815191153\\
1.54177077077077	1.78037327866012\\
1.53976976976977	1.78822390858003\\
1.53776876876877	1.79613406689838\\
1.53576776776777	1.80410442163003\\
1.53376676676677	1.8121356509769\\
1.53176576576577	1.82022844352826\\
1.52976476476476	1.82838349846595\\
1.52776376376376	1.83660152577458\\
1.52576276276276	1.84488324645706\\
1.52376176176176	1.85322939275529\\
1.52176076076076	1.86164070837656\\
1.51975975975976	1.87011794872547\\
1.51775875875876	1.8786618811418\\
1.51575775775776	1.88727328514436\\
1.51375675675676	1.89595295268109\\
1.51175575575576	1.90470168838552\\
1.50975475475475	1.91352030983992\\
1.50775375375375	1.92240964784515\\
1.50575275275275	1.93137054669764\\
1.50375175175175	1.94040386447357\\
1.50175075075075	1.94951047332053\\
1.49974974974975	1.95869125975695\\
1.49774874874875	1.96794712497937\\
1.49574774774775	1.97727898517815\\
1.49374674674675	1.98668777186146\\
1.49174574574575	1.99617443218815\\
1.48974474474474	2.00573992930968\\
1.48774374374374	2.01538524272137\\
1.48574274274274	2.02511136862338\\
1.48374174174174	2.03491932029162\\
1.48174074074074	2.04481012845904\\
1.47973973973974	2.05478484170759\\
1.47773873873874	2.06484452687125\\
1.47573773773774	2.07499026945043\\
1.47373673673674	2.08522317403829\\
1.47173573573574	2.09554436475921\\
1.46973473473473	2.10595498571994\\
1.46773373373373	2.11645620147392\\
1.46573273273273	2.1270491974991\\
1.46373173173173	2.13773518068986\\
1.46173073073073	2.14851537986346\\
1.45972972972973	2.15939104628167\\
1.45772872872873	2.17036345418791\\
1.45572772772773	2.18143390136068\\
1.45372672672673	2.19260370968384\\
1.45172572572573	2.20387422573421\\
1.44972472472472	2.21524682138733\\
1.44772372372372	2.22672289444205\\
1.44572272272272	2.23830386926441\\
1.44372172172172	2.24999119745196\\
1.44172072072072	2.26178635851894\\
1.43971971971972	2.27369086060327\\
1.43771871871872	2.28570624119629\\
1.43571771771772	2.29783406789598\\
1.43371671671672	2.31007593918467\\
1.43171571571572	2.32243348523223\\
1.42971471471471	2.3349083687257\\
1.42771371371371	2.34750228572652\\
1.42571271271271	2.36021696655637\\
1.42371171171171	2.37305417671284\\
1.42171071071071	2.38601571781631\\
1.41970970970971	2.39910342858903\\
1.41770870870871	2.41231918586798\\
1.41570770770771	2.425664905653\\
1.41370670670671	2.43914254419131\\
1.41170570570571	2.45275409910059\\
1.4097047047047	2.46650161053166\\
1.4077037037037	2.48038716237286\\
1.4057027027027	2.49441288349796\\
1.4037017017017	2.50858094905933\\
1.4017007007007	2.52289358182856\\
1.3996996996997	2.53735305358659\\
1.3976986986987	2.55196168656565\\
1.3956976976977	2.56672185494542\\
1.3936966966967	2.58163598640556\\
1.3916956956957	2.596706563738\\
1.38969469469469	2.61193612652086\\
1.38769369369369	2.62732727285765\\
1.38569269269269	2.64288266118458\\
1.38369169169169	2.65860501214918\\
1.38169069069069	2.67449711056387\\
1.37968968968969	2.69056180743811\\
1.37768868868869	2.70680202209301\\
1.37568768768769	2.72322074436241\\
1.37368668668669	2.73982103688511\\
1.37168568568569	2.75660603749239\\
1.36968468468468	2.77357896169623\\
1.36768368368368	2.79074310528325\\
1.36568268268268	2.80810184701957\\
1.36368168168168	2.82565865147309\\
1.36168068068068	2.84341707195907\\
1.35967967967968	2.86138075361552\\
1.35767867867868	2.8795534366158\\
1.35567767767768	2.897938959526\\
1.35367667667668	2.91654126281448\\
1.35167567567568	2.93536439252318\\
1.34967467467467	2.95441250410879\\
1.34767367367367	2.97368986646435\\
1.34567267267267	2.99320086613115\\
1.34367167167167	3.01295001171228\\
1.34167067067067	3.03294193849989\\
1.33966966966967	3.05318141332856\\
1.33766866866867	3.07367333966874\\
1.33566766766767	3.09442276297494\\
1.33366666666667	3.11543487630405\\
1.33166566566567	3.13671502622082\\
1.32966466466466	3.15826871900926\\
1.32766366366366	3.18010162720798\\
1.32566266266266	3.20221959649225\\
1.32366166166166	3.22462865292393\\
1.32166066066066	3.24733501059447\\
1.31965965965966	3.27034507968709\\
1.31765865865866	3.29366547498602\\
1.31565765765766	3.3173030248648\\
1.31365665665666	3.34126478078507\\
1.31165565565566	3.36555802734337\\
1.30965465465465	3.39019029290425\\
1.30765365365365	3.41516936086166\\
1.30565265265265	3.44050328157582\\
1.30365165165165	3.46620038503339\\
1.30165065065065	3.49226929428828\\
1.29964964964965	3.51871893973928\\
1.29764864864865	3.54555857431154\\
1.29564764764765	3.57279778961113\\
1.29364664664665	3.60044653313033\\
1.29164564564565	3.62851512658887\\
1.28964464464464	3.65701428550212\\
1.28764364364364	3.68595514007984\\
1.28564264264264	3.71534925756575\\
1.28364164164164	3.74520866614196\\
1.28164064064064	3.77554588053341\\
1.27963963963964	3.8063739294618\\
1.27763863863864	3.83770638511476\\
1.27563763763764	3.86955739481314\\
1.27363663663664	3.90194171507925\\
1.27163563563564	3.93487474833166\\
1.26963463463463	3.96837258245773\\
1.26763363363363	4.00245203354201\\
1.26563263263263	4.0371306920633\\
1.26363163163163	4.07242697290971\\
1.26163063063063	4.1083601696004\\
1.25962962962963	4.14495051315523\\
1.25762862862863	4.18221923610375\\
1.25562762762763	4.2201886421902\\
1.25362662662663	4.25888218240455\\
1.25162562562563	4.29832453804851\\
1.24962462462462	4.3385417116457\\
1.24762362362362	4.37956112661421\\
1.24562262262262	4.42141173674849\\
1.24362162162162	4.46412414671097\\
1.24162062062062	4.50773074490293\\
1.23961961961962	4.55226585029879\\
1.23761861861862	4.59776587506006\\
1.23561761761762	4.64426950504303\\
1.23361661661662	4.691817900635\\
1.23161561561562	4.74045492077664\\
1.22961461461461	4.79022737348809\\
1.22761361361361	4.84118529680129\\
1.22561261261261	4.89338227469139\\
1.22361161161161	4.94687579342665\\
1.22161061061061	5.00172764477914\\
1.21960960960961	5.05800438376529\\
1.21760860860861	5.11577785010376\\
1.21560760760761	5.17512576444211\\
1.21360660660661	5.23613241272076\\
1.21160560560561	5.29888943492147\\
1.2096046046046	5.36349673806305\\
1.2076036036036	5.43006355788043\\
1.2056026026026	5.49870969942892\\
1.2036016016016	5.56956699427966\\
1.2016006006006	5.64278102157492\\
1.1995995995996	5.71851315269701\\
1.1975985985986	5.79694299565693\\
1.1955975975976	5.87827133699868\\
1.1935965965966	5.96272370795543\\
1.1915955955956	6.05055474068017\\
1.18959459459459	6.14205353365235\\
1.18759359359359	6.23755031880098\\
1.18559259259259	6.33742482525877\\
1.18359159159159	6.44211687892237\\
1.18159059059059	6.55213998265171\\
1.17958958958959	6.66809891806412\\
1.17758858858859	6.79071284026372\\
1.17558758758759	6.92084596231455\\
1.17358658658659	7.05954884592043\\
1.17158558558559	7.20811460761059\\
1.16958458458458	7.36815611345958\\
1.16758358358358	7.54171220049271\\
1.16558258258258	7.731391718598\\
1.16358158158158	7.94055790259263\\
1.16158058058058	8.17352102326626\\
1.15957957957958	8.43557085497145\\
1.15757857857858	8.73221158711389\\
1.15557757757758	9.06567786943307\\
1.15357657657658	9.42554882065707\\
1.15157557557558	9.77882345874648\\
1.14957457457457	10.085095839374\\
1.14757357357357	10.3277825033724\\
1.14557257257257	10.5136864889125\\
1.14357157157157	10.6559098575457\\
1.14157057057057	10.7656851732283\\
1.13956956956957	10.8511707835453\\
1.13756856856857	10.9180932738611\\
1.13556756756757	10.9704996531655\\
1.13356656656657	11.0113067316053\\
1.13156556556557	11.0426650038902\\
1.12956456456456	11.0661955492475\\
1.12756356356356	11.0831456418277\\
1.12556256256256	11.0944928278493\\
1.12356156156156	11.1010160081054\\
1.12156056056056	11.1033450074279\\
1.11955955955956	11.1019958472413\\
1.11755855855856	11.0973963158655\\
1.11555755755756	11.0899048277162\\
1.11355655655656	11.0798245542482\\
1.11155555555556	11.0674141663557\\
1.10955455455455	11.0528961096787\\
1.10755355355355	11.0364630572782\\
1.10555255255255	11.0182829975474\\
1.10355155155155	10.9985032873879\\
1.10155055055055	10.9772539117925\\
1.09954954954955	10.954650128255\\
1.09754854854855	10.9307946295692\\
1.09554754754755	10.9057793260963\\
1.09354654654655	10.8796868247794\\
1.09154554554555	10.8525916645506\\
1.08954454454454	10.8245613545727\\
1.08754354354354	10.7956572517882\\
1.08554254254254	10.7659353066321\\
1.08354154154154	10.7354466999177\\
1.08154054054054	10.7042383893615\\
1.07953953953954	10.6723535806713\\
1.07753853853854	10.6398321353227\\
1.07553753753754	10.6067109249501\\
1.07353653653654	10.5730241404999\\
1.07153553553554	10.5388035628921\\
1.06953453453453	10.5040788007847\\
1.06753353353353	10.4688775001165\\
1.06553253253253	10.4332255293414\\
1.06353153153153	10.3971471436547\\
1.06153053053053	10.3606651309958\\
1.05952952952953	10.3238009421937\\
1.05752852852853	10.2865748072688\\
1.05552752752753	10.249005839613\\
1.05352652652653	10.211112129526\\
1.05152552552553	10.1729108283744\\
1.04952452452452	10.1344182244759\\
1.04752352352352	10.0956498116527\\
1.04552252252252	10.0566203512813\\
1.04352152152152	10.0173439285552\\
1.04152052052052	9.97783400358723\\
1.03951951951952	9.93810345789879\\
1.03751851851852	9.89816463677827\\
1.03551751751752	9.85802938792995\\
1.03351651651652	9.81770909678716\\
1.03151551551552	9.77721471881808\\
1.02951451451451	9.73655680911658\\
1.02751351351351	9.69574554953528\\
1.02551251251251	9.65479077359227\\
1.02351151151151	9.61370198935419\\
1.02151051051051	9.57248840048009\\
1.01950950950951	9.53115892558782\\
1.01750850850851	9.4897222160904\\
1.01550750750751	9.44818667263244\\
1.01350650650651	9.40656046024507\\
1.01150550550551	9.36485152232549\\
1.0095045045045	9.32306759353662\\
1.0075035035035	9.28121621171329\\
1.0055025025025	9.23930472885358\\
1.0035015015015	9.19734032126558\\
1.0015005005005	9.15532999893446\\
0.9994994994995	9.11328061416824\\
0.997498498498499	9.07119886957528\\
0.995497497497498	9.02909132542256\\
0.993496496496497	8.98696440641902\\
0.991495495495496	8.94482440796449\\
0.989494494494495	8.90267750190203\\
0.987493493493494	8.86052974180725\\
0.985492492492492	8.81838706784713\\
0.983491491491492	8.77625531123658\\
0.981490490490491	8.73414019832016\\
0.97948948948949	8.69204735430388\\
0.977488488488489	8.64998230665985\\
0.975487487487487	8.60795048822573\\
0.973486486486487	8.56595724001863\\
0.971485485485486	8.52400781378226\\
0.969484484484485	8.48210737428489\\
0.967483483483484	8.44026100138449\\
0.965482482482482	8.39847369187618\\
0.963481481481481	8.35675036113688\\
0.961480480480481	8.31509584458044\\
0.95947947947948	8.27351489893643\\
0.957478478478479	8.23201220336466\\
0.955477477477477	8.19059236041698\\
0.953476476476476	8.14925989685765\\
0.951475475475476	8.10801926435234\\
0.949474474474475	8.06687484003613\\
0.947473473473474	8.02583092696999\\
0.945472472472472	7.98489175449455\\
0.943471471471471	7.94406147849041\\
0.941470470470471	7.90334418155298\\
0.93946946946947	7.86274387308997\\
0.937468468468469	7.82226448934938\\
0.935467467467467	7.78190989338513\\
0.933466466466467	7.74168387496779\\
0.931465465465466	7.70159015044679\\
0.929464464464465	7.66163236257109\\
0.927463463463464	7.6218140802744\\
0.925462462462462	7.58213879843092\\
0.923461461461462	7.54260993758764\\
0.92146046046046	7.50323084367842\\
0.91945945945946	7.4640047877256\\
0.917458458458459	7.4249349655337\\
0.915457457457457	7.38602449738046\\
0.913456456456457	7.34727642770981\\
0.911455455455455	7.30869372483075\\
0.909454454454455	7.27027928062689\\
0.907453453453454	7.23203591028007\\
0.905452452452452	7.19396635201213\\
0.903451451451452	7.15607326684802\\
0.90145045045045	7.11835923840371\\
0.899449449449449	7.08082677270185\\
0.897448448448449	7.043478298018\\
0.895447447447447	7.00631616475987\\
0.893446446446447	6.96934264538231\\
0.891445445445445	6.93255993433971\\
0.889444444444444	6.89597014807808\\
0.887443443443444	6.85957532506846\\
0.885442442442442	6.82337742588298\\
0.883441441441442	6.78737833331513\\
0.88144044044044	6.75157985254511\\
0.879439439439439	6.71598371135132\\
0.877438438438438	6.68059156036847\\
0.875437437437438	6.64540497339306\\
0.873436436436437	6.61042544773634\\
0.871435435435435	6.57565440462497\\
0.869434434434434	6.54109318964939\\
0.867433433433433	6.50674307325956\\
0.865432432432433	6.47260525130784\\
0.863431431431432	6.43868084563838\\
0.86143043043043	6.40497090472245\\
0.859429429429429	6.37147640433884\\
0.857428428428428	6.33819824829839\\
0.855427427427428	6.30513726921162\\
0.853426426426427	6.27229422929833\\
0.851425425425425	6.2396698212378\\
0.849424424424424	6.20726466905827\\
0.847423423423423	6.17507932906421\\
0.845422422422422	6.14311429079995\\
0.843421421421422	6.11136997804777\\
0.84142042042042	6.07984674985918\\
0.839419419419419	6.04854490161727\\
0.837418418418418	6.01746466612856\\
0.835417417417417	5.9866062147425\\
0.833416416416417	5.95596965849672\\
0.831415415415415	5.92555504928613\\
0.829414414414414	5.89536238105416\\
0.827413413413413	5.86539159100388\\
0.825412412412412	5.83564256082752\\
0.823411411411412	5.8061151179521\\
0.82141041041041	5.77680903679952\\
0.819409409409409	5.74772404005915\\
0.817408408408409	5.71885979997107\\
0.815407407407407	5.69021593961802\\
0.813406406406406	5.66179203422449\\
0.811405405405405	5.63358761246093\\
0.809404404404404	5.60560215775151\\
0.807403403403404	5.57783510958371\\
0.805402402402402	5.55028586481798\\
0.803401401401401	5.52295377899615\\
0.8014004004004	5.49583816764668\\
0.799399399399399	5.46893830758562\\
0.797398398398398	5.4422534382116\\
0.795397397397397	5.4157827627936\\
0.793396396396396	5.38952544975015\\
0.791395395395395	5.36348063391873\\
0.789394394394394	5.33764741781409\\
0.787393393393393	5.31202487287455\\
0.785392392392392	5.286612040695\\
0.783391391391391	5.2614079342457\\
0.78139039039039	5.23641153907599\\
0.779389389389389	5.21162181450193\\
0.777388388388388	5.18703769477702\\
0.775387387387387	5.16265809024548\\
0.773386386386386	5.13848188847703\\
0.771385385385385	5.11450795538286\\
0.769384384384384	5.09073513631201\\
0.767383383383383	5.06716225712773\\
0.765382382382382	5.04378812526322\\
0.763381381381381	5.02061153075657\\
0.76138038038038	4.99763124726427\\
0.759379379379379	4.97484603305315\\
0.757378378378378	4.95225463197037\\
0.755377377377377	4.92985577439138\\
0.753376376376376	4.9076481781455\\
0.751375375375375	4.88563054941913\\
0.749374374374374	4.86380158363638\\
0.747373373373373	4.84215996631723\\
0.745372372372372	4.82070437391297\\
0.743371371371371	4.79943347461916\\
0.74137037037037	4.77834592916606\\
0.739369369369369	4.7574403915866\\
0.737368368368368	4.73671550996204\\
0.735367367367367	4.71616992714545\\
0.733366366366366	4.69580228146321\\
0.731365365365365	4.67561120739466\\
0.729364364364364	4.65559533623014\\
0.727363363363363	4.63575329670772\\
0.725362362362362	4.61608371562875\\
0.723361361361361	4.59658521845266\\
0.72136036036036	4.57725642987112\\
0.719359359359359	4.55809597436204\\
0.717358358358358	4.53910247672359\\
0.715357357357357	4.52027456258873\\
0.713356356356356	4.50161085892033\\
0.711355355355355	4.48310999448756\\
0.709354354354354	4.46477060032365\\
0.707353353353353	4.4465913101655\\
0.705352352352352	4.4285707608755\\
0.703351351351351	4.41070759284594\\
0.70135035035035	4.39300045038637\\
0.699349349349349	4.37544798209431\\
0.697348348348348	4.35804884120973\\
0.695347347347347	4.34080168595361\\
0.693346346346346	4.32370517985106\\
0.691345345345345	4.30675799203932\\
0.689344344344344	4.28995879756111\\
0.687343343343343	4.27330627764357\\
0.685342342342342	4.25679911996342\\
0.683341341341341	4.2404360188984\\
0.68134034034034	4.22421567576573\\
0.679339339339339	4.20813679904765\\
0.677338338338338	4.19219810460463\\
0.675337337337337	4.17639831587654\\
0.673336336336336	4.16073616407205\\
0.671335335335335	4.1452103883469\\
0.669334334334334	4.12981973597102\\
0.667333333333333	4.11456296248518\\
0.665332332332332	4.09943883184731\\
0.663331331331331	4.08444611656893\\
0.66133033033033	4.06958359784195\\
0.659329329329329	4.05485006565625\\
0.657328328328328	4.04024431890826\\
0.655327327327327	4.02576516550101\\
0.653326326326326	4.01141142243567\\
0.651325325325325	3.99718191589526\\
0.649324324324324	3.98307548132053\\
0.647323323323323	3.96909096347836\\
0.645322322322322	3.9552272165231\\
0.643321321321321	3.94148310405088\\
0.64132032032032	3.92785749914745\\
0.639319319319319	3.91434928442949\\
0.637318318318318	3.90095735207993\\
0.635317317317317	3.88768060387735\\
0.633316316316316	3.87451795121973\\
0.631315315315315	3.86146831514284\\
0.629314314314314	3.84853062633345\\
0.627313313313313	3.83570382513758\\
0.625312312312312	3.82298686156401\\
0.623311311311311	3.8103786952833\\
0.62131031031031	3.79787829562242\\
0.619309309309309	3.78548464155528\\
0.617308308308308	3.77319672168933\\
0.615307307307307	3.76101353424836\\
0.613306306306306	3.74893408705179\\
0.611305305305305	3.73695739749041\\
0.609304304304304	3.72508249249903\\
0.607303303303303	3.71330840852598\\
0.605302302302302	3.70163419149961\\
0.603301301301301	3.69005889679216\\
0.6013003003003	3.67858158918085\\
0.599299299299299	3.66720134280658\\
0.597298298298298	3.65591724113021\\
0.595297297297297	3.64472837688662\\
0.593296296296296	3.63363385203672\\
0.591295295295295	3.62263277771742\\
0.589294294294294	3.61172427418972\\
0.587293293293293	3.60090747078514\\
0.585292292292292	3.59018150585043\\
0.583291291291291	3.57954552669074\\
0.58129029029029	3.5689986895114\\
0.579289289289289	3.55854015935833\\
0.577288288288288	3.54816911005719\\
0.575287287287287	3.53788472415144\\
0.573286286286286	3.52768619283918\\
0.571285285285285	3.51757271590919\\
0.569284284284284	3.50754350167591\\
0.567283283283283	3.49759776691365\\
0.565282282282282	3.48773473679003\\
0.563281281281281	3.47795364479873\\
0.56128028028028	3.46825373269158\\
0.559279279279279	3.45863425041015\\
0.557278278278278	3.44909445601682\\
0.555277277277277	3.43963361562534\\
0.553276276276276	3.43025100333112\\
0.551275275275275	3.4209459011411\\
0.549274274274274	3.4117175989034\\
0.547273273273273	3.40256539423667\\
0.545272272272272	3.39348859245935\\
0.543271271271271	3.38448650651871\\
0.54127027027027	3.37555845691985\\
0.539269269269269	3.36670377165458\\
0.537268268268268	3.35792178613038\\
0.535267267267267	3.34921184309928\\
0.533266266266266	3.34057329258685\\
0.531265265265265	3.33200549182131\\
0.529264264264264	3.32350780516266\\
0.527263263263263	3.31507960403207\\
0.525262262262262	3.3067202668414\\
0.523261261261261	3.29842917892288\\
0.52126026026026	3.29020573245915\\
0.519259259259259	3.28204932641341\\
0.517258258258258	3.27395936645997\\
0.515257257257257	3.26593526491502\\
0.513256256256256	3.25797644066775\\
0.511255255255255	3.25008231911185\\
0.509254254254254	3.2422523320773\\
0.507253253253253	3.23448591776258\\
0.505252252252252	3.22678252066726\\
0.503251251251251	3.21914159152503\\
0.50125025025025	3.21156258723706\\
0.499249249249249	3.20404497080593\\
0.497248248248248	3.19658821126991\\
0.495247247247247	3.18919178363769\\
0.493246246246246	3.18185516882367\\
0.491245245245245	3.17457785358364\\
0.489244244244244	3.16735933045098\\
0.487243243243243	3.16019909767341\\
0.485242242242242	3.1530966591501\\
0.483241241241241	3.14605152436948\\
0.48124024024024	3.13906320834742\\
0.479239239239239	3.13213123156602\\
0.477238238238238	3.12525511991287\\
0.475237237237237	3.11843440462088\\
0.473236236236236	3.11166862220865\\
0.471235235235235	3.10495731442138\\
0.469234234234234	3.09830002817228\\
0.467233233233233	3.09169631548463\\
0.465232232232232	3.08514573343424\\
0.463231231231231	3.07864784409264\\
0.46123023023023	3.07220221447066\\
0.459229229229229	3.06580841646268\\
0.457228228228228	3.0594660267914\\
0.455227227227227	3.05317462695311\\
0.453226226226226	3.04693380316362\\
0.451225225225225	3.04074314630469\\
0.449224224224224	3.03460225187099\\
0.447223223223223	3.02851071991764\\
0.445222222222222	3.02246815500838\\
0.443221221221221	3.01647416616415\\
0.44122022022022	3.01052836681233\\
0.439219219219219	3.00463037473647\\
0.437218218218218	2.99877981202665\\
0.435217217217217	2.99297630503026\\
0.433216216216216	2.98721948430346\\
0.431215215215215	2.98150898456305\\
0.429214214214214	2.97584444463903\\
0.427213213213213	2.97022550742754\\
0.425212212212212	2.96465181984442\\
0.423211211211211	2.95912303277933\\
0.42121021021021	2.9536388010503\\
0.419209209209209	2.94819878335889\\
0.417208208208208	2.94280264224584\\
0.415207207207207	2.93745004404722\\
0.413206206206206	2.93214065885111\\
0.411205205205205	2.92687416045481\\
0.409204204204204	2.92165022632252\\
0.407203203203203	2.91646853754356\\
0.405202202202202	2.91132877879103\\
0.403201201201201	2.90623063828105\\
0.4012002002002	2.90117380773238\\
0.399199199199199	2.89615798232668\\
0.397198198198198	2.89118286066907\\
0.395197197197197	2.88624814474932\\
0.393196196196196	2.88135353990345\\
0.391195195195195	2.87649875477577\\
0.389194194194194	2.87168350128146\\
0.387193193193193	2.86690749456953\\
0.385192192192192	2.86217045298635\\
0.383191191191191	2.85747209803948\\
0.38119019019019	2.85281215436207\\
0.379189189189189	2.84819034967768\\
0.377188188188188	2.8436064147655\\
0.375187187187187	2.83906008342603\\
0.373186186186186	2.83455109244719\\
0.371185185185185	2.83007918157088\\
0.369184184184184	2.8256440934599\\
0.367183183183183	2.82124557366538\\
0.365182182182182	2.81688337059452\\
0.363181181181181	2.8125572354788\\
0.36118018018018	2.80826692234262\\
0.359179179179179	2.80401218797226\\
0.357178178178178	2.7997927918853\\
0.355177177177177	2.79560849630042\\
0.353176176176176	2.79145906610758\\
0.351175175175175	2.78734426883856\\
0.349174174174174	2.78326387463793\\
0.347173173173173	2.77921765623436\\
0.345172172172172	2.7752053889123\\
0.343171171171171	2.77122685048404\\
0.34117017017017	2.76728182126214\\
0.339169169169169	2.76337008403217\\
0.337168168168168	2.75949142402589\\
0.335167167167167	2.75564562889469\\
0.333166166166166	2.75183248868342\\
0.331165165165165	2.74805179580458\\
0.329164164164164	2.74430334501281\\
0.327163163163163	2.74058693337971\\
0.325162162162162	2.73690236026907\\
0.323161161161161	2.73324942731229\\
0.32116016016016	2.72962793838427\\
0.319159159159159	2.72603769957949\\
0.317158158158158	2.72247851918849\\
0.315157157157157	2.71895020767466\\
0.313156156156156	2.71545257765126\\
0.311155155155155	2.71198544385881\\
0.309154154154154	2.70854862314279\\
0.307153153153153	2.70514193443161\\
0.305152152152152	2.70176519871483\\
0.303151151151151	2.69841823902177\\
0.30115015015015	2.69510088040033\\
0.299149149149149	2.69181294989609\\
0.297148148148148	2.68855427653177\\
0.295147147147147	2.68532469128685\\
0.293146146146146	2.68212402707755\\
0.291145145145145	2.67895211873705\\
0.289144144144144	2.67580880299597\\
0.287143143143143	2.67269391846313\\
0.285142142142142	2.66960730560654\\
0.283141141141141	2.66654880673469\\
0.28114014014014	2.66351826597806\\
0.279139139139139	2.66051552927086\\
0.277138138138138	2.6575404443331\\
0.275137137137137	2.65459286065282\\
0.273136136136136	2.65167262946857\\
0.271135135135135	2.64877960375221\\
0.269134134134134	2.64591363819181\\
0.267133133133133	2.64307458917494\\
0.265132132132132	2.64026231477204\\
0.263131131131131	2.63747667472011\\
0.26113013013013	2.63471753040661\\
0.259129129129129	2.63198474485357\\
0.257128128128128	2.62927818270189\\
0.255127127127127	2.62659771019592\\
0.253126126126126	2.62394319516819\\
0.251125125125125	2.62131450702442\\
0.249124124124124	2.61871151672865\\
0.247123123123123	2.61613409678866\\
0.245122122122122	2.61358212124155\\
0.243121121121121	2.61105546563952\\
0.24112012012012	2.60855400703586\\
0.239119119119119	2.60607762397114\\
0.237118118118118	2.6036261964596\\
0.235117117117117	2.60119960597569\\
0.233116116116116	2.59879773544083\\
0.231115115115115	2.5964204692104\\
0.229114114114114	2.59406769306082\\
0.227113113113113	2.5917392941769\\
0.225112112112112	2.58943516113929\\
0.223111111111111	2.58715518391221\\
0.22111011011011	2.58489925383124\\
0.219109109109109	2.5826672635914\\
0.217108108108108	2.5804591072353\\
0.215107107107107	2.57827468014152\\
0.213106106106106	2.57611387901315\\
0.211105105105105	2.57397660186648\\
0.209104104104104	2.57186274801988\\
0.207103103103103	2.56977221808279\\
0.205102102102102	2.56770491394497\\
0.203101101101101	2.56566073876577\\
0.2011001001001	2.56363959696371\\
0.199099099099099	2.56164139420605\\
0.197098098098098	2.55966603739871\\
0.195097097097097	2.55771343467612\\
0.193096096096096	2.55578349539142\\
0.191095095095095	2.55387613010669\\
0.189094094094094	2.55199125058334\\
0.187093093093093	2.55012876977269\\
0.185092092092092	2.54828860180666\\
0.183091091091091	2.5464706619886\\
0.18109009009009	2.54467486678429\\
0.179089089089089	2.54290113381304\\
0.177088088088088	2.54114938183894\\
0.175087087087087	2.53941953076226\\
0.173086086086086	2.537711501611\\
0.171085085085085	2.5360252165325\\
0.169084084084084	2.53436059878528\\
0.167083083083083	2.5327175727309\\
0.165082082082082	2.53109606382609\\
0.163081081081081	2.52949599861486\\
0.16108008008008	2.52791730472083\\
0.159079079079079	2.52635991083964\\
0.157078078078078	2.52482374673155\\
0.155077077077077	2.52330874321406\\
0.153076076076076	2.52181483215471\\
0.151075075075075	2.52034194646405\\
0.149074074074074	2.5188900200886\\
0.147073073073073	2.51745898800405\\
0.145072072072072	2.51604878620853\\
0.143071071071071	2.51465935171597\\
0.14107007007007	2.51329062254959\\
0.139069069069069	2.51194253773554\\
0.137068068068068	2.51061503729664\\
0.135067067067067	2.50930806224613\\
0.133066066066066	2.50802155458172\\
0.131065065065065	2.50675545727957\\
0.129064064064064	2.50550971428848\\
0.127063063063063	2.50428427052415\\
0.125062062062062	2.50307907186357\\
0.123061061061061	2.50189406513947\\
0.12106006006006	2.50072919813494\\
0.119059059059059	2.49958441957809\\
0.117058058058058	2.49845967913686\\
0.115057057057057	2.49735492741388\\
0.113056056056056	2.49627011594151\\
0.111055055055055	2.49520519717689\\
0.109054054054054	2.49416012449717\\
0.107053053053053	2.49313485219475\\
0.105052052052052	2.49212933547272\\
0.103051051051051	2.4911435304403\\
0.10105005005005	2.49017739410847\\
0.0990490490490491	2.4892308843856\\
0.0970480480480481	2.48830396007324\\
0.0950470470470471	2.48739658086198\\
0.093046046046046	2.48650870732742\\
0.091045045045045	2.48564030092622\\
0.089044044044044	2.48479132399221\\
0.0870430430430431	2.48396173973268\\
0.085042042042042	2.48315151222465\\
0.083041041041041	2.48236060641132\\
0.08104004004004	2.48158898809857\\
0.079039039039039	2.48083662395152\\
0.077038038038038	2.48010348149127\\
0.075037037037037	2.47938952909163\\
0.073036036036036	2.47869473597598\\
0.071035035035035	2.47801907221422\\
0.069034034034034	2.47736250871981\\
0.067033033033033	2.47672501724686\\
0.065032032032032	2.47610657038737\\
0.063031031031031	2.47550714156849\\
0.06103003003003	2.47492670504988\\
0.059029029029029	2.47436523592119\\
0.057028028028028	2.47382271009961\\
0.055027027027027	2.47329910432745\\
0.053026026026026	2.47279439616986\\
0.051025025025025	2.47230856401266\\
0.049024024024024	2.47184158706015\\
0.047023023023023	2.47139344533308\\
0.045022022022022	2.47096411966672\\
0.043021021021021	2.47055359170891\\
0.04102002002002	2.47016184391831\\
0.039019019019019	2.46978885956264\\
0.037018018018018	2.46943462271708\\
0.035017017017017	2.46909911826263\\
0.033016016016016	2.46878233188471\\
0.031015015015015	2.4684842500717\\
0.029014014014014	2.46820486011363\\
0.027013013013013	2.46794415010094\\
0.025012012012012	2.46770210892331\\
0.023011011011011	2.46747872626856\\
0.02101001001001	2.46727399262166\\
0.019009009009009	2.46708789926377\\
0.017008008008008	2.46692043827141\\
0.015007007007007	2.46677160251568\\
0.013006006006006	2.46664138566155\\
0.011005005005005	2.46652978216725\\
0.009004004004004	2.46643678728373\\
0.007003003003003	2.4663623970542\\
0.005002002002002	2.46630660831371\\
0.003001001001001	2.46626941868887\\
0.001	2.46625082659764\\
0.003001001001001	2.46626941868887\\
0.005002002002002	2.46630660831371\\
0.007003003003003	2.4663623970542\\
0.009004004004004	2.46643678728373\\
0.011005005005005	2.46652978216725\\
0.013006006006006	2.46664138566155\\
0.015007007007007	2.46677160251568\\
0.017008008008008	2.46692043827141\\
0.019009009009009	2.46708789926377\\
0.02101001001001	2.46727399262166\\
0.023011011011011	2.46747872626856\\
0.025012012012012	2.46770210892331\\
0.027013013013013	2.46794415010094\\
0.029014014014014	2.46820486011363\\
0.031015015015015	2.4684842500717\\
0.033016016016016	2.46878233188471\\
0.035017017017017	2.46909911826263\\
0.037018018018018	2.46943462271708\\
0.039019019019019	2.46978885956265\\
0.04102002002002	2.47016184391831\\
0.043021021021021	2.47055359170891\\
0.045022022022022	2.47096411966672\\
0.047023023023023	2.47139344533308\\
0.049024024024024	2.47184158706015\\
0.051025025025025	2.47230856401266\\
0.053026026026026	2.47279439616986\\
0.055027027027027	2.47329910432745\\
0.057028028028028	2.47382271009961\\
0.059029029029029	2.47436523592119\\
0.06103003003003	2.47492670504988\\
0.063031031031031	2.47550714156849\\
0.065032032032032	2.47610657038737\\
0.067033033033033	2.47672501724686\\
0.069034034034034	2.47736250871981\\
0.071035035035035	2.47801907221422\\
0.073036036036036	2.47869473597598\\
0.075037037037037	2.47938952909163\\
0.077038038038038	2.48010348149127\\
0.079039039039039	2.48083662395152\\
0.08104004004004	2.48158898809857\\
0.083041041041041	2.48236060641132\\
0.085042042042042	2.48315151222465\\
0.0870430430430431	2.48396173973268\\
0.089044044044044	2.48479132399221\\
0.091045045045045	2.48564030092622\\
0.093046046046046	2.48650870732742\\
0.0950470470470471	2.48739658086198\\
0.0970480480480481	2.48830396007324\\
0.0990490490490491	2.4892308843856\\
0.10105005005005	2.49017739410847\\
0.103051051051051	2.4911435304403\\
0.105052052052052	2.49212933547272\\
0.107053053053053	2.49313485219475\\
0.109054054054054	2.49416012449717\\
0.111055055055055	2.49520519717689\\
0.113056056056056	2.49627011594151\\
0.115057057057057	2.49735492741388\\
0.117058058058058	2.49845967913686\\
0.119059059059059	2.49958441957809\\
0.12106006006006	2.50072919813494\\
0.123061061061061	2.50189406513947\\
0.125062062062062	2.50307907186357\\
0.127063063063063	2.50428427052415\\
0.129064064064064	2.50550971428848\\
0.131065065065065	2.50675545727957\\
0.133066066066066	2.50802155458172\\
0.135067067067067	2.50930806224613\\
0.137068068068068	2.51061503729664\\
0.139069069069069	2.51194253773554\\
0.14107007007007	2.51329062254959\\
0.143071071071071	2.51465935171597\\
0.145072072072072	2.51604878620853\\
0.147073073073073	2.51745898800405\\
0.149074074074074	2.5188900200886\\
0.151075075075075	2.52034194646405\\
0.153076076076076	2.52181483215471\\
0.155077077077077	2.52330874321406\\
0.157078078078078	2.52482374673155\\
0.159079079079079	2.52635991083964\\
0.16108008008008	2.52791730472083\\
0.163081081081081	2.52949599861486\\
0.165082082082082	2.53109606382609\\
0.167083083083083	2.5327175727309\\
0.169084084084084	2.53436059878528\\
0.171085085085085	2.5360252165325\\
0.173086086086086	2.537711501611\\
0.175087087087087	2.53941953076226\\
0.177088088088088	2.54114938183894\\
0.179089089089089	2.54290113381304\\
0.18109009009009	2.54467486678429\\
0.183091091091091	2.5464706619886\\
0.185092092092092	2.54828860180666\\
0.187093093093093	2.55012876977269\\
0.189094094094094	2.55199125058334\\
0.191095095095095	2.55387613010669\\
0.193096096096096	2.55578349539142\\
0.195097097097097	2.55771343467612\\
0.197098098098098	2.55966603739871\\
0.199099099099099	2.56164139420605\\
0.2011001001001	2.56363959696371\\
0.203101101101101	2.56566073876577\\
0.205102102102102	2.56770491394497\\
0.207103103103103	2.56977221808279\\
0.209104104104104	2.57186274801988\\
0.211105105105105	2.57397660186648\\
0.213106106106106	2.57611387901315\\
0.215107107107107	2.57827468014152\\
0.217108108108108	2.5804591072353\\
0.219109109109109	2.5826672635914\\
0.22111011011011	2.58489925383124\\
0.223111111111111	2.5871551839122\\
0.225112112112112	2.58943516113929\\
0.227113113113113	2.5917392941769\\
0.229114114114114	2.59406769306082\\
0.231115115115115	2.5964204692104\\
0.233116116116116	2.59879773544083\\
0.235117117117117	2.60119960597569\\
0.237118118118118	2.6036261964596\\
0.239119119119119	2.60607762397114\\
0.24112012012012	2.60855400703586\\
0.243121121121121	2.61105546563952\\
0.245122122122122	2.61358212124155\\
0.247123123123123	2.61613409678866\\
0.249124124124124	2.61871151672865\\
0.251125125125125	2.62131450702442\\
0.253126126126126	2.62394319516819\\
0.255127127127127	2.62659771019592\\
0.257128128128128	2.62927818270189\\
0.259129129129129	2.63198474485357\\
0.26113013013013	2.63471753040661\\
0.263131131131131	2.63747667472011\\
0.265132132132132	2.64026231477204\\
0.267133133133133	2.64307458917494\\
0.269134134134134	2.64591363819181\\
0.271135135135135	2.64877960375221\\
0.273136136136136	2.65167262946857\\
0.275137137137137	2.65459286065282\\
0.277138138138138	2.6575404443331\\
0.279139139139139	2.66051552927086\\
0.28114014014014	2.66351826597806\\
0.283141141141141	2.66654880673469\\
0.285142142142142	2.66960730560654\\
0.287143143143143	2.67269391846312\\
0.289144144144144	2.67580880299597\\
0.291145145145145	2.67895211873705\\
0.293146146146146	2.68212402707755\\
0.295147147147147	2.68532469128685\\
0.297148148148148	2.68855427653177\\
0.299149149149149	2.69181294989609\\
0.30115015015015	2.69510088040032\\
0.303151151151151	2.69841823902177\\
0.305152152152152	2.70176519871483\\
0.307153153153153	2.70514193443161\\
0.309154154154154	2.70854862314279\\
0.311155155155155	2.71198544385881\\
0.313156156156156	2.71545257765126\\
0.315157157157157	2.71895020767466\\
0.317158158158158	2.72247851918849\\
0.319159159159159	2.72603769957949\\
0.32116016016016	2.72962793838427\\
0.323161161161161	2.73324942731229\\
0.325162162162162	2.73690236026907\\
0.327163163163163	2.74058693337971\\
0.329164164164164	2.74430334501281\\
0.331165165165165	2.74805179580458\\
0.333166166166166	2.75183248868342\\
0.335167167167167	2.75564562889469\\
0.337168168168168	2.75949142402589\\
0.339169169169169	2.76337008403217\\
0.34117017017017	2.76728182126213\\
0.343171171171171	2.77122685048404\\
0.345172172172172	2.7752053889123\\
0.347173173173173	2.77921765623436\\
0.349174174174174	2.78326387463793\\
0.351175175175175	2.78734426883856\\
0.353176176176176	2.79145906610758\\
0.355177177177177	2.79560849630042\\
0.357178178178178	2.7997927918853\\
0.359179179179179	2.80401218797226\\
0.36118018018018	2.80826692234262\\
0.363181181181181	2.8125572354788\\
0.365182182182182	2.81688337059452\\
0.367183183183183	2.82124557366539\\
0.369184184184184	2.8256440934599\\
0.371185185185185	2.83007918157088\\
0.373186186186186	2.83455109244719\\
0.375187187187187	2.83906008342603\\
0.377188188188188	2.8436064147655\\
0.379189189189189	2.84819034967768\\
0.38119019019019	2.85281215436207\\
0.383191191191191	2.85747209803948\\
0.385192192192192	2.86217045298635\\
0.387193193193193	2.86690749456953\\
0.389194194194194	2.87168350128146\\
0.391195195195195	2.87649875477577\\
0.393196196196196	2.88135353990345\\
0.395197197197197	2.88624814474932\\
0.397198198198198	2.89118286066907\\
0.399199199199199	2.89615798232668\\
0.4012002002002	2.90117380773238\\
0.403201201201201	2.90623063828105\\
0.405202202202202	2.91132877879103\\
0.407203203203203	2.91646853754356\\
0.409204204204204	2.92165022632252\\
0.411205205205205	2.92687416045481\\
0.413206206206206	2.93214065885111\\
0.415207207207207	2.93745004404722\\
0.417208208208208	2.94280264224584\\
0.419209209209209	2.94819878335889\\
0.42121021021021	2.9536388010503\\
0.423211211211211	2.95912303277933\\
0.425212212212212	2.96465181984442\\
0.427213213213213	2.97022550742754\\
0.429214214214214	2.97584444463903\\
0.431215215215215	2.98150898456305\\
0.433216216216216	2.98721948430346\\
0.435217217217217	2.99297630503026\\
0.437218218218218	2.99877981202665\\
0.439219219219219	3.00463037473647\\
0.44122022022022	3.01052836681233\\
0.443221221221221	3.01647416616415\\
0.445222222222222	3.02246815500838\\
0.447223223223223	3.02851071991764\\
0.449224224224224	3.03460225187099\\
0.451225225225225	3.04074314630469\\
0.453226226226226	3.04693380316362\\
0.455227227227227	3.05317462695311\\
0.457228228228228	3.0594660267914\\
0.459229229229229	3.06580841646268\\
0.46123023023023	3.07220221447066\\
0.463231231231231	3.07864784409264\\
0.465232232232232	3.08514573343424\\
0.467233233233233	3.09169631548463\\
0.469234234234234	3.09830002817228\\
0.471235235235235	3.10495731442138\\
0.473236236236236	3.11166862220865\\
0.475237237237237	3.11843440462088\\
0.477238238238238	3.12525511991287\\
0.479239239239239	3.13213123156602\\
0.48124024024024	3.13906320834742\\
0.483241241241241	3.14605152436948\\
0.485242242242242	3.1530966591501\\
0.487243243243243	3.16019909767341\\
0.489244244244244	3.16735933045099\\
0.491245245245245	3.17457785358364\\
0.493246246246246	3.18185516882367\\
0.495247247247247	3.18919178363769\\
0.497248248248248	3.19658821126991\\
0.499249249249249	3.20404497080593\\
0.50125025025025	3.21156258723706\\
0.503251251251251	3.21914159152503\\
0.505252252252252	3.22678252066727\\
0.507253253253253	3.23448591776258\\
0.509254254254254	3.2422523320773\\
0.511255255255255	3.25008231911185\\
0.513256256256256	3.25797644066775\\
0.515257257257257	3.26593526491502\\
0.517258258258258	3.27395936645997\\
0.519259259259259	3.28204932641341\\
0.52126026026026	3.29020573245915\\
0.523261261261261	3.29842917892288\\
0.525262262262262	3.3067202668414\\
0.527263263263263	3.31507960403207\\
0.529264264264264	3.32350780516266\\
0.531265265265265	3.33200549182131\\
0.533266266266266	3.34057329258685\\
0.535267267267267	3.34921184309928\\
0.537268268268268	3.35792178613038\\
0.539269269269269	3.36670377165458\\
0.54127027027027	3.37555845691985\\
0.543271271271271	3.38448650651871\\
0.545272272272272	3.39348859245935\\
0.547273273273273	3.40256539423666\\
0.549274274274274	3.41171759890339\\
0.551275275275275	3.4209459011411\\
0.553276276276276	3.43025100333112\\
0.555277277277277	3.43963361562534\\
0.557278278278278	3.44909445601682\\
0.559279279279279	3.45863425041015\\
0.56128028028028	3.46825373269158\\
0.563281281281281	3.47795364479873\\
0.565282282282282	3.48773473679003\\
0.567283283283283	3.49759776691365\\
0.569284284284284	3.50754350167591\\
0.571285285285285	3.51757271590919\\
0.573286286286286	3.52768619283918\\
0.575287287287287	3.53788472415144\\
0.577288288288288	3.54816911005719\\
0.579289289289289	3.55854015935833\\
0.58129029029029	3.5689986895114\\
0.583291291291291	3.57954552669074\\
0.585292292292292	3.59018150585043\\
0.587293293293293	3.60090747078514\\
0.589294294294294	3.61172427418972\\
0.591295295295295	3.62263277771742\\
0.593296296296296	3.63363385203673\\
0.595297297297297	3.64472837688662\\
0.597298298298298	3.65591724113021\\
0.599299299299299	3.66720134280658\\
0.6013003003003	3.67858158918086\\
0.603301301301301	3.69005889679216\\
0.605302302302302	3.70163419149961\\
0.607303303303303	3.71330840852598\\
0.609304304304304	3.72508249249903\\
0.611305305305305	3.73695739749041\\
0.613306306306306	3.74893408705179\\
0.615307307307307	3.76101353424836\\
0.617308308308308	3.77319672168933\\
0.619309309309309	3.78548464155528\\
0.62131031031031	3.79787829562242\\
0.623311311311311	3.8103786952833\\
0.625312312312312	3.82298686156401\\
0.627313313313313	3.83570382513758\\
0.629314314314314	3.84853062633345\\
0.631315315315315	3.86146831514284\\
0.633316316316316	3.87451795121973\\
0.635317317317317	3.88768060387736\\
0.637318318318318	3.90095735207993\\
0.639319319319319	3.91434928442949\\
0.64132032032032	3.92785749914745\\
0.643321321321321	3.94148310405088\\
0.645322322322322	3.9552272165231\\
0.647323323323323	3.96909096347836\\
0.649324324324324	3.98307548132053\\
0.651325325325325	3.99718191589526\\
0.653326326326326	4.01141142243566\\
0.655327327327327	4.02576516550101\\
0.657328328328328	4.04024431890826\\
0.659329329329329	4.05485006565625\\
0.66133033033033	4.06958359784195\\
0.663331331331331	4.08444611656893\\
0.665332332332332	4.09943883184732\\
0.667333333333333	4.11456296248518\\
0.669334334334334	4.12981973597102\\
0.671335335335335	4.1452103883469\\
0.673336336336336	4.16073616407205\\
0.675337337337337	4.17639831587654\\
0.677338338338338	4.19219810460463\\
0.679339339339339	4.20813679904765\\
0.68134034034034	4.22421567576573\\
0.683341341341341	4.2404360188984\\
0.685342342342342	4.25679911996342\\
0.687343343343343	4.27330627764358\\
0.689344344344344	4.28995879756111\\
0.691345345345345	4.30675799203932\\
0.693346346346346	4.32370517985106\\
0.695347347347347	4.34080168595361\\
0.697348348348348	4.35804884120973\\
0.699349349349349	4.37544798209431\\
0.70135035035035	4.39300045038637\\
0.703351351351351	4.41070759284594\\
0.705352352352352	4.4285707608755\\
0.707353353353353	4.44659131016549\\
0.709354354354354	4.46477060032365\\
0.711355355355355	4.48310999448756\\
0.713356356356356	4.50161085892033\\
0.715357357357357	4.52027456258873\\
0.717358358358358	4.53910247672359\\
0.719359359359359	4.55809597436204\\
0.72136036036036	4.57725642987112\\
0.723361361361361	4.59658521845266\\
0.725362362362362	4.61608371562875\\
0.727363363363363	4.63575329670772\\
0.729364364364364	4.65559533623014\\
0.731365365365365	4.67561120739466\\
0.733366366366366	4.69580228146321\\
0.735367367367367	4.71616992714545\\
0.737368368368368	4.73671550996204\\
0.739369369369369	4.7574403915866\\
0.74137037037037	4.77834592916606\\
0.743371371371371	4.79943347461916\\
0.745372372372372	4.82070437391297\\
0.747373373373373	4.84215996631723\\
0.749374374374374	4.86380158363638\\
0.751375375375375	4.88563054941912\\
0.753376376376376	4.9076481781455\\
0.755377377377377	4.92985577439138\\
0.757378378378378	4.95225463197037\\
0.759379379379379	4.97484603305315\\
0.76138038038038	4.99763124726427\\
0.763381381381381	5.02061153075657\\
0.765382382382382	5.04378812526322\\
0.767383383383383	5.06716225712773\\
0.769384384384384	5.09073513631201\\
0.771385385385385	5.11450795538286\\
0.773386386386386	5.13848188847703\\
0.775387387387387	5.16265809024548\\
0.777388388388388	5.18703769477702\\
0.779389389389389	5.21162181450193\\
0.78139039039039	5.23641153907599\\
0.783391391391391	5.2614079342457\\
0.785392392392392	5.286612040695\\
0.787393393393393	5.31202487287455\\
0.789394394394394	5.33764741781409\\
0.791395395395395	5.36348063391873\\
0.793396396396396	5.38952544975015\\
0.795397397397397	5.4157827627936\\
0.797398398398398	5.4422534382116\\
0.799399399399399	5.46893830758562\\
0.8014004004004	5.49583816764668\\
0.803401401401401	5.52295377899615\\
0.805402402402402	5.55028586481798\\
0.807403403403404	5.57783510958371\\
0.809404404404404	5.60560215775151\\
0.811405405405405	5.63358761246093\\
0.813406406406406	5.66179203422449\\
0.815407407407407	5.69021593961802\\
0.817408408408409	5.71885979997108\\
0.819409409409409	5.74772404005916\\
0.82141041041041	5.77680903679952\\
0.823411411411412	5.8061151179521\\
0.825412412412412	5.83564256082753\\
0.827413413413413	5.86539159100389\\
0.829414414414414	5.89536238105417\\
0.831415415415415	5.92555504928614\\
0.833416416416417	5.95596965849673\\
0.835417417417417	5.98660621474251\\
0.837418418418418	6.01746466612857\\
0.839419419419419	6.04854490161728\\
0.84142042042042	6.07984674985919\\
0.843421421421422	6.11136997804779\\
0.845422422422422	6.14311429079996\\
0.847423423423423	6.17507932906423\\
0.849424424424424	6.20726466905828\\
0.851425425425425	6.23966982123782\\
0.853426426426427	6.27229422929835\\
0.855427427427428	6.30513726921164\\
0.857428428428428	6.33819824829841\\
0.859429429429429	6.37147640433887\\
0.86143043043043	6.40497090472248\\
0.863431431431432	6.43868084563841\\
0.865432432432433	6.47260525130787\\
0.867433433433433	6.5067430732596\\
0.869434434434434	6.54109318964943\\
0.871435435435435	6.57565440462502\\
0.873436436436437	6.61042544773638\\
0.875437437437438	6.64540497339311\\
0.877438438438438	6.68059156036852\\
0.879439439439439	6.71598371135137\\
0.88144044044044	6.75157985254516\\
0.883441441441442	6.78737833331519\\
0.885442442442442	6.82337742588305\\
0.887443443443444	6.85957532506854\\
0.889444444444444	6.89597014807817\\
0.891445445445445	6.9325599343398\\
0.893446446446447	6.9693426453824\\
0.895447447447447	7.00631616475996\\
0.897448448448449	7.04347829801811\\
0.899449449449449	7.08082677270196\\
0.90145045045045	7.11835923840383\\
0.903451451451452	7.15607326684814\\
0.905452452452452	7.19396635201227\\
0.907453453453454	7.23203591028021\\
0.909454454454455	7.27027928062704\\
0.911455455455455	7.30869372483091\\
0.913456456456457	7.34727642770997\\
0.915457457457457	7.38602449738065\\
0.917458458458459	7.42493496553389\\
0.91945945945946	7.46400478772581\\
0.92146046046046	7.50323084367865\\
0.923461461461462	7.54260993758787\\
0.925462462462462	7.58213879843118\\
0.927463463463464	7.62181408027467\\
0.929464464464465	7.66163236257137\\
0.931465465465466	7.7015901504471\\
0.933466466466467	7.74168387496811\\
0.935467467467467	7.78190989338547\\
0.937468468468469	7.82226448934973\\
0.93946946946947	7.86274387309033\\
0.941470470470471	7.90334418155336\\
0.943471471471471	7.9440614784908\\
0.945472472472472	7.98489175449496\\
0.947473473473474	8.02583092697042\\
0.949474474474475	8.06687484003659\\
0.951475475475476	8.10801926435281\\
0.953476476476476	8.14925989685814\\
0.955477477477477	8.19059236041754\\
0.957478478478479	8.23201220336523\\
0.95947947947948	8.27351489893701\\
0.961480480480481	8.31509584458105\\
0.963481481481481	8.35675036113752\\
0.965482482482482	8.39847369187684\\
0.967483483483484	8.44026100138518\\
0.969484484484485	8.48210737428564\\
0.971485485485486	8.52400781378303\\
0.973486486486487	8.5659572400194\\
0.975487487487487	8.60795048822658\\
0.977488488488489	8.64998230666071\\
0.97948948948949	8.69204735430478\\
0.981490490490491	8.73414019832111\\
0.983491491491492	8.77625531123748\\
0.985492492492492	8.81838706784812\\
0.987493493493494	8.86052974180825\\
0.989494494494495	8.90267750190305\\
0.991495495495496	8.94482440796557\\
0.993496496496497	8.98696440642007\\
0.995497497497498	9.02909132542369\\
0.997498498498499	9.07119886957639\\
0.9994994994995	9.11328061416941\\
1.0015005005005	9.1553299989357\\
1.0035015015015	9.19734032126679\\
1.0055025025025	9.23930472885485\\
1.0075035035035	9.28121621171459\\
1.0095045045045	9.32306759353791\\
1.01150550550551	9.36485152232684\\
1.01350650650651	9.40656046024642\\
1.01550750750751	9.44818667263382\\
1.01750850850851	9.48972221609179\\
1.01950950950951	9.53115892558918\\
1.02151051051051	9.57248840048148\\
1.02351151151151	9.61370198935562\\
1.02551251251251	9.65479077359368\\
1.02751351351351	9.69574554953668\\
1.02951451451451	9.73655680911796\\
1.03151551551552	9.77721471881944\\
1.03351651651652	9.81770909678848\\
1.03551751751752	9.8580293879313\\
1.03751851851852	9.89816463677964\\
1.03951951951952	9.93810345790006\\
1.04152052052052	9.97783400358851\\
1.04352152152152	10.0173439285565\\
1.04552252252252	10.0566203512825\\
1.04752352352352	10.0956498116538\\
1.04952452452452	10.1344182244769\\
1.05152552552553	10.1729108283753\\
1.05352652652653	10.2111121295268\\
1.05552752752753	10.2490058396138\\
1.05752852852853	10.2865748072693\\
1.05952952952953	10.3238009421941\\
1.06153053053053	10.3606651309961\\
1.06353153153153	10.3971471436547\\
1.06553253253253	10.4332255293412\\
1.06753353353353	10.4688775001161\\
1.06953453453453	10.5040788007841\\
1.07153553553554	10.5388035628911\\
1.07353653653654	10.5730241404986\\
1.07553753753754	10.6067109249484\\
1.07753853853854	10.6398321353206\\
1.07953953953954	10.6723535806687\\
1.08154054054054	10.7042383893585\\
1.08354154154154	10.7354466999139\\
1.08554254254254	10.7659353066277\\
1.08754354354354	10.7956572517829\\
1.08954454454454	10.8245613545666\\
1.09154554554555	10.8525916645433\\
1.09354654654655	10.8796868247709\\
1.09554754754755	10.9057793260862\\
1.09754854854855	10.9307946295574\\
1.09954954954955	10.9546501282411\\
1.10155055055055	10.977253911776\\
1.10355155155155	10.9985032873682\\
1.10555255255255	11.018282997524\\
1.10755355355355	11.0364630572498\\
1.10955455455455	11.0528961096444\\
1.11155555555556	11.0674141663137\\
1.11355655655656	11.0798245541954\\
1.11555755755756	11.0899048276502\\
1.11755855855856	11.0973963157812\\
1.11955955955956	11.1019958471314\\
1.12156056056056	11.1033450072829\\
1.12356156156156	11.1010160079079\\
1.12556256256256	11.0944928275732\\
1.12756356356356	11.0831456414304\\
1.12956456456456	11.0661955493281\\
1.13156556556557	11.0426650039885\\
1.13356656656657	11.0113067317227\\
1.13556756756757	10.9704996533002\\
1.13756856856857	10.9180932740057\\
1.13956956956957	10.851170783684\\
1.14157057057057	10.7656851733357\\
1.14357157157157	10.6559098576013\\
1.14557257257257	10.5136864889236\\
1.14757357357357	10.3277825033691\\
1.14957457457457	10.0850958393719\\
1.15157557557558	9.77882345876005\\
1.15357657657658	9.4255488206557\\
1.15557757757758	9.06567786943005\\
1.15757857857858	8.73221158711927\\
1.15957957957958	8.43557085497143\\
1.16158058058058	8.17352102326566\\
1.16358158158158	7.94055790258506\\
1.16558258258258	7.73139171857224\\
1.16758358358358	7.54171220044968\\
1.16958458458458	7.36815611340433\\
1.17158558558559	7.20811460755065\\
1.17358658658659	7.05954884586038\\
1.17558758758759	6.920845962258\\
1.17758858858859	6.79071284021158\\
1.17958958958959	6.66809891828086\\
1.18159059059059	6.5521399828197\\
1.18359159159159	6.44211687905191\\
1.18559259259259	6.33742482536127\\
1.18759359359359	6.23755031888207\\
1.18959459459459	6.14205353371684\\
1.1915955955956	6.05055474073197\\
1.1935965965966	5.96272370799754\\
1.1955975975976	5.87827133703299\\
1.1975985985986	5.79694299568498\\
1.1995995995996	5.71851315272013\\
1.2016006006006	5.64278102159432\\
1.2036016016016	5.56956699429594\\
1.2056026026026	5.49870969944294\\
1.2076036036036	5.43006355789162\\
1.2096046046046	5.36349673807252\\
1.21160560560561	5.29888943492992\\
1.21360660660661	5.23613241272774\\
1.21560760760761	5.17512576444773\\
1.21760860860861	5.11577785010889\\
1.21960960960961	5.05800438376999\\
1.22161061061061	5.00172764478287\\
1.22361161161161	4.94687579342958\\
1.22561261261261	4.893382274694\\
1.22761361361361	4.84118529680365\\
1.22961461461461	4.79022737349008\\
1.23161561561562	4.74045492077849\\
1.23361661661662	4.69181790063651\\
1.23561761761762	4.64426950504435\\
1.23761861861862	4.59776587506133\\
1.23961961961962	4.55226585029966\\
1.24162062062062	4.50773074490402\\
1.24362162162162	4.46412414671178\\
1.24562262262262	4.42141173674934\\
1.24762362362362	4.37956112661489\\
1.24962462462462	4.33854171164635\\
1.25162562562563	4.29832453804909\\
1.25362662662663	4.25888218240506\\
1.25562762762763	4.22018864219059\\
1.25762862862863	4.18221923610408\\
1.25962962962963	4.14495051315561\\
1.26163063063063	4.10836016960067\\
1.26363163163163	4.07242697290993\\
1.26563263263263	4.03713069206356\\
1.26763363363363	4.00245203354218\\
1.26963463463463	3.96837258245794\\
1.27163563563564	3.93487474833178\\
1.27363663663664	3.90194171507933\\
1.27563763763764	3.86955739481326\\
1.27763863863864	3.83770638511491\\
1.27963963963964	3.80637392946193\\
1.28164064064064	3.77554588053348\\
1.28364164164164	3.74520866614197\\
1.28564264264264	3.71534925756577\\
1.28764364364364	3.68595514007989\\
1.28964464464464	3.65701428550217\\
1.29164564564565	3.62851512658886\\
1.29364664664665	3.60044653313035\\
1.29564764764765	3.57279778961117\\
1.29764864864865	3.54555857431154\\
1.29964964964965	3.51871893973931\\
1.30165065065065	3.49226929428829\\
1.30365165165165	3.4662003850334\\
1.30565265265265	3.44050328157582\\
1.30765365365365	3.41516936086173\\
1.30965465465465	3.39019029290426\\
1.31165565565566	3.36555802734339\\
1.31365665665666	3.34126478078509\\
1.31565765765766	3.31730302486482\\
1.31765865865866	3.29366547498604\\
1.31965965965966	3.2703450796871\\
1.32166066066066	3.24733501059448\\
1.32366166166166	3.22462865292393\\
1.32566266266266	3.20221959649224\\
1.32766366366366	3.18010162720797\\
1.32966466466466	3.15826871900925\\
1.33166566566567	3.13671502622083\\
1.33366666666667	3.11543487630403\\
1.33566766766767	3.09442276297498\\
1.33766866866867	3.07367333966876\\
1.33966966966967	3.05318141332856\\
1.34167067067067	3.0329419384999\\
1.34367167167167	3.01295001171229\\
1.34567267267267	2.99320086613115\\
1.34767367367367	2.97368986646435\\
1.34967467467467	2.95441250410879\\
1.35167567567568	2.93536439252316\\
1.35367667667668	2.91654126281449\\
1.35567767767768	2.89793895952601\\
1.35767867867868	2.87955343661581\\
1.35967967967968	2.86138075361549\\
1.36168068068068	2.84341707195906\\
1.36368168168168	2.82565865147309\\
1.36568268268268	2.80810184701956\\
1.36768368368368	2.79074310528326\\
1.36968468468468	2.77357896169624\\
1.37168568568569	2.75660603749236\\
1.37368668668669	2.7398210368851\\
1.37568768768769	2.72322074436242\\
1.37768868868869	2.706802022093\\
1.37968968968969	2.6905618074381\\
1.38169069069069	2.67449711056386\\
1.38369169169169	2.65860501214919\\
1.38569269269269	2.64288266118457\\
1.38769369369369	2.62732727285764\\
1.38969469469469	2.61193612652085\\
1.3916956956957	2.596706563738\\
1.3936966966967	2.58163598640555\\
1.3956976976977	2.56672185494541\\
1.3976986986987	2.55196168656566\\
1.3996996996997	2.53735305358658\\
1.4017007007007	2.52289358182854\\
1.4037017017017	2.50858094905934\\
1.4057027027027	2.49441288349796\\
1.4077037037037	2.48038716237287\\
1.4097047047047	2.46650161053165\\
1.41170570570571	2.45275409910059\\
1.41370670670671	2.43914254419131\\
1.41570770770771	2.42566490565299\\
1.41770870870871	2.41231918586798\\
1.41970970970971	2.39910342858903\\
1.42171071071071	2.38601571781631\\
1.42371171171171	2.37305417671283\\
1.42571271271271	2.36021696655636\\
1.42771371371371	2.34750228572652\\
1.42971471471471	2.33490836872569\\
1.43171571571572	2.32243348523222\\
1.43371671671672	2.31007593918467\\
1.43571771771772	2.29783406789597\\
1.43771871871872	2.28570624119629\\
1.43971971971972	2.27369086060326\\
1.44172072072072	2.26178635851895\\
1.44372172172172	2.24999119745196\\
1.44572272272272	2.23830386926441\\
1.44772372372372	2.22672289444204\\
1.44972472472472	2.21524682138733\\
1.45172572572573	2.20387422573421\\
1.45372672672673	2.19260370968385\\
1.45572772772773	2.18143390136068\\
1.45772872872873	2.17036345418791\\
1.45972972972973	2.15939104628168\\
1.46173073073073	2.14851537986346\\
1.46373173173173	2.13773518068985\\
1.46573273273273	2.1270491974991\\
1.46773373373373	2.11645620147393\\
1.46973473473473	2.10595498571994\\
1.47173573573574	2.09554436475921\\
1.47373673673674	2.08522317403829\\
1.47573773773774	2.07499026945043\\
1.47773873873874	2.06484452687125\\
1.47973973973974	2.05478484170759\\
1.48174074074074	2.04481012845904\\
1.48374174174174	2.03491932029162\\
1.48574274274274	2.02511136862338\\
1.48774374374374	2.01538524272137\\
1.48974474474474	2.00573992930968\\
1.49174574574575	1.99617443218814\\
1.49374674674675	1.98668777186146\\
1.49574774774775	1.97727898517815\\
1.49774874874875	1.96794712497937\\
1.49974974974975	1.95869125975695\\
1.50175075075075	1.94951047332054\\
1.50375175175175	1.94040386447357\\
1.50575275275275	1.93137054669764\\
1.50775375375375	1.92240964784515\\
1.50975475475475	1.91352030983992\\
1.51175575575576	1.90470168838552\\
1.51375675675676	1.89595295268108\\
1.51575775775776	1.88727328514436\\
1.51775875875876	1.8786618811418\\
1.51975975975976	1.87011794872547\\
1.52176076076076	1.86164070837656\\
1.52376176176176	1.85322939275529\\
1.52576276276276	1.84488324645706\\
1.52776376376376	1.83660152577458\\
1.52976476476476	1.82838349846595\\
1.53176576576577	1.82022844352827\\
1.53376676676677	1.8121356509769\\
1.53576776776777	1.80410442163003\\
1.53776876876877	1.79613406689837\\
1.53976976976977	1.78822390858003\\
1.54177077077077	1.78037327866012\\
1.54377177177177	1.7725815191153\\
1.54577277277277	1.7648479817228\\
1.54777377377377	1.75717202787402\\
1.54977477477477	1.74955302839246\\
1.55177577577578	1.74199036335592\\
1.55377677677678	1.73448342192279\\
1.55577777777778	1.72703160216241\\
1.55777877877878	1.71963431088923\\
1.55977977977978	1.71229096350089\\
1.56178078078078	1.70500098381987\\
1.56378178178178	1.69776380393882\\
1.56578278278278	1.69057886406934\\
1.56778378378378	1.68344561239412\\
1.56978478478478	1.67636350492249\\
1.57178578578579	1.66933200534906\\
1.57378678678679	1.6623505849156\\
1.57578778778779	1.65541872227592\\
1.57778878878879	1.64853590336369\\
1.57978978978979	1.64170162126322\\
1.58179079079079	1.63491537608298\\
1.58379179179179	1.62817667483196\\
1.58579279279279	1.62148503129857\\
1.58779379379379	1.61483996593225\\
1.58979479479479	1.60824100572759\\
1.5917957957958	1.60168768411087\\
1.5937967967968	1.59517954082907\\
1.5957977977978	1.58871612184122\\
1.5977987987988	1.58229697921196\\
1.5997997997998	1.57592167100746\\
1.6018008008008	1.5695897611934\\
1.6038018018018	1.56330081953511\\
1.6058028028028	1.55705442149984\\
1.6078038038038	1.5508501481609\\
1.6098048048048	1.54468758610394\\
1.61180580580581	1.53856632733499\\
1.61380680680681	1.53248596919052\\
1.61580780780781	1.5264461142492\\
1.61780880880881	1.52044637024551\\
1.61980980980981	1.51448634998508\\
1.62181081081081	1.50856567126173\\
1.62381181181181	1.50268395677615\\
1.62581281281281	1.49684083405623\\
1.62781381381381	1.49103593537891\\
1.62981481481481	1.48526889769366\\
1.63181581581582	1.47953936254736\\
1.63381681681682	1.47384697601079\\
1.63581781781782	1.46819138860639\\
1.63781881881882	1.46257225523757\\
1.63981981981982	1.45698923511932\\
1.64182082082082	1.45144199171013\\
1.64382182182182	1.44593019264532\\
1.64582282282282	1.44045350967154\\
1.64782382382382	1.43501161858259\\
1.64982482482482	1.42960419915644\\
1.65182582582583	1.42423093509343\\
1.65382682682683	1.41889151395565\\
1.65582782782783	1.41358562710747\\
1.65782882882883	1.40831296965719\\
1.65982982982983	1.40307324039972\\
1.66183083083083	1.39786614176041\\
1.66383183183183	1.39269137973989\\
1.66583283283283	1.38754866385988\\
1.66783383383383	1.38243770711011\\
1.66983483483483	1.37735822589608\\
1.67183583583584	1.37230993998789\\
1.67383683683684	1.36729257246993\\
1.67583783783784	1.36230584969153\\
1.67783883883884	1.35734950121845\\
1.67983983983984	1.35242325978532\\
1.68184084084084	1.34752686124888\\
1.68384184184184	1.34266004454205\\
1.68584284284284	1.33782255162887\\
1.68784384384384	1.33301412746019\\
1.68984484484484	1.32823451993021\\
1.69184584584585	1.3234834798337\\
1.69384684684685	1.31876076082405\\
1.69584784784785	1.31406611937205\\
1.69784884884885	1.30939931472531\\
1.69984984984985	1.30476010886852\\
1.70185085085085	1.30014826648428\\
1.70385185185185	1.2955635549147\\
1.70585285285285	1.29100574412359\\
1.70785385385385	1.28647460665934\\
1.70985485485485	1.28196991761848\\
1.71185585585586	1.27749145460974\\
1.71385685685686	1.27303899771885\\
1.71585785785786	1.2686123294739\\
1.71785885885886	1.26421123481122\\
1.71985985985986	1.25983550104195\\
1.72186086086086	1.25548491781909\\
1.72386186186186	1.25115927710513\\
1.72586286286286	1.24685837314021\\
1.72786386386386	1.24258200241086\\
1.72986486486486	1.23832996361919\\
1.73186586586587	1.23410205765262\\
1.73386686686687	1.22989808755416\\
1.73586786786787	1.22571785849308\\
1.73786886886887	1.22156117773615\\
1.73986986986987	1.21742785461931\\
1.74187087087087	1.21331770051982\\
1.74387187187187	1.20923052882883\\
1.74587287287287	1.20516615492447\\
1.74787387387387	1.20112439614527\\
1.74987487487487	1.19710507176411\\
1.75187587587588	1.1931080029625\\
1.75387687687688	1.18913301280539\\
1.75587787787788	1.18517992621624\\
1.75787887887888	1.18124856995259\\
1.75987987987988	1.17733877258199\\
1.76188088088088	1.17345036445832\\
1.76388188188188	1.16958317769844\\
1.76588288288288	1.16573704615927\\
1.76788388388388	1.16191180541522\\
1.76988488488488	1.15810729273591\\
1.77188588588589	1.15432334706433\\
1.77388688688689	1.15055980899526\\
1.77588788788789	1.14681652075412\\
1.77788888888889	1.14309332617601\\
1.77988988988989	1.13939007068523\\
1.78189089089089	1.135706601275\\
1.78389189189189	1.13204276648753\\
1.78589289289289	1.12839841639443\\
1.78789389389389	1.12477340257735\\
1.78989489489489	1.12116757810898\\
1.7918958958959	1.11758079753431\\
1.7938968968969	1.11401291685215\\
1.7958978978979	1.11046379349699\\
1.7978988988989	1.10693328632108\\
1.7998998998999	1.10342125557682\\
1.8019009009009	1.09992756289934\\
1.8039019019019	1.09645207128948\\
1.8059029029029	1.09299464509686\\
1.8079039039039	1.08955515000332\\
1.8099049049049	1.08613345300656\\
1.81190590590591	1.08272942240401\\
1.81390690690691	1.07934292777699\\
1.81590790790791	1.07597383997505\\
1.81790890890891	1.07262203110055\\
1.81990990990991	1.0692873744935\\
1.82191091091091	1.06596974471657\\
1.82391191191191	1.06266901754038\\
1.82591291291291	1.05938506992893\\
1.82791391391391	1.05611778002531\\
1.82991491491491	1.05286702713758\\
1.83191591591592	1.04963269172487\\
1.83391691691692	1.04641465538363\\
1.83591791791792	1.04321280083417\\
1.83791891891892	1.04002701190732\\
1.83991991991992	1.03685717353131\\
1.84192092092092	1.03370317171879\\
1.84392192192192	1.03056489355413\\
1.84592292292292	1.02744222718081\\
1.84792392392392	1.02433506178903\\
1.84992492492492	1.02124328760346\\
1.85192592592593	1.01816679587124\\
1.85392692692693	1.01510547885006\\
1.85592792792793	1.01205922979643\\
1.85792892892893	1.00902794295417\\
1.85992992992993	1.00601151354299\\
1.86193093093093	1.00300983774725\\
1.86393193193193	1.00002281270489\\
1.86593293293293	0.997050336496537\\
1.86793393393393	0.994092308134657\\
1.86993493493493	0.991148627552996\\
1.87193593593594	0.988219195596068\\
1.87393693693694	0.985303914008826\\
1.87593793793794	0.98240268542646\\
1.87793893893894	0.979515413364348\\
1.87993993993994	0.976642002208128\\
1.88194094094094	0.973782357203919\\
1.88394194194194	0.970936384448661\\
1.88594294294294	0.968103990880596\\
1.88794394394394	0.965285084269871\\
1.88994494494494	0.962479573209272\\
1.89194594594595	0.959687367105072\\
1.89394694694695	0.95690837616802\\
1.89594794794795	0.95414251140443\\
1.89794894894895	0.951389684607397\\
1.89994994994995	0.948649808348138\\
1.90195095095095	0.945922795967432\\
1.90395195195195	0.943208561567184\\
1.90595295295295	0.940507020002097\\
1.90795395395395	0.937818086871453\\
1.90995495495495	0.935141678511002\\
1.91195595595596	0.932477711984959\\
1.91395695695696	0.929826105078103\\
1.91595795795796	0.927186776287982\\
1.91795895895896	0.924559644817213\\
1.91995995995996	0.921944630565892\\
1.92196096096096	0.919341654124094\\
1.92396196196196	0.916750636764472\\
1.92596296296296	0.914171500434952\\
1.92796396396396	0.911604167751524\\
1.92996496496496	0.90904856199112\\
1.93196596596597	0.90650460708459\\
1.93396696696697	0.903972227609762\\
1.93596796796797	0.901451348784593\\
1.93796896896897	0.898941896460406\\
1.93996996996997	0.896443797115217\\
1.94197097097097	0.893956977847136\\
1.94397197197197	0.89148136636787\\
1.94597297297297	0.889016890996282\\
1.94797397397397	0.886563480652056\\
1.94997497497497	0.884121064849428\\
1.95197597597598	0.881689573690997\\
1.95397697697698	0.879268937861615\\
1.95597797797798	0.876859088622353\\
1.95797897897898	0.874459957804543\\
1.95997997997998	0.872071477803891\\
1.96198098098098	0.869693581574668\\
1.96398198198198	0.867326202623965\\
1.96598298298298	0.864969275006027\\
1.96798398398398	0.862622733316652\\
1.96998498498498	0.860286512687662\\
1.97198598598599	0.857960548781435\\
1.97398698698699	0.855644777785516\\
1.97598798798799	0.853339136407278\\
1.97798898898899	0.851043561868664\\
1.97998998998999	0.84875799190098\\
1.98199099099099	0.846482364739761\\
1.98399199199199	0.844216619119692\\
1.98599299299299	0.841960694269592\\
1.98799399399399	0.839714529907464\\
1.98999499499499	0.837478066235595\\
1.991995995996	0.835251243935723\\
1.993996996997	0.833034004164258\\
1.995997997998	0.830826288547562\\
1.997998998999	0.828628039177282\\
2	0.826439198605741\\
};
\addlegendentry{$\alpha\text{ = 0.003}$};

\addplot [color=red,solid]
  table[row sep=crcr]{%
0.001	2.24531183920681\\
0.003001001001001	2.24532501030795\\
0.005002002002002	2.24535135614285\\
0.007003003003003	2.24539087738637\\
0.009004004004004	2.24544357505089\\
0.011005005005005	2.24550945048629\\
0.013006006006006	2.24558850538003\\
0.015007007007007	2.2456807417572\\
0.017008008008008	2.24578616198054\\
0.019009009009009	2.24590476875057\\
0.02101001001001	2.2460365651056\\
0.023011011011011	2.24618155442188\\
0.025012012012012	2.24633974041364\\
0.027013013013013	2.24651112713326\\
0.029014014014014	2.24669571897134\\
0.031015015015015	2.24689352065683\\
0.033016016016016	2.24710453725721\\
0.035017017017017	2.2473287741786\\
0.037018018018018	2.24756623716589\\
0.039019019019019	2.24781693230298\\
0.04102002002002	2.2480808660129\\
0.043021021021021	2.24835804505801\\
0.045022022022022	2.24864847654019\\
0.047023023023023	2.24895216790106\\
0.049024024024024	2.24926912692217\\
0.051025025025025	2.24959936172525\\
0.053026026026026	2.24994288077242\\
0.055027027027027	2.25029969286644\\
0.057028028028028	2.25066980715097\\
0.059029029029029	2.25105323311079\\
0.06103003003003	2.25144998057216\\
0.063031031031031	2.251860059703\\
0.065032032032032	2.25228348101321\\
0.067033033033033	2.25272025535501\\
0.069034034034034	2.25317039392318\\
0.071035035035035	2.25363390825544\\
0.073036036036036	2.25411081023271\\
0.075037037037037	2.25460111207949\\
0.077038038038038	2.25510482636419\\
0.079039039039039	2.25562196599946\\
0.08104004004004	2.25615254424257\\
0.083041041041041	2.2566965746958\\
0.085042042042042	2.25725407130675\\
0.0870430430430431	2.25782504836879\\
0.089044044044044	2.25840952052142\\
0.091045045045045	2.25900750275068\\
0.093046046046046	2.25961901038956\\
0.0950470470470471	2.2602440591184\\
0.0970480480480481	2.26088266496537\\
0.0990490490490491	2.26153484430684\\
0.10105005005005	2.26220061386784\\
0.103051051051051	2.26287999072255\\
0.105052052052052	2.26357299229469\\
0.107053053053053	2.26427963635806\\
0.109054054054054	2.26499994103695\\
0.111055055055055	2.26573392480663\\
0.113056056056056	2.2664816064939\\
0.115057057057057	2.26724300527748\\
0.117058058058058	2.26801814068861\\
0.119059059059059	2.2688070326115\\
0.12106006006006	2.26960970128386\\
0.123061061061061	2.27042616729744\\
0.125062062062062	2.27125645159852\\
0.127063063063063	2.27210057548851\\
0.129064064064064	2.2729585606244\\
0.131065065065065	2.27383042901941\\
0.133066066066066	2.27471620304348\\
0.135067067067067	2.27561590542383\\
0.137068068068068	2.27652955924556\\
0.139069069069069	2.27745718795221\\
0.14107007007007	2.27839881534633\\
0.143071071071071	2.27935446559007\\
0.145072072072072	2.28032416320575\\
0.147073073073073	2.2813079330765\\
0.149074074074074	2.2823058004468\\
0.151075075075075	2.28331779092314\\
0.153076076076076	2.28434393047458\\
0.155077077077077	2.2853842454334\\
0.157078078078078	2.2864387624957\\
0.159079079079079	2.28750750872201\\
0.16108008008008	2.28859051153797\\
0.163081081081081	2.28968779873488\\
0.165082082082082	2.29079939847038\\
0.167083083083083	2.29192533926911\\
0.169084084084084	2.29306565002327\\
0.171085085085085	2.29422035999335\\
0.173086086086086	2.29538949880873\\
0.175087087087087	2.29657309646832\\
0.177088088088088	2.29777118334124\\
0.179089089089089	2.29898379016748\\
0.18109009009009	2.30021094805848\\
0.183091091091091	2.3014526884979\\
0.185092092092092	2.30270904334217\\
0.187093093093093	2.30398004482125\\
0.189094094094094	2.30526572553919\\
0.191095095095095	2.30656611847486\\
0.193096096096096	2.3078812569826\\
0.195097097097097	2.30921117479287\\
0.197098098098098	2.31055590601292\\
0.199099099099099	2.31191548512743\\
0.2011001001001	2.31328994699921\\
0.203101101101101	2.31467932686982\\
0.205102102102102	2.31608366036028\\
0.207103103103103	2.31750298347167\\
0.209104104104104	2.31893733258583\\
0.211105105105105	2.32038674446598\\
0.213106106106106	2.32185125625742\\
0.215107107107107	2.32333090548814\\
0.217108108108108	2.32482573006945\\
0.219109109109109	2.3263357682967\\
0.22111011011011	2.32786105884984\\
0.223111111111111	2.32940164079409\\
0.225112112112112	2.33095755358058\\
0.227113113113113	2.33252883704696\\
0.229114114114114	2.33411553141805\\
0.231115115115115	2.33571767730639\\
0.233116116116116	2.33733531571296\\
0.235117117117117	2.33896848802767\\
0.237118118118118	2.34061723603005\\
0.239119119119119	2.34228160188979\\
0.24112012012012	2.34396162816734\\
0.243121121121121	2.3456573578145\\
0.245122122122122	2.34736883417498\\
0.247123123123123	2.34909610098494\\
0.249124124124124	2.35083920237358\\
0.251125125125125	2.35259818286365\\
0.253126126126126	2.35437308737202\\
0.255127127127127	2.35616396121016\\
0.257128128128128	2.35797085008468\\
0.259129129129129	2.35979380009785\\
0.26113013013013	2.361632857748\\
0.263131131131131	2.36348806993015\\
0.265132132132132	2.36535948393632\\
0.267133133133133	2.36724714745608\\
0.269134134134134	2.36915110857694\\
0.271135135135135	2.37107141578483\\
0.273136136136136	2.37300811796443\\
0.275137137137137	2.37496126439963\\
0.277138138138138	2.37693090477389\\
0.279139139139139	2.3789170891706\\
0.28114014014014	2.38091986807339\\
0.283141141141141	2.38293929236656\\
0.285142142142142	2.38497541333528\\
0.287143143143143	2.38702828266594\\
0.289144144144144	2.38909795244642\\
0.291145145145145	2.39118447516633\\
0.293146146146146	2.39328790371722\\
0.295147147147147	2.39540829139287\\
0.297148148148148	2.39754569188941\\
0.299149149149149	2.39970015930551\\
0.30115015015015	2.40187174814251\\
0.303151151151151	2.40406051330459\\
0.305152152152152	2.40626651009881\\
0.307153153153153	2.40848979423522\\
0.309154154154154	2.41073042182688\\
0.311155155155155	2.41298844938993\\
0.313156156156156	2.41526393384354\\
0.315157157157157	2.41755693250983\\
0.317158158158158	2.41986750311395\\
0.319159159159159	2.42219570378384\\
0.32116016016016	2.42454159305024\\
0.323161161161161	2.42690522984637\\
0.325162162162162	2.42928667350794\\
0.327163163163163	2.43168598377275\\
0.329164164164164	2.43410322078063\\
0.331165165165165	2.43653844507298\\
0.333166166166166	2.43899171759253\\
0.335167167167167	2.44146309968304\\
0.337168168168168	2.44395265308879\\
0.339169169169169	2.44646043995426\\
0.34117017017017	2.44898652282357\\
0.343171171171171	2.45153096464009\\
0.345172172172172	2.45409382874579\\
0.347173173173173	2.45667517888069\\
0.349174174174174	2.45927507918226\\
0.351175175175175	2.46189359418469\\
0.353176176176176	2.46453078881826\\
0.355177177177177	2.46718672840847\\
0.357178178178178	2.46986147867538\\
0.359179179179179	2.47255510573263\\
0.36118018018018	2.47526767608654\\
0.363181181181181	2.47799925663536\\
0.365182182182182	2.48074991466796\\
0.367183183183183	2.48351971786306\\
0.369184184184184	2.48630873428796\\
0.371185185185185	2.48911703239753\\
0.373186186186186	2.49194468103283\\
0.375187187187187	2.49479174942\\
0.377188188188188	2.49765830716893\\
0.379189189189189	2.50054442427174\\
0.38119019019019	2.50345017110157\\
0.383191191191191	2.50637561841092\\
0.385192192192192	2.50932083733017\\
0.387193193193193	2.51228589936599\\
0.389194194194194	2.51527087639962\\
0.391195195195195	2.51827584068515\\
0.393196196196196	2.52130086484768\\
0.395197197197197	2.52434602188164\\
0.397198198198198	2.52741138514852\\
0.399199199199199	2.53049702837518\\
0.4012002002002	2.53360302565161\\
0.403201201201201	2.53672945142889\\
0.405202202202202	2.53987638051684\\
0.407203203203203	2.5430438880819\\
0.409204204204204	2.54623204964461\\
0.411205205205205	2.54944094107724\\
0.413206206206206	2.55267063860138\\
0.415207207207207	2.55592121878515\\
0.417208208208208	2.55919275854066\\
0.419209209209209	2.56248533512125\\
0.42121021021021	2.56579902611862\\
0.423211211211211	2.5691339094599\\
0.425212212212212	2.57249006340478\\
0.427213213213213	2.57586756654213\\
0.429214214214214	2.57926649778723\\
0.431215215215215	2.58268693637817\\
0.433216216216216	2.58612896187271\\
0.435217217217217	2.58959265414469\\
0.437218218218218	2.59307809338052\\
0.439219219219219	2.59658536007575\\
0.44122022022022	2.60011453503109\\
0.443221221221221	2.60366569934868\\
0.445222222222222	2.60723893442824\\
0.447223223223223	2.61083432196303\\
0.449224224224224	2.6144519439356\\
0.451225225225225	2.61809188261386\\
0.453226226226226	2.62175422054646\\
0.455227227227227	2.62543904055852\\
0.457228228228228	2.62914642574713\\
0.459229229229229	2.63287645947667\\
0.46123023023023	2.63662922537399\\
0.463231231231231	2.6404048073238\\
0.465232232232232	2.64420328946335\\
0.467233233233233	2.64802475617771\\
0.469234234234234	2.65186929209432\\
0.471235235235235	2.65573698207794\\
0.473236236236236	2.65962791122494\\
0.475237237237237	2.66354216485811\\
0.477238238238238	2.6674798285206\\
0.479239239239239	2.67144098797059\\
0.48124024024024	2.67542572917503\\
0.483241241241241	2.6794341383039\\
0.485242242242242	2.68346630172388\\
0.487243243243243	2.68752230599222\\
0.489244244244244	2.69160223785023\\
0.491245245245245	2.69570618421695\\
0.493246246246246	2.69983423218241\\
0.495247247247247	2.70398646900075\\
0.497248248248248	2.70816298208363\\
0.499249249249249	2.71236385899276\\
0.50125025025025	2.71658918743319\\
0.503251251251251	2.72083905524574\\
0.505252252252252	2.72511355039968\\
0.507253253253253	2.72941276098514\\
0.509254254254254	2.73373677520538\\
0.511255255255255	2.73808568136899\\
0.513256256256256	2.74245956788206\\
0.515257257257257	2.74685852323975\\
0.517258258258258	2.75128263601831\\
0.519259259259259	2.75573199486657\\
0.52126026026026	2.76020668849746\\
0.523261261261261	2.7647068056793\\
0.525262262262262	2.7692324352271\\
0.527263263263263	2.77378366599334\\
0.529264264264264	2.77836058685909\\
0.531265265265265	2.78296328672481\\
0.533266266266266	2.78759185450068\\
0.535267267267267	2.79224637909749\\
0.537268268268268	2.79692694941639\\
0.539269269269269	2.80163365433968\\
0.54127027027027	2.80636658272019\\
0.543271271271271	2.81112582337186\\
0.545272272272272	2.8159114650589\\
0.547273273273273	2.82072359648557\\
0.549274274274274	2.82556230628578\\
0.551275275275275	2.83042768301206\\
0.553276276276276	2.83531981512494\\
0.555277277277277	2.84023879098183\\
0.557278278278278	2.84518469882601\\
0.559279279279279	2.85015762677506\\
0.56128028028028	2.85515766280965\\
0.563281281281281	2.86018489476189\\
0.565282282282282	2.86523941030352\\
0.567283283283283	2.87032129693414\\
0.569284284284284	2.8754306419689\\
0.571285285285285	2.88056753252661\\
0.573286286286286	2.88573205551748\\
0.575287287287287	2.89092429763021\\
0.577288288288288	2.89614434531994\\
0.579289289289289	2.90139228479489\\
0.58129029029029	2.90666820200411\\
0.583291291291291	2.9119721826239\\
0.585292292292292	2.91730431204494\\
0.587293293293293	2.92266467535885\\
0.589294294294294	2.92805335734482\\
0.591295295295295	2.93347044245589\\
0.593296296296296	2.93891601480548\\
0.595297297297297	2.94439015815331\\
0.597298298298298	2.94989295589147\\
0.599299299299299	2.95542449103017\\
0.6013003003003	2.96098484618408\\
0.603301301301301	2.9665741035571\\
0.605302302302302	2.97219234492846\\
0.607303303303303	2.97783965163813\\
0.609304304304304	2.9835161045717\\
0.611305305305305	2.98922178414568\\
0.613306306306306	2.99495677029298\\
0.615307307307307	3.00072114244721\\
0.617308308308308	3.00651497952761\\
0.619309309309309	3.01233835992432\\
0.62131031031031	3.01819136148219\\
0.623311311311311	3.02407406148594\\
0.625312312312312	3.02998653664403\\
0.627313313313313	3.03592886307336\\
0.629314314314314	3.04190111628347\\
0.631315315315315	3.04790337116041\\
0.633316316316316	3.05393570195103\\
0.635317317317317	3.05999818224662\\
0.637318318318318	3.06609088496749\\
0.639319319319319	3.07221388234589\\
0.64132032032032	3.07836724591038\\
0.643321321321321	3.08455104646924\\
0.645322322322322	3.09076535409429\\
0.647323323323323	3.09701023810401\\
0.649324324324324	3.10328576704756\\
0.651325325325325	3.10959200868791\\
0.653326326326326	3.11592902998548\\
0.655327327327327	3.12229689708111\\
0.657328328328328	3.12869567527976\\
0.659329329329329	3.13512542903381\\
0.66133033033033	3.14158622192604\\
0.663331331331331	3.14807811665312\\
0.665332332332332	3.15460117500861\\
0.667333333333333	3.16115545786661\\
0.669334334334334	3.16774102516454\\
0.671335335335335	3.17435793588662\\
0.673336336336336	3.18100624804674\\
0.675337337337337	3.18768601867232\\
0.677338338338338	3.19439730378687\\
0.679339339339339	3.20114015839388\\
0.68134034034034	3.20791463645951\\
0.683341341341341	3.21472079089654\\
0.685342342342342	3.22155867354709\\
0.687343343343343	3.22842833516667\\
0.689344344344344	3.23532982540748\\
0.691345345345345	3.24226319280124\\
0.693346346346346	3.24922848474394\\
0.695347347347347	3.25622574747878\\
0.697348348348348	3.26325502608008\\
0.699349349349349	3.2703163644369\\
0.70135035035035	3.27740980523751\\
0.703351351351351	3.2845353899529\\
0.705352352352352	3.29169315882081\\
0.707353353353353	3.29888315083068\\
0.709354354354354	3.30610540370709\\
0.711355355355355	3.31335995389499\\
0.713356356356356	3.32064683654373\\
0.715357357357357	3.32796608549211\\
0.717358358358358	3.33531773325344\\
0.719359359359359	3.3427018110002\\
0.72136036036036	3.35011834854898\\
0.723361361361361	3.35756737434714\\
0.725362362362362	3.36504891545645\\
0.727363363363363	3.37256299754044\\
0.729364364364364	3.38010964484955\\
0.731365365365365	3.38768888020728\\
0.733366366366366	3.39530072499636\\
0.735367367367367	3.40294519914589\\
0.737368368368368	3.41062232111713\\
0.739369369369369	3.41833210789121\\
0.74137037037037	3.42607457495572\\
0.743371371371371	3.43384973629272\\
0.745372372372372	3.44165760436589\\
0.747373373373373	3.44949819010835\\
0.749374374374374	3.45737150291142\\
0.751375375375375	3.46527755061244\\
0.753376376376376	3.47321633948349\\
0.755377377377377	3.48118787422101\\
0.757378378378378	3.48919215793374\\
0.759379379379379	3.49722919213378\\
0.76138038038038	3.50529897672524\\
0.763381381381381	3.51340150999453\\
0.765382382382382	3.52153678860154\\
0.767383383383383	3.52970480756922\\
0.769384384384384	3.5379055602758\\
0.771385385385385	3.54613903844523\\
0.773386386386386	3.55440523213939\\
0.775387387387387	3.56270412974959\\
0.777388388388388	3.57103571799014\\
0.779389389389389	3.5793999818895\\
0.78139039039039	3.58779690478394\\
0.783391391391391	3.59622646831199\\
0.785392392392392	3.60468865240678\\
0.787393393393393	3.61318343529078\\
0.789394394394394	3.62171079347044\\
0.791395395395395	3.63027070173059\\
0.793396396396396	3.63886313313029\\
0.795397397397397	3.64748805899838\\
0.797398398398398	3.65614544892936\\
0.799399399399399	3.66483527077917\\
0.8014004004004	3.67355749066359\\
0.803401401401401	3.68231207295383\\
0.805402402402402	3.69109898027474\\
0.807403403403404	3.6999181735033\\
0.809404404404404	3.70876961176619\\
0.811405405405405	3.71765325243952\\
0.813406406406406	3.72656905114694\\
0.815407407407407	3.73551696176026\\
0.817408408408409	3.74449693639854\\
0.819409409409409	3.75350892542999\\
0.82141041041041	3.76255287747141\\
0.823411411411412	3.77162873938956\\
0.825412412412412	3.78073645630408\\
0.827413413413413	3.7898759715883\\
0.829414414414414	3.79904722687261\\
0.831415415415415	3.80825016204685\\
0.833416416416417	3.81748471526383\\
0.835417417417417	3.82675082294347\\
0.837418418418418	3.83604841977636\\
0.839419419419419	3.84537743872889\\
0.84142042042042	3.85473781104844\\
0.843421421421422	3.86412946626733\\
0.845422422422422	3.87355233221138\\
0.847423423423423	3.88300633500323\\
0.849424424424424	3.892491399071\\
0.851425425425425	3.90200744715444\\
0.853426426426427	3.91155440031325\\
0.855427427427428	3.92113217793317\\
0.857428428428428	3.93074069773691\\
0.859429429429429	3.94037987579016\\
0.86143043043043	3.95004962651236\\
0.863431431431432	3.95974986268519\\
0.865432432432433	3.96948049546365\\
0.867433433433433	3.97924143429941\\
0.869434434434434	3.98903258729111\\
0.871435435435435	3.99885386068961\\
0.873436436436437	4.00870515924469\\
0.875437437437438	4.01858638613326\\
0.877438438438438	4.02849744297128\\
0.879439439439439	4.03843822982624\\
0.88144044044044	4.04840864522979\\
0.883441441441442	4.05840858619081\\
0.885442442442442	4.06843794820877\\
0.887443443443444	4.07849662528747\\
0.889444444444444	4.08858450994902\\
0.891445445445445	4.0987014932482\\
0.893446446446447	4.10884746478709\\
0.895447447447447	4.11902231273001\\
0.897448448448449	4.12922592381877\\
0.899449449449449	4.13945818338815\\
0.90145045045045	4.14971897538176\\
0.903451451451452	4.16000818236803\\
0.905452452452452	4.17032568555661\\
0.907453453453454	4.1806713648149\\
0.909454454454455	4.19104509868492\\
0.911455455455455	4.20144676440038\\
0.913456456456457	4.21187623790397\\
0.915457457457457	4.22233339386489\\
0.917458458458459	4.2328181056966\\
0.91945945945946	4.2433302455748\\
0.92146046046046	4.25386968445551\\
0.923461461461462	4.26443629209347\\
0.925462462462462	4.27502993706067\\
0.927463463463464	4.28565048676497\\
0.929464464464465	4.29629780746905\\
0.931465465465466	4.30697176430937\\
0.933466466466467	4.31767222131533\\
0.935467467467467	4.3283990414286\\
0.937468468468469	4.33915208652255\\
0.93946946946947	4.34993121742176\\
0.941470470470471	4.36073629392171\\
0.943471471471471	4.37156717480855\\
0.945472472472472	4.38242371787894\\
0.947473473473474	4.39330577995998\\
0.949474474474475	4.40421321692928\\
0.951475475475476	4.41514588373498\\
0.953476476476476	4.42610363441594\\
0.955477477477477	4.43708632212191\\
0.957478478478479	4.44809379913375\\
0.95947947947948	4.45912591688373\\
0.961480480480481	4.47018252597578\\
0.963481481481481	4.48126347620582\\
0.965482482482482	4.49236861658205\\
0.967483483483484	4.50349779534522\\
0.969484484484485	4.51465085998901\\
0.971485485485486	4.52582765728019\\
0.973486486486487	4.53702803327895\\
0.975487487487487	4.54825183335906\\
0.977488488488489	4.55949890222804\\
0.97948948948949	4.57076908394726\\
0.981490490490491	4.582062221952\\
0.983491491491492	4.59337815907141\\
0.985492492492492	4.60471673754842\\
0.987493493493494	4.61607779905955\\
0.989494494494495	4.6274611847346\\
0.991495495495496	4.63886673517633\\
0.993496496496497	4.65029429047992\\
0.995497497497498	4.66174369025238\\
0.997498498498499	4.67321477363185\\
0.9994994994995	4.68470737930671\\
1.0015005005005	4.6962213455346\\
1.0035015015015	4.70775651016129\\
1.0055025025025	4.7193127106394\\
1.0075035035035	4.73088978404695\\
1.0095045045045	4.74248756710574\\
1.01150550550551	4.75410589619961\\
1.01350650650651	4.76574460739246\\
1.01550750750751	4.77740353644614\\
1.01750850850851	4.78908251883813\\
1.01950950950951	4.80078138977898\\
1.02151051051051	4.81249998422968\\
1.02351151151151	4.82423813691867\\
1.02551251251251	4.83599568235875\\
1.02751351351351	4.84777245486374\\
1.02951451451451	4.85956828856489\\
1.03151551551552	4.87138301742711\\
1.03351651651652	4.88321647526497\\
1.03551751751752	4.89506849575841\\
1.03751851851852	4.90693891246825\\
1.03951951951952	4.9188275588515\\
1.04152052052052	4.93073426827632\\
1.04352152152152	4.94265887403681\\
1.04552252252252	4.9546012093675\\
1.04752352352352	4.96656110745761\\
1.04952452452452	4.97853840146504\\
1.05152552552553	4.99053292453005\\
1.05352652652653	5.00254450978873\\
1.05552752752753	5.01457299038616\\
1.05752852852853	5.0266181994893\\
1.05952952952953	5.0386799702996\\
1.06153053053053	5.05075813606532\\
1.06353153153153	5.06285253009357\\
1.06553253253253	5.07496298576206\\
1.06753353353353	5.08708933653056\\
1.06953453453453	5.09923141595204\\
1.07153553553554	5.11138905768355\\
1.07353653653654	5.1235620954968\\
1.07553753753754	5.13575036328836\\
1.07753853853854	5.1479536950897\\
1.07953953953954	5.16017192507676\\
1.08154054054054	5.17240488757935\\
1.08354154154154	5.18465241709013\\
1.08554254254254	5.19691434827338\\
1.08754354354354	5.20919051597338\\
1.08954454454454	5.22148075522247\\
1.09154554554555	5.23378490124887\\
1.09354654654655	5.24610278948412\\
1.09554754754755	5.25843425557017\\
1.09754854854855	5.27077913536625\\
1.09954954954955	5.28313726495532\\
1.10155055055055	5.29550848065022\\
1.10355155155155	5.30789261899957\\
1.10555255255255	5.32028951679319\\
1.10755355355355	5.33269901106735\\
1.10955455455455	5.3451209391096\\
1.11155555555556	5.35755513846325\\
1.11355655655656	5.37000144693162\\
1.11555755755756	5.38245970258188\\
1.11755855855856	5.39492974374853\\
1.11955955955956	5.40741140903663\\
1.12156056056056	5.41990453732463\\
1.12356156156156	5.43240896776687\\
1.12556256256256	5.44492453979579\\
1.12756356356356	5.45745109312371\\
1.12956456456456	5.46998846774437\\
1.13156556556557	5.48253650393406\\
1.13356656656657	5.49509504225241\\
1.13556756756757	5.50766392354288\\
1.13756856856857	5.52024298893284\\
1.13956956956957	5.53283207983338\\
1.14157057057057	5.54543103793868\\
1.14357157157157	5.55803970522515\\
1.14557257257257	5.57065792395004\\
1.14757357357357	5.5832855366499\\
1.14957457457457	5.59592238613854\\
1.15157557557558	5.60856831550469\\
1.15357657657658	5.62122316810929\\
1.15557757757758	5.63388678758242\\
1.15757857857858	5.64655901781987\\
1.15957957957958	5.65923970297931\\
1.16158058058058	5.6719286874762\\
1.16358158158158	5.68462581597916\\
1.16558258258258	5.69733093340514\\
1.16758358358358	5.71004388491406\\
1.16958458458458	5.72276451590322\\
1.17158558558559	5.73549267200119\\
1.17358658658659	5.74822819906144\\
1.17558758758759	5.76097094315544\\
1.17758858858859	5.77372075056555\\
1.17958958958959	5.78647746777732\\
1.18159059059059	5.79924094147151\\
1.18359159159159	5.81201101851572\\
1.18559259259259	5.82478754595551\\
1.18759359359359	5.8375703710052\\
1.18959459459459	5.85035934103817\\
1.1915955955956	5.86315430357684\\
1.1935965965966	5.87595510628214\\
1.1955975975976	5.88876159694251\\
1.1975985985986	5.90157362346261\\
1.1995995995996	5.91439103385138\\
1.2016006006006	5.92721367620984\\
1.2036016016016	5.94004139871827\\
1.2056026026026	5.952874049623\\
1.2076036036036	5.96571147722273\\
1.2096046046046	5.97855352985431\\
1.21160560560561	5.99140005587805\\
1.21360660660661	6.00425090366255\\
1.21560760760761	6.01710592156899\\
1.21760860860861	6.02996495793488\\
1.21960960960961	6.04282786105727\\
1.22161061061061	6.05569447917552\\
1.22361161161161	6.06856466045333\\
1.22561261261261	6.08143825296033\\
1.22761361361361	6.09431510465308\\
1.22961461461461	6.10719506335539\\
1.23161561561562	6.12007797673811\\
1.23361661661662	6.13296369229827\\
1.23561761761762	6.14585205733752\\
1.23761861861862	6.15874291894005\\
1.23961961961962	6.17163612394965\\
1.24162062062062	6.18453151894622\\
1.24362162162162	6.19742895022153\\
1.24562262262262	6.2103282637542\\
1.24762362362362	6.22322930518397\\
1.24962462462462	6.2361319197852\\
1.25162562562563	6.24903595243957\\
1.25362662662663	6.26194124760788\\
1.25562762762763	6.27484764930117\\
1.25762862862863	6.28775500105079\\
1.25962962962963	6.30066314587768\\
1.26163063063063	6.31357192626069\\
1.26363163163163	6.32648118410394\\
1.26563263263263	6.33939076070318\\
1.26763363363363	6.35230049671117\\
1.26963463463463	6.36521023210195\\
1.27163563563564	6.37811980613404\\
1.27363663663664	6.39102905731254\\
1.27563763763764	6.40393782334999\\
1.27763863863864	6.41684594112608\\
1.27963963963964	6.42975324664609\\
1.28164064064064	6.442659574998\\
1.28364164164164	6.45556476030829\\
1.28564264264264	6.46846863569631\\
1.28764364364364	6.48137103322725\\
1.28964464464464	6.49427178386349\\
1.29164564564565	6.50717071741452\\
1.29364664664665	6.52006766248515\\
1.29564764764765	6.53296244642204\\
1.29764864864865	6.54585489525852\\
1.29964964964965	6.55874483365746\\
1.30165065065065	6.57163208485242\\
1.30365165165165	6.58451647058664\\
1.30565265265265	6.59739781105006\\
1.30765365365365	6.61027592481419\\
1.30965465465465	6.62315062876469\\
1.31165565565566	6.63602173803165\\
1.31365665665666	6.64888906591744\\
1.31565765765766	6.66175242382194\\
1.31765865865866	6.67461162116524\\
1.31965965965966	6.68746646530739\\
1.32166066066066	6.70031676146545\\
1.32366166166166	6.71316231262734\\
1.32566266266266	6.72600291946259\\
1.32766366366366	6.73883838022978\\
1.32966466466466	6.75166849068049\\
1.33166566566567	6.76449304395958\\
1.33366666666667	6.77731183050176\\
1.33566766766767	6.79012463792402\\
1.33766866866867	6.80293125091402\\
1.33966966966967	6.81573145111398\\
1.34167067067067	6.82852501700005\\
1.34367167167167	6.84131172375675\\
1.34567267267267	6.85409134314641\\
1.34767367367367	6.86686364337323\\
1.34967467467467	6.87962838894174\\
1.35167567567568	6.89238534050934\\
1.35367667667668	6.90513425473265\\
1.35567767767768	6.91787488410733\\
1.35767867867868	6.93060697680093\\
1.35967967967968	6.94333027647863\\
1.36168068068068	6.95604452212116\\
1.36368168168168	6.96874944783477\\
1.36568268268268	6.9814447826526\\
1.36768368368368	6.99413025032712\\
1.36968468468468	7.00680556911297\\
1.37168568568569	7.01947045153984\\
1.37368668668669	7.03212460417466\\
1.37568768768769	7.04476772737254\\
1.37768868868869	7.05739951501582\\
1.37968968968969	7.07001965424042\\
1.38169069069069	7.08262782514895\\
1.38369169169169	7.09522370050938\\
1.38569269269269	7.10780694543889\\
1.38769369369369	7.12037721707146\\
1.38969469469469	7.13293416420862\\
1.3916956956957	7.14547742695196\\
1.3936966966967	7.15800663631642\\
1.3956976976977	7.17052141382302\\
1.3976986986987	7.18302137106969\\
1.3996996996997	7.19550610927879\\
1.4017007007007	7.20797521881965\\
1.4037017017017	7.22042827870449\\
1.4057027027027	7.23286485605589\\
1.4077037037037	7.24528450554378\\
1.4097047047047	7.25768676878977\\
1.41170570570571	7.27007117373655\\
1.41370670670671	7.28243723397975\\
1.41570770770771	7.29478444805955\\
1.41770870870871	7.30711229870892\\
1.41970970970971	7.31942025205543\\
1.42171071071071	7.33170775677278\\
1.42371171171171	7.34397424317841\\
1.42571271271271	7.35621912227275\\
1.42771371371371	7.36844178471554\\
1.42971471471471	7.38064159973405\\
1.43171571571572	7.39281791395773\\
1.43371671671672	7.40497005017288\\
1.43571771771772	7.41709730599094\\
1.43771871871872	7.42919895242253\\
1.43971971971972	7.44127423234937\\
1.44172072072072	7.45332235888482\\
1.44372172172172	7.46534251361298\\
1.44572272272272	7.47733384469535\\
1.44772372372372	7.48929546483251\\
1.44972472472472	7.50122644906729\\
1.45172572572573	7.51312583241396\\
1.45372672672673	7.52499260729645\\
1.45572772772773	7.53682572077655\\
1.45772872872873	7.54862407155068\\
1.45972972972973	7.5603865066913\\
1.46173073073073	7.57211181810602\\
1.46373173173173	7.583798738684\\
1.46573273273273	7.59544593809546\\
1.46773373373373	7.60705201820537\\
1.46973473473473	7.61861550805742\\
1.47173573573574	7.63013485837816\\
1.47373673673674	7.64160843554424\\
1.47573773773774	7.6530345149474\\
1.47773873873874	7.66441127368265\\
1.47973973973974	7.67573678247325\\
1.48174074074074	7.68700899673365\\
1.48374174174174	7.69822574665529\\
1.48574274274274	7.70938472618223\\
1.48774374374374	7.72048348072135\\
1.48974474474474	7.7315193934056\\
1.49174574574575	7.74248966969749\\
1.49374674674675	7.75339132008184\\
1.49574774774775	7.76422114055095\\
1.49774874874875	7.77497569052939\\
1.49974974974975	7.7856512678171\\
1.50175075075075	7.7962438800453\\
1.50375175175175	7.80674921203525\\
1.50575275275275	7.8171625883203\\
1.50775375375375	7.8274789299286\\
1.50975475475475	7.8376927043183\\
1.51175575575576	7.84779786709556\\
1.51375675675676	7.85778779381037\\
1.51575775775776	7.86765519969158\\
1.51775875875876	7.8773920446163\\
1.51975975975976	7.88698941986351\\
1.52176076076076	7.89643741220756\\
1.52376176176176	7.90572493956935\\
1.52576276276276	7.91483955061686\\
1.52776376376376	7.92376717818251\\
1.52976476476476	7.93249183282325\\
1.53176576576577	7.94099521779888\\
1.53376676676677	7.9492562394138\\
1.53576776776777	7.9572503758159\\
1.53776876876877	7.96494885092521\\
1.53976976976977	7.97231753469822\\
1.54177077077077	7.97931545032478\\
1.54377177177177	7.98589270212008\\
1.54577277277277	7.99198752380137\\
1.54777377377377	7.99752194372687\\
1.54977477477477	8.00239518351986\\
1.55177577577578	8.00647315112991\\
1.55377677677678	8.00957077445189\\
1.55577777777778	8.0114201361813\\
1.55777877877878	8.01160736658756\\
1.55977977977978	8.00943007774844\\
1.56178078078078	8.0035028594365\\
1.56378178178178	7.99017225784444\\
1.56578278278278	7.93481605863578\\
1.56378178178178	7.84179159028189\\
1.56178078078078	7.7906169528163\\
1.55977977977978	7.7467588227964\\
1.55777877877878	7.7065621982891\\
1.55577777777778	7.66864006935677\\
1.55377677677678	7.63228840402724\\
1.55177577577578	7.59709163807264\\
1.54977477477477	7.56278010777926\\
1.54777377377377	7.52916694054577\\
1.54577277277277	7.49611618739056\\
1.54377177177177	7.46352515408021\\
1.54177077077077	7.4313138915607\\
1.53976976976977	7.39941859179885\\
1.53776876876877	7.36778725007328\\
1.53576776776777	7.33637671012997\\
1.53376676676677	7.30515058876433\\
1.53176576576577	7.27407777950526\\
1.52976476476476	7.24313134915017\\
1.52776376376376	7.21228770772915\\
1.52576276276276	7.18152597308806\\
1.52376176176176	7.15082747674019\\
1.52176076076076	7.12017537406287\\
1.51975975975976	7.08955433275548\\
1.51775875875876	7.05895028081184\\
1.51575775775776	7.02835020030213\\
1.51375675675676	6.99774195680045\\
1.51175575575576	6.96711415681595\\
1.50975475475475	6.93645602740391\\
1.50775375375375	6.90575731347355\\
1.50575275275275	6.87500818929203\\
1.50375175175175	6.84419918142941\\
1.50175075075075	6.81332110094693\\
1.49974974974975	6.78236498305916\\
1.49774874874875	6.75132203282837\\
1.49574774774775	6.72018357570245\\
1.49374674674675	6.68894101190372\\
1.49174574574575	6.65758577382788\\
1.48974474474474	6.62610928573009\\
1.48774374374374	6.59450292506481\\
1.48574274274274	6.56275798491451\\
1.48374174174174	6.53086563699094\\
1.48174074074074	6.4988168947283\\
1.47973973973974	6.46660257600737\\
1.47773873873874	6.43421326505897\\
1.47573773773774	6.40163927309244\\
1.47373673673674	6.36887059718187\\
1.47173573573574	6.33589687691755\\
1.46973473473473	6.3027073482948\\
1.46773373373373	6.26929079426306\\
1.46573273273273	6.23563549129507\\
1.46373173173173	6.20172915125676\\
1.46173073073073	6.16755885775953\\
1.45972972972973	6.13311099605535\\
1.45772872872873	6.09837117538572\\
1.45572772772773	6.06332414251249\\
1.45372672672673	6.02795368493463\\
1.45172572572573	5.99224252201752\\
1.44972472472472	5.95617218192097\\
1.44772372372372	5.91972286178804\\
1.44572272272272	5.88287326812946\\
1.44372172172172	5.84560043367468\\
1.44172072072072	5.80787950612475\\
1.43971971971972	5.76968350317959\\
1.43771871871872	5.73098302684698\\
1.43571771771772	5.69174592828149\\
1.43371671671672	5.6519369121049\\
1.43171571571572	5.61151706613595\\
1.42971471471471	5.57044329843909\\
1.42771371371371	5.52866765818768\\
1.42571271271271	5.48613650947284\\
1.42371171171171	5.4427895170125\\
1.42171071071071	5.39855838846691\\
1.41970970970971	5.35336529778076\\
1.41770870870871	5.30712088458297\\
1.41570770770771	5.25972168124772\\
1.41370670670671	5.21104675364575\\
1.41170570570571	5.1609532401404\\
1.4097047047047	5.10927031196487\\
1.4077037037037	5.05579081315952\\
1.4057027027027	5.00025938736344\\
1.4037017017017	4.94235509856113\\
1.4017007007007	4.88166506092591\\
1.3996996996997	4.81764264024293\\
1.3976986986987	4.7495375191897\\
1.3956976976977	4.67627028300945\\
1.3936966966967	4.59618589955958\\
1.3916956956957	4.50650243630702\\
1.38969469469469	4.4018092068894\\
1.38769369369369	4.26821555548623\\
1.38569269269269	3.98981874689335\\
1.38769369369369	3.65601647729803\\
1.38969469469469	3.53881663847835\\
1.3916956956957	3.45053652392801\\
1.3936966966967	3.37728575185072\\
1.3956976976977	3.31365392034329\\
1.3976986986987	3.25685939218749\\
1.3996996996997	3.20524744185646\\
1.4017007007007	3.15773897353061\\
1.4037017017017	3.11358399915878\\
1.4057027027027	3.0722362289502\\
1.4077037037037	3.0332831338002\\
1.4097047047047	2.99640414839136\\
1.41170570570571	2.96134430163844\\
1.41370670670671	2.92789683733943\\
1.41570770770771	2.89589134482168\\
1.41770870870871	2.86518539459804\\
1.41970970970971	2.83565850311439\\
1.42171071071071	2.80720767161568\\
1.42371171171171	2.77974402779403\\
1.42571271271271	2.7531902538042\\
1.42771371371371	2.72747858668349\\
1.42971471471471	2.70254924277219\\
1.43171571571572	2.67834916119695\\
1.43371671671672	2.65483099080973\\
1.43571771771772	2.63195226532727\\
1.43771871871872	2.60967472561112\\
1.43971971971972	2.58796375823898\\
1.44172072072072	2.56678792686607\\
1.44372172172172	2.54611857829928\\
1.44572272272272	2.5259295092262\\
1.44772372372372	2.50619668255131\\
1.44972472472472	2.48689798461619\\
1.45172572572573	2.46801301631671\\
1.45372672672673	2.4495229125076\\
1.45572772772773	2.43141018515472\\
1.45772872872873	2.413658586518\\
1.45972972972973	2.39625298933372\\
1.46173073073073	2.3791792814777\\
1.46373173173173	2.36242427302874\\
1.46573273273273	2.34597561399464\\
1.46773373373373	2.32982172124022\\
1.46973473473473	2.31395171339321\\
1.47173573573574	2.29835535268586\\
1.47373673673674	2.28302299285086\\
1.47573773773774	2.26794553232108\\
1.47773873873874	2.25311437208248\\
1.47973973973974	2.23852137763326\\
1.48174074074074	2.22415884456602\\
1.48374174174174	2.21001946736226\\
1.48574274274274	2.1960963110394\\
1.48774374374374	2.18238278533833\\
1.48974474474474	2.1688726211783\\
1.49174574574575	2.15555984914065\\
1.49374674674675	2.14243877977223\\
1.49574774774775	2.129503985523\\
1.49774874874875	2.11675028415514\\
1.49974974974975	2.10417272348081\\
1.50175075075075	2.09176656729933\\
1.50375175175175	2.07952728242237\\
1.50575275275275	2.06745052668354\\
1.50775375375375	2.05553213784479\\
1.50975475475475	2.04376812331859\\
1.51175575575576	2.03215465063236\\
1.51375675675676	2.02068803857219\\
1.51575775775776	2.00936474894702\\
1.51775875875876	1.99818137892097\\
1.51975975975976	1.98713465386691\\
1.52176076076076	1.97622142069782\\
1.52376176176176	1.96543864163844\\
1.52576276276276	1.954783388402\\
1.52776376376376	1.94425283673929\\
1.52976476476476	1.93384426133282\\
1.53176576576577	1.92355503100902\\
1.53376676676677	1.91338260424383\\
1.53576776776777	1.90332452494217\\
1.53776876876877	1.89337841846817\\
1.53976976976977	1.88354198791136\\
1.54177077077077	1.87381301056899\\
1.54377177177177	1.86418933463185\\
1.54577277277277	1.85466887605736\\
1.54777377377377	1.84524961561856\\
1.54977477477477	1.83592959611641\\
1.55177577577578	1.82670691974441\\
1.55377677677678	1.81757974559637\\
1.55577777777778	1.80854628730608\\
1.55777877877878	1.79960481081288\\
1.55977977977978	1.79075363224288\\
1.56178078078078	1.7819911158998\\
1.56378178178178	1.77331567235831\\
1.56578278278278	1.76472575665317\\
1.56778378378378	1.7562198665591\\
1.56978478478478	1.74779654095505\\
1.57178578578579	1.73945435826846\\
1.57378678678679	1.73119193499455\\
1.57578778778779	1.72300792428601\\
1.57778878878879	1.71490101460987\\
1.57978978978979	1.70686992846649\\
1.58179079079079	1.6989134211684\\
1.58379179179179	1.69103027967434\\
1.58579279279279	1.683219321477\\
1.58779379379379	1.67547939353975\\
1.58979479479479	1.66780937128137\\
1.5917957957958	1.66020815760493\\
1.5937967967968	1.65267468196919\\
1.5957977977978	1.64520789949996\\
1.5977987987988	1.6378067901395\\
1.5997997997998	1.63047035783182\\
1.6018008008008	1.62319762974223\\
1.6038018018018	1.61598765550916\\
1.6058028028028	1.60883950652705\\
1.6078038038038	1.60175227525805\\
1.6098048048048	1.59472507457196\\
1.61180580580581	1.58775703711222\\
1.61380680680681	1.58084731468751\\
1.61580780780781	1.57399507768668\\
1.61780880880881	1.56719951451716\\
1.61980980980981	1.56045983106447\\
1.62181081081081	1.55377525017299\\
1.62381181181181	1.54714501114595\\
1.62581281281281	1.54056836926465\\
1.62781381381381	1.53404459532546\\
1.62981481481481	1.52757297519397\\
1.63181581581582	1.52115280937559\\
1.63381681681682	1.51478341260178\\
1.63581781781782	1.50846411343121\\
1.63781881881882	1.50219425386522\\
1.63981981981982	1.49597318897697\\
1.64182082082082	1.48980028655377\\
1.64382182182182	1.48367492675175\\
1.64582282282282	1.4775965017626\\
1.64782382382382	1.471564415492\\
1.64982482482482	1.46557808324857\\
1.65182582582583	1.45963693144403\\
1.65382682682683	1.45374039730281\\
1.65582782782783	1.44788792858182\\
1.65782882882883	1.4420789832993\\
1.65982982982983	1.4363130294728\\
1.66183083083083	1.43058954486532\\
1.66383183183183	1.42490801674034\\
1.66583283283283	1.41926794162402\\
1.66783383383383	1.41366882507555\\
1.66983483483483	1.40811018146458\\
1.67183583583584	1.40259153375559\\
1.67383683683684	1.39711241329927\\
1.67583783783784	1.39167235963005\\
1.67783883883884	1.38627092027022\\
1.67983983983984	1.38090765053976\\
1.68184084084084	1.37558211337223\\
1.68384184184184	1.37029387913593\\
1.68584284284284	1.3650425254608\\
1.68784384384384	1.3598276370701\\
1.68984484484484	1.35464880561739\\
1.69184584584585	1.34950562952811\\
1.69384684684685	1.34439771384588\\
1.69584784784785	1.33932467008329\\
1.69784884884885	1.33428611607692\\
1.69984984984985	1.32928167584657\\
1.70185085085085	1.32431097945849\\
1.70385185185185	1.31937366289247\\
1.70585285285285	1.31446936791271\\
1.70785385385385	1.30959774194222\\
1.70985485485485	1.30475843794085\\
1.71185585585586	1.29995111428651\\
1.71385685685686	1.29517543465973\\
1.71585785785786	1.29043106793149\\
1.71785885885886	1.28571768805385\\
1.71985985985986	1.2810349739537\\
1.72186086086086	1.27638260942931\\
1.72386186186186	1.2717602830496\\
1.72586286286286	1.26716768805616\\
1.72786386386386	1.26260452226772\\
1.72986486486486	1.25807048798725\\
1.73186586586587	1.25356529191134\\
1.73386686686687	1.24908864504205\\
1.73586786786787	1.24464026260095\\
1.73786886886887	1.24021986394534\\
1.73986986986987	1.23582717248667\\
1.74187087087087	1.23146191561096\\
1.74387187187187	1.22712382460122\\
1.74587287287287	1.22281263456185\\
1.74787387387387	1.21852808434485\\
1.74987487487487	1.21426991647792\\
1.75187587587588	1.21003787709428\\
1.75387687687688	1.20583171586422\\
1.75587787787788	1.20165118592828\\
1.75787887887888	1.19749604383212\\
1.75987987987988	1.19336604946287\\
1.76188088088088	1.18926096598702\\
1.76388188188188	1.18518055978984\\
1.76588288288288	1.18112460041616\\
1.76788388388388	1.17709286051257\\
1.76988488488488	1.17308511577103\\
1.77188588588589	1.1691011448736\\
1.77388688688689	1.16514072943876\\
1.77588788788789	1.16120365396864\\
1.77788888888889	1.15728970579765\\
1.77988988988989	1.15339867504229\\
1.78189089089089	1.14953035455197\\
1.78389189189189	1.14568453986101\\
1.78589289289289	1.14186102914181\\
1.78789389389389	1.13805962315883\\
1.78989489489489	1.13428012522387\\
1.7918958958959	1.13052234115207\\
1.7938968968969	1.12678607921915\\
1.7958978978979	1.1230711501193\\
1.7978988988989	1.11937736692424\\
1.7998998998999	1.115704545043\\
1.8019009009009	1.11205250218262\\
1.8039019019019	1.10842105830972\\
1.8059029029029	1.10481003561288\\
1.8079039039039	1.10121925846574\\
1.8099049049049	1.09764855339101\\
1.81190590590591	1.09409774902513\\
1.81390690690691	1.09056667608373\\
1.81590790790791	1.0870551673278\\
1.81790890890891	1.08356305753052\\
1.81990990990991	1.08009018344486\\
1.82191091091091	1.07663638377171\\
1.82391191191191	1.07320149912884\\
1.82591291291291	1.06978537202033\\
1.82791391391391	1.06638784680673\\
1.82991491491491	1.06300876967572\\
1.83191591591592	1.0596479886135\\
1.83391691691692	1.05630535337658\\
1.83591791791792	1.05298071546431\\
1.83791891891892	1.04967392809178\\
1.83991991991992	1.04638484616341\\
1.84192092092092	1.04311332624693\\
1.84392192192192	1.03985922654798\\
1.84592292292292	1.03662240688511\\
1.84792392392392	1.03340272866534\\
1.84992492492492	1.03020005486012\\
1.85192592592593	1.02701424998181\\
1.85392692692693	1.0238451800606\\
1.85592792792793	1.02069271262183\\
1.85792892892893	1.01755671666381\\
1.85992992992993	1.01443706263595\\
1.86193093093093	1.01133362241745\\
1.86393193193193	1.00824626929625\\
1.86593293293293	1.00517487794846\\
1.86793393393393	1.0021193244181\\
1.86993493493493	0.999079486097336\\
1.87193593593594	0.99605524170691\\
1.87393693693694	0.993046471277089\\
1.87593793793794	0.990053056128838\\
1.87793893893894	0.987074878855448\\
1.87993993993994	0.984111823304392\\
1.88194094094094	0.981163774559567\\
1.88394194194194	0.978230618923857\\
1.88594294294294	0.975312243901975\\
1.88794394394394	0.972408538183651\\
1.88994494494494	0.969519391627072\\
1.89194594594595	0.966644695242669\\
1.89394694694695	0.963784341177131\\
1.89594794794795	0.960938222697756\\
1.89794894894895	0.958106234177033\\
1.89994994994995	0.955288271077506\\
1.90195095095095	0.952484229936916\\
1.90395195195195	0.949694008353585\\
1.90595295295295	0.946917504972026\\
1.90795395395395	0.944154619468865\\
1.90995495495495	0.941405252538943\\
1.91195595595596	0.938669305881676\\
1.91395695695696	0.93594668218765\\
1.91595795795796	0.933237285125439\\
1.91795895895896	0.930541019328647\\
1.91995995995996	0.927857790383156\\
1.92196096096096	0.925187504814601\\
1.92396196196196	0.922530070076051\\
1.92596296296296	0.91988539453588\\
1.92796396396396	0.917253387465853\\
1.92996496496496	0.914633959029418\\
1.93196596596597	0.912027020270146\\
1.93396696696697	0.90943248310042\\
1.93596796796797	0.906850260290258\\
1.93796896896897	0.904280265456345\\
1.93996996996997	0.901722413051227\\
1.94197097097097	0.899176618352692\\
1.94397197197197	0.896642797453313\\
1.94597297297297	0.894120867250159\\
1.94797397397397	0.891610745434673\\
1.94997497497497	0.889112350482706\\
1.95197597597598	0.886625601644712\\
1.95397697697698	0.884150418936097\\
1.95597797797798	0.881686723127719\\
1.95797897897898	0.879234435736532\\
1.95997997997998	0.876793479016395\\
1.96198098098098	0.874363775948989\\
1.96398198198198	0.871945250234913\\
1.96598298298298	0.86953782628489\\
1.96798398398398	0.867141429211119\\
1.96998498498498	0.86475598481876\\
1.97198598598599	0.862381419597546\\
1.97398698698699	0.860017660713532\\
1.97598798798799	0.857664636000947\\
1.97798898898899	0.855322273954191\\
1.97998998998999	0.852990503719952\\
1.98199099099099	0.850669255089422\\
1.98399199199199	0.848358458490655\\
1.98599299299299	0.846058044981013\\
1.98799399399399	0.84376794623975\\
1.98999499499499	0.841488094560689\\
1.991995995996	0.839218422845007\\
1.993996996997	0.836958864594142\\
1.995997997998	0.834709353902781\\
1.997998998999	0.832469825451977\\
2	0.830240214502336\\
};
\addlegendentry{$\alpha\text{ = 0.03}$};

\end{axis}
\end{tikzpicture}%
	% This file was created by matlab2tikz.
% Minimal pgfplots version: 1.3
%
%The latest updates can be retrieved from
%  http://www.mathworks.com/matlabcentral/fileexchange/22022-matlab2tikz
%where you can also make suggestions and rate matlab2tikz.
%
\begin{tikzpicture}

\begin{axis}[%
width=0.95092\figurewidth,
height=\figureheight,
at={(0\figurewidth,0\figureheight)},
scale only axis,
every outer x axis line/.append style={black},
every x tick label/.append style={font=\color{black}},
xmin=0,
xmax=2,
xlabel={$\omega\text{/}\omega{}_\text{0}$},
every outer y axis line/.append style={black},
every y tick label/.append style={font=\color{black}},
ymin=0,
ymax=3.14159265358979,
ytick={0,1.5707963267949,3.14159265358979},
yticklabels={{-pi},{-pi/2},{0}},
ylabel={$\phi$},
axis x line*=bottom,
axis y line*=left,
legend style={legend cell align=left,align=left,draw=black}
]
\addplot [color=black,solid]
  table[row sep=crcr]{%
0.001	3.14139265339246\\
0.003001001001001	3.14099244805621\\
0.005002002002002	3.14059222849248\\
0.007003003003003	3.14019198521348\\
0.009004004004004	3.13979170872954\\
0.011005005005005	3.13939138954844\\
0.013006006006006	3.1389910181746\\
0.015007007007007	3.1385905851084\\
0.017008008008008	3.13819008084538\\
0.019009009009009	3.13778949587555\\
0.02101001001001	3.1373888206826\\
0.023011011011011	3.1369880457432\\
0.025012012012012	3.13658716152621\\
0.027013013013013	3.13618615849199\\
0.029014014014014	3.13578502709159\\
0.031015015015015	3.13538375776605\\
0.033016016016016	3.1349823409456\\
0.035017017017017	3.13458076704896\\
0.037018018018018	3.13417902648256\\
0.039019019019019	3.13377710963977\\
0.04102002002002	3.13337500690017\\
0.043021021021021	3.13297270862875\\
0.045022022022022	3.13257020517521\\
0.047023023023023	3.13216748687311\\
0.049024024024024	3.13176454403918\\
0.051025025025025	3.13136136697249\\
0.053026026026026	3.13095794595371\\
0.055027027027027	3.13055427124433\\
0.057028028028028	3.13015033308587\\
0.059029029029029	3.12974612169909\\
0.06103003003003	3.12934162728322\\
0.063031031031031	3.12893684001516\\
0.065032032032032	3.12853175004868\\
0.067033033033033	3.12812634751363\\
0.069034034034034	3.12772062251516\\
0.071035035035035	3.12731456513286\\
0.073036036036036	3.12690816542\\
0.075037037037037	3.12650141340269\\
0.077038038038038	3.12609429907907\\
0.079039039039039	3.12568681241848\\
0.08104004004004	3.12527894336064\\
0.083041041041041	3.12487068181481\\
0.085042042042042	3.12446201765895\\
0.0870430430430431	3.12405294073888\\
0.089044044044044	3.12364344086745\\
0.091045045045045	3.12323350782364\\
0.093046046046046	3.12282313135173\\
0.0950470470470471	3.12241230116045\\
0.0970480480480481	3.12200100692207\\
0.0990490490490491	3.12158923827155\\
0.10105005005005	3.12117698480564\\
0.103051051051051	3.12076423608201\\
0.105052052052052	3.12035098161831\\
0.107053053053053	3.11993721089132\\
0.109054054054054	3.119522913336\\
0.111055055055055	3.1191080783446\\
0.113056056056056	3.11869269526569\\
0.115057057057057	3.11827675340329\\
0.117058058058058	3.11786024201588\\
0.119059059059059	3.11744315031547\\
0.12106006006006	3.11702546746664\\
0.123061061061061	3.11660718258558\\
0.125062062062062	3.11618828473911\\
0.127063063063063	3.11576876294371\\
0.129064064064064	3.11534860616453\\
0.131065065065065	3.11492780331438\\
0.133066066066066	3.11450634325271\\
0.135067067067067	3.11408421478463\\
0.137068068068068	3.11366140665986\\
0.139069069069069	3.1132379075717\\
0.14107007007007	3.11281370615597\\
0.143071071071071	3.11238879098998\\
0.145072072072072	3.11196315059144\\
0.147073073073073	3.11153677341738\\
0.149074074074074	3.11110964786309\\
0.151075075075075	3.110681762261\\
0.153076076076076	3.11025310487958\\
0.155077077077077	3.10982366392219\\
0.157078078078078	3.109393427526\\
0.159079079079079	3.10896238376079\\
0.16108008008008	3.10853052062784\\
0.163081081081081	3.10809782605872\\
0.165082082082082	3.10766428791414\\
0.167083083083083	3.10722989398274\\
0.169084084084084	3.10679463197987\\
0.171085085085085	3.10635848954641\\
0.173086086086086	3.10592145424747\\
0.175087087087087	3.10548351357122\\
0.177088088088088	3.10504465492755\\
0.179089089089089	3.10460486564686\\
0.18109009009009	3.10416413297872\\
0.183091091091091	3.10372244409058\\
0.185092092092092	3.10327978606645\\
0.187093093093093	3.10283614590556\\
0.189094094094094	3.102391510521\\
0.191095095095095	3.10194586673837\\
0.193096096096096	3.10149920129437\\
0.195097097097097	3.1010515008354\\
0.197098098098098	3.10060275191616\\
0.199099099099099	3.10015294099818\\
0.2011001001001	3.0997020544484\\
0.203101101101101	3.09925007853768\\
0.205102102102102	3.09879699943928\\
0.207103103103103	3.0983428032274\\
0.209104104104104	3.09788747587563\\
0.211105105105105	3.09743100325537\\
0.213106106106106	3.09697337113431\\
0.215107107107107	3.09651456517481\\
0.217108108108108	3.09605457093228\\
0.219109109109109	3.09559337385358\\
0.22111011011011	3.09513095927536\\
0.223111111111111	3.09466731242235\\
0.225112112112112	3.09420241840572\\
0.227113113113113	3.0937362622213\\
0.229114114114114	3.09326882874789\\
0.231115115115115	3.09280010274546\\
0.233116116116116	3.09233006885336\\
0.235117117117117	3.09185871158854\\
0.237118118118118	3.09138601534365\\
0.239119119119119	3.09091196438526\\
0.24112012012012	3.09043654285187\\
0.243121121121121	3.08995973475206\\
0.245122122122122	3.08948152396255\\
0.247123123123123	3.08900189422616\\
0.249124124124124	3.0885208291499\\
0.251125125125125	3.08803831220287\\
0.253126126126126	3.08755432671426\\
0.255127127127127	3.08706885587121\\
0.257128128128128	3.08658188271673\\
0.259129129129129	3.08609339014755\\
0.26113013013013	3.08560336091192\\
0.263131131131131	3.08511177760744\\
0.265132132132132	3.08461862267877\\
0.267133133133133	3.08412387841538\\
0.269134134134134	3.08362752694926\\
0.271135135135135	3.08312955025254\\
0.273136136136136	3.08262993013512\\
0.275137137137137	3.0821286482423\\
0.277138138138138	3.08162568605225\\
0.279139139139139	3.08112102487361\\
0.28114014014014	3.08061464584289\\
0.283141141141141	3.08010652992196\\
0.285142142142142	3.07959665789542\\
0.287143143143143	3.07908501036798\\
0.289144144144144	3.07857156776174\\
0.291145145145145	3.07805631031351\\
0.293146146146146	3.07753921807202\\
0.295147147147147	3.0770202708951\\
0.297148148148148	3.07649944844685\\
0.299149149149149	3.07597673019474\\
0.30115015015015	3.07545209540664\\
0.303151151151151	3.07492552314785\\
0.305152152152152	3.07439699227806\\
0.307153153153153	3.07386648144826\\
0.309154154154154	3.07333396909761\\
0.311155155155155	3.07279943345026\\
0.313156156156156	3.07226285251206\\
0.315157157157157	3.07172420406735\\
0.317158158158158	3.07118346567557\\
0.319159159159159	3.07064061466785\\
0.32116016016016	3.0700956281436\\
0.323161161161161	3.06954848296696\\
0.325162162162162	3.06899915576324\\
0.327163163163163	3.06844762291532\\
0.329164164164164	3.06789386055994\\
0.331165165165165	3.06733784458392\\
0.333166166166166	3.06677955062043\\
0.335167167167167	3.06621895404504\\
0.337168168168168	3.06565602997181\\
0.339169169169169	3.06509075324928\\
0.34117017017017	3.06452309845639\\
0.343171171171171	3.06395303989836\\
0.345172172172172	3.06338055160244\\
0.347173173173173	3.06280560731364\\
0.349174174174174	3.06222818049038\\
0.351175175175175	3.06164824430004\\
0.353176176176176	3.06106577161445\\
0.355177177177177	3.06048073500529\\
0.357178178178178	3.05989310673946\\
0.359179179179179	3.05930285877428\\
0.36118018018018	3.05870996275267\\
0.363181181181181	3.05811438999826\\
0.365182182182182	3.05751611151033\\
0.367183183183183	3.05691509795877\\
0.369184184184184	3.05631131967884\\
0.371185185185185	3.05570474666596\\
0.373186186186186	3.05509534857025\\
0.375187187187187	3.05448309469115\\
0.377188188188188	3.0538679539718\\
0.379189189189189	3.05324989499336\\
0.38119019019019	3.05262888596931\\
0.383191191191191	3.05200489473951\\
0.385192192192192	3.05137788876422\\
0.387193193193193	3.05074783511807\\
0.389194194194194	3.05011470048379\\
0.391195195195195	3.04947845114593\\
0.393196196196196	3.04883905298441\\
0.395197197197197	3.04819647146797\\
0.397198198198198	3.04755067164752\\
0.399199199199199	3.04690161814931\\
0.4012002002002	3.04624927516801\\
0.403201201201201	3.04559360645968\\
0.405202202202202	3.04493457533455\\
0.407203203203203	3.04427214464971\\
0.409204204204204	3.04360627680166\\
0.411205205205205	3.04293693371868\\
0.413206206206206	3.04226407685311\\
0.415207207207207	3.04158766717342\\
0.417208208208208	3.04090766515619\\
0.419209209209209	3.04022403077789\\
0.42121021021021	3.03953672350654\\
0.423211211211211	3.03884570229314\\
0.425212212212212	3.03815092556304\\
0.427213213213213	3.03745235120704\\
0.429214214214214	3.03674993657237\\
0.431215215215215	3.03604363845347\\
0.433216216216216	3.03533341308264\\
0.435217217217217	3.0346192161204\\
0.437218218218218	3.03390100264579\\
0.439219219219219	3.03317872714637\\
0.44122022022022	3.03245234350805\\
0.443221221221221	3.03172180500479\\
0.445222222222222	3.03098706428798\\
0.447223223223223	3.03024807337564\\
0.449224224224224	3.0295047836415\\
0.451225225225225	3.02875714580369\\
0.453226226226226	3.02800510991336\\
0.455227227227227	3.02724862534295\\
0.457228228228228	3.02648764077428\\
0.459229229229229	3.02572210418639\\
0.46123023023023	3.02495196284313\\
0.463231231231231	3.02417716328045\\
0.465232232232232	3.02339765129351\\
0.467233233233233	3.02261337192344\\
0.469234234234234	3.02182426944387\\
0.471235235235235	3.02103028734718\\
0.473236236236236	3.02023136833042\\
0.475237237237237	3.01942745428098\\
0.477238238238238	3.01861848626195\\
0.479239239239239	3.01780440449712\\
0.48124024024024	3.01698514835573\\
0.483241241241241	3.01616065633689\\
0.485242242242242	3.01533086605356\\
0.487243243243243	3.01449571421635\\
0.489244244244244	3.01365513661684\\
0.491245245245245	3.01280906811059\\
0.493246246246246	3.01195744259982\\
0.495247247247247	3.0111001930156\\
0.497248248248248	3.01023725129979\\
0.499249249249249	3.00936854838647\\
0.50125025025025	3.00849401418305\\
0.503251251251251	3.00761357755089\\
0.505252252252252	3.00672716628553\\
0.507253253253253	3.00583470709649\\
0.509254254254254	3.00493612558659\\
0.511255255255255	3.00403134623082\\
0.513256256256256	3.00312029235477\\
0.515257257257257	3.0022028861125\\
0.517258258258258	3.00127904846399\\
0.519259259259259	3.00034869915202\\
0.52126026026026	2.99941175667858\\
0.523261261261261	2.99846813828068\\
0.525262262262262	2.99751775990564\\
0.527263263263263	2.99656053618585\\
0.529264264264264	2.99559638041285\\
0.531265265265265	2.99462520451094\\
0.533266266266266	2.99364691901003\\
0.535267267267267	2.99266143301801\\
0.537268268268268	2.99166865419239\\
0.539269269269269	2.99066848871123\\
0.54127027027027	2.98966084124354\\
0.543271271271271	2.98864561491882\\
0.545272272272272	2.98762271129596\\
0.547273273273273	2.98659203033142\\
0.549274274274274	2.98555347034661\\
0.551275275275275	2.98450692799447\\
0.553276276276276	2.98345229822535\\
0.555277277277277	2.98238947425195\\
0.557278278278278	2.98131834751346\\
0.559279279279279	2.9802388076389\\
0.56128028028028	2.97915074240943\\
0.563281281281281	2.97805403771988\\
0.565282282282282	2.97694857753924\\
0.567283283283283	2.97583424387024\\
0.569284284284284	2.9747109167079\\
0.571285285285285	2.97357847399707\\
0.573286286286286	2.97243679158891\\
0.575287287287287	2.97128574319631\\
0.577288288288288	2.97012520034817\\
0.579289289289289	2.96895503234253\\
0.58129029029029	2.96777510619856\\
0.583291291291291	2.96658528660731\\
0.585292292292292	2.9653854358812\\
0.587293293293293	2.96417541390228\\
0.589294294294294	2.96295507806911\\
0.591295295295295	2.96172428324233\\
0.593296296296296	2.9604828816888\\
0.595297297297297	2.95923072302434\\
0.597298298298298	2.95796765415494\\
0.599299299299299	2.9566935192165\\
0.6013003003003	2.95540815951297\\
0.603301301301301	2.95411141345289\\
0.605302302302302	2.95280311648424\\
0.607303303303303	2.95148310102767\\
0.609304304304304	2.95015119640783\\
0.611305305305305	2.94880722878299\\
0.613306306306306	2.94745102107277\\
0.615307307307307	2.94608239288396\\
0.617308308308308	2.94470116043428\\
0.619309309309309	2.94330713647421\\
0.62131031031031	2.94190013020664\\
0.623311311311311	2.9404799472044\\
0.625312312312312	2.93904638932552\\
0.627313313313313	2.9375992546262\\
0.629314314314314	2.93613833727147\\
0.631315315315315	2.93466342744328\\
0.633316316316316	2.93317431124621\\
0.635317317317317	2.9316707706105\\
0.637318318318318	2.93015258319245\\
0.639319319319319	2.92861952227203\\
0.64132032032032	2.92707135664763\\
0.643321321321321	2.925507850528\\
0.645322322322322	2.92392876342098\\
0.647323323323323	2.92233385001931\\
0.649324324324324	2.92072286008304\\
0.651325325325325	2.91909553831881\\
0.653326326326326	2.91745162425554\\
0.655327327327327	2.91579085211677\\
0.657328328328328	2.91411295068919\\
0.659329329329329	2.91241764318755\\
0.66133033033033	2.91070464711566\\
0.663331331331331	2.90897367412339\\
0.665332332332332	2.90722442985953\\
0.667333333333333	2.90545661382049\\
0.669334334334334	2.9036699191945\\
0.671335335335335	2.90186403270143\\
0.673336336336336	2.90003863442777\\
0.675337337337337	2.89819339765695\\
0.677338338338338	2.89632798869463\\
0.679339339339339	2.89444206668885\\
0.68134034034034	2.89253528344488\\
0.683341341341341	2.89060728323469\\
0.685342342342342	2.88865770260069\\
0.687343343343343	2.8866861701537\\
0.689344344344344	2.88469230636491\\
0.691345345345345	2.88267572335164\\
0.693346346346346	2.88063602465675\\
0.695347347347347	2.87857280502136\\
0.697348348348348	2.87648565015086\\
0.699349349349349	2.87437413647381\\
0.70135035035035	2.87223783089365\\
0.703351351351351	2.87007629053291\\
0.705352352352352	2.86788906246968\\
0.707353353353353	2.86567568346606\\
0.709354354354354	2.86343567968855\\
0.711355355355355	2.86116856641974\\
0.713356356356356	2.85887384776141\\
0.715357357357357	2.85655101632854\\
0.717358358358358	2.854199552934\\
0.719359359359359	2.85181892626362\\
0.72136036036036	2.84940859254138\\
0.723361361361361	2.84696799518432\\
0.725362362362362	2.84449656444703\\
0.727363363363363	2.84199371705511\\
0.729364364364364	2.83945885582768\\
0.731365365365365	2.83689136928793\\
0.733366366366366	2.83429063126233\\
0.735367367367367	2.83165600046712\\
0.737368368368368	2.82898682008236\\
0.739369369369369	2.82628241731313\\
0.74137037037037	2.82354210293707\\
0.743371371371371	2.82076517083845\\
0.745372372372372	2.81795089752775\\
0.747373373373373	2.81509854164707\\
0.749374374374374	2.81220734346025\\
0.751375375375375	2.8092765243278\\
0.753376376376376	2.80630528616587\\
0.755377377377377	2.80329281088922\\
0.757378378378378	2.80023825983729\\
0.759379379379379	2.79714077318353\\
0.76138038038038	2.79399946932674\\
0.763381381381381	2.7908134442651\\
0.765382382382382	2.7875817709514\\
0.767383383383383	2.7843034986297\\
0.769384384384384	2.78097765215285\\
0.771385385385385	2.77760323128038\\
0.773386386386386	2.77417920995649\\
0.775387387387387	2.77070453556768\\
0.777388388388388	2.76717812817986\\
0.779389389389389	2.76359887975414\\
0.78139039039039	2.75996565334173\\
0.783391391391391	2.7562772822568\\
0.785392392392392	2.75253256922817\\
0.787393393393393	2.74873028552804\\
0.789394394394394	2.7448691700797\\
0.791395395395395	2.74094792854198\\
0.793396396396396	2.73696523237195\\
0.795397397397397	2.73291971786532\\
0.797398398398398	2.72880998517481\\
0.799399399399399	2.72463459730611\\
0.8014004004004	2.72039207909302\\
0.803401401401401	2.71608091615076\\
0.805402402402402	2.71169955380906\\
0.807403403403404	2.70724639602508\\
0.809404404404404	2.70271980427774\\
0.811405405405405	2.6981180964434\\
0.813406406406406	2.69343954565555\\
0.815407407407407	2.68868237914793\\
0.817408408408409	2.68384477708559\\
0.819409409409409	2.67892487138345\\
0.82141041041041	2.67392074451521\\
0.823411411411412	2.6688304283156\\
0.825412412412412	2.66365190277943\\
0.827413413413413	2.65838309485911\\
0.829414414414414	2.65302187726615\\
0.831415415415415	2.6475660672793\\
0.833416416416417	2.64201342556562\\
0.835417417417417	2.63636165501918\\
0.837418418418418	2.63060839962178\\
0.839419419419419	2.62475124333624\\
0.84142042042042	2.61878770903456\\
0.843421421421422	2.61271525747328\\
0.845422422422422	2.60653128632303\\
0.847423423423423	2.60023312926227\\
0.849424424424424	2.59381805514714\\
0.851425425425425	2.58728326726859\\
0.853426426426427	2.58062590270965\\
0.855427427427428	2.57384303181966\\
0.857428428428428	2.56693165781881\\
0.859429429429429	2.55988871654964\\
0.86143043043043	2.55271107639998\\
0.863431431431432	2.54539553840911\\
0.865432432432433	2.53793883658886\\
0.867433433433433	2.53033763847461\\
0.869434434434434	2.5225885459417\\
0.871435435435435	2.51468809630553\\
0.873436436436437	2.50663276374668\\
0.875437437437438	2.49841896108171\\
0.877438438438438	2.49004304192445\\
0.879439439439439	2.48150130327149\\
0.88144044044044	2.47278998854698\\
0.883441441441442	2.46390529115788\\
0.885442442442442	2.45484335859631\\
0.887443443443444	2.44560029713964\\
0.889444444444444	2.43617217719478\\
0.891445445445445	2.42655503934318\\
0.893446446446447	2.41674490113235\\
0.895447447447447	2.40673776467428\\
0.897448448448449	2.39652962510576\\
0.899449449449449	2.38611647996477\\
0.90145045045045	2.37549433954743\\
0.903451451451452	2.36465923829222\\
0.905452452452452	2.35360724726137\\
0.907453453453454	2.34233448776041\\
0.909454454454455	2.33083714615937\\
0.911455455455455	2.31911148995458\\
0.913456456456457	2.30715388511896\\
0.915457457457457	2.29496081477714\\
0.917458458458459	2.28252889922682\\
0.91945945945946	2.26985491732914\\
0.92146046046046	2.2569358292709\\
0.923461461461462	2.24376880068889\\
0.925462462462462	2.23035122813002\\
0.927463463463464	2.2166807658008\\
0.929464464464465	2.2027553535419\\
0.931465465465466	2.18857324593165\\
0.933466466466467	2.17413304240908\\
0.935467467467467	2.15943371826116\\
0.937468468468469	2.14447465630583\\
0.93946946946947	2.129255679051\\
0.941470470470471	2.11377708110185\\
0.943471471471471	2.09803966151301\\
0.945472472472472	2.08204475580207\\
0.947473473473474	2.06579426724569\\
0.949474474474475	2.04929069710121\\
0.951475475475476	2.03253717331967\\
0.953476476476476	2.01553747733388\\
0.955477477477477	1.99829606843431\\
0.957478478478479	1.98081810527493\\
0.95947947947948	1.96310946400272\\
0.961480480480481	1.94517675255564\\
0.963481481481481	1.92702732060472\\
0.965482482482482	1.90866926473422\\
0.967483483483484	1.89011142839447\\
0.969484484484485	1.87136339627606\\
0.971485485485486	1.85243548276582\\
0.973486486486487	1.83333871424649\\
0.975487487487487	1.81408480504872\\
0.977488488488489	1.79468612699487\\
0.97948948948949	1.77515567254217\\
0.981490490490491	1.75550701164459\\
0.983491491491492	1.73575424258517\\
0.985492492492492	1.71591193711628\\
0.987493493493494	1.69599508035888\\
0.989494494494495	1.67601900604241\\
0.991495495495496	1.65599932772309\\
0.993496496496497	1.63595186673723\\
0.995497497497498	1.61589257769981\\
0.997498498498499	1.59583747241601\\
0.9994994994995	1.57580254310816\\
1.0015005005005	1.55580368587807\\
1.0035015015015	1.53585662531612\\
1.0055025025025	1.51597684114493\\
1.0075035035035	1.49617949774196\\
1.0095045045045	1.47647937731793\\
1.01150550550551	1.45689081745583\\
1.01350650650651	1.43742765362027\\
1.01550750750751	1.41810316714556\\
1.01750850850851	1.39893003910996\\
1.01950950950951	1.37992031037368\\
1.02151051051051	1.361085347979\\
1.02351151151151	1.34243581796161\\
1.02551251251251	1.3239816645601\\
1.02751351351351	1.30573209567527\\
1.02951451451451	1.28769557438368\\
1.03151551551552	1.26987981619878\\
1.03351651651652	1.25229179173686\\
1.03551751751752	1.2349377343846\\
1.03751851851852	1.21782315251679\\
1.03951951951952	1.2009528458097\\
1.04152052052052	1.18433092515221\\
1.04352152152152	1.16796083567531\\
1.04552252252252	1.15184538241717\\
1.04752352352352	1.13598675815028\\
1.04952452452452	1.12038657291498\\
1.05152552552553	1.10504588483803\\
1.05352652652653	1.08996523184017\\
1.05552752752753	1.07514466386773\\
1.05752852852853	1.06058377530526\\
1.05952952952953	1.04628173729108\\
1.06153053053053	1.03223732965992\\
1.06353153153153	1.01844897229913\\
1.06553253253253	1.00491475571215\\
1.06753353353353	0.991632470637938\\
1.06953453453453	0.978599636593122\\
1.07153553553554	0.965813529234884\\
1.07353653653654	0.95327120645404\\
1.07553753753754	0.94096953316369\\
1.07753853853854	0.928905204727068\\
1.07953953953954	0.917074769024326\\
1.08154054054054	0.905474647140967\\
1.08354154154154	0.894101152702286\\
1.08554254254254	0.8829505098676\\
1.08754354354354	0.872018870018862\\
1.08954454454454	0.861302327184632\\
1.09154554554555	0.85079693224207\\
1.09354654654655	0.84049870595041\\
1.09554754754755	0.830403650865033\\
1.09754854854855	0.820507762191643\\
1.09954954954955	0.810807037635121\\
1.10155055055055	0.801297486303219\\
1.10355155155155	0.791975136718422\\
1.10555255255255	0.782836044000069\\
1.10755355355355	0.773876296268031\\
1.10955455455455	0.765092020322019\\
1.11155555555556	0.756479386654147\\
1.11355655655656	0.748034613835993\\
1.11555755755756	0.739753972337945\\
1.11755855855856	0.731633787817681\\
1.11955955955956	0.723670443926316\\
1.12156056056056	0.71586038467251\\
1.12356156156156	0.708200116380481\\
1.12556256256256	0.70068620928029\\
1.12756356356356	0.693315298762861\\
1.12956456456456	0.686084086333148\\
1.13156556556557	0.678989340285726\\
1.13356656656657	0.67202789613754\\
1.13556756756757	0.66519665683551\\
1.13756856856857	0.658492592766519\\
1.13956956956957	0.651912741590798\\
1.14157057057057	0.645454207916164\\
1.14357157157157	0.639114162834213\\
1.14557257257257	0.632889843332928\\
1.14757357357357	0.626778551600902\\
1.14957457457457	0.620777654238623\\
1.15157557557558	0.614884581388536\\
1.15357657657658	0.609096825794068\\
1.15557757757758	0.603411941800502\\
1.15757857857858	0.59782754430539\\
1.15957957957958	0.592341307668089\\
1.16158058058058	0.586950964585328\\
1.16358158158158	0.581654304940091\\
1.16558258258258	0.576449174630544\\
1.16758358358358	0.571333474383738\\
1.16958458458458	0.56630515856022\\
1.17158558558559	0.561362233952648\\
1.17358658658659	0.55650275858392\\
1.17558758758759	0.551724840507232\\
1.17758858858859	0.547026636611261\\
1.17958958958959	0.542406351434345\\
1.18159059059059	0.537862235988674\\
1.18359159159159	0.533392586597078\\
1.18559259259259	0.528995743744805\\
1.18759359359359	0.524670090947181\\
1.18959459459459	0.520414053634472\\
1.1915955955956	0.516226098055669\\
1.1935965965966	0.512104730201603\\
1.1955975975976	0.508048494748287\\
1.1975985985986	0.504055974021196\\
1.1995995995996	0.500125786981147\\
1.2016006006006	0.496256588231593\\
1.2036016016016	0.492447067048278\\
1.2056026026026	0.488695946430685\\
1.2076036036036	0.485001982176303\\
1.2096046046046	0.481363961976532\\
1.21160560560561	0.477780704535533\\
1.21360660660661	0.474251058710383\\
1.21560760760761	0.470773902673599\\
1.21760860860861	0.467348143097543\\
1.21960960960961	0.463972714359752\\
1.22161061061061	0.460646577769657\\
1.22361161161161	0.457368720816496\\
1.22561261261261	0.454138156437046\\
1.22761361361361	0.450953922303868\\
1.22961461461461	0.447815080133473\\
1.23161561561562	0.444720715013415\\
1.23361661661662	0.441669934748652\\
1.23561761761762	0.438661869226346\\
1.23761861861862	0.435695669798757\\
1.23961961961962	0.432770508683847\\
1.24162062062062	0.429885578383264\\
1.24362162162162	0.427040091116962\\
1.24562262262262	0.424233278274563\\
1.24762362362362	0.421464389882329\\
1.24962462462462	0.418732694086071\\
1.25162562562563	0.416037476649036\\
1.25362662662663	0.413378040464556\\
1.25562762762763	0.410753705083238\\
1.25762862862863	0.408163806253828\\
1.25962962962963	0.405607695477774\\
1.26163063063063	0.403084739577103\\
1.26363163163163	0.400594320274829\\
1.26563263263263	0.398135833788018\\
1.26763363363363	0.395708690432745\\
1.26963463463463	0.393312314240929\\
1.27163563563564	0.390946142588408\\
1.27363663663664	0.388609625834026\\
1.27563763763764	0.386302226969628\\
1.27763863863864	0.384023421280054\\
1.27963963963964	0.381772696013653\\
1.28164064064064	0.379549550062105\\
1.28364164164164	0.377353493650066\\
1.28564264264264	0.375184048033752\\
1.28764364364364	0.373040745208526\\
1.28964464464464	0.370923127625006\\
1.29164564564565	0.368830747913698\\
1.29364664664665	0.366763168617479\\
1.29564764764765	0.364719961932168\\
1.29764864864865	0.362700709454525\\
1.29964964964965	0.360705001937652\\
1.30165065065065	0.35873243905363\\
1.30365165165165	0.35678262916295\\
1.30565265265265	0.354855189090661\\
1.30765365365365	0.352949743909121\\
1.30965465465465	0.351065926726885\\
1.31165565565566	0.349203378483813\\
1.31365665665666	0.347361747752015\\
1.31565765765766	0.345540690542534\\
1.31765865865866	0.343739870117597\\
1.31965965965966	0.341958956808198\\
1.32166066066066	0.340197627836929\\
1.32366166166166	0.33845556714585\\
1.32566266266266	0.336732465229188\\
1.32766366366366	0.335028018970912\\
1.32966466466466	0.333341931486696\\
1.33166566566567	0.331673911970553\\
1.33366666666667	0.330023675545622\\
1.33566766766767	0.328390943119181\\
1.33766866866867	0.326775441241743\\
1.33966966966967	0.32517690197007\\
1.34167067067067	0.323595062733938\\
1.34367167167167	0.32202966620667\\
1.34567267267267	0.320480460179203\\
1.34767367367367	0.318947197437632\\
1.34967467467467	0.317429635644107\\
1.35167567567568	0.31592753722104\\
1.35367667667668	0.314440669238393\\
1.35567767767768	0.31296880330411\\
1.35767867867868	0.311511715457482\\
1.35967967967968	0.310069186065413\\
1.36168068068068	0.308640999721445\\
1.36368168168168	0.307226945147604\\
1.36568268268268	0.305826815098722\\
1.36768368368368	0.304440406269451\\
1.36968468468468	0.303067519203692\\
1.37168568568569	0.301707958206427\\
1.37368668668669	0.300361531257913\\
1.37568768768769	0.299028049930108\\
1.37768868868869	0.297707329305337\\
1.37968968968969	0.296399187897042\\
1.38169069069069	0.295103447572634\\
1.38369169169169	0.293819933478363\\
1.38569269269269	0.292548473966109\\
1.38769369369369	0.291288900522089\\
1.38969469469469	0.290041047697403\\
1.3916956956957	0.288804753040362\\
1.3936966966967	0.287579857030548\\
1.3956976976977	0.286366203014559\\
1.3976986986987	0.285163637143408\\
1.3996996996997	0.283972008311488\\
1.4017007007007	0.282791168097087\\
1.4037017017017	0.281620970704425\\
1.4057027027027	0.280461272907108\\
1.4077037037037	0.279311933993047\\
1.4097047047047	0.278172815710702\\
1.41170570570571	0.277043782216711\\
1.41370670670671	0.275924700024794\\
1.41570770770771	0.274815437955914\\
1.41770870870871	0.273715867089691\\
1.41970970970971	0.272625860716988\\
1.42171071071071	0.271545294293665\\
1.42371171171171	0.270474045395475\\
1.42571271271271	0.269411993674026\\
1.42771371371371	0.268359020813854\\
1.42971471471471	0.267315010490496\\
1.43171571571572	0.266279848329586\\
1.43371671671672	0.265253421866946\\
1.43571771771772	0.264235620509611\\
1.43771871871872	0.263226335497799\\
1.43971971971972	0.26222545986778\\
1.44172072072072	0.261232888415627\\
1.44372172172172	0.260248517661808\\
1.44572272272272	0.259272245816653\\
1.44772372372372	0.258303972746582\\
1.44972472472472	0.257343599941144\\
1.45172572572573	0.256391030480843\\
1.45372672672673	0.25544616900568\\
1.45572772772773	0.254508921684443\\
1.45772872872873	0.253579196184698\\
1.45972972972973	0.252656901643488\\
1.46173073073073	0.251741948638665\\
1.46373173173173	0.25083424916093\\
1.46573273273273	0.249933716586462\\
1.46773373373373	0.249040265650208\\
1.46973473473473	0.24815381241974\\
1.47173573573574	0.247274274269743\\
1.47373673673674	0.246401569857044\\
1.47573773773774	0.245535619096214\\
1.47773873873874	0.24467634313572\\
1.47973973973974	0.243823664334594\\
1.48174074074074	0.242977506239643\\
1.48374174174174	0.242137793563141\\
1.48574274274274	0.241304452161023\\
1.48774374374374	0.240477409011561\\
1.48974474474474	0.239656592194508\\
1.49174574574575	0.238841930870686\\
1.49374674674675	0.238033355262039\\
1.49574774774775	0.237230796632097\\
1.49774874874875	0.236434187266891\\
1.49974974974975	0.235643460456243\\
1.50175075075075	0.234858550475495\\
1.50375175175175	0.234079392567606\\
1.50575275275275	0.233305922925646\\
1.50775375375375	0.232538078675649\\
1.50975475475475	0.231775797859846\\
1.51175575575576	0.231019019420247\\
1.51375675675676	0.230267683182559\\
1.51575775775776	0.229521729840457\\
1.51775875875876	0.228781100940189\\
1.51975975975976	0.228045738865481\\
1.52176076076076	0.227315586822776\\
1.52376176176176	0.226590588826777\\
1.52576276276276	0.225870689686281\\
1.52776376376376	0.225155834990319\\
1.52976476476476	0.224445971094568\\
1.53176576576577	0.223741045108051\\
1.53376676676677	0.223041004880105\\
1.53576776776777	0.222345798987621\\
1.53776876876877	0.221655376722537\\
1.53976976976977	0.220969688079582\\
1.54177077077077	0.220288683744283\\
1.54377177177177	0.219612315081199\\
1.54577277277277	0.218940534122397\\
1.54777377377377	0.218273293556156\\
1.54977477477477	0.217610546715904\\
1.55177577577578	0.216952247569372\\
1.55377677677678	0.216298350707954\\
1.55577777777778	0.215648811336298\\
1.55777877877878	0.215003585262077\\
1.55977977977978	0.214362628885992\\
1.56178078078078	0.213725899191937\\
1.56378178178178	0.213093353737385\\
1.56578278278278	0.212464950643948\\
1.56778378378378	0.211840648588121\\
1.56978478478478	0.211220406792211\\
1.57178578578579	0.210604185015434\\
1.57378678678679	0.209991943545194\\
1.57578778778779	0.209383643188516\\
1.57778878878879	0.208779245263658\\
1.57978978978979	0.208178711591873\\
1.58179079079079	0.207582004489328\\
1.58379179179179	0.206989086759187\\
1.58579279279279	0.206399921683833\\
1.58779379379379	0.205814473017238\\
1.58979479479479	0.205232704977486\\
1.5917957957958	0.204654582239425\\
1.5937967967968	0.204080069927465\\
1.5957977977978	0.203509133608502\\
1.5977987987988	0.202941739284988\\
1.5997997997998	0.202377853388115\\
1.6018008008008	0.201817442771128\\
1.6038018018018	0.201260474702774\\
1.6058028028028	0.200706916860852\\
1.6078038038038	0.200156737325901\\
1.6098048048048	0.199609904574985\\
1.61180580580581	0.199066387475609\\
1.61380680680681	0.198526155279732\\
1.61580780780781	0.197989177617897\\
1.61780880880881	0.197455424493465\\
1.61980980980981	0.196924866276948\\
1.62181081081081	0.196397473700455\\
1.62381181181181	0.195873217852227\\
1.62581281281281	0.195352070171277\\
1.62781381381381	0.19483400244212\\
1.62981481481481	0.194318986789604\\
1.63181581581582	0.193806995673824\\
1.63381681681682	0.193298001885133\\
1.63581781781782	0.19279197853924\\
1.63781881881882	0.192288899072392\\
1.63981981981982	0.191788737236639\\
1.64182082082082	0.19129146709519\\
1.64382182182182	0.190797063017841\\
1.64582282282282	0.190305499676488\\
1.64782382382382	0.189816752040716\\
1.64982482482482	0.189330795373464\\
1.65182582582583	0.18884760522677\\
1.65382682682683	0.188367157437577\\
1.65582782782783	0.187889428123628\\
1.65782882882883	0.187414393679417\\
1.65982982982983	0.186942030772217\\
1.66183083083083	0.186472316338173\\
1.66383183183183	0.18600522757846\\
1.66583283283283	0.185540741955514\\
1.66783383383383	0.185078837189314\\
1.66983483483483	0.184619491253736\\
1.67183583583584	0.184162682372969\\
1.67383683683684	0.183708389017985\\
1.67583783783784	0.183256589903069\\
1.67783883883884	0.182807263982415\\
1.67983983983984	0.182360390446769\\
1.68184084084084	0.181915948720131\\
1.68384184184184	0.181473918456514\\
1.68584284284284	0.181034279536755\\
1.68784384384384	0.180597012065374\\
1.68984484484484	0.180162096367496\\
1.69184584584585	0.17972951298581\\
1.69384684684685	0.179299242677587\\
1.69584784784785	0.178871266411743\\
1.69784884884885	0.17844556536595\\
1.69984984984985	0.178022120923798\\
1.70185085085085	0.17760091467199\\
1.70385185185185	0.177181928397604\\
1.70585285285285	0.176765144085375\\
1.70785385385385	0.176350543915042\\
1.70985485485485	0.175938110258719\\
1.71185585585586	0.175527825678323\\
1.71385685685686	0.175119672923036\\
1.71585785785786	0.174713634926803\\
1.71785885885886	0.174309694805883\\
1.71985985985986	0.173907835856423\\
1.72186086086086	0.173508041552082\\
1.72386186186186	0.173110295541687\\
1.72586286286286	0.172714581646926\\
1.72786386386386	0.172320883860082\\
1.72986486486486	0.171929186341794\\
1.73186586586587	0.171539473418858\\
1.73386686686687	0.171151729582065\\
1.73586786786787	0.170765939484065\\
1.73786886886887	0.170382087937273\\
1.73986986986987	0.170000159911796\\
1.74187087087087	0.169620140533405\\
1.74387187187187	0.169242015081527\\
1.74587287287287	0.168865768987275\\
1.74787387387387	0.168491387831503\\
1.74987487487487	0.168118857342896\\
1.75187587587588	0.167748163396083\\
1.75387687687688	0.167379292009785\\
1.75587787787788	0.167012229344983\\
1.75787887887888	0.166646961703126\\
1.75987987987988	0.16628347552435\\
1.76188088088088	0.165921757385737\\
1.76388188188188	0.165561793999597\\
1.76588288288288	0.165203572211767\\
1.76788388388388	0.16484707899995\\
1.76988488488488	0.164492301472066\\
1.77188588588589	0.164139226864636\\
1.77388688688689	0.163787842541183\\
1.77588788788789	0.163438135990664\\
1.77788888888889	0.163090094825916\\
1.77988988988989	0.162743706782136\\
1.78189089089089	0.162398959715374\\
1.78389189189189	0.162055841601052\\
1.78589289289289	0.161714340532503\\
1.78789389389389	0.161374444719536\\
1.78989489489489	0.161036142487014\\
1.7918958958959	0.160699422273459\\
1.7938968968969	0.160364272629674\\
1.7958978978979	0.160030682217389\\
1.7978988988989	0.159698639807919\\
1.7998998998999	0.159368134280848\\
1.8019009009009	0.159039154622731\\
1.8039019019019	0.158711689925808\\
1.8059029029029	0.15838572938675\\
1.8079039039039	0.158061262305406\\
1.8099049049049	0.157738278083581\\
1.81190590590591	0.157416766223824\\
1.81390690690691	0.157096716328238\\
1.81590790790791	0.156778118097302\\
1.81790890890891	0.156460961328714\\
1.81990990990991	0.156145235916247\\
1.82191091091091	0.155830931848624\\
1.82391191191191	0.155518039208403\\
1.82591291291291	0.155206548170887\\
1.82791391391391	0.15489644900304\\
1.82991491491491	0.154587732062422\\
1.83191591591592	0.15428038779614\\
1.83391691691692	0.153974406739812\\
1.83591791791792	0.153669779516544\\
1.83791891891892	0.153366496835925\\
1.83991991991992	0.153064549493029\\
1.84192092092092	0.15276392836744\\
1.84392192192192	0.152464624422279\\
1.84592292292292	0.152166628703259\\
1.84792392392392	0.151869932337735\\
1.84992492492492	0.151574526533786\\
1.85192592592593	0.151280402579289\\
1.85392692692693	0.150987551841026\\
1.85592792792793	0.150695965763788\\
1.85792892892893	0.150405635869498\\
1.85992992992993	0.150116553756343\\
1.86193093093093	0.149828711097921\\
1.86393193193193	0.149542099642394\\
1.86593293293293	0.14925671121166\\
1.86793393393393	0.148972537700528\\
1.86993493493493	0.148689571075906\\
1.87193593593594	0.148407803376009\\
1.87393693693694	0.148127226709559\\
1.87593793793794	0.147847833255017\\
1.87793893893894	0.147569615259807\\
1.87993993993994	0.14729256503956\\
1.88194094094094	0.147016674977366\\
1.88394194194194	0.146741937523037\\
1.88594294294294	0.146468345192374\\
1.88794394394394	0.146195890566452\\
1.88994494494494	0.145924566290904\\
1.89194594594595	0.14565436507523\\
1.89394694694695	0.145385279692093\\
1.89594794794795	0.145117302976647\\
1.89794894894895	0.144850427825855\\
1.89994994994995	0.14458464719783\\
1.90195095095095	0.144319954111173\\
1.90395195195195	0.144056341644328\\
1.90595295295295	0.143793802934942\\
1.90795395395395	0.143532331179228\\
1.90995495495495	0.143271919631351\\
1.91195595595596	0.1430125616028\\
1.91395695695696	0.142754250461793\\
1.91595795795796	0.142496979632663\\
1.91795895895896	0.142240742595278\\
1.91995995995996	0.141985532884446\\
1.92196096096096	0.141731344089342\\
1.92396196196196	0.141478169852936\\
1.92596296296296	0.14122600387143\\
1.92796396396396	0.140974839893702\\
1.92996496496496	0.140724671720754\\
1.93196596596597	0.140475493205171\\
1.93396696696697	0.140227298250587\\
1.93596796796797	0.139980080811153\\
1.93796896896897	0.139733834891013\\
1.93996996996997	0.139488554543793\\
1.94197097097097	0.139244233872087\\
1.94397197197197	0.139000867026953\\
1.94597297297297	0.138758448207418\\
1.94797397397397	0.138516971659986\\
1.94997497497497	0.138276431678152\\
1.95197597597598	0.138036822601924\\
1.95397697697698	0.137798138817345\\
1.95597797797798	0.137560374756033\\
1.95797897897898	0.137323524894712\\
1.95997997997998	0.137087583754759\\
1.96198098098098	0.136852545901752\\
1.96398198198198	0.136618405945023\\
1.96598298298298	0.136385158537224\\
1.96798398398398	0.136152798373881\\
1.96998498498498	0.135921320192978\\
1.97198598598599	0.135690718774519\\
1.97398698698699	0.135460988940119\\
1.97598798798799	0.135232125552585\\
1.97798898898899	0.135004123515505\\
1.97998998998999	0.134776977772849\\
1.98199099099099	0.134550683308563\\
1.98399199199199	0.134325235146176\\
1.98599299299299	0.134100628348413\\
1.98799399399399	0.133876858016804\\
1.98999499499499	0.133653919291306\\
1.991995995996	0.133431807349924\\
1.993996996997	0.133210517408342\\
1.995997997998	0.132990044719551\\
1.997998998999	0.132770384573487\\
2	0.132551532296674\\
1.997998998999	0.132770384573487\\
1.995997997998	0.132990044719551\\
1.993996996997	0.133210517408342\\
1.991995995996	0.133431807349924\\
1.98999499499499	0.133653919291306\\
1.98799399399399	0.133876858016804\\
1.98599299299299	0.134100628348413\\
1.98399199199199	0.134325235146176\\
1.98199099099099	0.134550683308563\\
1.97998998998999	0.134776977772849\\
1.97798898898899	0.135004123515505\\
1.97598798798799	0.135232125552585\\
1.97398698698699	0.135460988940119\\
1.97198598598599	0.135690718774519\\
1.96998498498498	0.135921320192978\\
1.96798398398398	0.136152798373881\\
1.96598298298298	0.136385158537224\\
1.96398198198198	0.136618405945023\\
1.96198098098098	0.136852545901752\\
1.95997997997998	0.137087583754759\\
1.95797897897898	0.137323524894712\\
1.95597797797798	0.137560374756033\\
1.95397697697698	0.137798138817345\\
1.95197597597598	0.138036822601923\\
1.94997497497497	0.138276431678152\\
1.94797397397397	0.138516971659986\\
1.94597297297297	0.138758448207418\\
1.94397197197197	0.139000867026953\\
1.94197097097097	0.139244233872087\\
1.93996996996997	0.139488554543793\\
1.93796896896897	0.139733834891013\\
1.93596796796797	0.139980080811152\\
1.93396696696697	0.140227298250587\\
1.93196596596597	0.140475493205171\\
1.92996496496496	0.140724671720754\\
1.92796396396396	0.140974839893702\\
1.92596296296296	0.14122600387143\\
1.92396196196196	0.141478169852936\\
1.92196096096096	0.141731344089342\\
1.91995995995996	0.141985532884446\\
1.91795895895896	0.142240742595278\\
1.91595795795796	0.142496979632663\\
1.91395695695696	0.142754250461793\\
1.91195595595596	0.143012561602801\\
1.90995495495495	0.143271919631351\\
1.90795395395395	0.143532331179228\\
1.90595295295295	0.143793802934942\\
1.90395195195195	0.144056341644328\\
1.90195095095095	0.144319954111173\\
1.89994994994995	0.14458464719783\\
1.89794894894895	0.144850427825855\\
1.89594794794795	0.145117302976647\\
1.89394694694695	0.145385279692093\\
1.89194594594595	0.14565436507523\\
1.88994494494494	0.145924566290904\\
1.88794394394394	0.146195890566451\\
1.88594294294294	0.146468345192374\\
1.88394194194194	0.146741937523037\\
1.88194094094094	0.147016674977366\\
1.87993993993994	0.14729256503956\\
1.87793893893894	0.147569615259807\\
1.87593793793794	0.147847833255017\\
1.87393693693694	0.148127226709559\\
1.87193593593594	0.148407803376009\\
1.86993493493493	0.148689571075906\\
1.86793393393393	0.148972537700527\\
1.86593293293293	0.149256711211661\\
1.86393193193193	0.149542099642394\\
1.86193093093093	0.149828711097921\\
1.85992992992993	0.150116553756343\\
1.85792892892893	0.150405635869498\\
1.85592792792793	0.150695965763788\\
1.85392692692693	0.150987551841026\\
1.85192592592593	0.151280402579289\\
1.84992492492492	0.151574526533786\\
1.84792392392392	0.151869932337735\\
1.84592292292292	0.152166628703259\\
1.84392192192192	0.152464624422279\\
1.84192092092092	0.15276392836744\\
1.83991991991992	0.153064549493029\\
1.83791891891892	0.153366496835925\\
1.83591791791792	0.153669779516544\\
1.83391691691692	0.153974406739811\\
1.83191591591592	0.15428038779614\\
1.82991491491491	0.154587732062422\\
1.82791391391391	0.15489644900304\\
1.82591291291291	0.155206548170888\\
1.82391191191191	0.155518039208403\\
1.82191091091091	0.155830931848624\\
1.81990990990991	0.156145235916247\\
1.81790890890891	0.156460961328714\\
1.81590790790791	0.156778118097302\\
1.81390690690691	0.157096716328238\\
1.81190590590591	0.157416766223824\\
1.8099049049049	0.157738278083581\\
1.8079039039039	0.158061262305406\\
1.8059029029029	0.15838572938675\\
1.8039019019019	0.158711689925808\\
1.8019009009009	0.15903915462273\\
1.7998998998999	0.159368134280848\\
1.7978988988989	0.159698639807919\\
1.7958978978979	0.160030682217389\\
1.7938968968969	0.160364272629674\\
1.7918958958959	0.160699422273459\\
1.78989489489489	0.161036142487014\\
1.78789389389389	0.161374444719536\\
1.78589289289289	0.161714340532503\\
1.78389189189189	0.162055841601052\\
1.78189089089089	0.162398959715374\\
1.77988988988989	0.162743706782136\\
1.77788888888889	0.163090094825916\\
1.77588788788789	0.163438135990664\\
1.77388688688689	0.163787842541183\\
1.77188588588589	0.164139226864636\\
1.76988488488488	0.164492301472066\\
1.76788388388388	0.16484707899995\\
1.76588288288288	0.165203572211766\\
1.76388188188188	0.165561793999597\\
1.76188088088088	0.165921757385737\\
1.75987987987988	0.16628347552435\\
1.75787887887888	0.166646961703126\\
1.75587787787788	0.167012229344983\\
1.75387687687688	0.167379292009785\\
1.75187587587588	0.167748163396083\\
1.74987487487487	0.168118857342896\\
1.74787387387387	0.168491387831503\\
1.74587287287287	0.168865768987275\\
1.74387187187187	0.169242015081528\\
1.74187087087087	0.169620140533405\\
1.73986986986987	0.170000159911796\\
1.73786886886887	0.170382087937273\\
1.73586786786787	0.170765939484065\\
1.73386686686687	0.171151729582065\\
1.73186586586587	0.171539473418858\\
1.72986486486486	0.171929186341794\\
1.72786386386386	0.172320883860082\\
1.72586286286286	0.172714581646926\\
1.72386186186186	0.173110295541687\\
1.72186086086086	0.173508041552082\\
1.71985985985986	0.173907835856423\\
1.71785885885886	0.174309694805883\\
1.71585785785786	0.174713634926803\\
1.71385685685686	0.175119672923035\\
1.71185585585586	0.175527825678323\\
1.70985485485485	0.175938110258719\\
1.70785385385385	0.176350543915042\\
1.70585285285285	0.176765144085375\\
1.70385185185185	0.177181928397604\\
1.70185085085085	0.17760091467199\\
1.69984984984985	0.178022120923798\\
1.69784884884885	0.17844556536595\\
1.69584784784785	0.178871266411743\\
1.69384684684685	0.179299242677587\\
1.69184584584585	0.17972951298581\\
1.68984484484484	0.180162096367496\\
1.68784384384384	0.180597012065374\\
1.68584284284284	0.181034279536755\\
1.68384184184184	0.181473918456514\\
1.68184084084084	0.181915948720131\\
1.67983983983984	0.182360390446769\\
1.67783883883884	0.182807263982415\\
1.67583783783784	0.183256589903069\\
1.67383683683684	0.183708389017985\\
1.67183583583584	0.184162682372969\\
1.66983483483483	0.184619491253736\\
1.66783383383383	0.185078837189314\\
1.66583283283283	0.185540741955514\\
1.66383183183183	0.18600522757846\\
1.66183083083083	0.186472316338173\\
1.65982982982983	0.186942030772217\\
1.65782882882883	0.187414393679417\\
1.65582782782783	0.187889428123628\\
1.65382682682683	0.188367157437577\\
1.65182582582583	0.18884760522677\\
1.64982482482482	0.189330795373464\\
1.64782382382382	0.189816752040716\\
1.64582282282282	0.190305499676488\\
1.64382182182182	0.190797063017841\\
1.64182082082082	0.19129146709519\\
1.63981981981982	0.191788737236639\\
1.63781881881882	0.192288899072392\\
1.63581781781782	0.192791978539241\\
1.63381681681682	0.193298001885133\\
1.63181581581582	0.193806995673824\\
1.62981481481481	0.194318986789604\\
1.62781381381381	0.19483400244212\\
1.62581281281281	0.195352070171277\\
1.62381181181181	0.195873217852227\\
1.62181081081081	0.196397473700455\\
1.61980980980981	0.196924866276948\\
1.61780880880881	0.197455424493465\\
1.61580780780781	0.197989177617897\\
1.61380680680681	0.198526155279732\\
1.61180580580581	0.199066387475609\\
1.6098048048048	0.199609904574985\\
1.6078038038038	0.200156737325901\\
1.6058028028028	0.200706916860852\\
1.6038018018018	0.201260474702774\\
1.6018008008008	0.201817442771128\\
1.5997997997998	0.202377853388114\\
1.5977987987988	0.202941739284988\\
1.5957977977978	0.203509133608502\\
1.5937967967968	0.204080069927464\\
1.5917957957958	0.204654582239425\\
1.58979479479479	0.205232704977487\\
1.58779379379379	0.205814473017238\\
1.58579279279279	0.206399921683833\\
1.58379179179179	0.206989086759187\\
1.58179079079079	0.207582004489328\\
1.57978978978979	0.208178711591873\\
1.57778878878879	0.208779245263658\\
1.57578778778779	0.209383643188516\\
1.57378678678679	0.209991943545194\\
1.57178578578579	0.210604185015434\\
1.56978478478478	0.211220406792211\\
1.56778378378378	0.211840648588121\\
1.56578278278278	0.212464950643948\\
1.56378178178178	0.213093353737385\\
1.56178078078078	0.213725899191937\\
1.55977977977978	0.214362628885992\\
1.55777877877878	0.215003585262078\\
1.55577777777778	0.215648811336297\\
1.55377677677678	0.216298350707954\\
1.55177577577578	0.216952247569371\\
1.54977477477477	0.217610546715904\\
1.54777377377377	0.218273293556156\\
1.54577277277277	0.218940534122397\\
1.54377177177177	0.2196123150812\\
1.54177077077077	0.220288683744284\\
1.53976976976977	0.220969688079582\\
1.53776876876877	0.221655376722537\\
1.53576776776777	0.222345798987622\\
1.53376676676677	0.223041004880105\\
1.53176576576577	0.223741045108051\\
1.52976476476476	0.224445971094568\\
1.52776376376376	0.225155834990319\\
1.52576276276276	0.225870689686281\\
1.52376176176176	0.226590588826776\\
1.52176076076076	0.227315586822775\\
1.51975975975976	0.228045738865481\\
1.51775875875876	0.228781100940189\\
1.51575775775776	0.229521729840457\\
1.51375675675676	0.230267683182558\\
1.51175575575576	0.231019019420248\\
1.50975475475475	0.231775797859847\\
1.50775375375375	0.232538078675649\\
1.50575275275275	0.233305922925645\\
1.50375175175175	0.234079392567606\\
1.50175075075075	0.234858550475495\\
1.49974974974975	0.235643460456243\\
1.49774874874875	0.236434187266891\\
1.49574774774775	0.237230796632098\\
1.49374674674675	0.238033355262038\\
1.49174574574575	0.238841930870686\\
1.48974474474474	0.239656592194508\\
1.48774374374374	0.240477409011562\\
1.48574274274274	0.241304452161023\\
1.48374174174174	0.242137793563141\\
1.48174074074074	0.242977506239644\\
1.47973973973974	0.243823664334594\\
1.47773873873874	0.244676343135719\\
1.47573773773774	0.245535619096215\\
1.47373673673674	0.246401569857045\\
1.47173573573574	0.247274274269744\\
1.46973473473473	0.24815381241974\\
1.46773373373373	0.249040265650207\\
1.46573273273273	0.249933716586462\\
1.46373173173173	0.250834249160929\\
1.46173073073073	0.251741948638665\\
1.45972972972973	0.252656901643487\\
1.45772872872873	0.2535791961847\\
1.45572772772773	0.254508921684442\\
1.45372672672673	0.25544616900568\\
1.45172572572573	0.256391030480844\\
1.44972472472472	0.257343599941144\\
1.44772372372372	0.258303972746581\\
1.44572272272272	0.259272245816654\\
1.44372172172172	0.260248517661808\\
1.44172072072072	0.261232888415625\\
1.43971971971972	0.26222545986778\\
1.43771871871872	0.263226335497799\\
1.43571771771772	0.264235620509611\\
1.43371671671672	0.265253421866946\\
1.43171571571572	0.266279848329587\\
1.42971471471471	0.267315010490496\\
1.42771371371371	0.268359020813855\\
1.42571271271271	0.269411993674026\\
1.42371171171171	0.270474045395474\\
1.42171071071071	0.271545294293666\\
1.41970970970971	0.272625860716988\\
1.41770870870871	0.273715867089692\\
1.41570770770771	0.274815437955915\\
1.41370670670671	0.275924700024794\\
1.41170570570571	0.277043782216711\\
1.4097047047047	0.278172815710701\\
1.4077037037037	0.279311933993046\\
1.4057027027027	0.280461272907111\\
1.4037017017017	0.281620970704426\\
1.4017007007007	0.282791168097089\\
1.3996996996997	0.283972008311488\\
1.3976986986987	0.285163637143409\\
1.3956976976977	0.28636620301456\\
1.3936966966967	0.287579857030547\\
1.3916956956957	0.288804753040363\\
1.38969469469469	0.290041047697405\\
1.38769369369369	0.291288900522089\\
1.38569269269269	0.292548473966109\\
1.38369169169169	0.293819933478364\\
1.38169069069069	0.295103447572635\\
1.37968968968969	0.296399187897042\\
1.37768868868869	0.297707329305339\\
1.37568768768769	0.29902804993011\\
1.37368668668669	0.300361531257911\\
1.37168568568569	0.301707958206426\\
1.36968468468468	0.303067519203691\\
1.36768368368368	0.304440406269451\\
1.36568268268268	0.30582681509872\\
1.36368168168168	0.307226945147603\\
1.36168068068068	0.308640999721448\\
1.35967967967968	0.310069186065409\\
1.35767867867868	0.311511715457485\\
1.35567767767768	0.312968803304113\\
1.35367667667668	0.314440669238395\\
1.35167567567568	0.315927537221042\\
1.34967467467467	0.31742963564411\\
1.34767367367367	0.31894719743763\\
1.34567267267267	0.320480460179204\\
1.34367167167167	0.32202966620667\\
1.34167067067067	0.323595062733936\\
1.33966966966967	0.325176901970068\\
1.33766866866867	0.326775441241744\\
1.33566766766767	0.328390943119178\\
1.33366666666667	0.330023675545623\\
1.33166566566567	0.331673911970559\\
1.32966466466466	0.333341931486698\\
1.32766366366366	0.335028018970908\\
1.32566266266266	0.336732465229191\\
1.32366166166166	0.338455567145844\\
1.32166066066066	0.340197627836929\\
1.31965965965966	0.341958956808195\\
1.31765865865866	0.343739870117597\\
1.31565765765766	0.345540690542537\\
1.31365665665666	0.347361747752016\\
1.31165565565566	0.349203378483813\\
1.30965465465465	0.351065926726882\\
1.30765365365365	0.352949743909115\\
1.30565265265265	0.354855189090659\\
1.30365165165165	0.356782629162946\\
1.30165065065065	0.358732439053636\\
1.29964964964965	0.360705001937655\\
1.29764864864865	0.362700709454523\\
1.29564764764765	0.364719961932174\\
1.29364664664665	0.366763168617481\\
1.29164564564565	0.368830747913692\\
1.28964464464464	0.370923127625007\\
1.28764364364364	0.37304074520852\\
1.28564264264264	0.375184048033757\\
1.28364164164164	0.377353493650069\\
1.28164064064064	0.379549550062106\\
1.27963963963964	0.381772696013648\\
1.27763863863864	0.38402342128006\\
1.27563763763764	0.38630222696962\\
1.27363663663664	0.388609625834034\\
1.27163563563564	0.390946142588398\\
1.26963463463463	0.393312314240928\\
1.26763363363363	0.395708690432736\\
1.26563263263263	0.398135833788002\\
1.26363163163163	0.40059432027482\\
1.26163063063063	0.403084739577092\\
1.25962962962963	0.40560769547777\\
1.25762862862863	0.40816380625382\\
1.25562762762763	0.410753705083251\\
1.25362662662663	0.413378040464568\\
1.25162562562563	0.416037476649027\\
1.24962462462462	0.418732694086074\\
1.24762362362362	0.421464389882335\\
1.24562262262262	0.424233278274569\\
1.24362162162162	0.427040091116974\\
1.24162062062062	0.429885578383252\\
1.23961961961962	0.432770508683844\\
1.23761861861862	0.43569566979875\\
1.23561761761762	0.43866186922635\\
1.23361661661662	0.441669934748654\\
1.23161561561562	0.444720715013409\\
1.22961461461461	0.447815080133468\\
1.22761361361361	0.45095392230387\\
1.22561261261261	0.45413815643704\\
1.22361161161161	0.457368720816522\\
1.22161061061061	0.460646577769676\\
1.21960960960961	0.463972714359722\\
1.21760860860861	0.46734814309754\\
1.21560760760761	0.470773902673594\\
1.21360660660661	0.474251058710352\\
1.21160560560561	0.477780704535539\\
1.2096046046046	0.481363961976544\\
1.2076036036036	0.48500198217627\\
1.2056026026026	0.488695946430681\\
1.2036016016016	0.492447067048272\\
1.2016006006006	0.49625658823162\\
1.1995995995996	0.500125786981153\\
1.1975985985986	0.504055974021195\\
1.1955975975976	0.50804849474827\\
1.1935965965966	0.512104730201609\\
1.1915955955956	0.516226098055677\\
1.18959459459459	0.520414053634463\\
1.18759359359359	0.52467009094717\\
1.18559259259259	0.528995743744805\\
1.18359159159159	0.533392586597055\\
1.18159059059059	0.53786223598866\\
1.17958958958959	0.542406351434354\\
1.17758858858859	0.547026636611229\\
1.17558758758759	0.55172484050719\\
1.17358658658659	0.556502758583932\\
1.17158558558559	0.561362233952623\\
1.16958458458458	0.566305158560179\\
1.16758358358358	0.57133347438373\\
1.16558258258258	0.576449174630519\\
1.16358158158158	0.581654304940104\\
1.16158058058058	0.586950964585334\\
1.15957957957958	0.592341307668098\\
1.15757857857858	0.597827544305363\\
1.15557757757758	0.603411941800479\\
1.15357657657658	0.609096825794119\\
1.15157557557558	0.614884581388611\\
1.14957457457457	0.620777654238669\\
1.14757357357357	0.626778551600811\\
1.14557257257257	0.632889843332884\\
1.14357157157157	0.639114162834241\\
1.14157057057057	0.645454207916129\\
1.13956956956957	0.651912741590783\\
1.13756856856857	0.658492592766608\\
1.13556756756757	0.665196656835571\\
1.13356656656657	0.672027896137656\\
1.13156556556557	0.67898934028578\\
1.12956456456456	0.686084086333095\\
1.12756356356356	0.693315298762961\\
1.12556256256256	0.700686209280218\\
1.12356156156156	0.708200116380513\\
1.12156056056056	0.715860384672585\\
1.11955955955956	0.723670443926364\\
1.11755855855856	0.73163378781766\\
1.11555755755756	0.739753972338002\\
1.11355655655656	0.748034613835973\\
1.11155555555556	0.756479386653977\\
1.10955455455455	0.765092020322066\\
1.10755355355355	0.773876296267979\\
1.10555255255255	0.782836044000189\\
1.10355155155155	0.791975136718327\\
1.10155055055055	0.801297486303024\\
1.09954954954955	0.810807037635046\\
1.09754854854855	0.820507762191481\\
1.09554754754755	0.830403650864957\\
1.09354654654655	0.840498705950312\\
1.09154554554555	0.850796932241954\\
1.08954454454454	0.861302327184478\\
1.08754354354354	0.87201887001894\\
1.08554254254254	0.882950509867741\\
1.08354154154154	0.894101152702422\\
1.08154054054054	0.905474647140931\\
1.07953953953954	0.91707476902431\\
1.07753853853854	0.928905204727374\\
1.07553753753754	0.940969533163908\\
1.07353653653654	0.953271206454607\\
1.07153553553554	0.965813529235197\\
1.06953453453453	0.978599636593509\\
1.06753353353353	0.991632470637553\\
1.06553253253253	1.00491475571217\\
1.06353153153153	1.01844897229973\\
1.06153053053053	1.03223732966048\\
1.05952952952953	1.0462817372912\\
1.05752852852853	1.0605837753056\\
1.05552752752753	1.07514466386783\\
1.05352652652653	1.08996523184012\\
1.05152552552553	1.10504588483737\\
1.04952452452452	1.12038657291498\\
1.04752352352352	1.13598675815077\\
1.04552252252252	1.15184538241718\\
1.04352152152152	1.1679608356747\\
1.04152052052052	1.18433092515202\\
1.03951951951952	1.20095284581013\\
1.03751851851852	1.21782315251699\\
1.03551751751752	1.23493773438423\\
1.03351651651652	1.25229179173672\\
1.03151551551552	1.26987981619861\\
1.02951451451451	1.28769557438398\\
1.02751351351351	1.30573209567558\\
1.02551251251251	1.32398166456007\\
1.02351151151151	1.34243581796184\\
1.02151051051051	1.36108534797889\\
1.01950950950951	1.3799203103739\\
1.01750850850851	1.39893003911177\\
1.01550750750751	1.41810316714225\\
1.01350650650651	1.43742765362121\\
1.01150550550551	1.45689081745632\\
1.0095045045045	1.4764793773157\\
1.0075035035035	1.49617949774302\\
1.0055025025025	1.51597684114222\\
1.0035015015015	1.53585662531656\\
1.0015005005005	1.5558036858778\\
0.9994994994995	1.57580254310804\\
0.997498498498499	1.59583747241556\\
0.995497497497498	1.61589257770013\\
0.993496496496497	1.63595186673656\\
0.991495495495496	1.65599932772357\\
0.989494494494495	1.67601900604206\\
0.987493493493494	1.69599508035874\\
0.985492492492492	1.71591193711438\\
0.983491491491492	1.73575424258509\\
0.981490490490491	1.75550701164341\\
0.97948948948949	1.77515567253914\\
0.977488488488489	1.79468612699533\\
0.975487487487487	1.81408480504725\\
0.973486486486487	1.83333871424548\\
0.971485485485486	1.8524354827682\\
0.969484484484485	1.87136339627559\\
0.967483483483484	1.89011142839614\\
0.965482482482482	1.908669264734\\
0.963481481481481	1.92702732060574\\
0.961480480480481	1.94517675255172\\
0.95947947947948	1.96310946399977\\
0.957478478478479	1.98081810527932\\
0.955477477477477	1.99829606843894\\
0.953476476476476	2.01553747733488\\
0.951475475475476	2.03253717331755\\
0.949474474474475	2.04929069710375\\
0.947473473473474	2.06579426724985\\
0.945472472472472	2.0820447558087\\
0.943471471471471	2.09803966151142\\
0.941470470470471	2.11377708109967\\
0.93946946946947	2.12925567904709\\
0.937468468468469	2.1444746563027\\
0.935467467467467	2.15943371825821\\
0.933466466466467	2.1741330424148\\
0.931465465465466	2.18857324593274\\
0.929464464464465	2.20275535353719\\
0.927463463463464	2.21668076580052\\
0.925462462462462	2.23035122812945\\
0.923461461461462	2.24376880068908\\
0.92146046046046	2.25693582927147\\
0.91945945945946	2.26985491733216\\
0.917458458458459	2.28252889922744\\
0.915457457457457	2.29496081477511\\
0.913456456456457	2.30715388511769\\
0.911455455455455	2.31911148995568\\
0.909454454454455	2.33083714615998\\
0.907453453453454	2.34233448775827\\
0.905452452452452	2.35360724726105\\
0.903451451451452	2.36465923829138\\
0.90145045045045	2.37549433954705\\
0.899449449449449	2.38611647996455\\
0.897448448448449	2.39652962510408\\
0.895447447447447	2.40673776467613\\
0.893446446446447	2.41674490113274\\
0.891445445445445	2.42655503934425\\
0.889444444444444	2.43617217719477\\
0.887443443443444	2.44560029713964\\
0.885442442442442	2.45484335859522\\
0.883441441441442	2.4639052911574\\
0.88144044044044	2.47278998854796\\
0.879439439439439	2.48150130327174\\
0.877438438438438	2.49004304192399\\
0.875437437437438	2.49841896108285\\
0.873436436436437	2.50663276374734\\
0.871435435435435	2.51468809630804\\
0.869434434434434	2.52258854594058\\
0.867433433433433	2.53033763847437\\
0.865432432432433	2.53793883658904\\
0.863431431431432	2.54539553841015\\
0.86143043043043	2.55271107639875\\
0.859429429429429	2.5598887165492\\
0.857428428428428	2.56693165781893\\
0.855427427427428	2.57384303182049\\
0.853426426426427	2.58062590270886\\
0.851425425425425	2.58728326727017\\
0.849424424424424	2.59381805514824\\
0.847423423423423	2.60023312926248\\
0.845422422422422	2.60653128632129\\
0.843421421421422	2.6127152574732\\
0.84142042042042	2.61878770903603\\
0.839419419419419	2.62475124333713\\
0.837418418418418	2.63060839962242\\
0.835417417417417	2.63636165501818\\
0.833416416416417	2.64201342556559\\
0.831415415415415	2.64756606727907\\
0.829414414414414	2.65302187726672\\
0.827413413413413	2.65838309485962\\
0.825412412412412	2.66365190277893\\
0.823411411411412	2.66883042831547\\
0.82141041041041	2.67392074451535\\
0.819409409409409	2.67892487138312\\
0.817408408408409	2.68384477708609\\
0.815407407407407	2.68868237914732\\
0.813406406406406	2.69343954565538\\
0.811405405405405	2.69811809644331\\
0.809404404404404	2.70271980427799\\
0.807403403403404	2.70724639602516\\
0.805402402402402	2.71169955380869\\
0.803401401401401	2.71608091615078\\
0.8014004004004	2.72039207909304\\
0.799399399399399	2.72463459730597\\
0.797398398398398	2.72880998517508\\
0.795397397397397	2.73291971786505\\
0.793396396396396	2.73696523237187\\
0.791395395395395	2.7409479285421\\
0.789394394394394	2.74486917007949\\
0.787393393393393	2.74873028552773\\
0.785392392392392	2.75253256922838\\
0.783391391391391	2.75627728225684\\
0.78139039039039	2.75996565334137\\
0.779389389389389	2.76359887975402\\
0.777388388388388	2.76717812818005\\
0.775387387387387	2.77070453556757\\
0.773386386386386	2.77417920995657\\
0.771385385385385	2.77760323128025\\
0.769384384384384	2.78097765215268\\
0.767383383383383	2.78430349862988\\
0.765382382382382	2.78758177095165\\
0.763381381381381	2.79081344426529\\
0.76138038038038	2.7939994693266\\
0.759379379379379	2.7971407731836\\
0.757378378378378	2.80023825983739\\
0.755377377377377	2.80329281088917\\
0.753376376376376	2.80630528616598\\
0.751375375375375	2.80927652432777\\
0.749374374374374	2.81220734346024\\
0.747373373373373	2.81509854164724\\
0.745372372372372	2.81795089752765\\
0.743371371371371	2.82076517083857\\
0.74137037037037	2.823542102937\\
0.739369369369369	2.82628241731301\\
0.737368368368368	2.82898682008237\\
0.735367367367367	2.83165600046733\\
0.733366366366366	2.8342906312621\\
0.731365365365365	2.83689136928789\\
0.729364364364364	2.83945885582761\\
0.727363363363363	2.84199371705514\\
0.725362362362362	2.84449656444697\\
0.723361361361361	2.8469679951843\\
0.72136036036036	2.84940859254131\\
0.719359359359359	2.85181892626383\\
0.717358358358358	2.85419955293389\\
0.715357357357357	2.85655101632856\\
0.713356356356356	2.85887384776137\\
0.711355355355355	2.86116856641979\\
0.709354354354354	2.86343567968856\\
0.707353353353353	2.86567568346602\\
0.705352352352352	2.86788906246986\\
0.703351351351351	2.87007629053294\\
0.70135035035035	2.87223783089368\\
0.699349349349349	2.87437413647384\\
0.697348348348348	2.87648565015084\\
0.695347347347347	2.8785728050214\\
0.693346346346346	2.88063602465676\\
0.691345345345345	2.88267572335171\\
0.689344344344344	2.88469230636507\\
0.687343343343343	2.88668617015361\\
0.685342342342342	2.88865770260078\\
0.683341341341341	2.89060728323467\\
0.68134034034034	2.89253528344492\\
0.679339339339339	2.89444206668894\\
0.677338338338338	2.89632798869462\\
0.675337337337337	2.89819339765698\\
0.673336336336336	2.90003863442777\\
0.671335335335335	2.90186403270136\\
0.669334334334334	2.90366991919453\\
0.667333333333333	2.90545661382046\\
0.665332332332332	2.90722442985957\\
0.663331331331331	2.90897367412331\\
0.66133033033033	2.91070464711574\\
0.659329329329329	2.91241764318759\\
0.657328328328328	2.91411295068922\\
0.655327327327327	2.91579085211677\\
0.653326326326326	2.91745162425548\\
0.651325325325325	2.91909553831885\\
0.649324324324324	2.92072286008305\\
0.647323323323323	2.92233385001934\\
0.645322322322322	2.92392876342104\\
0.643321321321321	2.92550785052798\\
0.64132032032032	2.92707135664764\\
0.639319319319319	2.92861952227201\\
0.637318318318318	2.93015258319248\\
0.635317317317317	2.9316707706105\\
0.633316316316316	2.93317431124619\\
0.631315315315315	2.93466342744329\\
0.629314314314314	2.93613833727148\\
0.627313313313313	2.93759925462619\\
0.625312312312312	2.93904638932551\\
0.623311311311311	2.94047994720443\\
0.62131031031031	2.94190013020664\\
0.619309309309309	2.94330713647421\\
0.617308308308308	2.94470116043428\\
0.615307307307307	2.94608239288395\\
0.613306306306306	2.9474510210728\\
0.611305305305305	2.94880722878295\\
0.609304304304304	2.95015119640784\\
0.607303303303303	2.95148310102766\\
0.605302302302302	2.95280311648428\\
0.603301301301301	2.95411141345288\\
0.6013003003003	2.95540815951296\\
0.599299299299299	2.95669351921647\\
0.597298298298298	2.95796765415492\\
0.595297297297297	2.95923072302435\\
0.593296296296296	2.96048288168882\\
0.591295295295295	2.96172428324233\\
0.589294294294294	2.96295507806911\\
0.587293293293293	2.96417541390227\\
0.585292292292292	2.96538543588119\\
0.583291291291291	2.96658528660729\\
0.58129029029029	2.96777510619857\\
0.579289289289289	2.96895503234254\\
0.577288288288288	2.97012520034818\\
0.575287287287287	2.97128574319634\\
0.573286286286286	2.97243679158892\\
0.571285285285285	2.97357847399709\\
0.569284284284284	2.97471091670787\\
0.567283283283283	2.97583424387024\\
0.565282282282282	2.97694857753925\\
0.563281281281281	2.9780540377199\\
0.56128028028028	2.97915074240942\\
0.559279279279279	2.9802388076389\\
0.557278278278278	2.98131834751344\\
0.555277277277277	2.98238947425195\\
0.553276276276276	2.98345229822534\\
0.551275275275275	2.98450692799448\\
0.549274274274274	2.9855534703466\\
0.547273273273273	2.98659203033142\\
0.545272272272272	2.98762271129596\\
0.543271271271271	2.98864561491884\\
0.54127027027027	2.98966084124356\\
0.539269269269269	2.99066848871124\\
0.537268268268268	2.99166865419238\\
0.535267267267267	2.99266143301801\\
0.533266266266266	2.99364691901002\\
0.531265265265265	2.99462520451094\\
0.529264264264264	2.99559638041287\\
0.527263263263263	2.99656053618585\\
0.525262262262262	2.99751775990564\\
0.523261261261261	2.9984681382807\\
0.52126026026026	2.99941175667858\\
0.519259259259259	3.00034869915201\\
0.517258258258258	3.00127904846398\\
0.515257257257257	3.00220288611248\\
0.513256256256256	3.00312029235476\\
0.511255255255255	3.00403134623082\\
0.509254254254254	3.00493612558658\\
0.507253253253253	3.0058347070965\\
0.505252252252252	3.00672716628553\\
0.503251251251251	3.00761357755089\\
0.50125025025025	3.00849401418306\\
0.499249249249249	3.00936854838648\\
0.497248248248248	3.01023725129979\\
0.495247247247247	3.01110019301558\\
0.493246246246246	3.01195744259983\\
0.491245245245245	3.01280906811059\\
0.489244244244244	3.01365513661684\\
0.487243243243243	3.01449571421636\\
0.485242242242242	3.01533086605355\\
0.483241241241241	3.01616065633688\\
0.48124024024024	3.01698514835573\\
0.479239239239239	3.01780440449711\\
0.477238238238238	3.01861848626195\\
0.475237237237237	3.01942745428098\\
0.473236236236236	3.02023136833042\\
0.471235235235235	3.02103028734718\\
0.469234234234234	3.02182426944388\\
0.467233233233233	3.02261337192344\\
0.465232232232232	3.02339765129351\\
0.463231231231231	3.02417716328045\\
0.46123023023023	3.02495196284313\\
0.459229229229229	3.02572210418639\\
0.457228228228228	3.02648764077429\\
0.455227227227227	3.02724862534296\\
0.453226226226226	3.02800510991337\\
0.451225225225225	3.02875714580369\\
0.449224224224224	3.0295047836415\\
0.447223223223223	3.03024807337564\\
0.445222222222222	3.03098706428797\\
0.443221221221221	3.03172180500479\\
0.44122022022022	3.03245234350805\\
0.439219219219219	3.03317872714636\\
0.437218218218218	3.03390100264579\\
0.435217217217217	3.0346192161204\\
0.433216216216216	3.03533341308264\\
0.431215215215215	3.03604363845347\\
0.429214214214214	3.03674993657237\\
0.427213213213213	3.03745235120704\\
0.425212212212212	3.03815092556305\\
0.423211211211211	3.03884570229314\\
0.42121021021021	3.03953672350654\\
0.419209209209209	3.0402240307779\\
0.417208208208208	3.04090766515619\\
0.415207207207207	3.04158766717342\\
0.413206206206206	3.04226407685311\\
0.411205205205205	3.04293693371869\\
0.409204204204204	3.04360627680166\\
0.407203203203203	3.04427214464971\\
0.405202202202202	3.04493457533454\\
0.403201201201201	3.04559360645968\\
0.4012002002002	3.04624927516801\\
0.399199199199199	3.04690161814931\\
0.397198198198198	3.04755067164752\\
0.395197197197197	3.04819647146797\\
0.393196196196196	3.04883905298441\\
0.391195195195195	3.04947845114593\\
0.389194194194194	3.05011470048379\\
0.387193193193193	3.05074783511807\\
0.385192192192192	3.05137788876422\\
0.383191191191191	3.05200489473951\\
0.38119019019019	3.05262888596931\\
0.379189189189189	3.05324989499336\\
0.377188188188188	3.0538679539718\\
0.375187187187187	3.05448309469115\\
0.373186186186186	3.05509534857026\\
0.371185185185185	3.05570474666596\\
0.369184184184184	3.05631131967884\\
0.367183183183183	3.05691509795877\\
0.365182182182182	3.05751611151033\\
0.363181181181181	3.05811438999826\\
0.36118018018018	3.05870996275267\\
0.359179179179179	3.05930285877428\\
0.357178178178178	3.05989310673946\\
0.355177177177177	3.06048073500529\\
0.353176176176176	3.06106577161445\\
0.351175175175175	3.06164824430004\\
0.349174174174174	3.06222818049038\\
0.347173173173173	3.06280560731364\\
0.345172172172172	3.06338055160244\\
0.343171171171171	3.06395303989836\\
0.34117017017017	3.06452309845639\\
0.339169169169169	3.06509075324927\\
0.337168168168168	3.06565602997181\\
0.335167167167167	3.06621895404504\\
0.333166166166166	3.06677955062043\\
0.331165165165165	3.06733784458393\\
0.329164164164164	3.06789386055994\\
0.327163163163163	3.06844762291532\\
0.325162162162162	3.06899915576324\\
0.323161161161161	3.06954848296696\\
0.32116016016016	3.0700956281436\\
0.319159159159159	3.07064061466785\\
0.317158158158158	3.07118346567557\\
0.315157157157157	3.07172420406735\\
0.313156156156156	3.07226285251206\\
0.311155155155155	3.07279943345026\\
0.309154154154154	3.07333396909761\\
0.307153153153153	3.07386648144826\\
0.305152152152152	3.07439699227806\\
0.303151151151151	3.07492552314785\\
0.30115015015015	3.07545209540664\\
0.299149149149149	3.07597673019474\\
0.297148148148148	3.07649944844685\\
0.295147147147147	3.0770202708951\\
0.293146146146146	3.07753921807202\\
0.291145145145145	3.07805631031351\\
0.289144144144144	3.07857156776174\\
0.287143143143143	3.07908501036798\\
0.285142142142142	3.07959665789542\\
0.283141141141141	3.08010652992196\\
0.28114014014014	3.08061464584289\\
0.279139139139139	3.08112102487361\\
0.277138138138138	3.08162568605225\\
0.275137137137137	3.0821286482423\\
0.273136136136136	3.08262993013512\\
0.271135135135135	3.08312955025254\\
0.269134134134134	3.08362752694926\\
0.267133133133133	3.08412387841538\\
0.265132132132132	3.08461862267877\\
0.263131131131131	3.08511177760744\\
0.26113013013013	3.08560336091192\\
0.259129129129129	3.08609339014755\\
0.257128128128128	3.08658188271673\\
0.255127127127127	3.08706885587121\\
0.253126126126126	3.08755432671426\\
0.251125125125125	3.08803831220288\\
0.249124124124124	3.0885208291499\\
0.247123123123123	3.08900189422616\\
0.245122122122122	3.08948152396255\\
0.243121121121121	3.08995973475206\\
0.24112012012012	3.09043654285187\\
0.239119119119119	3.09091196438526\\
0.237118118118118	3.09138601534366\\
0.235117117117117	3.09185871158854\\
0.233116116116116	3.09233006885336\\
0.231115115115115	3.09280010274546\\
0.229114114114114	3.09326882874789\\
0.227113113113113	3.0937362622213\\
0.225112112112112	3.09420241840572\\
0.223111111111111	3.09466731242235\\
0.22111011011011	3.09513095927536\\
0.219109109109109	3.09559337385358\\
0.217108108108108	3.09605457093228\\
0.215107107107107	3.09651456517481\\
0.213106106106106	3.09697337113431\\
0.211105105105105	3.09743100325537\\
0.209104104104104	3.09788747587563\\
0.207103103103103	3.0983428032274\\
0.205102102102102	3.09879699943928\\
0.203101101101101	3.09925007853768\\
0.2011001001001	3.0997020544484\\
0.199099099099099	3.10015294099818\\
0.197098098098098	3.10060275191616\\
0.195097097097097	3.1010515008354\\
0.193096096096096	3.10149920129437\\
0.191095095095095	3.10194586673837\\
0.189094094094094	3.102391510521\\
0.187093093093093	3.10283614590556\\
0.185092092092092	3.10327978606645\\
0.183091091091091	3.10372244409058\\
0.18109009009009	3.10416413297872\\
0.179089089089089	3.10460486564686\\
0.177088088088088	3.10504465492755\\
0.175087087087087	3.10548351357122\\
0.173086086086086	3.10592145424747\\
0.171085085085085	3.1063584895464\\
0.169084084084084	3.10679463197987\\
0.167083083083083	3.10722989398274\\
0.165082082082082	3.10766428791414\\
0.163081081081081	3.10809782605872\\
0.16108008008008	3.10853052062784\\
0.159079079079079	3.10896238376079\\
0.157078078078078	3.109393427526\\
0.155077077077077	3.10982366392219\\
0.153076076076076	3.11025310487958\\
0.151075075075075	3.110681762261\\
0.149074074074074	3.11110964786309\\
0.147073073073073	3.11153677341738\\
0.145072072072072	3.11196315059144\\
0.143071071071071	3.11238879098998\\
0.14107007007007	3.11281370615597\\
0.139069069069069	3.1132379075717\\
0.137068068068068	3.11366140665986\\
0.135067067067067	3.11408421478463\\
0.133066066066066	3.11450634325271\\
0.131065065065065	3.11492780331438\\
0.129064064064064	3.11534860616453\\
0.127063063063063	3.11576876294371\\
0.125062062062062	3.11618828473911\\
0.123061061061061	3.11660718258558\\
0.12106006006006	3.11702546746664\\
0.119059059059059	3.11744315031547\\
0.117058058058058	3.11786024201588\\
0.115057057057057	3.11827675340329\\
0.113056056056056	3.11869269526569\\
0.111055055055055	3.1191080783446\\
0.109054054054054	3.119522913336\\
0.107053053053053	3.11993721089132\\
0.105052052052052	3.12035098161831\\
0.103051051051051	3.12076423608201\\
0.10105005005005	3.12117698480564\\
0.0990490490490491	3.12158923827155\\
0.0970480480480481	3.12200100692207\\
0.0950470470470471	3.12241230116045\\
0.093046046046046	3.12282313135173\\
0.091045045045045	3.12323350782364\\
0.089044044044044	3.12364344086745\\
0.0870430430430431	3.12405294073888\\
0.085042042042042	3.12446201765895\\
0.083041041041041	3.12487068181481\\
0.08104004004004	3.12527894336064\\
0.079039039039039	3.12568681241848\\
0.077038038038038	3.12609429907907\\
0.075037037037037	3.12650141340269\\
0.073036036036036	3.12690816542\\
0.071035035035035	3.12731456513286\\
0.069034034034034	3.12772062251516\\
0.067033033033033	3.12812634751363\\
0.065032032032032	3.12853175004868\\
0.063031031031031	3.12893684001516\\
0.06103003003003	3.12934162728322\\
0.059029029029029	3.12974612169909\\
0.057028028028028	3.13015033308587\\
0.055027027027027	3.13055427124433\\
0.053026026026026	3.13095794595371\\
0.051025025025025	3.13136136697249\\
0.049024024024024	3.13176454403918\\
0.047023023023023	3.13216748687311\\
0.045022022022022	3.13257020517521\\
0.043021021021021	3.13297270862875\\
0.04102002002002	3.13337500690017\\
0.039019019019019	3.13377710963977\\
0.037018018018018	3.13417902648256\\
0.035017017017017	3.13458076704896\\
0.033016016016016	3.1349823409456\\
0.031015015015015	3.13538375776605\\
0.029014014014014	3.13578502709159\\
0.027013013013013	3.13618615849199\\
0.025012012012012	3.13658716152621\\
0.023011011011011	3.1369880457432\\
0.02101001001001	3.1373888206826\\
0.019009009009009	3.13778949587555\\
0.017008008008008	3.13819008084538\\
0.015007007007007	3.1385905851084\\
0.013006006006006	3.1389910181746\\
0.011005005005005	3.13939138954844\\
0.009004004004004	3.13979170872954\\
0.007003003003003	3.14019198521348\\
0.005002002002002	3.14059222849248\\
0.003001001001001	3.14099244805621\\
0.001	3.14139265339246\\
0.003001001001001	3.14099244805621\\
0.005002002002002	3.14059222849248\\
0.007003003003003	3.14019198521348\\
0.009004004004004	3.13979170872954\\
0.011005005005005	3.13939138954844\\
0.013006006006006	3.1389910181746\\
0.015007007007007	3.1385905851084\\
0.017008008008008	3.13819008084538\\
0.019009009009009	3.13778949587555\\
0.02101001001001	3.1373888206826\\
0.023011011011011	3.1369880457432\\
0.025012012012012	3.13658716152621\\
0.027013013013013	3.13618615849199\\
0.029014014014014	3.13578502709159\\
0.031015015015015	3.13538375776605\\
0.033016016016016	3.1349823409456\\
0.035017017017017	3.13458076704896\\
0.037018018018018	3.13417902648256\\
0.039019019019019	3.13377710963977\\
0.04102002002002	3.13337500690017\\
0.043021021021021	3.13297270862875\\
0.045022022022022	3.13257020517521\\
0.047023023023023	3.13216748687311\\
0.049024024024024	3.13176454403918\\
0.051025025025025	3.13136136697249\\
0.053026026026026	3.13095794595371\\
0.055027027027027	3.13055427124433\\
0.057028028028028	3.13015033308587\\
0.059029029029029	3.12974612169909\\
0.06103003003003	3.12934162728322\\
0.063031031031031	3.12893684001516\\
0.065032032032032	3.12853175004868\\
0.067033033033033	3.12812634751363\\
0.069034034034034	3.12772062251516\\
0.071035035035035	3.12731456513286\\
0.073036036036036	3.12690816542\\
0.075037037037037	3.12650141340269\\
0.077038038038038	3.12609429907907\\
0.079039039039039	3.12568681241848\\
0.08104004004004	3.12527894336064\\
0.083041041041041	3.12487068181481\\
0.085042042042042	3.12446201765895\\
0.0870430430430431	3.12405294073888\\
0.089044044044044	3.12364344086745\\
0.091045045045045	3.12323350782364\\
0.093046046046046	3.12282313135173\\
0.0950470470470471	3.12241230116045\\
0.0970480480480481	3.12200100692207\\
0.0990490490490491	3.12158923827155\\
0.10105005005005	3.12117698480564\\
0.103051051051051	3.12076423608201\\
0.105052052052052	3.12035098161831\\
0.107053053053053	3.11993721089132\\
0.109054054054054	3.119522913336\\
0.111055055055055	3.1191080783446\\
0.113056056056056	3.11869269526569\\
0.115057057057057	3.11827675340329\\
0.117058058058058	3.11786024201588\\
0.119059059059059	3.11744315031547\\
0.12106006006006	3.11702546746664\\
0.123061061061061	3.11660718258558\\
0.125062062062062	3.11618828473911\\
0.127063063063063	3.11576876294371\\
0.129064064064064	3.11534860616453\\
0.131065065065065	3.11492780331438\\
0.133066066066066	3.11450634325271\\
0.135067067067067	3.11408421478463\\
0.137068068068068	3.11366140665986\\
0.139069069069069	3.1132379075717\\
0.14107007007007	3.11281370615597\\
0.143071071071071	3.11238879098998\\
0.145072072072072	3.11196315059144\\
0.147073073073073	3.11153677341738\\
0.149074074074074	3.11110964786309\\
0.151075075075075	3.110681762261\\
0.153076076076076	3.11025310487958\\
0.155077077077077	3.10982366392219\\
0.157078078078078	3.109393427526\\
0.159079079079079	3.10896238376079\\
0.16108008008008	3.10853052062784\\
0.163081081081081	3.10809782605872\\
0.165082082082082	3.10766428791414\\
0.167083083083083	3.10722989398274\\
0.169084084084084	3.10679463197987\\
0.171085085085085	3.10635848954641\\
0.173086086086086	3.10592145424747\\
0.175087087087087	3.10548351357122\\
0.177088088088088	3.10504465492755\\
0.179089089089089	3.10460486564686\\
0.18109009009009	3.10416413297872\\
0.183091091091091	3.10372244409058\\
0.185092092092092	3.10327978606645\\
0.187093093093093	3.10283614590556\\
0.189094094094094	3.102391510521\\
0.191095095095095	3.10194586673837\\
0.193096096096096	3.10149920129437\\
0.195097097097097	3.1010515008354\\
0.197098098098098	3.10060275191616\\
0.199099099099099	3.10015294099818\\
0.2011001001001	3.0997020544484\\
0.203101101101101	3.09925007853768\\
0.205102102102102	3.09879699943928\\
0.207103103103103	3.0983428032274\\
0.209104104104104	3.09788747587563\\
0.211105105105105	3.09743100325537\\
0.213106106106106	3.09697337113431\\
0.215107107107107	3.09651456517481\\
0.217108108108108	3.09605457093228\\
0.219109109109109	3.09559337385358\\
0.22111011011011	3.09513095927536\\
0.223111111111111	3.09466731242235\\
0.225112112112112	3.09420241840572\\
0.227113113113113	3.0937362622213\\
0.229114114114114	3.09326882874789\\
0.231115115115115	3.09280010274546\\
0.233116116116116	3.09233006885336\\
0.235117117117117	3.09185871158854\\
0.237118118118118	3.09138601534365\\
0.239119119119119	3.09091196438526\\
0.24112012012012	3.09043654285187\\
0.243121121121121	3.08995973475206\\
0.245122122122122	3.08948152396255\\
0.247123123123123	3.08900189422616\\
0.249124124124124	3.0885208291499\\
0.251125125125125	3.08803831220287\\
0.253126126126126	3.08755432671426\\
0.255127127127127	3.08706885587121\\
0.257128128128128	3.08658188271673\\
0.259129129129129	3.08609339014755\\
0.26113013013013	3.08560336091192\\
0.263131131131131	3.08511177760744\\
0.265132132132132	3.08461862267877\\
0.267133133133133	3.08412387841538\\
0.269134134134134	3.08362752694926\\
0.271135135135135	3.08312955025254\\
0.273136136136136	3.08262993013512\\
0.275137137137137	3.0821286482423\\
0.277138138138138	3.08162568605225\\
0.279139139139139	3.08112102487361\\
0.28114014014014	3.08061464584289\\
0.283141141141141	3.08010652992196\\
0.285142142142142	3.07959665789542\\
0.287143143143143	3.07908501036798\\
0.289144144144144	3.07857156776174\\
0.291145145145145	3.07805631031351\\
0.293146146146146	3.07753921807202\\
0.295147147147147	3.0770202708951\\
0.297148148148148	3.07649944844685\\
0.299149149149149	3.07597673019474\\
0.30115015015015	3.07545209540664\\
0.303151151151151	3.07492552314785\\
0.305152152152152	3.07439699227806\\
0.307153153153153	3.07386648144826\\
0.309154154154154	3.07333396909761\\
0.311155155155155	3.07279943345026\\
0.313156156156156	3.07226285251206\\
0.315157157157157	3.07172420406735\\
0.317158158158158	3.07118346567557\\
0.319159159159159	3.07064061466785\\
0.32116016016016	3.0700956281436\\
0.323161161161161	3.06954848296696\\
0.325162162162162	3.06899915576324\\
0.327163163163163	3.06844762291532\\
0.329164164164164	3.06789386055994\\
0.331165165165165	3.06733784458392\\
0.333166166166166	3.06677955062043\\
0.335167167167167	3.06621895404504\\
0.337168168168168	3.06565602997181\\
0.339169169169169	3.06509075324928\\
0.34117017017017	3.06452309845639\\
0.343171171171171	3.06395303989836\\
0.345172172172172	3.06338055160244\\
0.347173173173173	3.06280560731364\\
0.349174174174174	3.06222818049038\\
0.351175175175175	3.06164824430004\\
0.353176176176176	3.06106577161445\\
0.355177177177177	3.06048073500529\\
0.357178178178178	3.05989310673946\\
0.359179179179179	3.05930285877427\\
0.36118018018018	3.05870996275267\\
0.363181181181181	3.05811438999826\\
0.365182182182182	3.05751611151033\\
0.367183183183183	3.05691509795877\\
0.369184184184184	3.05631131967885\\
0.371185185185185	3.05570474666596\\
0.373186186186186	3.05509534857025\\
0.375187187187187	3.05448309469115\\
0.377188188188188	3.0538679539718\\
0.379189189189189	3.05324989499336\\
0.38119019019019	3.05262888596931\\
0.383191191191191	3.05200489473951\\
0.385192192192192	3.05137788876422\\
0.387193193193193	3.05074783511807\\
0.389194194194194	3.05011470048379\\
0.391195195195195	3.04947845114593\\
0.393196196196196	3.04883905298441\\
0.395197197197197	3.04819647146797\\
0.397198198198198	3.04755067164752\\
0.399199199199199	3.04690161814931\\
0.4012002002002	3.04624927516801\\
0.403201201201201	3.04559360645968\\
0.405202202202202	3.04493457533455\\
0.407203203203203	3.04427214464971\\
0.409204204204204	3.04360627680166\\
0.411205205205205	3.04293693371868\\
0.413206206206206	3.04226407685311\\
0.415207207207207	3.04158766717342\\
0.417208208208208	3.04090766515619\\
0.419209209209209	3.04022403077789\\
0.42121021021021	3.03953672350654\\
0.423211211211211	3.03884570229315\\
0.425212212212212	3.03815092556304\\
0.427213213213213	3.03745235120704\\
0.429214214214214	3.03674993657236\\
0.431215215215215	3.03604363845347\\
0.433216216216216	3.03533341308264\\
0.435217217217217	3.0346192161204\\
0.437218218218218	3.03390100264579\\
0.439219219219219	3.03317872714636\\
0.44122022022022	3.03245234350806\\
0.443221221221221	3.03172180500479\\
0.445222222222222	3.03098706428797\\
0.447223223223223	3.03024807337564\\
0.449224224224224	3.0295047836415\\
0.451225225225225	3.0287571458037\\
0.453226226226226	3.02800510991337\\
0.455227227227227	3.02724862534295\\
0.457228228228228	3.02648764077428\\
0.459229229229229	3.02572210418639\\
0.46123023023023	3.02495196284313\\
0.463231231231231	3.02417716328045\\
0.465232232232232	3.02339765129351\\
0.467233233233233	3.02261337192344\\
0.469234234234234	3.02182426944388\\
0.471235235235235	3.02103028734719\\
0.473236236236236	3.02023136833042\\
0.475237237237237	3.01942745428099\\
0.477238238238238	3.01861848626195\\
0.479239239239239	3.01780440449712\\
0.48124024024024	3.01698514835573\\
0.483241241241241	3.01616065633689\\
0.485242242242242	3.01533086605356\\
0.487243243243243	3.01449571421634\\
0.489244244244244	3.01365513661684\\
0.491245245245245	3.01280906811059\\
0.493246246246246	3.01195744259982\\
0.495247247247247	3.0111001930156\\
0.497248248248248	3.01023725129979\\
0.499249249249249	3.00936854838647\\
0.50125025025025	3.00849401418305\\
0.503251251251251	3.0076135775509\\
0.505252252252252	3.00672716628554\\
0.507253253253253	3.00583470709649\\
0.509254254254254	3.00493612558659\\
0.511255255255255	3.00403134623082\\
0.513256256256256	3.00312029235477\\
0.515257257257257	3.0022028861125\\
0.517258258258258	3.00127904846399\\
0.519259259259259	3.00034869915202\\
0.52126026026026	2.99941175667858\\
0.523261261261261	2.99846813828069\\
0.525262262262262	2.99751775990565\\
0.527263263263263	2.99656053618585\\
0.529264264264264	2.99559638041285\\
0.531265265265265	2.99462520451094\\
0.533266266266266	2.99364691901003\\
0.535267267267267	2.99266143301801\\
0.537268268268268	2.99166865419239\\
0.539269269269269	2.99066848871123\\
0.54127027027027	2.98966084124354\\
0.543271271271271	2.98864561491882\\
0.545272272272272	2.98762271129596\\
0.547273273273273	2.98659203033143\\
0.549274274274274	2.98555347034661\\
0.551275275275275	2.98450692799447\\
0.553276276276276	2.98345229822536\\
0.555277277277277	2.98238947425195\\
0.557278278278278	2.98131834751347\\
0.559279279279279	2.9802388076389\\
0.56128028028028	2.97915074240943\\
0.563281281281281	2.97805403771987\\
0.565282282282282	2.97694857753924\\
0.567283283283283	2.97583424387024\\
0.569284284284284	2.9747109167079\\
0.571285285285285	2.97357847399708\\
0.573286286286286	2.97243679158891\\
0.575287287287287	2.97128574319631\\
0.577288288288288	2.97012520034817\\
0.579289289289289	2.96895503234253\\
0.58129029029029	2.96777510619856\\
0.583291291291291	2.9665852866073\\
0.585292292292292	2.9653854358812\\
0.587293293293293	2.96417541390228\\
0.589294294294294	2.96295507806911\\
0.591295295295295	2.96172428324233\\
0.593296296296296	2.96048288168881\\
0.595297297297297	2.95923072302433\\
0.597298298298298	2.95796765415494\\
0.599299299299299	2.9566935192165\\
0.6013003003003	2.95540815951296\\
0.603301301301301	2.95411141345288\\
0.605302302302302	2.95280311648424\\
0.607303303303303	2.95148310102768\\
0.609304304304304	2.95015119640782\\
0.611305305305305	2.94880722878298\\
0.613306306306306	2.94745102107278\\
0.615307307307307	2.94608239288396\\
0.617308308308308	2.94470116043429\\
0.619309309309309	2.94330713647422\\
0.62131031031031	2.94190013020665\\
0.623311311311311	2.9404799472044\\
0.625312312312312	2.93904638932553\\
0.627313313313313	2.93759925462621\\
0.629314314314314	2.93613833727147\\
0.631315315315315	2.93466342744329\\
0.633316316316316	2.93317431124621\\
0.635317317317317	2.93167077061051\\
0.637318318318318	2.93015258319244\\
0.639319319319319	2.92861952227201\\
0.64132032032032	2.92707135664763\\
0.643321321321321	2.92550785052798\\
0.645322322322322	2.92392876342098\\
0.647323323323323	2.92233385001933\\
0.649324324324324	2.92072286008304\\
0.651325325325325	2.91909553831882\\
0.653326326326326	2.91745162425556\\
0.655327327327327	2.91579085211678\\
0.657328328328328	2.91411295068921\\
0.659329329329329	2.91241764318755\\
0.66133033033033	2.91070464711569\\
0.663331331331331	2.90897367412339\\
0.665332332332332	2.90722442985953\\
0.667333333333333	2.90545661382049\\
0.669334334334334	2.90366991919451\\
0.671335335335335	2.90186403270142\\
0.673336336336336	2.90003863442778\\
0.675337337337337	2.89819339765696\\
0.677338338338338	2.89632798869461\\
0.679339339339339	2.89444206668883\\
0.68134034034034	2.89253528344491\\
0.683341341341341	2.89060728323472\\
0.685342342342342	2.88865770260069\\
0.687343343343343	2.88668617015372\\
0.689344344344344	2.88469230636492\\
0.691345345345345	2.88267572335164\\
0.693346346346346	2.88063602465675\\
0.695347347347347	2.87857280502135\\
0.697348348348348	2.87648565015086\\
0.699349349349349	2.87437413647378\\
0.70135035035035	2.87223783089366\\
0.703351351351351	2.87007629053294\\
0.705352352352352	2.8678890624697\\
0.707353353353353	2.86567568346604\\
0.709354354354354	2.86343567968858\\
0.711355355355355	2.86116856641975\\
0.713356356356356	2.85887384776139\\
0.715357357357357	2.85655101632857\\
0.717358358358358	2.85419955293396\\
0.719359359359359	2.85181892626363\\
0.72136036036036	2.84940859254139\\
0.723361361361361	2.84696799518429\\
0.725362362362362	2.84449656444701\\
0.727363363363363	2.84199371705509\\
0.729364364364364	2.83945885582772\\
0.731365365365365	2.83689136928792\\
0.733366366366366	2.8342906312623\\
0.735367367367367	2.83165600046711\\
0.737368368368368	2.82898682008231\\
0.739369369369369	2.82628241731315\\
0.74137037037037	2.82354210293706\\
0.743371371371371	2.82076517083844\\
0.745372372372372	2.81795089752772\\
0.747373373373373	2.81509854164712\\
0.749374374374374	2.81220734346021\\
0.751375375375375	2.80927652432779\\
0.753376376376376	2.80630528616591\\
0.755377377377377	2.80329281088922\\
0.757378378378378	2.80023825983735\\
0.759379379379379	2.79714077318346\\
0.76138038038038	2.79399946932674\\
0.763381381381381	2.79081344426517\\
0.765382382382382	2.7875817709514\\
0.767383383383383	2.78430349862967\\
0.769384384384384	2.78097765215292\\
0.771385385385385	2.77760323128036\\
0.773386386386386	2.77417920995648\\
0.775387387387387	2.77070453556767\\
0.777388388388388	2.76717812817979\\
0.779389389389389	2.76359887975414\\
0.78139039039039	2.75996565334176\\
0.783391391391391	2.75627728225678\\
0.785392392392392	2.75253256922815\\
0.787393393393393	2.74873028552804\\
0.789394394394394	2.74486917007966\\
0.791395395395395	2.74094792854198\\
0.793396396396396	2.73696523237188\\
0.795397397397397	2.73291971786536\\
0.797398398398398	2.7288099851748\\
0.799399399399399	2.72463459730613\\
0.8014004004004	2.72039207909307\\
0.803401401401401	2.71608091615077\\
0.805402402402402	2.71169955380902\\
0.807403403403404	2.70724639602511\\
0.809404404404404	2.70271980427774\\
0.811405405405405	2.69811809644354\\
0.813406406406406	2.69343954565554\\
0.815407407407407	2.688682379148\\
0.817408408408409	2.6838447770855\\
0.819409409409409	2.67892487138353\\
0.82141041041041	2.67392074451521\\
0.823411411411412	2.66883042831556\\
0.825412412412412	2.66365190277938\\
0.827413413413413	2.65838309485902\\
0.829414414414414	2.65302187726612\\
0.831415415415415	2.64756606727911\\
0.833416416416417	2.64201342556566\\
0.835417417417417	2.63636165501906\\
0.837418418418418	2.63060839962186\\
0.839419419419419	2.62475124333612\\
0.84142042042042	2.61878770903466\\
0.843421421421422	2.61271525747323\\
0.845422422422422	2.60653128632299\\
0.847423423423423	2.60023312926223\\
0.849424424424424	2.59381805514728\\
0.851425425425425	2.58728326726838\\
0.853426426426427	2.58062590270946\\
0.855427427427428	2.5738430318197\\
0.857428428428428	2.56693165781866\\
0.859429429429429	2.55988871654974\\
0.86143043043043	2.55271107639999\\
0.863431431431432	2.54539553840912\\
0.865432432432433	2.53793883658872\\
0.867433433433433	2.53033763847482\\
0.869434434434434	2.52258854594157\\
0.871435435435435	2.51468809630545\\
0.873436436436437	2.5066327637466\\
0.875437437437438	2.49841896108178\\
0.877438438438438	2.49004304192464\\
0.879439439439439	2.48150130327144\\
0.88144044044044	2.47278998854683\\
0.883441441441442	2.46390529115791\\
0.885442442442442	2.45484335859633\\
0.887443443443444	2.44560029713968\\
0.889444444444444	2.4361721771945\\
0.891445445445445	2.42655503934315\\
0.893446446446447	2.41674490113233\\
0.895447447447447	2.40673776467446\\
0.897448448448449	2.39652962510562\\
0.899449449449449	2.38611647996501\\
0.90145045045045	2.3754943395472\\
0.903451451451452	2.36465923829223\\
0.905452452452452	2.3536072472614\\
0.907453453453454	2.3423344877604\\
0.909454454454455	2.3308371461594\\
0.911455455455455	2.31911148995435\\
0.913456456456457	2.30715388511884\\
0.915457457457457	2.29496081477727\\
0.917458458458459	2.28252889922664\\
0.91945945945946	2.26985491732875\\
0.92146046046046	2.25693582927058\\
0.923461461461462	2.24376880068864\\
0.925462462462462	2.23035122812985\\
0.927463463463464	2.21668076580102\\
0.929464464464465	2.20275535354195\\
0.931465465465466	2.18857324593182\\
0.933466466466467	2.17413304240898\\
0.935467467467467	2.15943371826129\\
0.937468468468469	2.14447465630534\\
0.93946946946947	2.12925567905119\\
0.941470470470471	2.11377708110172\\
0.943471471471471	2.09803966151297\\
0.945472472472472	2.08204475580148\\
0.947473473473474	2.06579426724575\\
0.949474474474475	2.04929069710074\\
0.951475475475476	2.03253717331982\\
0.953476476476476	2.01553747733437\\
0.955477477477477	1.99829606843571\\
0.957478478478479	1.98081810527506\\
0.95947947947948	1.96310946400339\\
0.961480480480481	1.94517675255561\\
0.963481481481481	1.92702732060488\\
0.965482482482482	1.90866926473453\\
0.967483483483484	1.89011142839464\\
0.969484484484485	1.87136339627577\\
0.971485485485486	1.85243548276606\\
0.973486486486487	1.83333871424612\\
0.975487487487487	1.81408480504874\\
0.977488488488489	1.79468612699495\\
0.97948948948949	1.77515567254207\\
0.981490490490491	1.7555070116444\\
0.983491491491492	1.73575424258552\\
0.985492492492492	1.71591193711604\\
0.987493493493494	1.69599508035887\\
0.989494494494495	1.67601900604269\\
0.991495495495496	1.65599932772314\\
0.993496496496497	1.63595186673716\\
0.995497497497498	1.61589257769982\\
0.997498498498499	1.59583747241597\\
0.9994994994995	1.57580254310816\\
1.0015005005005	1.55580368587808\\
1.0035015015015	1.53585662531611\\
1.0055025025025	1.51597684114512\\
1.0075035035035	1.49617949774214\\
1.0095045045045	1.47647937731788\\
1.01150550550551	1.45689081745585\\
1.01350650650651	1.43742765362012\\
1.01550750750751	1.41810316714581\\
1.01750850850851	1.39893003911018\\
1.01950950950951	1.37992031037385\\
1.02151051051051	1.36108534797822\\
1.02351151151151	1.34243581796205\\
1.02551251251251	1.32398166455994\\
1.02751351351351	1.30573209567626\\
1.02951451451451	1.28769557438493\\
1.03151551551552	1.26987981619895\\
1.03351651651652	1.2522917917375\\
1.03551751751752	1.23493773438592\\
1.03751851851852	1.21782315251821\\
1.03951951951952	1.20095284580965\\
1.04152052052052	1.18433092515328\\
1.04352152152152	1.16796083567433\\
1.04552252252252	1.15184538241445\\
1.04752352352352	1.13598675815048\\
1.04952452452452	1.12038657291538\\
1.05152552552553	1.10504588483313\\
1.05352652652653	1.08996523184254\\
1.05552752752753	1.07514466387103\\
1.05752852852853	1.06058377530694\\
1.05952952952953	1.04628173728946\\
1.06153053053053	1.03223732965871\\
1.06353153153153	1.01844897229936\\
1.06553253253253	1.00491475571407\\
1.06753353353353	0.991632470633991\\
1.06953453453453	0.97859963659687\\
1.07153553553554	0.965813529232984\\
1.07353653653654	0.953271206455268\\
1.07553753753754	0.940969533163603\\
1.07753853853854	0.928905204729994\\
1.07953953953954	0.917074769024029\\
1.08154054054054	0.905474647145299\\
1.08354154154154	0.894101152703482\\
1.08554254254254	0.882950509870768\\
1.08754354354354	0.872018870020422\\
1.08954454454454	0.861302327179482\\
1.09154554554555	0.850796932239104\\
1.09354654654655	0.84049870594945\\
1.09554754754755	0.830403650868458\\
1.09754854854855	0.820507762193135\\
1.09954954954955	0.810807037633617\\
1.10155055055055	0.801297486301803\\
1.10355155155155	0.791975136719675\\
1.10555255255255	0.782836043994573\\
1.10755355355355	0.7738762962668\\
1.10955455455455	0.765092020323392\\
1.11155555555556	0.756479386652764\\
1.11355655655656	0.748034613839555\\
1.11555755755756	0.739753972334547\\
1.11755855855856	0.731633787818405\\
1.11955955955956	0.723670443925751\\
1.12156056056056	0.715860384672898\\
1.12356156156156	0.708200116378634\\
1.12556256256256	0.700686209281248\\
1.12756356356356	0.693315298760466\\
1.12956456456456	0.686084086334539\\
1.13156556556557	0.678989340287814\\
1.13356656656657	0.672027896137248\\
1.13556756756757	0.665196656836929\\
1.13756856856857	0.658492592770999\\
1.13956956956957	0.651912741590729\\
1.14157057057057	0.645454207908363\\
1.14357157157157	0.639114162834275\\
1.14557257257257	0.632889843339666\\
1.14757357357357	0.626778551604267\\
1.14957457457457	0.620777654235695\\
1.15157557557558	0.614884581394168\\
1.15357657657658	0.609096825799563\\
1.15557757757758	0.60341194181076\\
1.15757857857858	0.597827544310626\\
1.15957957957958	0.592341307662408\\
1.16158058058058	0.586950964585334\\
1.16358158158158	0.581654304940104\\
1.16558258258258	0.576449174630519\\
1.16758358358358	0.57133347438373\\
1.16958458458458	0.566305158560179\\
1.17158558558559	0.561362233952623\\
1.17358658658659	0.556502758583932\\
1.17558758758759	0.55172484050719\\
1.17758858858859	0.547026636611229\\
1.17958958958959	0.542406351434354\\
1.18159059059059	0.53786223598866\\
1.18359159159159	0.533392586597055\\
1.18559259259259	0.528995743744805\\
1.18759359359359	0.52467009094717\\
1.18959459459459	0.520414053634463\\
1.1915955955956	0.516226098055677\\
1.1935965965966	0.512104730201608\\
1.1955975975976	0.50804849474827\\
1.1975985985986	0.504055974021195\\
1.1995995995996	0.500125786981153\\
1.2016006006006	0.49625658823162\\
1.2036016016016	0.492447067048272\\
1.2056026026026	0.488695946430682\\
1.2076036036036	0.485001982176269\\
1.2096046046046	0.481363961976544\\
1.21160560560561	0.477780704535539\\
1.21360660660661	0.474251058710352\\
1.21560760760761	0.470773902673594\\
1.21760860860861	0.46734814309754\\
1.21960960960961	0.463972714359722\\
1.22161061061061	0.460646577769676\\
1.22361161161161	0.457368720816522\\
1.22561261261261	0.45413815643704\\
1.22761361361361	0.45095392230387\\
1.22961461461461	0.447815080133468\\
1.23161561561562	0.444720715013409\\
1.23361661661662	0.441669934748654\\
1.23561761761762	0.43866186922635\\
1.23761861861862	0.43569566979875\\
1.23961961961962	0.432770508683844\\
1.24162062062062	0.429885578383252\\
1.24362162162162	0.427040091116973\\
1.24562262262262	0.424233278274568\\
1.24762362362362	0.421464389882335\\
1.24962462462462	0.418732694086074\\
1.25162562562563	0.416037476649027\\
1.25362662662663	0.413378040464567\\
1.25562762762763	0.410753705083251\\
1.25762862862863	0.40816380625382\\
1.25962962962963	0.405607695477769\\
1.26163063063063	0.403084739577092\\
1.26363163163163	0.40059432027482\\
1.26563263263263	0.398135833788002\\
1.26763363363363	0.395708690432736\\
1.26963463463463	0.393312314240928\\
1.27163563563564	0.390946142588399\\
1.27363663663664	0.388609625834034\\
1.27563763763764	0.38630222696962\\
1.27763863863864	0.38402342128006\\
1.27963963963964	0.381772696013648\\
1.28164064064064	0.379549550062106\\
1.28364164164164	0.377353493650069\\
1.28564264264264	0.375184048033757\\
1.28764364364364	0.37304074520852\\
1.28964464464464	0.370923127625007\\
1.29164564564565	0.368830747913692\\
1.29364664664665	0.366763168617481\\
1.29564764764765	0.364719961932174\\
1.29764864864865	0.362700709454523\\
1.29964964964965	0.360705001937655\\
1.30165065065065	0.358732439053636\\
1.30365165165165	0.356782629162946\\
1.30565265265265	0.354855189090659\\
1.30765365365365	0.352949743909115\\
1.30965465465465	0.351065926726882\\
1.31165565565566	0.349203378483813\\
1.31365665665666	0.347361747752016\\
1.31565765765766	0.345540690542538\\
1.31765865865866	0.343739870117597\\
1.31965965965966	0.341958956808195\\
1.32166066066066	0.340197627836929\\
1.32366166166166	0.338455567145844\\
1.32566266266266	0.336732465229191\\
1.32766366366366	0.335028018970908\\
1.32966466466466	0.333341931486698\\
1.33166566566567	0.331673911970559\\
1.33366666666667	0.330023675545623\\
1.33566766766767	0.328390943119178\\
1.33766866866867	0.326775441241744\\
1.33966966966967	0.325176901970068\\
1.34167067067067	0.323595062733936\\
1.34367167167167	0.32202966620667\\
1.34567267267267	0.320480460179204\\
1.34767367367367	0.31894719743763\\
1.34967467467467	0.31742963564411\\
1.35167567567568	0.315927537221042\\
1.35367667667668	0.314440669238395\\
1.35567767767768	0.312968803304113\\
1.35767867867868	0.311511715457485\\
1.35967967967968	0.310069186065409\\
1.36168068068068	0.308640999721448\\
1.36368168168168	0.307226945147603\\
1.36568268268268	0.30582681509872\\
1.36768368368368	0.304440406269451\\
1.36968468468468	0.303067519203691\\
1.37168568568569	0.301707958206426\\
1.37368668668669	0.300361531257911\\
1.37568768768769	0.29902804993011\\
1.37768868868869	0.297707329305339\\
1.37968968968969	0.296399187897042\\
1.38169069069069	0.295103447572635\\
1.38369169169169	0.293819933478364\\
1.38569269269269	0.292548473966109\\
1.38769369369369	0.29128890052209\\
1.38969469469469	0.290041047697405\\
1.3916956956957	0.288804753040364\\
1.3936966966967	0.287579857030547\\
1.3956976976977	0.28636620301456\\
1.3976986986987	0.285163637143409\\
1.3996996996997	0.283972008311488\\
1.4017007007007	0.282791168097089\\
1.4037017017017	0.281620970704426\\
1.4057027027027	0.280461272907111\\
1.4077037037037	0.279311933993046\\
1.4097047047047	0.278172815710701\\
1.41170570570571	0.277043782216711\\
1.41370670670671	0.275924700024794\\
1.41570770770771	0.274815437955915\\
1.41770870870871	0.273715867089692\\
1.41970970970971	0.272625860716988\\
1.42171071071071	0.271545294293666\\
1.42371171171171	0.270474045395474\\
1.42571271271271	0.269411993674026\\
1.42771371371371	0.268359020813855\\
1.42971471471471	0.267315010490496\\
1.43171571571572	0.266279848329587\\
1.43371671671672	0.265253421866946\\
1.43571771771772	0.264235620509611\\
1.43771871871872	0.263226335497799\\
1.43971971971972	0.262225459867781\\
1.44172072072072	0.261232888415625\\
1.44372172172172	0.260248517661808\\
1.44572272272272	0.259272245816654\\
1.44772372372372	0.258303972746581\\
1.44972472472472	0.257343599941144\\
1.45172572572573	0.256391030480844\\
1.45372672672673	0.25544616900568\\
1.45572772772773	0.254508921684443\\
1.45772872872873	0.2535791961847\\
1.45972972972973	0.252656901643487\\
1.46173073073073	0.251741948638665\\
1.46373173173173	0.250834249160929\\
1.46573273273273	0.249933716586463\\
1.46773373373373	0.249040265650207\\
1.46973473473473	0.24815381241974\\
1.47173573573574	0.247274274269744\\
1.47373673673674	0.246401569857045\\
1.47573773773774	0.245535619096215\\
1.47773873873874	0.244676343135719\\
1.47973973973974	0.243823664334594\\
1.48174074074074	0.242977506239644\\
1.48374174174174	0.242137793563141\\
1.48574274274274	0.241304452161023\\
1.48774374374374	0.240477409011562\\
1.48974474474474	0.239656592194508\\
1.49174574574575	0.238841930870686\\
1.49374674674675	0.238033355262038\\
1.49574774774775	0.237230796632098\\
1.49774874874875	0.236434187266891\\
1.49974974974975	0.235643460456243\\
1.50175075075075	0.234858550475495\\
1.50375175175175	0.234079392567606\\
1.50575275275275	0.233305922925645\\
1.50775375375375	0.232538078675649\\
1.50975475475475	0.231775797859847\\
1.51175575575576	0.231019019420248\\
1.51375675675676	0.230267683182558\\
1.51575775775776	0.229521729840457\\
1.51775875875876	0.228781100940189\\
1.51975975975976	0.22804573886548\\
1.52176076076076	0.227315586822775\\
1.52376176176176	0.226590588826776\\
1.52576276276276	0.225870689686281\\
1.52776376376376	0.225155834990319\\
1.52976476476476	0.224445971094568\\
1.53176576576577	0.223741045108051\\
1.53376676676677	0.223041004880105\\
1.53576776776777	0.222345798987622\\
1.53776876876877	0.221655376722537\\
1.53976976976977	0.220969688079582\\
1.54177077077077	0.220288683744283\\
1.54377177177177	0.2196123150812\\
1.54577277277277	0.218940534122397\\
1.54777377377377	0.218273293556156\\
1.54977477477477	0.217610546715904\\
1.55177577577578	0.216952247569371\\
1.55377677677678	0.216298350707954\\
1.55577777777778	0.215648811336297\\
1.55777877877878	0.215003585262078\\
1.55977977977978	0.214362628885992\\
1.56178078078078	0.213725899191937\\
1.56378178178178	0.213093353737385\\
1.56578278278278	0.212464950643948\\
1.56778378378378	0.211840648588121\\
1.56978478478478	0.211220406792211\\
1.57178578578579	0.210604185015434\\
1.57378678678679	0.209991943545194\\
1.57578778778779	0.209383643188516\\
1.57778878878879	0.208779245263658\\
1.57978978978979	0.208178711591872\\
1.58179079079079	0.207582004489328\\
1.58379179179179	0.206989086759187\\
1.58579279279279	0.206399921683833\\
1.58779379379379	0.205814473017238\\
1.58979479479479	0.205232704977487\\
1.5917957957958	0.204654582239425\\
1.5937967967968	0.204080069927464\\
1.5957977977978	0.203509133608502\\
1.5977987987988	0.202941739284988\\
1.5997997997998	0.202377853388114\\
1.6018008008008	0.201817442771128\\
1.6038018018018	0.201260474702774\\
1.6058028028028	0.200706916860852\\
1.6078038038038	0.200156737325901\\
1.6098048048048	0.199609904574985\\
1.61180580580581	0.199066387475609\\
1.61380680680681	0.198526155279732\\
1.61580780780781	0.197989177617897\\
1.61780880880881	0.197455424493465\\
1.61980980980981	0.196924866276948\\
1.62181081081081	0.196397473700455\\
1.62381181181181	0.195873217852227\\
1.62581281281281	0.195352070171277\\
1.62781381381381	0.19483400244212\\
1.62981481481481	0.194318986789604\\
1.63181581581582	0.193806995673824\\
1.63381681681682	0.193298001885133\\
1.63581781781782	0.192791978539241\\
1.63781881881882	0.192288899072392\\
1.63981981981982	0.191788737236639\\
1.64182082082082	0.19129146709519\\
1.64382182182182	0.190797063017841\\
1.64582282282282	0.190305499676488\\
1.64782382382382	0.189816752040716\\
1.64982482482482	0.189330795373464\\
1.65182582582583	0.18884760522677\\
1.65382682682683	0.188367157437577\\
1.65582782782783	0.187889428123628\\
1.65782882882883	0.187414393679417\\
1.65982982982983	0.186942030772217\\
1.66183083083083	0.186472316338173\\
1.66383183183183	0.18600522757846\\
1.66583283283283	0.185540741955514\\
1.66783383383383	0.185078837189314\\
1.66983483483483	0.184619491253736\\
1.67183583583584	0.184162682372969\\
1.67383683683684	0.183708389017985\\
1.67583783783784	0.183256589903069\\
1.67783883883884	0.182807263982415\\
1.67983983983984	0.182360390446769\\
1.68184084084084	0.181915948720131\\
1.68384184184184	0.181473918456514\\
1.68584284284284	0.181034279536755\\
1.68784384384384	0.180597012065374\\
1.68984484484484	0.180162096367496\\
1.69184584584585	0.17972951298581\\
1.69384684684685	0.179299242677587\\
1.69584784784785	0.178871266411743\\
1.69784884884885	0.17844556536595\\
1.69984984984985	0.178022120923798\\
1.70185085085085	0.17760091467199\\
1.70385185185185	0.177181928397604\\
1.70585285285285	0.176765144085375\\
1.70785385385385	0.176350543915042\\
1.70985485485485	0.175938110258719\\
1.71185585585586	0.175527825678323\\
1.71385685685686	0.175119672923035\\
1.71585785785786	0.174713634926803\\
1.71785885885886	0.174309694805883\\
1.71985985985986	0.173907835856423\\
1.72186086086086	0.173508041552082\\
1.72386186186186	0.173110295541687\\
1.72586286286286	0.172714581646926\\
1.72786386386386	0.172320883860082\\
1.72986486486486	0.171929186341794\\
1.73186586586587	0.171539473418858\\
1.73386686686687	0.171151729582065\\
1.73586786786787	0.170765939484065\\
1.73786886886887	0.170382087937273\\
1.73986986986987	0.170000159911796\\
1.74187087087087	0.169620140533405\\
1.74387187187187	0.169242015081528\\
1.74587287287287	0.168865768987275\\
1.74787387387387	0.168491387831504\\
1.74987487487487	0.168118857342896\\
1.75187587587588	0.167748163396083\\
1.75387687687688	0.167379292009785\\
1.75587787787788	0.167012229344983\\
1.75787887887888	0.166646961703126\\
1.75987987987988	0.16628347552435\\
1.76188088088088	0.165921757385737\\
1.76388188188188	0.165561793999597\\
1.76588288288288	0.165203572211766\\
1.76788388388388	0.16484707899995\\
1.76988488488488	0.164492301472066\\
1.77188588588589	0.164139226864636\\
1.77388688688689	0.163787842541183\\
1.77588788788789	0.163438135990664\\
1.77788888888889	0.163090094825916\\
1.77988988988989	0.162743706782136\\
1.78189089089089	0.162398959715374\\
1.78389189189189	0.162055841601052\\
1.78589289289289	0.161714340532503\\
1.78789389389389	0.161374444719536\\
1.78989489489489	0.161036142487014\\
1.7918958958959	0.160699422273459\\
1.7938968968969	0.160364272629674\\
1.7958978978979	0.160030682217389\\
1.7978988988989	0.159698639807919\\
1.7998998998999	0.159368134280848\\
1.8019009009009	0.15903915462273\\
1.8039019019019	0.158711689925808\\
1.8059029029029	0.15838572938675\\
1.8079039039039	0.158061262305406\\
1.8099049049049	0.157738278083581\\
1.81190590590591	0.157416766223824\\
1.81390690690691	0.157096716328238\\
1.81590790790791	0.156778118097302\\
1.81790890890891	0.156460961328714\\
1.81990990990991	0.156145235916247\\
1.82191091091091	0.155830931848624\\
1.82391191191191	0.155518039208403\\
1.82591291291291	0.155206548170888\\
1.82791391391391	0.15489644900304\\
1.82991491491491	0.154587732062422\\
1.83191591591592	0.15428038779614\\
1.83391691691692	0.153974406739811\\
1.83591791791792	0.153669779516544\\
1.83791891891892	0.153366496835925\\
1.83991991991992	0.153064549493029\\
1.84192092092092	0.15276392836744\\
1.84392192192192	0.152464624422279\\
1.84592292292292	0.152166628703259\\
1.84792392392392	0.151869932337735\\
1.84992492492492	0.151574526533786\\
1.85192592592593	0.151280402579289\\
1.85392692692693	0.150987551841026\\
1.85592792792793	0.150695965763788\\
1.85792892892893	0.150405635869498\\
1.85992992992993	0.150116553756343\\
1.86193093093093	0.149828711097921\\
1.86393193193193	0.149542099642394\\
1.86593293293293	0.14925671121166\\
1.86793393393393	0.148972537700527\\
1.86993493493493	0.148689571075906\\
1.87193593593594	0.148407803376009\\
1.87393693693694	0.148127226709559\\
1.87593793793794	0.147847833255017\\
1.87793893893894	0.147569615259807\\
1.87993993993994	0.14729256503956\\
1.88194094094094	0.147016674977366\\
1.88394194194194	0.146741937523037\\
1.88594294294294	0.146468345192374\\
1.88794394394394	0.146195890566451\\
1.88994494494494	0.145924566290904\\
1.89194594594595	0.14565436507523\\
1.89394694694695	0.145385279692093\\
1.89594794794795	0.145117302976647\\
1.89794894894895	0.144850427825855\\
1.89994994994995	0.14458464719783\\
1.90195095095095	0.144319954111173\\
1.90395195195195	0.144056341644328\\
1.90595295295295	0.143793802934941\\
1.90795395395395	0.143532331179228\\
1.90995495495495	0.143271919631351\\
1.91195595595596	0.1430125616028\\
1.91395695695696	0.142754250461793\\
1.91595795795796	0.142496979632663\\
1.91795895895896	0.142240742595278\\
1.91995995995996	0.141985532884446\\
1.92196096096096	0.141731344089342\\
1.92396196196196	0.141478169852936\\
1.92596296296296	0.14122600387143\\
1.92796396396396	0.140974839893702\\
1.92996496496496	0.140724671720754\\
1.93196596596597	0.140475493205171\\
1.93396696696697	0.140227298250587\\
1.93596796796797	0.139980080811152\\
1.93796896896897	0.139733834891013\\
1.93996996996997	0.139488554543793\\
1.94197097097097	0.139244233872087\\
1.94397197197197	0.139000867026953\\
1.94597297297297	0.138758448207418\\
1.94797397397397	0.138516971659986\\
1.94997497497497	0.138276431678152\\
1.95197597597598	0.138036822601923\\
1.95397697697698	0.137798138817345\\
1.95597797797798	0.137560374756033\\
1.95797897897898	0.137323524894712\\
1.95997997997998	0.137087583754759\\
1.96198098098098	0.136852545901752\\
1.96398198198198	0.136618405945023\\
1.96598298298298	0.136385158537224\\
1.96798398398398	0.136152798373881\\
1.96998498498498	0.135921320192978\\
1.97198598598599	0.135690718774519\\
1.97398698698699	0.135460988940119\\
1.97598798798799	0.135232125552585\\
1.97798898898899	0.135004123515505\\
1.97998998998999	0.134776977772849\\
1.98199099099099	0.134550683308563\\
1.98399199199199	0.134325235146176\\
1.98599299299299	0.134100628348413\\
1.98799399399399	0.133876858016804\\
1.98999499499499	0.133653919291306\\
1.991995995996	0.133431807349924\\
1.993996996997	0.133210517408342\\
1.995997997998	0.132990044719551\\
1.997998998999	0.132770384573487\\
2	0.132551532296674\\
};
\addlegendentry{$\alpha\text{ = 0}$};

\addplot [color=blue,solid]
  table[row sep=crcr]{%
0.001	3.14139535352239\\
0.003001001001001	3.14100055139566\\
0.005002002002002	3.1406057357822\\
0.007003003003003	3.14021089768826\\
0.009004004004004	3.1398160281185\\
0.011005005005005	3.1394211180753\\
0.013006006006006	3.13902615855813\\
0.015007007007007	3.13863114056288\\
0.017008008008008	3.13823605508126\\
0.019009009009009	3.1378408931001\\
0.02101001001001	3.13744564560073\\
0.023011011011011	3.1370503035583\\
0.025012012012012	3.13665485794117\\
0.027013013013013	3.13625929971022\\
0.029014014014014	3.13586361981822\\
0.031015015015015	3.13546780920916\\
0.033016016016016	3.1350718588176\\
0.035017017017017	3.13467575956801\\
0.037018018018018	3.13427950237411\\
0.039019019019019	3.13388307813823\\
0.04102002002002	3.13348647775061\\
0.043021021021021	3.13308969208878\\
0.045022022022022	3.13269271201684\\
0.047023023023023	3.13229552838485\\
0.049024024024024	3.13189813202812\\
0.051025025025025	3.13150051376656\\
0.053026026026026	3.13110266440401\\
0.055027027027027	3.13070457472752\\
0.057028028028028	3.13030623550675\\
0.059029029029029	3.1299076374932\\
0.06103003003003	3.1295087714196\\
0.063031031031031	3.12910962799919\\
0.065032032032032	3.12871019792503\\
0.067033033033033	3.12831047186931\\
0.069034034034034	3.12791044048267\\
0.071035035035035	3.12751009439349\\
0.073036036036036	3.12710942420718\\
0.075037037037037	3.12670842050551\\
0.077038038038038	3.12630707384584\\
0.079039039039039	3.1259053747605\\
0.08104004004004	3.12550331375597\\
0.083041041041041	3.12510088131225\\
0.085042042042042	3.12469806788209\\
0.0870430430430431	3.12429486389028\\
0.089044044044044	3.12389125973291\\
0.091045045045045	3.12348724577666\\
0.093046046046046	3.12308281235802\\
0.0950470470470471	3.12267794978258\\
0.0970480480480481	3.12227264832429\\
0.0990490490490491	3.12186689822466\\
0.10105005005005	3.12146068969206\\
0.103051051051051	3.12105401290092\\
0.105052052052052	3.12064685799099\\
0.107053053053053	3.12023921506656\\
0.109054054054054	3.11983107419564\\
0.111055055055055	3.11942242540927\\
0.113056056056056	3.11901325870066\\
0.115057057057057	3.1186035640244\\
0.117058058058058	3.1181933312957\\
0.119059059059059	3.11778255038956\\
0.12106006006006	3.11737121113998\\
0.123061061061061	3.1169593033391\\
0.125062062062062	3.11654681673643\\
0.127063063063063	3.11613374103799\\
0.129064064064064	3.11572006590549\\
0.131065065065065	3.11530578095548\\
0.133066066066066	3.1148908757585\\
0.135067067067067	3.11447533983823\\
0.137068068068068	3.11405916267065\\
0.139069069069069	3.11364233368311\\
0.14107007007007	3.11322484225352\\
0.143071071071071	3.11280667770944\\
0.145072072072072	3.11238782932718\\
0.147073073073073	3.1119682863309\\
0.149074074074074	3.11154803789173\\
0.151075075075075	3.11112707312684\\
0.153076076076076	3.1107053810985\\
0.155077077077077	3.11028295081317\\
0.157078078078078	3.10985977122057\\
0.159079079079079	3.10943583121271\\
0.16108008008008	3.10901111962292\\
0.163081081081081	3.10858562522495\\
0.165082082082082	3.10815933673191\\
0.167083083083083	3.10773224279536\\
0.169084084084084	3.10730433200425\\
0.171085085085085	3.10687559288399\\
0.173086086086086	3.10644601389537\\
0.175087087087087	3.10601558343359\\
0.177088088088088	3.10558428982718\\
0.179089089089089	3.10515212133702\\
0.18109009009009	3.1047190661552\\
0.183091091091091	3.10428511240405\\
0.185092092092092	3.10385024813499\\
0.187093093093093	3.10341446132751\\
0.189094094094094	3.102977739888\\
0.191095095095095	3.1025400716487\\
0.193096096096096	3.10210144436658\\
0.195097097097097	3.10166184572216\\
0.197098098098098	3.10122126331841\\
0.199099099099099	3.10077968467958\\
0.2011001001001	3.10033709725004\\
0.203101101101101	3.09989348839308\\
0.205102102102102	3.09944884538975\\
0.207103103103103	3.09900315543761\\
0.209104104104104	3.09855640564956\\
0.211105105105105	3.09810858305257\\
0.213106106106106	3.09765967458643\\
0.215107107107107	3.09720966710252\\
0.217108108108108	3.09675854736253\\
0.219109109109109	3.09630630203714\\
0.22111011011011	3.09585291770475\\
0.223111111111111	3.09539838085014\\
0.225112112112112	3.09494267786315\\
0.227113113113113	3.09448579503733\\
0.229114114114114	3.09402771856856\\
0.231115115115115	3.09356843455369\\
0.233116116116116	3.09310792898914\\
0.235117117117117	3.09264618776946\\
0.237118118118118	3.09218319668595\\
0.239119119119119	3.09171894142516\\
0.24112012012012	3.09125340756748\\
0.243121121121121	3.0907865805856\\
0.245122122122122	3.09031844584306\\
0.247123123123123	3.08984898859272\\
0.249124124124124	3.0893781939752\\
0.251125125125125	3.08890604701736\\
0.253126126126126	3.08843253263071\\
0.255127127127127	3.08795763560983\\
0.257128128128128	3.08748134063073\\
0.259129129129129	3.08700363224925\\
0.26113013013013	3.08652449489939\\
0.263131131131131	3.08604391289165\\
0.265132132132132	3.08556187041132\\
0.267133133133133	3.0850783515168\\
0.269134134134134	3.08459334013781\\
0.271135135135135	3.08410682007368\\
0.273136136136136	3.08361877499156\\
0.275137137137137	3.08312918842458\\
0.277138138138138	3.08263804377009\\
0.279139139139139	3.08214532428776\\
0.28114014014014	3.08165101309773\\
0.283141141141141	3.08115509317869\\
0.285142142142142	3.08065754736601\\
0.287143143143143	3.08015835834973\\
0.289144144144144	3.07965750867263\\
0.291145145145145	3.07915498072824\\
0.293146146146146	3.07865075675878\\
0.295147147147147	3.07814481885315\\
0.297148148148148	3.07763714894482\\
0.299149149149149	3.07712772880975\\
0.30115015015015	3.07661654006427\\
0.303151151151151	3.07610356416289\\
0.305152152152152	3.07558878239611\\
0.307153153153153	3.07507217588826\\
0.309154154154154	3.07455372559518\\
0.311155155155155	3.07403341230199\\
0.313156156156156	3.07351121662078\\
0.315157157157157	3.07298711898825\\
0.317158158158158	3.07246109966336\\
0.319159159159159	3.07193313872494\\
0.32116016016016	3.07140321606922\\
0.323161161161161	3.0708713114074\\
0.325162162162162	3.07033740426312\\
0.327163163163163	3.06980147396996\\
0.329164164164164	3.06926349966885\\
0.331165165165165	3.06872346030547\\
0.333166166166166	3.0681813346276\\
0.335167167167167	3.06763710118248\\
0.337168168168168	3.06709073831407\\
0.339169169169169	3.06654222416029\\
0.34117017017017	3.06599153665027\\
0.343171171171171	3.0654386535015\\
0.345172172172172	3.06488355221697\\
0.347173173173173	3.06432621008227\\
0.349174174174174	3.06376660416265\\
0.351175175175175	3.0632047113\\
0.353176176176176	3.06264050810987\\
0.355177177177177	3.0620739709784\\
0.357178178178178	3.06150507605916\\
0.359179179179179	3.06093379927004\\
0.36118018018018	3.06036011629003\\
0.363181181181181	3.05978400255599\\
0.365182182182182	3.05920543325932\\
0.367183183183183	3.0586243833427\\
0.369184184184184	3.05804082749662\\
0.371185185185185	3.05745474015602\\
0.373186186186186	3.05686609549675\\
0.375187187187187	3.0562748674321\\
0.377188188188188	3.05568102960918\\
0.379189189189189	3.05508455540529\\
0.38119019019019	3.05448541792427\\
0.383191191191191	3.05388358999274\\
0.385192192192192	3.05327904415632\\
0.387193193193193	3.05267175267579\\
0.389194194194194	3.05206168752321\\
0.391195195195195	3.05144882037794\\
0.393196196196196	3.05083312262265\\
0.395197197197197	3.05021456533926\\
0.397198198198198	3.04959311930482\\
0.399199199199199	3.04896875498729\\
0.4012002002002	3.04834144254137\\
0.403201201201201	3.04771115180413\\
0.405202202202202	3.04707785229069\\
0.407203203203203	3.04644151318974\\
0.409204204204204	3.04580210335915\\
0.411205205205205	3.0451595913213\\
0.413206206206206	3.04451394525855\\
0.415207207207207	3.04386513300849\\
0.417208208208208	3.04321312205923\\
0.419209209209209	3.04255787954456\\
0.42121021021021	3.04189937223907\\
0.423211211211211	3.04123756655317\\
0.425212212212212	3.04057242852807\\
0.427213213213213	3.03990392383069\\
0.429214214214214	3.03923201774843\\
0.431215215215215	3.038556675184\\
0.433216216216216	3.03787786065\\
0.435217217217217	3.03719553826359\\
0.437218218218218	3.03650967174096\\
0.439219219219219	3.03582022439178\\
0.44122022022022	3.03512715911359\\
0.443221221221221	3.03443043838603\\
0.445222222222222	3.03373002426506\\
0.447223223223223	3.03302587837707\\
0.449224224224224	3.03231796191291\\
0.451225225225225	3.03160623562184\\
0.453226226226226	3.03089065980536\\
0.455227227227227	3.030171194311\\
0.457228228228228	3.029447798526\\
0.459229229229229	3.02872043137088\\
0.46123023023023	3.02798905129296\\
0.463231231231231	3.02725361625976\\
0.465232232232232	3.02651408375228\\
0.467233233233233	3.02577041075828\\
0.469234234234234	3.02502255376532\\
0.471235235235235	3.02427046875386\\
0.473236236236236	3.02351411119014\\
0.475237237237237	3.02275343601899\\
0.477238238238238	3.02198839765661\\
0.479239239239239	3.02121894998313\\
0.48124024024024	3.02044504633518\\
0.483241241241241	3.01966663949824\\
0.485242242242242	3.01888368169903\\
0.487243243243243	3.01809612459762\\
0.489244244244244	3.0173039192796\\
0.491245245245245	3.016507016248\\
0.493246246246246	3.0157053654152\\
0.495247247247247	3.01489891609467\\
0.497248248248248	3.01408761699263\\
0.499249249249249	3.01327141619956\\
0.50125025025025	3.01245026118163\\
0.503251251251251	3.01162409877199\\
0.505252252252252	3.01079287516198\\
0.507253253253253	3.00995653589215\\
0.509254254254254	3.00911502584323\\
0.511255255255255	3.00826828922691\\
0.513256256256256	3.00741626957661\\
0.515257257257257	3.00655890973798\\
0.517258258258258	3.0056961518594\\
0.519259259259259	3.00482793738227\\
0.52126026026026	3.00395420703125\\
0.523261261261261	3.00307490080425\\
0.525262262262262	3.00218995796249\\
0.527263263263263	3.0012993170202\\
0.529264264264264	3.00040291573436\\
0.531265265265265	2.99950069109425\\
0.533266266266266	2.99859257931084\\
0.535267267267267	2.9976785158061\\
0.537268268268268	2.99675843520215\\
0.539269269269269	2.99583227131028\\
0.54127027027027	2.99489995711982\\
0.543271271271271	2.9939614247869\\
0.545272272272272	2.99301660562306\\
0.547273273273273	2.99206543008374\\
0.549274274274274	2.99110782775661\\
0.551275275275275	2.99014372734977\\
0.553276276276276	2.98917305667981\\
0.555277277277277	2.98819574265978\\
0.557278278278278	2.98721171128695\\
0.559279279279279	2.98622088763046\\
0.56128028028028	2.98522319581887\\
0.563281281281281	2.98421855902752\\
0.565282282282282	2.9832068994658\\
0.567283283283283	2.98218813836422\\
0.569284284284284	2.98116219596144\\
0.571285285285285	2.98012899149106\\
0.573286286286286	2.97908844316837\\
0.575287287287287	2.97804046817691\\
0.577288288288288	2.9769849826549\\
0.579289289289289	2.97592190168156\\
0.58129029029029	2.97485113926331\\
0.583291291291291	2.97377260831982\\
0.585292292292292	2.97268622066992\\
0.587293293293293	2.97159188701743\\
0.589294294294294	2.97048951693683\\
0.591295295295295	2.96937901885886\\
0.593296296296296	2.96826030005589\\
0.595297297297297	2.96713326662735\\
0.597298298298298	2.96599782348488\\
0.599299299299299	2.96485387433747\\
0.6013003003003	2.96370132167646\\
0.603301301301301	2.96254006676045\\
0.605302302302302	2.96137000960011\\
0.607303303303303	2.96019104894286\\
0.609304304304304	2.95900308225757\\
0.611305305305305	2.95780600571898\\
0.613306306306306	2.95659971419227\\
0.615307307307307	2.95538410121736\\
0.617308308308308	2.95415905899329\\
0.619309309309309	2.9529244783624\\
0.62131031031031	2.95168024879454\\
0.623311311311311	2.95042625837121\\
0.625312312312312	2.94916239376969\\
0.627313313313313	2.94788854024701\\
0.629314314314314	2.94660458162404\\
0.631315315315315	2.94531040026947\\
0.633316316316316	2.94400587708373\\
0.635317317317317	2.94269089148311\\
0.637318318318318	2.9413653213836\\
0.639319319319319	2.94002904318495\\
0.64132032032032	2.93868193175466\\
0.643321321321321	2.93732386041209\\
0.645322322322322	2.93595470091243\\
0.647323323323323	2.93457432343096\\
0.649324324324324	2.93318259654716\\
0.651325325325325	2.93177938722911\\
0.653326326326326	2.93036456081779\\
0.655327327327327	2.92893798101153\\
0.657328328328328	2.9274995098507\\
0.659329329329329	2.92604900770239\\
0.66133033033033	2.92458633324528\\
0.663331331331331	2.92311134345467\\
0.665332332332332	2.92162389358772\\
0.667333333333333	2.92012383716884\\
0.669334334334334	2.91861102597529\\
0.671335335335335	2.91708531002304\\
0.673336336336336	2.91554653755283\\
0.675337337337337	2.91399455501655\\
0.677338338338338	2.91242920706379\\
0.679339339339339	2.91085033652891\\
0.68134034034034	2.90925778441812\\
0.683341341341341	2.90765138989723\\
0.685342342342342	2.90603099027936\\
0.687343343343343	2.90439642101356\\
0.689344344344344	2.90274751567343\\
0.691345345345345	2.90108410594616\\
0.693346346346346	2.89940602162229\\
0.695347347347347	2.89771309058575\\
0.697348348348348	2.89600513880452\\
0.699349349349349	2.89428199032149\\
0.70135035035035	2.89254346724645\\
0.703351351351351	2.89078938974799\\
0.705352352352352	2.88901957604662\\
0.707353353353353	2.88723384240788\\
0.709354354354354	2.88543200313667\\
0.711355355355355	2.88361387057186\\
0.713356356356356	2.88177925508189\\
0.715357357357357	2.87992796506071\\
0.717358358358358	2.87805980692489\\
0.719359359359359	2.87617458511136\\
0.72136036036036	2.87427210207567\\
0.723361361361361	2.87235215829156\\
0.725362362362362	2.87041455225095\\
0.727363363363363	2.86845908046524\\
0.729364364364364	2.86648553746696\\
0.731365365365365	2.86449371581302\\
0.733366366366366	2.86248340608824\\
0.735367367367367	2.86045439691027\\
0.737368368368368	2.85840647493553\\
0.739369369369369	2.85633942486595\\
0.74137037037037	2.85425302945674\\
0.743371371371371	2.85214706952558\\
0.745372372372372	2.85002132396236\\
0.747373373373373	2.84787556974054\\
0.749374374374374	2.84570958192913\\
0.751375375375375	2.84352313370624\\
0.753376376376376	2.84131599637346\\
0.755377377377377	2.83908793937151\\
0.757378378378378	2.83683873029693\\
0.759379379379379	2.83456813492032\\
0.76138038038038	2.83227591720495\\
0.763381381381381	2.82996183932754\\
0.765382382382382	2.82762566169936\\
0.767383383383383	2.82526714298892\\
0.769384384384384	2.82288604014587\\
0.771385385385385	2.82048210842547\\
0.773386386386386	2.81805510141505\\
0.775387387387387	2.81560477106079\\
0.777388388388388	2.81313086769604\\
0.779389389389389	2.81063314007061\\
0.78139039039039	2.80811133538092\\
0.783391391391391	2.8055651993013\\
0.785392392392392	2.80299447601622\\
0.787393393393393	2.80039890825328\\
0.789394394394394	2.79777823731752\\
0.791395395395395	2.79513220312561\\
0.793396396396396	2.79246054424183\\
0.795397397397397	2.78976299791409\\
0.797398398398398	2.7870393001106\\
0.799399399399399	2.78428918555746\\
0.8014004004004	2.78151238777595\\
0.803401401401401	2.77870863912099\\
0.805402402402402	2.77587767081931\\
0.807403403403404	2.773019213008\\
0.809404404404404	2.77013299477302\\
0.811405405405405	2.76721874418761\\
0.813406406406406	2.76427618835054\\
0.815407407407407	2.76130505342422\\
0.817408408408409	2.75830506467188\\
0.819409409409409	2.75527594649474\\
0.82141041041041	2.75221742246774\\
0.823411411411412	2.74912921537503\\
0.825412412412412	2.74601104724403\\
0.827413413413413	2.74286263937807\\
0.829414414414414	2.73968371238822\\
0.831415415415415	2.73647398622269\\
0.833416416416417	2.7332331801951\\
0.835417417417417	2.72996101301019\\
0.837418418418418	2.72665720278851\\
0.839419419419419	2.72332146708643\\
0.84142042042042	2.71995352291592\\
0.843421421421422	2.71655308676033\\
0.845422422422422	2.71311987458683\\
0.847423423423423	2.70965360185619\\
0.849424424424424	2.70615398352901\\
0.851425425425425	2.70262073406709\\
0.853426426426427	2.69905356743224\\
0.855427427427428	2.69545219707998\\
0.857428428428428	2.69181633594834\\
0.859429429429429	2.68814569644281\\
0.86143043043043	2.68443999041559\\
0.863431431431432	2.68069892913891\\
0.865432432432433	2.67692222327482\\
0.867433433433433	2.67310958283573\\
0.869434434434434	2.66926071714253\\
0.871435435435435	2.66537533477288\\
0.873436436436437	2.66145314350516\\
0.875437437437438	2.65749385025457\\
0.877438438438438	2.65349716100088\\
0.879439439439439	2.64946278071055\\
0.88144044044044	2.64539041324967\\
0.883441441441442	2.64127976128817\\
0.885442442442442	2.63713052619734\\
0.887443443443444	2.63294240793693\\
0.889444444444444	2.62871510493424\\
0.891445445445445	2.62444831395271\\
0.893446446446447	2.62014172995136\\
0.895447447447447	2.61579504593568\\
0.897448448448449	2.61140795279478\\
0.899449449449449	2.60698013913134\\
0.90145045045045	2.60251129107874\\
0.903451451451452	2.59800109210691\\
0.905452452452452	2.59344922281804\\
0.907453453453454	2.58885536072819\\
0.909454454454455	2.58421918020309\\
0.911455455455455	2.57954035156185\\
0.913456456456457	2.57481854179366\\
0.915457457457457	2.57005341364675\\
0.917458458458459	2.56524462550489\\
0.91945945945946	2.56039183109055\\
0.92146046046046	2.5554946791534\\
0.923461461461462	2.55055281314362\\
0.925462462462462	2.54556587086975\\
0.927463463463464	2.54053348414031\\
0.929464464464465	2.53545527838896\\
0.931465465465466	2.53033087228239\\
0.933466466466467	2.52515987731053\\
0.935467467467467	2.51994189735835\\
0.937468468468469	2.51467652825871\\
0.93946946946947	2.5093633573254\\
0.941470470470471	2.50400196286574\\
0.943471471471471	2.4985919136719\\
0.945472472472472	2.49313276849004\\
0.947473473473474	2.4876240754664\\
0.949474474474475	2.48206537156934\\
0.951475475475476	2.4764561819862\\
0.953476476476476	2.47079601949394\\
0.955477477477477	2.46508438380226\\
0.957478478478479	2.45932076086783\\
0.95947947947948	2.45350462217834\\
0.961480480480481	2.44763542400466\\
0.963481481481481	2.44171260661956\\
0.965482482482482	2.43573559348118\\
0.967483483483484	2.42970379037921\\
0.969484484484485	2.42361658454187\\
0.971485485485486	2.41747334370121\\
0.973486486486487	2.41127341511444\\
0.975487487487487	2.40501612453859\\
0.977488488488489	2.3987007751556\\
0.97948948948949	2.39232664644478\\
0.981490490490491	2.38589299299922\\
0.983491491491492	2.37939904328264\\
0.985492492492492	2.3728439983225\\
0.987493493493494	2.36622703033528\\
0.989494494494495	2.35954728127923\\
0.991495495495496	2.35280386132943\\
0.993496496496497	2.34599584726975\\
0.995497497497498	2.33912228079578\\
0.997498498498499	2.33218216672205\\
0.9994994994995	2.32517447108656\\
1.0015005005005	2.31809811914493\\
1.0035015015015	2.31095199324555\\
1.0055025025025	2.3037349305767\\
1.0075035035035	2.29644572077537\\
1.0095045045045	2.28908310338693\\
1.01150550550551	2.28164576516331\\
1.01350650650651	2.2741323371867\\
1.01550750750751	2.26654139180398\\
1.01750850850851	2.25887143935588\\
1.01950950950951	2.25112092468341\\
1.02151051051051	2.24328822339182\\
1.02351151151151	2.23537163785095\\
1.02551251251251	2.22736939290801\\
1.02751351351351	2.21927963128662\\
1.02951451451451	2.21110040864305\\
1.03151551551552	2.20282968824729\\
1.03351651651652	2.19446533525307\\
1.03551751751752	2.18600511051674\\
1.03751851851852	2.17744666392046\\
1.03951951951952	2.1687875271497\\
1.04152052052052	2.16002510586923\\
1.04352152152152	2.15115667123488\\
1.04552252252252	2.14217935067049\\
1.04752352352352	2.13309011783091\\
1.04952452452452	2.12388578166102\\
1.05152552552553	2.11456297444982\\
1.05352652652653	2.10511813876416\\
1.05552752752753	2.09554751313183\\
1.05752852852853	2.08584711632472\\
1.05952952952953	2.0760127300723\\
1.06153053053053	2.0660398800103\\
1.06353153153153	2.05592381464097\\
1.06553253253253	2.04565948204722\\
1.06753353353353	2.03524150406275\\
1.06953453453453	2.02466414755325\\
1.07153553553554	2.01392129240745\\
1.07353653653654	2.0030063957699\\
1.07553753753754	1.99191245196783\\
1.07753853853854	1.98063194748792\\
1.07953953953954	1.96915681024377\\
1.08154054054054	1.95747835223427\\
1.08354154154154	1.9455872045229\\
1.08554254254254	1.93347324325937\\
1.08754354354354	1.92112550520879\\
1.08954454454454	1.90853209093655\\
1.09154554554555	1.89568005340364\\
1.09354654654655	1.88255526923397\\
1.09554754754755	1.86914228929565\\
1.09754854854855	1.85542416445224\\
1.09954954954955	1.84138224133693\\
1.10155055055055	1.82699592171357\\
1.10355155155155	1.81224237731654\\
1.10555255255255	1.79709620987772\\
1.10755355355355	1.78152904316773\\
1.10955455455455	1.7655090300448\\
1.11155555555556	1.74900025234959\\
1.11355655655656	1.73196198448063\\
1.11555755755756	1.71434778186288\\
1.11755855855856	1.69610434214704\\
1.11955955955956	1.67717006815557\\
1.12156056056056	1.65747323476741\\
1.12356156156156	1.63692962320198\\
1.12556256256256	1.61543942951113\\
1.12756356356356	1.59288317020212\\
1.12956456456456	1.56911618241541\\
1.13156556556557	1.54396112719004\\
1.13356656656657	1.51719762095649\\
1.13556756756757	1.48854770528344\\
1.13756856856857	1.45765530203617\\
1.13956956956957	1.42405721776195\\
1.14157057057057	1.38714340590126\\
1.14357157157157	1.34610822698043\\
1.14557257257257	1.29991434850867\\
1.14757357357357	1.24736742438305\\
1.14957457457457	1.18763014735822\\
1.15157557557558	1.1218025992781\\
1.15357657657658	1.05489287820954\\
1.15557757757758	0.993768043049956\\
1.15757857857858	0.941866760835263\\
1.15957957957958	0.898730679207927\\
1.16158058058058	0.862599887365965\\
1.16358158158158	0.831820789390028\\
1.16558258258258	0.805137648808037\\
1.16758358358358	0.78164278613518\\
1.16958458458458	0.76068051118818\\
1.17158558558559	0.741769877691948\\
1.17358658658659	0.724551062826376\\
1.17558758758759	0.708749392807511\\
1.17758858858859	0.69415118827119\\
1.17958958958959	0.680587327412524\\
1.18159059059059	0.667921860026986\\
1.18359159159159	0.656043970176366\\
1.18559259259259	0.644862196833434\\
1.18759359359359	0.634300204833358\\
1.18959459459459	0.624293639460629\\
1.1915955955956	0.614787751557958\\
1.1935965965966	0.605735579418075\\
1.1955975975976	0.597096539104735\\
1.1975985985986	0.588835318591881\\
1.1995995995996	0.580921000849804\\
1.2016006006006	0.573326361541258\\
1.2036016016016	0.566027301377046\\
1.2056026026026	0.559002383399708\\
1.2076036036036	0.5522324528173\\
1.2096046046046	0.545700321952439\\
1.21160560560561	0.539390509782426\\
1.21360660660661	0.533289019344151\\
1.21560760760761	0.527383154157643\\
1.21760860860861	0.521661360357111\\
1.21960960960961	0.516113092497219\\
1.22161061061061	0.510728698544341\\
1.22361161161161	0.505499320863926\\
1.22561261261261	0.500416810593352\\
1.22761361361361	0.495473653289642\\
1.22961461461461	0.490662904114939\\
1.23161561561562	0.485978131127532\\
1.23361661661662	0.481413365485481\\
1.23561761761762	0.476963057579308\\
1.23761861861862	0.472622038264034\\
1.23961961961962	0.468385484490267\\
1.24162062062062	0.464248888751868\\
1.24362162162162	0.460208031851045\\
1.24562262262262	0.456258958551971\\
1.24762362362362	0.452397955769846\\
1.24962462462462	0.448621532978288\\
1.25162562562563	0.44492640457036\\
1.25362662662663	0.441309473944776\\
1.25562762762763	0.437767819116385\\
1.25762862862863	0.434298679679949\\
1.25962962962963	0.430899444976052\\
1.26163063063063	0.427567643327686\\
1.26363163163163	0.42430093223377\\
1.26563263263263	0.421097089418823\\
1.26763363363363	0.417954004649356\\
1.26963463463463	0.414869672241155\\
1.27163563563564	0.411842184185858\\
1.27363663663664	0.408869723838354\\
1.27563763763764	0.405950560109289\\
1.27763863863864	0.403083042116097\\
1.27963963963964	0.400265594247712\\
1.28164064064064	0.397496711607084\\
1.28364164164164	0.394774955795834\\
1.28564264264264	0.392098951010619\\
1.28764364364364	0.389467380424838\\
1.28964464464464	0.386878982829244\\
1.29164564564565	0.384332549511222\\
1.29364664664665	0.381826921351181\\
1.29564764764765	0.379360986118942\\
1.29764864864865	0.37693367595296\\
1.29964964964965	0.374543965008984\\
1.30165065065065	0.372190867262825\\
1.30365165165165	0.369873434456691\\
1.30565265265265	0.36759075417735\\
1.30765365365365	0.365341948055667\\
1.30965465465465	0.363126170079299\\
1.31165565565566	0.360942605009108\\
1.31365665665666	0.358790466892396\\
1.31565765765766	0.356668997665494\\
1.31765865865866	0.354577465839506\\
1.31965965965966	0.352515165263284\\
1.32166066066066	0.35048141395814\\
1.32366166166166	0.348475553019246\\
1.32566266266266	0.346496945579551\\
1.32766366366366	0.344544975831589\\
1.32966466466466	0.3426190481033\\
1.33166566566567	0.340718585984778\\
1.33366666666667	0.338843031501795\\
1.33566766766767	0.336991844334128\\
1.33766866866867	0.335164501074648\\
1.33966966966967	0.333360494527734\\
1.34167067067067	0.331579333043635\\
1.34367167167167	0.329820539887145\\
1.34567267267267	0.328083652638191\\
1.34767367367367	0.326368222622495\\
1.34967467467467	0.324673814370504\\
1.35167567567568	0.323000005102945\\
1.35367667667668	0.321346384241357\\
1.35567767767768	0.319712552942193\\
1.35767867867868	0.318098123653095\\
1.35967967967968	0.316502719690272\\
1.36168068068068	0.314925974835357\\
1.36368168168168	0.313367532951217\\
1.36568268268268	0.311827047615166\\
1.36768368368368	0.310304181768944\\
1.36968468468468	0.308798607384367\\
1.37168568568569	0.307310005143958\\
1.37368668668669	0.305838064135634\\
1.37568768768769	0.304382481560729\\
1.37768868868869	0.302942962454736\\
1.37968968968969	0.301519219419993\\
1.38169069069069	0.300110972369777\\
1.38369169169169	0.298717948283233\\
1.38569269269269	0.297339880970457\\
1.38769369369369	0.295976510847467\\
1.38969469469469	0.294627584720271\\
1.3916956956957	0.293292855577888\\
1.3936966966967	0.291972082393627\\
1.3956976976977	0.290665029934376\\
1.3976986986987	0.289371468577485\\
1.3996996996997	0.28809117413484\\
1.4017007007007	0.286823927683849\\
1.4037017017017	0.28556951540495\\
1.4057027027027	0.284327728425404\\
1.4077037037037	0.283098362669029\\
1.4097047047047	0.281881218711635\\
1.41170570570571	0.280676101641892\\
1.41370670670671	0.279482820927385\\
1.41570770770771	0.278301190285603\\
1.41770870870871	0.277131027559721\\
1.41970970970971	0.275972154598837\\
1.42171071071071	0.274824397142598\\
1.42371171171171	0.273687584709914\\
1.42571271271271	0.272561550491703\\
1.42771371371371	0.271446131247353\\
1.42971471471471	0.270341167204893\\
1.43171571571572	0.269246501964572\\
1.43371671671672	0.268161982405828\\
1.43571771771772	0.267087458597444\\
1.43771871871872	0.266022783710723\\
1.43971971971972	0.264967813935681\\
1.44172072072072	0.263922408399979\\
1.44372172172172	0.262886429090622\\
1.44572272272272	0.261859740778229\\
1.44772372372372	0.26084221094379\\
1.44972472472472	0.259833709707849\\
1.45172572572573	0.258834109761955\\
1.45372672672673	0.257843286302351\\
1.45572772772773	0.256861116965772\\
1.45772872872873	0.255887481767305\\
1.45972972972973	0.25492226304019\\
1.46173073073073	0.25396534537755\\
1.46373173173173	0.253016615575905\\
1.46573273273273	0.252075962580468\\
1.46773373373373	0.251143277432103\\
1.46973473473473	0.250218453215926\\
1.47173573573574	0.249301385011471\\
1.47373673673674	0.248391969844352\\
1.47573773773774	0.247490106639388\\
1.47773873873874	0.246595696175127\\
1.47973973973974	0.24570864103972\\
1.48174074074074	0.244828845588102\\
1.48374174174174	0.243956215900422\\
1.48574274274274	0.24309065974169\\
1.48774374374374	0.242232086522598\\
1.48974474474474	0.241380407261463\\
1.49174574574575	0.240535534547268\\
1.49374674674675	0.239697382503747\\
1.49574774774775	0.238865866754495\\
1.49774874874875	0.238040904389058\\
1.49974974974975	0.237222413929973\\
1.50175075075075	0.236410315300723\\
1.50375175175175	0.235604529794585\\
1.50575275275275	0.234804980044324\\
1.50775375375375	0.234011589992736\\
1.50975475475475	0.233224284863967\\
1.51175575575576	0.232442991135635\\
1.51375675675676	0.231667636511684\\
1.51575775775776	0.230898149895969\\
1.51775875875876	0.230134461366553\\
1.51975975975976	0.229376502150673\\
1.52176076076076	0.228624204600372\\
1.52376176176176	0.227877502168768\\
1.52576276276276	0.227136329386944\\
1.52776376376376	0.226400621841433\\
1.52976476476476	0.225670316152286\\
1.53176576576577	0.224945349951705\\
1.53376676676677	0.224225661863218\\
1.53576776776777	0.223511191481383\\
1.53776876876877	0.222801879352007\\
1.53976976976977	0.22209766695286\\
1.54177077077077	0.221398496674877\\
1.54377177177177	0.220704311803816\\
1.54577277277277	0.220015056502386\\
1.54777377377377	0.2193306757928\\
1.54977477477477	0.218651115539769\\
1.55177577577578	0.217976322433899\\
1.55377677677678	0.217306243975501\\
1.55577777777778	0.216640828458785\\
1.55777877877878	0.215980024956442\\
1.55977977977978	0.215323783304584\\
1.56178078078078	0.214672054088052\\
1.56378178178178	0.214024788626072\\
1.56578278278278	0.213381938958239\\
1.56778378378378	0.212743457830844\\
1.56978478478478	0.212109298683513\\
1.57178578578579	0.211479415636155\\
1.57378678678679	0.210853763476217\\
1.57578778778779	0.210232297646233\\
1.57778878878879	0.209614974231653\\
1.57978978978979	0.209001749948955\\
1.58179079079079	0.208392582134029\\
1.58379179179179	0.207787428730817\\
1.58579279279279	0.207186248280211\\
1.58779379379379	0.206588999909209\\
1.58979479479479	0.2059956433203\\
1.5917957957958	0.205406138781095\\
1.5937967967968	0.20482044711418\\
1.5957977977978	0.2042385296872\\
1.5977987987988	0.203660348403154\\
1.5997997997998	0.203085865690908\\
1.6018008008008	0.20251504449591\\
1.6038018018018	0.201947848271108\\
1.6058028028028	0.201384240968063\\
1.6078038038038	0.200824187028254\\
1.6098048048048	0.20026765137457\\
1.61180580580581	0.199714599402982\\
1.61380680680681	0.199164996974385\\
1.61580780780781	0.198618810406631\\
1.61780880880881	0.198076006466706\\
1.61980980980981	0.19753655236309\\
1.62181081081081	0.197000415738264\\
1.62381181181181	0.196467564661382\\
1.62581281281281	0.195937967621086\\
1.62781381381381	0.195411593518479\\
1.62981481481481	0.194888411660231\\
1.63181581581582	0.194368391751833\\
1.63381681681682	0.193851503890993\\
1.63581781781782	0.193337718561152\\
1.63781881881882	0.192827006625143\\
1.63981981981982	0.192319339318977\\
1.64182082082082	0.191814688245745\\
1.64382182182182	0.191313025369651\\
1.64582282282282	0.190814323010158\\
1.64782382382382	0.190318553836254\\
1.64982482482482	0.189825690860829\\
1.65182582582583	0.189335707435163\\
1.65382682682683	0.18884857724352\\
1.65582782782783	0.188364274297852\\
1.65782882882883	0.187882772932605\\
1.65982982982983	0.187404047799619\\
1.66183083083083	0.186928073863139\\
1.66383183183183	0.186454826394906\\
1.66583283283283	0.185984280969361\\
1.66783383383383	0.185516413458921\\
1.66983483483483	0.185051200029359\\
1.67183583583584	0.184588617135268\\
1.67383683683684	0.184128641515606\\
1.67583783783784	0.18367125018933\\
1.67783883883884	0.18321642045111\\
1.67983983983984	0.182764129867126\\
1.68184084084084	0.182314356270934\\
1.68384184184184	0.181867077759424\\
1.68584284284284	0.181422272688835\\
1.68784384384384	0.18097991967086\\
1.68984484484484	0.180539997568809\\
1.69184584584585	0.180102485493854\\
1.69384684684685	0.179667362801331\\
1.69584784784785	0.179234609087116\\
1.69784884884885	0.17880420418407\\
1.69984984984985	0.178376128158538\\
1.70185085085085	0.177950361306918\\
1.70385185185185	0.177526884152291\\
1.70585285285285	0.17710567744111\\
1.70785385385385	0.176686722139945\\
1.70985485485485	0.176269999432296\\
1.71185585585586	0.175855490715446\\
1.71385685685686	0.175443177597387\\
1.71585785785786	0.175033041893784\\
1.71785885885886	0.174625065625006\\
1.71985985985986	0.1742192310132\\
1.72186086086086	0.173815520479419\\
1.72386186186186	0.173413916640797\\
1.72586286286286	0.173014402307779\\
1.72786386386386	0.172616960481391\\
1.72986486486486	0.172221574350565\\
1.73186586586587	0.171828227289498\\
1.73386686686687	0.171436902855073\\
1.73586786786787	0.171047584784307\\
1.73786886886887	0.170660256991853\\
1.73986986986987	0.170274903567538\\
1.74187087087087	0.169891508773949\\
1.74387187187187	0.169510057044054\\
1.74587287287287	0.169130532978863\\
1.74787387387387	0.168752921345131\\
1.74987487487487	0.168377207073099\\
1.75187587587588	0.168003375254268\\
1.75387687687688	0.167631411139218\\
1.75587787787788	0.167261300135453\\
1.75787887887888	0.166893027805288\\
1.75987987987988	0.166526579863771\\
1.76188088088088	0.166161942176636\\
1.76388188188188	0.165799100758288\\
1.76588288288288	0.165438041769825\\
1.76788388388388	0.16507875151709\\
1.76988488488488	0.164721216448755\\
1.77188588588589	0.164365423154434\\
1.77388688688689	0.164011358362828\\
1.77588788788789	0.163659008939897\\
1.77788888888889	0.163308361887068\\
1.77988988988989	0.162959404339464\\
1.78189089089089	0.162612123564162\\
1.78389189189189	0.162266506958485\\
1.78589289289289	0.161922542048313\\
1.78789389389389	0.161580216486422\\
1.78989489489489	0.161239518050859\\
1.7918958958959	0.160900434643323\\
1.7938968968969	0.160562954287593\\
1.7958978978979	0.160227065127962\\
1.7978988988989	0.159892755427708\\
1.7998998998999	0.159560013567581\\
1.8019009009009	0.159228828044319\\
1.8039019019019	0.15889918746918\\
1.8059029029029	0.158571080566504\\
1.8079039039039	0.158244496172291\\
1.8099049049049	0.157919423232804\\
1.81190590590591	0.157595850803192\\
1.81390690690691	0.157273768046134\\
1.81590790790791	0.156953164230502\\
1.81790890890891	0.15663402873005\\
1.81990990990991	0.156316351022113\\
1.82191091091091	0.156000120686334\\
1.82391191191191	0.155685327403406\\
1.82591291291291	0.155371960953834\\
1.82791391391391	0.155060011216716\\
1.82991491491491	0.154749468168537\\
1.83191591591592	0.154440321881989\\
1.83391691691692	0.154132562524803\\
1.83591791791792	0.1538261803586\\
1.83791891891892	0.153521165737756\\
1.83991991991992	0.153217509108287\\
1.84192092092092	0.152915201006752\\
1.84392192192192	0.152614232059165\\
1.84592292292292	0.152314592979929\\
1.84792392392392	0.152016274570781\\
1.84992492492492	0.151719267719758\\
1.85192592592593	0.151423563400172\\
1.85392692692693	0.151129152669602\\
1.85592792792793	0.150836026668904\\
1.85792892892893	0.150544176621223\\
1.85992992992993	0.150253593831041\\
1.86193093093093	0.14996426968321\\
1.86393193193193	0.149676195642028\\
1.86593293293293	0.149389363250305\\
1.86793393393393	0.149103764128453\\
1.86993493493493	0.148819389973589\\
1.87193593593594	0.148536232558648\\
1.87393693693694	0.148254283731507\\
1.87593793793794	0.147973535414125\\
1.87793893893894	0.147693979601693\\
1.87993993993994	0.147415608361796\\
1.88194094094094	0.147138413833584\\
1.88394194194194	0.146862388226962\\
1.88594294294294	0.146587523821781\\
1.88794394394394	0.14631381296705\\
1.88994494494494	0.146041248080149\\
1.89194594594595	0.145769821646063\\
1.89394694694695	0.145499526216617\\
1.89594794794795	0.14523035440973\\
1.89794894894895	0.144962298908671\\
1.89994994994995	0.14469535246133\\
1.90195095095095	0.144429507879499\\
1.90395195195195	0.144164758038163\\
1.90595295295295	0.143901095874793\\
1.90795395395395	0.143638514388659\\
1.90995495495495	0.14337700664015\\
1.91195595595596	0.143116565750096\\
1.91395695695696	0.142857184899104\\
1.91595795795796	0.142598857326907\\
1.91795895895896	0.142341576331713\\
1.91995995995996	0.14208533526957\\
1.92196096096096	0.141830127553732\\
1.92396196196196	0.141575946654044\\
1.92596296296296	0.141322786096321\\
1.92796396396396	0.14107063946175\\
1.92996496496496	0.140819500386289\\
1.93196596596597	0.140569362560076\\
1.93396696696697	0.140320219726851\\
1.93596796796797	0.14007206568338\\
1.93796896896897	0.139824894278886\\
1.93996996996997	0.139578699414493\\
1.94197097097097	0.139333475042673\\
1.94397197197197	0.139089215166697\\
1.94597297297297	0.138845913840102\\
1.94797397397397	0.138603565166156\\
1.94997497497497	0.138362163297336\\
1.95197597597598	0.138121702434808\\
1.95397697697698	0.137882176827917\\
1.95597797797798	0.13764358077368\\
1.95797897897898	0.137405908616293\\
1.95997997997998	0.137169154746633\\
1.96198098098098	0.136933313601774\\
1.96398198198198	0.13669837966451\\
1.96598298298298	0.136464347462875\\
1.96798398398398	0.136231211569683\\
1.96998498498498	0.13599896660206\\
1.97198598598599	0.135767607220989\\
1.97398698698699	0.135537128130864\\
1.97598798798799	0.135307524079036\\
1.97798898898899	0.135078789855383\\
1.97998998998999	0.13485092029187\\
1.98199099099099	0.134623910262122\\
1.98399199199199	0.134397754681001\\
1.98599299299299	0.134172448504188\\
1.98799399399399	0.133947986727768\\
1.98999499499499	0.133724364387824\\
1.991995995996	0.133501576560035\\
1.993996996997	0.133279618359273\\
1.995997997998	0.133058484939216\\
1.997998998999	0.132838171491953\\
2	0.132618673247605\\
1.997998998999	0.132838171491953\\
1.995997997998	0.133058484939216\\
1.993996996997	0.133279618359273\\
1.991995995996	0.133501576560035\\
1.98999499499499	0.133724364387824\\
1.98799399399399	0.133947986727768\\
1.98599299299299	0.134172448504188\\
1.98399199199199	0.134397754681001\\
1.98199099099099	0.134623910262122\\
1.97998998998999	0.13485092029187\\
1.97798898898899	0.135078789855383\\
1.97598798798799	0.135307524079036\\
1.97398698698699	0.135537128130864\\
1.97198598598599	0.135767607220989\\
1.96998498498498	0.13599896660206\\
1.96798398398398	0.136231211569683\\
1.96598298298298	0.136464347462875\\
1.96398198198198	0.13669837966451\\
1.96198098098098	0.136933313601774\\
1.95997997997998	0.137169154746633\\
1.95797897897898	0.137405908616293\\
1.95597797797798	0.13764358077368\\
1.95397697697698	0.137882176827917\\
1.95197597597598	0.138121702434808\\
1.94997497497497	0.138362163297336\\
1.94797397397397	0.138603565166156\\
1.94597297297297	0.138845913840102\\
1.94397197197197	0.139089215166697\\
1.94197097097097	0.139333475042672\\
1.93996996996997	0.139578699414493\\
1.93796896896897	0.139824894278886\\
1.93596796796797	0.14007206568338\\
1.93396696696697	0.140320219726851\\
1.93196596596597	0.140569362560076\\
1.92996496496496	0.140819500386289\\
1.92796396396396	0.14107063946175\\
1.92596296296296	0.141322786096321\\
1.92396196196196	0.141575946654044\\
1.92196096096096	0.141830127553732\\
1.91995995995996	0.14208533526957\\
1.91795895895896	0.142341576331713\\
1.91595795795796	0.142598857326907\\
1.91395695695696	0.142857184899104\\
1.91195595595596	0.143116565750096\\
1.90995495495495	0.14337700664015\\
1.90795395395395	0.143638514388659\\
1.90595295295295	0.143901095874793\\
1.90395195195195	0.144164758038163\\
1.90195095095095	0.1444295078795\\
1.89994994994995	0.14469535246133\\
1.89794894894895	0.144962298908671\\
1.89594794794795	0.14523035440973\\
1.89394694694695	0.145499526216617\\
1.89194594594595	0.145769821646063\\
1.88994494494494	0.146041248080149\\
1.88794394394394	0.14631381296705\\
1.88594294294294	0.146587523821781\\
1.88394194194194	0.146862388226962\\
1.88194094094094	0.147138413833584\\
1.87993993993994	0.147415608361796\\
1.87793893893894	0.147693979601693\\
1.87593793793794	0.147973535414125\\
1.87393693693694	0.148254283731507\\
1.87193593593594	0.148536232558648\\
1.86993493493493	0.148819389973589\\
1.86793393393393	0.149103764128453\\
1.86593293293293	0.149389363250305\\
1.86393193193193	0.149676195642028\\
1.86193093093093	0.14996426968321\\
1.85992992992993	0.150253593831041\\
1.85792892892893	0.150544176621223\\
1.85592792792793	0.150836026668904\\
1.85392692692693	0.151129152669602\\
1.85192592592593	0.151423563400172\\
1.84992492492492	0.151719267719758\\
1.84792392392392	0.152016274570781\\
1.84592292292292	0.152314592979929\\
1.84392192192192	0.152614232059165\\
1.84192092092092	0.152915201006752\\
1.83991991991992	0.153217509108287\\
1.83791891891892	0.153521165737756\\
1.83591791791792	0.1538261803586\\
1.83391691691692	0.154132562524803\\
1.83191591591592	0.154440321881989\\
1.82991491491491	0.154749468168537\\
1.82791391391391	0.155060011216716\\
1.82591291291291	0.155371960953834\\
1.82391191191191	0.155685327403406\\
1.82191091091091	0.156000120686334\\
1.81990990990991	0.156316351022113\\
1.81790890890891	0.15663402873005\\
1.81590790790791	0.156953164230502\\
1.81390690690691	0.157273768046134\\
1.81190590590591	0.157595850803192\\
1.8099049049049	0.157919423232804\\
1.8079039039039	0.158244496172291\\
1.8059029029029	0.158571080566504\\
1.8039019019019	0.15889918746918\\
1.8019009009009	0.159228828044319\\
1.7998998998999	0.159560013567581\\
1.7978988988989	0.159892755427708\\
1.7958978978979	0.160227065127962\\
1.7938968968969	0.160562954287593\\
1.7918958958959	0.160900434643323\\
1.78989489489489	0.161239518050859\\
1.78789389389389	0.161580216486422\\
1.78589289289289	0.161922542048313\\
1.78389189189189	0.162266506958485\\
1.78189089089089	0.162612123564162\\
1.77988988988989	0.162959404339464\\
1.77788888888889	0.163308361887068\\
1.77588788788789	0.163659008939897\\
1.77388688688689	0.164011358362828\\
1.77188588588589	0.164365423154434\\
1.76988488488488	0.164721216448755\\
1.76788388388388	0.16507875151709\\
1.76588288288288	0.165438041769825\\
1.76388188188188	0.165799100758288\\
1.76188088088088	0.166161942176636\\
1.75987987987988	0.166526579863771\\
1.75787887887888	0.166893027805288\\
1.75587787787788	0.167261300135453\\
1.75387687687688	0.167631411139218\\
1.75187587587588	0.168003375254268\\
1.74987487487487	0.168377207073099\\
1.74787387387387	0.168752921345131\\
1.74587287287287	0.169130532978863\\
1.74387187187187	0.169510057044054\\
1.74187087087087	0.169891508773949\\
1.73986986986987	0.170274903567538\\
1.73786886886887	0.170660256991853\\
1.73586786786787	0.171047584784307\\
1.73386686686687	0.171436902855073\\
1.73186586586587	0.171828227289498\\
1.72986486486486	0.172221574350565\\
1.72786386386386	0.172616960481391\\
1.72586286286286	0.173014402307779\\
1.72386186186186	0.173413916640797\\
1.72186086086086	0.173815520479419\\
1.71985985985986	0.1742192310132\\
1.71785885885886	0.174625065625006\\
1.71585785785786	0.175033041893784\\
1.71385685685686	0.175443177597387\\
1.71185585585586	0.175855490715446\\
1.70985485485485	0.176269999432296\\
1.70785385385385	0.176686722139945\\
1.70585285285285	0.17710567744111\\
1.70385185185185	0.177526884152291\\
1.70185085085085	0.177950361306918\\
1.69984984984985	0.178376128158538\\
1.69784884884885	0.178804204184071\\
1.69584784784785	0.179234609087116\\
1.69384684684685	0.17966736280133\\
1.69184584584585	0.180102485493854\\
1.68984484484484	0.180539997568809\\
1.68784384384384	0.18097991967086\\
1.68584284284284	0.181422272688835\\
1.68384184184184	0.181867077759424\\
1.68184084084084	0.182314356270934\\
1.67983983983984	0.182764129867126\\
1.67783883883884	0.18321642045111\\
1.67583783783784	0.18367125018933\\
1.67383683683684	0.184128641515606\\
1.67183583583584	0.184588617135268\\
1.66983483483483	0.185051200029359\\
1.66783383383383	0.18551641345892\\
1.66583283283283	0.185984280969361\\
1.66383183183183	0.186454826394906\\
1.66183083083083	0.186928073863139\\
1.65982982982983	0.187404047799619\\
1.65782882882883	0.187882772932605\\
1.65582782782783	0.188364274297852\\
1.65382682682683	0.18884857724352\\
1.65182582582583	0.189335707435163\\
1.64982482482482	0.189825690860829\\
1.64782382382382	0.190318553836254\\
1.64582282282282	0.190814323010158\\
1.64382182182182	0.191313025369651\\
1.64182082082082	0.191814688245745\\
1.63981981981982	0.192319339318977\\
1.63781881881882	0.192827006625143\\
1.63581781781782	0.193337718561152\\
1.63381681681682	0.193851503890993\\
1.63181581581582	0.194368391751833\\
1.62981481481481	0.194888411660231\\
1.62781381381381	0.195411593518479\\
1.62581281281281	0.195937967621086\\
1.62381181181181	0.196467564661382\\
1.62181081081081	0.197000415738264\\
1.61980980980981	0.19753655236309\\
1.61780880880881	0.198076006466706\\
1.61580780780781	0.198618810406631\\
1.61380680680681	0.199164996974386\\
1.61180580580581	0.199714599402982\\
1.6098048048048	0.20026765137457\\
1.6078038038038	0.200824187028254\\
1.6058028028028	0.201384240968062\\
1.6038018018018	0.201947848271108\\
1.6018008008008	0.20251504449591\\
1.5997997997998	0.203085865690908\\
1.5977987987988	0.203660348403154\\
1.5957977977978	0.204238529687199\\
1.5937967967968	0.20482044711418\\
1.5917957957958	0.205406138781095\\
1.58979479479479	0.2059956433203\\
1.58779379379379	0.206588999909208\\
1.58579279279279	0.207186248280211\\
1.58379179179179	0.207787428730817\\
1.58179079079079	0.208392582134029\\
1.57978978978979	0.209001749948956\\
1.57778878878879	0.209614974231653\\
1.57578778778779	0.210232297646233\\
1.57378678678679	0.210853763476217\\
1.57178578578579	0.211479415636155\\
1.56978478478478	0.212109298683513\\
1.56778378378378	0.212743457830844\\
1.56578278278278	0.213381938958239\\
1.56378178178178	0.214024788626072\\
1.56178078078078	0.214672054088052\\
1.55977977977978	0.215323783304584\\
1.55777877877878	0.215980024956441\\
1.55577777777778	0.216640828458785\\
1.55377677677678	0.217306243975501\\
1.55177577577578	0.217976322433899\\
1.54977477477477	0.218651115539769\\
1.54777377377377	0.2193306757928\\
1.54577277277277	0.220015056502386\\
1.54377177177177	0.220704311803816\\
1.54177077077077	0.221398496674876\\
1.53976976976977	0.22209766695286\\
1.53776876876877	0.222801879352007\\
1.53576776776777	0.223511191481383\\
1.53376676676677	0.224225661863219\\
1.53176576576577	0.224945349951705\\
1.52976476476476	0.225670316152286\\
1.52776376376376	0.226400621841432\\
1.52576276276276	0.227136329386943\\
1.52376176176176	0.227877502168768\\
1.52176076076076	0.228624204600372\\
1.51975975975976	0.229376502150673\\
1.51775875875876	0.230134461366554\\
1.51575775775776	0.230898149895969\\
1.51375675675676	0.231667636511684\\
1.51175575575576	0.232442991135635\\
1.50975475475475	0.233224284863967\\
1.50775375375375	0.234011589992735\\
1.50575275275275	0.234804980044323\\
1.50375175175175	0.235604529794585\\
1.50175075075075	0.236410315300722\\
1.49974974974975	0.237222413929973\\
1.49774874874875	0.238040904389057\\
1.49574774774775	0.238865866754495\\
1.49374674674675	0.239697382503746\\
1.49174574574575	0.240535534547268\\
1.48974474474474	0.241380407261464\\
1.48774374374374	0.242232086522598\\
1.48574274274274	0.243090659741689\\
1.48374174174174	0.243956215900421\\
1.48174074074074	0.244828845588102\\
1.47973973973974	0.24570864103972\\
1.47773873873874	0.246595696175127\\
1.47573773773774	0.247490106639389\\
1.47373673673674	0.248391969844353\\
1.47173573573574	0.249301385011472\\
1.46973473473473	0.250218453215926\\
1.46773373373373	0.251143277432101\\
1.46573273273273	0.252075962580468\\
1.46373173173173	0.253016615575905\\
1.46173073073073	0.253965345377549\\
1.45972972972973	0.25492226304019\\
1.45772872872873	0.255887481767305\\
1.45572772772773	0.256861116965773\\
1.45372672672673	0.257843286302351\\
1.45172572572573	0.258834109761956\\
1.44972472472472	0.259833709707848\\
1.44772372372372	0.26084221094379\\
1.44572272272272	0.261859740778228\\
1.44372172172172	0.262886429090623\\
1.44172072072072	0.26392240839998\\
1.43971971971972	0.264967813935681\\
1.43771871871872	0.266022783710724\\
1.43571771771772	0.267087458597444\\
1.43371671671672	0.268161982405829\\
1.43171571571572	0.26924650196457\\
1.42971471471471	0.270341167204893\\
1.42771371371371	0.271446131247356\\
1.42571271271271	0.272561550491704\\
1.42371171171171	0.273687584709917\\
1.42171071071071	0.274824397142599\\
1.41970970970971	0.275972154598842\\
1.41770870870871	0.277131027559722\\
1.41570770770771	0.278301190285605\\
1.41370670670671	0.279482820927384\\
1.41170570570571	0.280676101641894\\
1.4097047047047	0.281881218711638\\
1.4077037037037	0.283098362669033\\
1.4057027027027	0.284327728425407\\
1.4037017017017	0.285569515404954\\
1.4017007007007	0.286823927683852\\
1.3996996996997	0.288091174134844\\
1.3976986986987	0.289371468577485\\
1.3956976976977	0.290665029934379\\
1.3936966966967	0.29197208239363\\
1.3916956956957	0.293292855577893\\
1.38969469469469	0.294627584720273\\
1.38769369369369	0.295976510847467\\
1.38569269269269	0.297339880970462\\
1.38369169169169	0.298717948283235\\
1.38169069069069	0.300110972369785\\
1.37968968968969	0.301519219419997\\
1.37768868868869	0.302942962454744\\
1.37568768768769	0.304382481560738\\
1.37368668668669	0.305838064135643\\
1.37168568568569	0.307310005143969\\
1.36968468468468	0.308798607384372\\
1.36768368368368	0.310304181768953\\
1.36568268268268	0.311827047615184\\
1.36368168168168	0.313367532951233\\
1.36168068068068	0.314925974835376\\
1.35967967967968	0.316502719690291\\
1.35767867867868	0.318098123653125\\
1.35567767767768	0.319712552942214\\
1.35367667667668	0.321346384241391\\
1.35167567567568	0.323000005102982\\
1.34967467467467	0.324673814370538\\
1.34767367367367	0.326368222622535\\
1.34567267267267	0.328083652638242\\
1.34367167167167	0.329820539887199\\
1.34167067067067	0.331579333043686\\
1.33966966966967	0.333360494527788\\
1.33766866866867	0.335164501074706\\
1.33566766766767	0.336991844334192\\
1.33366666666667	0.338843031501891\\
1.33166566566567	0.34071858598486\\
1.32966466466466	0.342619048103415\\
1.32766366366366	0.344544975831691\\
1.32566266266266	0.346496945579675\\
1.32366166166166	0.348475553019372\\
1.32166066066066	0.350481413958289\\
1.31965965965966	0.352515165263477\\
1.31765865865866	0.354577465839697\\
1.31565765765766	0.3566689976657\\
1.31365665665666	0.358790466892635\\
1.31165565565566	0.360942605009385\\
1.30965465465465	0.363126170079606\\
1.30765365365365	0.365341948056011\\
1.30565265265265	0.367590754177733\\
1.30365165165165	0.369873434457139\\
1.30165065065065	0.372190867263297\\
1.29964964964965	0.374543965009525\\
1.29764864864865	0.376933675953569\\
1.29564764764765	0.379360986119599\\
1.29364664664665	0.381826921351966\\
1.29164564564565	0.384332549512083\\
1.28964464464464	0.386878982830197\\
1.28764364364364	0.389467380425911\\
1.28564264264264	0.392098951011878\\
1.28364164164164	0.394774955797216\\
1.28164064064064	0.397496711608677\\
1.27963963963964	0.400265594249487\\
1.27763863863864	0.403083042118111\\
1.27563763763764	0.405950560111597\\
1.27363663663664	0.408869723840904\\
1.27163563563564	0.411842184188814\\
1.26963463463463	0.414869672244524\\
1.26763363363363	0.417954004653204\\
1.26563263263263	0.421097089423154\\
1.26363163163163	0.424300932238808\\
1.26163063063063	0.427567643333423\\
1.25962962962963	0.430899444982622\\
1.25762862862863	0.434298679687472\\
1.25562762762763	0.437767819125007\\
1.25362662662663	0.441309473954779\\
1.25162562562563	0.44492640458199\\
1.24962462462462	0.448621532991655\\
1.24762362362362	0.452397955785235\\
1.24562262262262	0.456258958569689\\
1.24362162162162	0.460208031871774\\
1.24162062062062	0.464248888776287\\
1.23961961961962	0.468385484518517\\
1.23761861861862	0.472622038297103\\
1.23561761761762	0.476963057618122\\
1.23361661661662	0.481413365530988\\
1.23161561561562	0.485978131181459\\
1.22961461461461	0.490662904178805\\
1.22761361361361	0.495473653365102\\
1.22561261261261	0.500416810683182\\
1.22361161161161	0.505499320971027\\
1.22161061061061	0.510728698672271\\
1.21960960960961	0.516113092651032\\
1.21760860860861	0.521661360543375\\
1.21560760760761	0.527383154382767\\
1.21360660660661	0.533289019617454\\
1.21160560560561	0.539390510115968\\
1.2096046046046	0.545700322361867\\
1.2076036036036	0.552232452813502\\
1.2056026026026	0.559002383395232\\
1.2036016016016	0.566027301371835\\
1.2016006006006	0.573326361535197\\
1.1995995995996	0.580921000842666\\
1.1975985985986	0.58883531858342\\
1.1955975975976	0.5970965390946\\
1.1935965965966	0.605735579405958\\
1.1915955955956	0.614787751543354\\
1.18959459459459	0.624293639442902\\
1.18759359359359	0.634300204811641\\
1.18559259259259	0.644862196806623\\
1.18359159159159	0.656043970143131\\
1.18159059059059	0.667921859985036\\
1.17958958958959	0.680587327359371\\
1.17758858858859	0.694151188202887\\
1.17558758758759	0.708749392718342\\
1.17358658658659	0.724551062708601\\
1.17158558558559	0.741769877533842\\
1.16958458458458	0.760680510972083\\
1.16758358358358	0.781642785835122\\
1.16558258258258	0.805137648382651\\
1.16358158158158	0.831820788775619\\
1.16158058058058	0.862599887392203\\
1.15957957957958	0.898730679365395\\
1.15757857857858	0.941866761318058\\
1.15557757757758	0.99376804304306\\
1.15357657657658	1.05489287820949\\
1.15157557557558	1.12180259927516\\
1.14957457457457	1.18763014397434\\
1.14757357357357	1.24736742454748\\
1.14557257257257	1.29991434872089\\
1.14357157157157	1.34610822705563\\
1.14157057057057	1.38714340592498\\
1.13956956956957	1.42405721776968\\
1.13756856856857	1.45765530203891\\
1.13556756756757	1.48854770567032\\
1.13356656656657	1.51719762117474\\
1.13156556556557	1.54396112732775\\
1.12956456456456	1.56911618250817\\
1.12756356356356	1.59288317026758\\
1.12556256256256	1.61543942955848\\
1.12356156156156	1.63692962323703\\
1.12156056056056	1.65747323479378\\
1.11955955955956	1.67717006817565\\
1.11755855855856	1.69610434216265\\
1.11555755755756	1.7143477818749\\
1.11355655655656	1.73196198449026\\
1.11155555555556	1.74900025235719\\
1.10955455455455	1.76550903005086\\
1.10755355355355	1.78152904317257\\
1.10555255255255	1.79709620988167\\
1.10355155155155	1.81224237731973\\
1.10155055055055	1.82699592171619\\
1.09954954954955	1.84138224133908\\
1.09754854854855	1.855424164454\\
1.09554754754755	1.8691422892971\\
1.09354654654655	1.88255526923517\\
1.09154554554555	1.89568005340463\\
1.08954454454454	1.90853209093738\\
1.08754354354354	1.92112550520946\\
1.08554254254254	1.93347324325994\\
1.08354154154154	1.94558720452335\\
1.08154054054054	1.95747835223464\\
1.07953953953954	1.96915681024408\\
1.07753853853854	1.98063194748816\\
1.07553753753754	1.99191245196804\\
1.07353653653654	2.00300639577007\\
1.07153553553554	2.01392129240758\\
1.06953453453453	2.02466414755334\\
1.06753353353353	2.03524150406283\\
1.06553253253253	2.04565948204727\\
1.06353153153153	2.05592381464102\\
1.06153053053053	2.06603988001033\\
1.05952952952953	2.07601273007231\\
1.05752852852853	2.08584711632472\\
1.05552752752753	2.09554751313183\\
1.05352652652653	2.10511813876416\\
1.05152552552553	2.1145629744498\\
1.04952452452452	2.123885781661\\
1.04752352352352	2.13309011783087\\
1.04552252252252	2.14217935067047\\
1.04352152152152	2.15115667123484\\
1.04152052052052	2.1600251058692\\
1.03951951951952	2.16878752714967\\
1.03751851851852	2.17744666392042\\
1.03551751751752	2.1860051105167\\
1.03351651651652	2.19446533525303\\
1.03151551551552	2.20282968824725\\
1.02951451451451	2.21110040864301\\
1.02751351351351	2.21927963128659\\
1.02551251251251	2.22736939290798\\
1.02351151151151	2.23537163785092\\
1.02151051051051	2.24328822339179\\
1.01950950950951	2.25112092468337\\
1.01750850850851	2.25887143935585\\
1.01550750750751	2.26654139180395\\
1.01350650650651	2.27413233718667\\
1.01150550550551	2.28164576516328\\
1.0095045045045	2.2890831033869\\
1.0075035035035	2.29644572077535\\
1.0055025025025	2.30373493057667\\
1.0035015015015	2.31095199324553\\
1.0015005005005	2.3180981191449\\
0.9994994994995	2.32517447108653\\
0.997498498498499	2.33218216672203\\
0.995497497497498	2.33912228079577\\
0.993496496496497	2.34599584726973\\
0.991495495495496	2.35280386132941\\
0.989494494494495	2.35954728127922\\
0.987493493493494	2.36622703033526\\
0.985492492492492	2.37284399832248\\
0.983491491491492	2.37939904328263\\
0.981490490490491	2.38589299299921\\
0.97948948948949	2.39232664644476\\
0.977488488488489	2.39870077515559\\
0.975487487487487	2.40501612453858\\
0.973486486486487	2.41127341511443\\
0.971485485485486	2.4174733437012\\
0.969484484484485	2.42361658454186\\
0.967483483483484	2.4297037903792\\
0.965482482482482	2.43573559348117\\
0.963481481481481	2.44171260661955\\
0.961480480480481	2.44763542400465\\
0.95947947947948	2.45350462217834\\
0.957478478478479	2.45932076086783\\
0.955477477477477	2.46508438380225\\
0.953476476476476	2.47079601949394\\
0.951475475475476	2.47645618198619\\
0.949474474474475	2.48206537156933\\
0.947473473473474	2.4876240754664\\
0.945472472472472	2.49313276849003\\
0.943471471471471	2.4985919136719\\
0.941470470470471	2.50400196286574\\
0.93946946946947	2.50936335732539\\
0.937468468468469	2.5146765282587\\
0.935467467467467	2.51994189735834\\
0.933466466466467	2.52515987731052\\
0.931465465465466	2.53033087228239\\
0.929464464464465	2.53545527838896\\
0.927463463463464	2.54053348414031\\
0.925462462462462	2.54556587086974\\
0.923461461461462	2.55055281314362\\
0.92146046046046	2.55549467915339\\
0.91945945945946	2.56039183109055\\
0.917458458458459	2.56524462550489\\
0.915457457457457	2.57005341364675\\
0.913456456456457	2.57481854179366\\
0.911455455455455	2.57954035156185\\
0.909454454454455	2.58421918020309\\
0.907453453453454	2.58885536088733\\
0.905452452452452	2.59344922297148\\
0.903451451451452	2.59800109225476\\
0.90145045045045	2.60251129122105\\
0.899449449449449	2.60698013926848\\
0.897448448448449	2.61140795292681\\
0.895447447447447	2.61579504606274\\
0.893446446446447	2.62014173007366\\
0.891445445445445	2.6244483140701\\
0.889444444444444	2.62871510504719\\
0.887443443443444	2.63294240804549\\
0.885442442442442	2.63713052630149\\
0.883441441441442	2.64127976138815\\
0.88144044044044	2.64539041334564\\
0.879439439439439	2.64946278080276\\
0.877438438438438	2.65349716108923\\
0.875437437437438	2.65749385033923\\
0.873436436436437	2.66145314358639\\
0.871435435435435	2.66537533485061\\
0.869434434434434	2.66926071721699\\
0.867433433433433	2.67310958290711\\
0.865432432432433	2.67692222334305\\
0.863431431431432	2.68069892920428\\
0.86143043043043	2.68443999047794\\
0.859429429429429	2.68814569650249\\
0.857428428428428	2.69181633600532\\
0.855427427427428	2.69545219713441\\
0.853426426426427	2.69905356748428\\
0.851425425425425	2.70262073411674\\
0.849424424424424	2.70615398357641\\
0.847423423423423	2.70965360190149\\
0.845422422422422	2.71311987462995\\
0.843421421421422	2.71655308680145\\
0.84142042042042	2.71995352295516\\
0.839419419419419	2.72332146712381\\
0.837418418418418	2.72665720282417\\
0.835417417417417	2.72996101304426\\
0.833416416416417	2.73323318022739\\
0.831415415415415	2.73647398625347\\
0.829414414414414	2.73968371241753\\
0.827413413413413	2.74286263940597\\
0.825412412412412	2.74601104727052\\
0.823411411411412	2.74912921540025\\
0.82141041041041	2.75221742249172\\
0.819409409409409	2.75527594651755\\
0.817408408408409	2.7583050646936\\
0.815407407407407	2.76130505344484\\
0.813406406406406	2.76427618837013\\
0.811405405405405	2.76721874420619\\
0.809404404404404	2.77013299479069\\
0.807403403403404	2.77301921302479\\
0.805402402402402	2.77587767083524\\
0.803401401401401	2.7787086391361\\
0.8014004004004	2.7815123877903\\
0.799399399399399	2.78428918557108\\
0.797398398398398	2.78703930012355\\
0.795397397397397	2.78976299792632\\
0.793396396396396	2.79246054425343\\
0.791395395395395	2.7951322031366\\
0.789394394394394	2.79777823732795\\
0.787393393393393	2.80039890826322\\
0.785392392392392	2.80299447602559\\
0.783391391391391	2.80556519931019\\
0.78139039039039	2.80811133538935\\
0.779389389389389	2.81063314007859\\
0.777388388388388	2.8131308677036\\
0.775387387387387	2.81560477106793\\
0.773386386386386	2.81805510142182\\
0.771385385385385	2.82048210843191\\
0.769384384384384	2.82288604015196\\
0.767383383383383	2.82526714299473\\
0.765382382382382	2.82762566170482\\
0.763381381381381	2.8299618393327\\
0.76138038038038	2.83227591720984\\
0.759379379379379	2.83456813492493\\
0.757378378378378	2.83683873030134\\
0.755377377377377	2.83908793937565\\
0.753376376376376	2.84131599637739\\
0.751375375375375	2.84352313370996\\
0.749374374374374	2.84570958193265\\
0.747373373373373	2.84787556974386\\
0.745372372372372	2.85002132396553\\
0.743371371371371	2.85214706952857\\
0.74137037037037	2.85425302945958\\
0.739369369369369	2.85633942486862\\
0.737368368368368	2.85840647493807\\
0.735367367367367	2.86045439691266\\
0.733366366366366	2.86248340609049\\
0.731365365365365	2.86449371581518\\
0.729364364364364	2.86648553746902\\
0.727363363363363	2.86845908046716\\
0.725362362362362	2.87041455225278\\
0.723361361361361	2.87235215829327\\
0.72136036036036	2.87427210207731\\
0.719359359359359	2.87617458511292\\
0.717358358358358	2.87805980692638\\
0.715357357357357	2.87992796506209\\
0.713356356356356	2.88177925508319\\
0.711355355355355	2.88361387057311\\
0.709354354354354	2.88543200313784\\
0.707353353353353	2.88723384240899\\
0.705352352352352	2.88901957604767\\
0.703351351351351	2.890789389749\\
0.70135035035035	2.89254346724738\\
0.699349349349349	2.89428199032238\\
0.697348348348348	2.89600513880535\\
0.695347347347347	2.89771309058657\\
0.693346346346346	2.89940602162305\\
0.691345345345345	2.90108410594688\\
0.689344344344344	2.90274751567412\\
0.687343343343343	2.90439642101423\\
0.685342342342342	2.90603099027996\\
0.683341341341341	2.90765138989779\\
0.68134034034034	2.90925778441868\\
0.679339339339339	2.91085033652944\\
0.677338338338338	2.91242920706429\\
0.675337337337337	2.91399455501701\\
0.673336336336336	2.91554653755327\\
0.671335335335335	2.91708531002346\\
0.669334334334334	2.91861102597569\\
0.667333333333333	2.92012383716922\\
0.665332332332332	2.92162389358808\\
0.663331331331331	2.92311134345501\\
0.66133033033033	2.9245863332456\\
0.659329329329329	2.9260490077027\\
0.657328328328328	2.92749950985099\\
0.655327327327327	2.9289379810118\\
0.653326326326326	2.93036456081805\\
0.651325325325325	2.93177938722937\\
0.649324324324324	2.9331825965474\\
0.647323323323323	2.93457432343117\\
0.645322322322322	2.93595470091264\\
0.643321321321321	2.93732386041229\\
0.64132032032032	2.93868193175486\\
0.639319319319319	2.94002904318513\\
0.637318318318318	2.94136532138377\\
0.635317317317317	2.94269089148327\\
0.633316316316316	2.94400587708389\\
0.631315315315315	2.94531040026961\\
0.629314314314314	2.94660458162419\\
0.627313313313313	2.94788854024715\\
0.625312312312312	2.94916239376982\\
0.623311311311311	2.95042625837134\\
0.62131031031031	2.95168024879465\\
0.619309309309309	2.95292447836251\\
0.617308308308308	2.9541590589934\\
0.615307307307307	2.95538410121747\\
0.613306306306306	2.95659971419236\\
0.611305305305305	2.95780600571906\\
0.609304304304304	2.95900308225765\\
0.607303303303303	2.96019104894294\\
0.605302302302302	2.96137000960018\\
0.603301301301301	2.96254006676052\\
0.6013003003003	2.96370132167653\\
0.599299299299299	2.96485387433754\\
0.597298298298298	2.96599782348495\\
0.595297297297297	2.96713326662741\\
0.593296296296296	2.96826030005595\\
0.591295295295295	2.96937901885891\\
0.589294294294294	2.97048951693688\\
0.587293293293293	2.97159188701747\\
0.585292292292292	2.97268622066996\\
0.583291291291291	2.97377260831986\\
0.58129029029029	2.97485113926335\\
0.579289289289289	2.9759219016816\\
0.577288288288288	2.97698498265494\\
0.575287287287287	2.97804046817695\\
0.573286286286286	2.97908844316841\\
0.571285285285285	2.98012899149109\\
0.569284284284284	2.98116219596147\\
0.567283283283283	2.98218813836425\\
0.565282282282282	2.98320689946583\\
0.563281281281281	2.98421855902755\\
0.56128028028028	2.98522319581889\\
0.559279279279279	2.98622088763048\\
0.557278278278278	2.98721171128697\\
0.555277277277277	2.9881957426598\\
0.553276276276276	2.98917305667983\\
0.551275275275275	2.99014372734979\\
0.549274274274274	2.99110782775663\\
0.547273273273273	2.99206543008376\\
0.545272272272272	2.99301660562308\\
0.543271271271271	2.99396142478692\\
0.54127027027027	2.99489995711984\\
0.539269269269269	2.9958322713103\\
0.537268268268268	2.99675843520217\\
0.535267267267267	2.99767851580611\\
0.533266266266266	2.99859257931085\\
0.531265265265265	2.99950069109426\\
0.529264264264264	3.00040291573438\\
0.527263263263263	3.00129931702021\\
0.525262262262262	3.0021899579625\\
0.523261261261261	3.00307490080427\\
0.52126026026026	3.00395420703125\\
0.519259259259259	3.00482793738228\\
0.517258258258258	3.00569615185941\\
0.515257257257257	3.00655890973799\\
0.513256256256256	3.00741626957662\\
0.511255255255255	3.00826828922692\\
0.509254254254254	3.00911502584323\\
0.507253253253253	3.00995653589216\\
0.505252252252252	3.01079287516199\\
0.503251251251251	3.011624098772\\
0.50125025025025	3.01245026118163\\
0.499249249249249	3.01327141617231\\
0.497248248248248	3.01408761696607\\
0.495247247247247	3.01489891606879\\
0.493246246246246	3.01570536538997\\
0.491245245245245	3.01650701622343\\
0.489244244244244	3.01730391925569\\
0.487243243243243	3.0180961245742\\
0.485242242242242	3.01888368167628\\
0.483241241241241	3.01966663947591\\
0.48124024024024	3.02044504631345\\
0.479239239239239	3.02121894996199\\
0.477238238238238	3.021988397636\\
0.475237237237237	3.02275343599894\\
0.473236236236236	3.0235141111705\\
0.471235235235235	3.02427046873466\\
0.469234234234234	3.02502255374667\\
0.467233233233233	3.02577041074009\\
0.465232232232232	3.0265140837345\\
0.463231231231231	3.02725361624245\\
0.46123023023023	3.02798905127596\\
0.459229229229229	3.02872043135427\\
0.457228228228228	3.0294477985098\\
0.455227227227227	3.0301711942952\\
0.453226226226226	3.03089065978995\\
0.451225225225225	3.03160623560686\\
0.449224224224224	3.03231796189821\\
0.447223223223223	3.03302587836276\\
0.445222222222222	3.03373002425113\\
0.443221221221221	3.03443043837232\\
0.44122022022022	3.03512715910024\\
0.439219219219219	3.03582022437874\\
0.437218218218218	3.03650967172827\\
0.435217217217217	3.03719553825116\\
0.433216216216216	3.03787786063783\\
0.431215215215215	3.03855667517213\\
0.429214214214214	3.03923201773675\\
0.427213213213213	3.0399039238193\\
0.425212212212212	3.04057242851693\\
0.423211211211211	3.04123756654227\\
0.42121021021021	3.04189937222841\\
0.419209209209209	3.04255787953416\\
0.417208208208208	3.04321312204904\\
0.415207207207207	3.04386513299852\\
0.413206206206206	3.0445139452488\\
0.411205205205205	3.04515959131178\\
0.409204204204204	3.04580210334988\\
0.407203203203203	3.04644151318062\\
0.405202202202202	3.04707785228175\\
0.403201201201201	3.04771115179541\\
0.4012002002002	3.04834144253281\\
0.399199199199199	3.04896875497894\\
0.397198198198198	3.0495931192966\\
0.395197197197197	3.05021456533123\\
0.393196196196196	3.05083312261479\\
0.391195195195195	3.05144882037025\\
0.389194194194194	3.0520616875157\\
0.387193193193193	3.05267175266846\\
0.385192192192192	3.05327904414912\\
0.383191191191191	3.05388358998566\\
0.38119019019019	3.0544854179174\\
0.379189189189189	3.05508455539853\\
0.377188188188188	3.05568102960257\\
0.375187187187187	3.05627486742563\\
0.373186186186186	3.0568660954904\\
0.371185185185185	3.05745474014978\\
0.369184184184184	3.05804082749057\\
0.367183183183183	3.05862438333676\\
0.365182182182182	3.05920543325348\\
0.363181181181181	3.0597840025503\\
0.36118018018018	3.0603601162844\\
0.359179179179179	3.06093379926455\\
0.357178178178178	3.06150507605375\\
0.355177177177177	3.06207397097313\\
0.353176176176176	3.06264050810469\\
0.351175175175175	3.06320471129486\\
0.349174174174174	3.06376660415769\\
0.347173173173173	3.0643262100774\\
0.345172172172172	3.06488355221217\\
0.343171171171171	3.06543865349681\\
0.34117017017017	3.06599153664564\\
0.339169169169169	3.06654222415577\\
0.337168168168168	3.06709073830966\\
0.335167167167167	3.06763710117816\\
0.333166166166166	3.06818133462336\\
0.331165165165165	3.06872346030131\\
0.329164164164164	3.06926349966473\\
0.327163163163163	3.06980147396593\\
0.325162162162162	3.0703374042592\\
0.323161161161161	3.07087131140353\\
0.32116016016016	3.07140321606543\\
0.319159159159159	3.0719331387212\\
0.317158158158158	3.07246109965973\\
0.315157157157157	3.07298711898465\\
0.313156156156156	3.07351121661727\\
0.311155155155155	3.07403341229856\\
0.309154154154154	3.07455372559179\\
0.307153153153153	3.07507217588494\\
0.305152152152152	3.07558878239281\\
0.303151151151151	3.07610356415966\\
0.30115015015015	3.07661654006115\\
0.299149149149149	3.07712772880668\\
0.297148148148148	3.07763714894182\\
0.295147147147147	3.07814481885015\\
0.293146146146146	3.07865075675586\\
0.291145145145145	3.07915498072534\\
0.289144144144144	3.07965750866982\\
0.287143143143143	3.08015835834698\\
0.285142142142142	3.08065754736329\\
0.283141141141141	3.08115509317604\\
0.28114014014014	3.08165101309514\\
0.279139139139139	3.08214532428519\\
0.277138138138138	3.08263804376759\\
0.275137137137137	3.08312918842209\\
0.273136136136136	3.0836187749891\\
0.271135135135135	3.08410682007131\\
0.269134134134134	3.08459334013544\\
0.267133133133133	3.08507835151446\\
0.265132132132132	3.08556187040907\\
0.263131131131131	3.08604391288943\\
0.26113013013013	3.08652449489723\\
0.259129129129129	3.08700363224709\\
0.257128128128128	3.08748134062862\\
0.255127127127127	3.08795763560776\\
0.253126126126126	3.08843253262868\\
0.251125125125125	3.08890604701535\\
0.249124124124124	3.08937819397322\\
0.247123123123123	3.08984898859081\\
0.245122122122122	3.09031844584114\\
0.243121121121121	3.09078658058375\\
0.24112012012012	3.09125340756566\\
0.239119119119119	3.09171894142337\\
0.237118118118118	3.09218319668417\\
0.235117117117117	3.09264618776774\\
0.233116116116116	3.09310792898744\\
0.231115115115115	3.09356843455202\\
0.229114114114114	3.09402771856691\\
0.227113113113113	3.09448579503572\\
0.225112112112112	3.09494267786156\\
0.223111111111111	3.09539838084857\\
0.22111011011011	3.09585291770321\\
0.219109109109109	3.09630630203562\\
0.217108108108108	3.09675854736105\\
0.215107107107107	3.09720966710106\\
0.213106106106106	3.09765967458499\\
0.211105105105105	3.09810858305115\\
0.209104104104104	3.09855640564819\\
0.207103103103103	3.09900315543625\\
0.205102102102102	3.09944884538839\\
0.203101101101101	3.09989348839308\\
0.2011001001001	3.10033709725004\\
0.199099099099099	3.10077968467958\\
0.197098098098098	3.10122126331841\\
0.195097097097097	3.10166184572216\\
0.193096096096096	3.10210144436658\\
0.191095095095095	3.1025400716487\\
0.189094094094094	3.102977739888\\
0.187093093093093	3.10341446132751\\
0.185092092092092	3.10385024813499\\
0.183091091091091	3.10428511240405\\
0.18109009009009	3.1047190661552\\
0.179089089089089	3.10515212133702\\
0.177088088088088	3.10558428982718\\
0.175087087087087	3.10601558343359\\
0.173086086086086	3.10644601389537\\
0.171085085085085	3.10687559288399\\
0.169084084084084	3.10730433200425\\
0.167083083083083	3.10773224279536\\
0.165082082082082	3.10815933673191\\
0.163081081081081	3.10858562522495\\
0.16108008008008	3.10901111962292\\
0.159079079079079	3.10943583121271\\
0.157078078078078	3.10985977122057\\
0.155077077077077	3.11028295081317\\
0.153076076076076	3.1107053810985\\
0.151075075075075	3.11112707312684\\
0.149074074074074	3.11154803789173\\
0.147073073073073	3.1119682863309\\
0.145072072072072	3.11238782932718\\
0.143071071071071	3.11280667770944\\
0.14107007007007	3.11322484225352\\
0.139069069069069	3.11364233368311\\
0.137068068068068	3.11405916267065\\
0.135067067067067	3.11447533983823\\
0.133066066066066	3.1148908757585\\
0.131065065065065	3.11530578095548\\
0.129064064064064	3.11572006590549\\
0.127063063063063	3.11613374103799\\
0.125062062062062	3.11654681673643\\
0.123061061061061	3.1169593033391\\
0.12106006006006	3.11737121113998\\
0.119059059059059	3.11778255038956\\
0.117058058058058	3.1181933312957\\
0.115057057057057	3.1186035640244\\
0.113056056056056	3.11901325870066\\
0.111055055055055	3.11942242540927\\
0.109054054054054	3.11983107419564\\
0.107053053053053	3.12023921506656\\
0.105052052052052	3.12064685799099\\
0.103051051051051	3.12105401290092\\
0.10105005005005	3.12146068969206\\
0.0990490490490491	3.12186689822466\\
0.0970480480480481	3.12227264832429\\
0.0950470470470471	3.12267794978258\\
0.093046046046046	3.12308281235802\\
0.091045045045045	3.12348724577666\\
0.089044044044044	3.12389125973291\\
0.0870430430430431	3.12429486389028\\
0.085042042042042	3.12469806788209\\
0.083041041041041	3.12510088131225\\
0.08104004004004	3.12550331375597\\
0.079039039039039	3.1259053747605\\
0.077038038038038	3.12630707384584\\
0.075037037037037	3.12670842050551\\
0.073036036036036	3.12710942420718\\
0.071035035035035	3.12751009439349\\
0.069034034034034	3.12791044048267\\
0.067033033033033	3.12831047186931\\
0.065032032032032	3.12871019792503\\
0.063031031031031	3.12910962799919\\
0.06103003003003	3.1295087714196\\
0.059029029029029	3.1299076374932\\
0.057028028028028	3.13030623550675\\
0.055027027027027	3.13070457472752\\
0.053026026026026	3.13110266440401\\
0.051025025025025	3.13150051376656\\
0.049024024024024	3.13189813202812\\
0.047023023023023	3.13229552838485\\
0.045022022022022	3.13269271201684\\
0.043021021021021	3.13308969208878\\
0.04102002002002	3.13348647775061\\
0.039019019019019	3.13388307813823\\
0.037018018018018	3.13427950237411\\
0.035017017017017	3.13467575956801\\
0.033016016016016	3.1350718588176\\
0.031015015015015	3.13546780920916\\
0.029014014014014	3.13586361981822\\
0.027013013013013	3.13625929971022\\
0.025012012012012	3.13665485794117\\
0.023011011011011	3.1370503035583\\
0.02101001001001	3.13744564560073\\
0.019009009009009	3.1378408931001\\
0.017008008008008	3.13823605508126\\
0.015007007007007	3.13863114056288\\
0.013006006006006	3.13902615855813\\
0.011005005005005	3.1394211180753\\
0.009004004004004	3.1398160281185\\
0.007003003003003	3.14021089768826\\
0.005002002002002	3.1406057357822\\
0.003001001001001	3.14100055139566\\
0.001	3.14139535352239\\
0.003001001001001	3.14100055139566\\
0.005002002002002	3.1406057357822\\
0.007003003003003	3.14021089768826\\
0.009004004004004	3.1398160281185\\
0.011005005005005	3.1394211180753\\
0.013006006006006	3.13902615855813\\
0.015007007007007	3.13863114056288\\
0.017008008008008	3.13823605508126\\
0.019009009009009	3.1378408931001\\
0.02101001001001	3.13744564560073\\
0.023011011011011	3.1370503035583\\
0.025012012012012	3.13665485794117\\
0.027013013013013	3.13625929971022\\
0.029014014014014	3.13586361981822\\
0.031015015015015	3.13546780920916\\
0.033016016016016	3.1350718588176\\
0.035017017017017	3.13467575956801\\
0.037018018018018	3.13427950237411\\
0.039019019019019	3.13388307813823\\
0.04102002002002	3.13348647775061\\
0.043021021021021	3.13308969208878\\
0.045022022022022	3.13269271201684\\
0.047023023023023	3.13229552838485\\
0.049024024024024	3.13189813202812\\
0.051025025025025	3.13150051376656\\
0.053026026026026	3.13110266440401\\
0.055027027027027	3.13070457472752\\
0.057028028028028	3.13030623550675\\
0.059029029029029	3.1299076374932\\
0.06103003003003	3.1295087714196\\
0.063031031031031	3.12910962799919\\
0.065032032032032	3.12871019792503\\
0.067033033033033	3.12831047186931\\
0.069034034034034	3.12791044048267\\
0.071035035035035	3.12751009439349\\
0.073036036036036	3.12710942420718\\
0.075037037037037	3.12670842050551\\
0.077038038038038	3.12630707384584\\
0.079039039039039	3.1259053747605\\
0.08104004004004	3.12550331375597\\
0.083041041041041	3.12510088131225\\
0.085042042042042	3.12469806788209\\
0.0870430430430431	3.12429486389028\\
0.089044044044044	3.12389125973291\\
0.091045045045045	3.12348724577666\\
0.093046046046046	3.12308281235802\\
0.0950470470470471	3.12267794978258\\
0.0970480480480481	3.12227264832429\\
0.0990490490490491	3.12186689822466\\
0.10105005005005	3.12146068969206\\
0.103051051051051	3.12105401290092\\
0.105052052052052	3.12064685799099\\
0.107053053053053	3.12023921506656\\
0.109054054054054	3.11983107419564\\
0.111055055055055	3.11942242540927\\
0.113056056056056	3.11901325870066\\
0.115057057057057	3.1186035640244\\
0.117058058058058	3.1181933312957\\
0.119059059059059	3.11778255038956\\
0.12106006006006	3.11737121113998\\
0.123061061061061	3.1169593033391\\
0.125062062062062	3.11654681673643\\
0.127063063063063	3.11613374103799\\
0.129064064064064	3.11572006590549\\
0.131065065065065	3.11530578095548\\
0.133066066066066	3.1148908757585\\
0.135067067067067	3.11447533983823\\
0.137068068068068	3.11405916267065\\
0.139069069069069	3.11364233368311\\
0.14107007007007	3.11322484225352\\
0.143071071071071	3.11280667770944\\
0.145072072072072	3.11238782932718\\
0.147073073073073	3.1119682863309\\
0.149074074074074	3.11154803789173\\
0.151075075075075	3.11112707312684\\
0.153076076076076	3.1107053810985\\
0.155077077077077	3.11028295081317\\
0.157078078078078	3.10985977122057\\
0.159079079079079	3.10943583121271\\
0.16108008008008	3.10901111962292\\
0.163081081081081	3.10858562522495\\
0.165082082082082	3.10815933673191\\
0.167083083083083	3.10773224279536\\
0.169084084084084	3.10730433200425\\
0.171085085085085	3.10687559288399\\
0.173086086086086	3.10644601389537\\
0.175087087087087	3.10601558343359\\
0.177088088088088	3.10558428982718\\
0.179089089089089	3.10515212133702\\
0.18109009009009	3.1047190661552\\
0.183091091091091	3.10428511240405\\
0.185092092092092	3.10385024813499\\
0.187093093093093	3.10341446132751\\
0.189094094094094	3.102977739888\\
0.191095095095095	3.1025400716487\\
0.193096096096096	3.10210144436658\\
0.195097097097097	3.10166184572216\\
0.197098098098098	3.10122126331841\\
0.199099099099099	3.10077968467958\\
0.2011001001001	3.10033709725004\\
0.203101101101101	3.09989348839308\\
0.205102102102102	3.09944884538975\\
0.207103103103103	3.09900315543761\\
0.209104104104104	3.09855640564956\\
0.211105105105105	3.09810858305257\\
0.213106106106106	3.09765967458643\\
0.215107107107107	3.09720966710252\\
0.217108108108108	3.09675854736253\\
0.219109109109109	3.09630630203714\\
0.22111011011011	3.09585291770475\\
0.223111111111111	3.09539838085014\\
0.225112112112112	3.09494267786315\\
0.227113113113113	3.09448579503733\\
0.229114114114114	3.09402771856856\\
0.231115115115115	3.09356843455369\\
0.233116116116116	3.09310792898914\\
0.235117117117117	3.09264618776946\\
0.237118118118118	3.09218319668595\\
0.239119119119119	3.09171894142516\\
0.24112012012012	3.09125340756748\\
0.243121121121121	3.09078658058331\\
0.245122122122122	3.09031844584071\\
0.247123123123123	3.08984898859033\\
0.249124124124124	3.08937819397278\\
0.251125125125125	3.08890604701487\\
0.253126126126126	3.0884325326282\\
0.255127127127127	3.08795763560725\\
0.257128128128128	3.08748134062807\\
0.259129129129129	3.08700363224656\\
0.26113013013013	3.08652449489665\\
0.263131131131131	3.08604391288884\\
0.265132132132132	3.08556187040847\\
0.267133133133133	3.08507835151392\\
0.269134134134134	3.08459334013485\\
0.271135135135135	3.08410682007067\\
0.273136136136136	3.08361877498852\\
0.275137137137137	3.08312918842144\\
0.277138138138138	3.08263804376691\\
0.279139139139139	3.0821453242845\\
0.28114014014014	3.08165101309439\\
0.283141141141141	3.0811550931753\\
0.285142142142142	3.08065754736256\\
0.287143143143143	3.0801583583462\\
0.289144144144144	3.07965750866902\\
0.291145145145145	3.07915498072459\\
0.293146146146146	3.07865075675504\\
0.295147147147147	3.07814481884932\\
0.297148148148148	3.07763714894092\\
0.299149149149149	3.07712772880575\\
0.30115015015015	3.07661654006021\\
0.303151151151151	3.07610356415871\\
0.305152152152152	3.07558878239185\\
0.307153153153153	3.07507217588397\\
0.309154154154154	3.07455372559077\\
0.311155155155155	3.07403341229747\\
0.313156156156156	3.07351121661613\\
0.315157157157157	3.07298711898352\\
0.317158158158158	3.07246109965858\\
0.319159159159159	3.07193313872006\\
0.32116016016016	3.07140321606423\\
0.323161161161161	3.07087131140227\\
0.325162162162162	3.07033740425788\\
0.327163163163163	3.06980147396458\\
0.329164164164164	3.06926349966342\\
0.331165165165165	3.06872346029988\\
0.333166166166166	3.06818133462194\\
0.335167167167167	3.06763710117667\\
0.337168168168168	3.0670907383082\\
0.339169169169169	3.06654222415421\\
0.34117017017017	3.06599153664407\\
0.343171171171171	3.06543865349519\\
0.345172172172172	3.06488355221051\\
0.347173173173173	3.06432621007566\\
0.349174174174174	3.06376660415593\\
0.351175175175175	3.06320471129312\\
0.353176176176176	3.06264050810281\\
0.355177177177177	3.0620739709712\\
0.357178178178178	3.0615050760518\\
0.359179179179179	3.06093379926249\\
0.36118018018018	3.06036011628234\\
0.363181181181181	3.05978400254813\\
0.365182182182182	3.05920543325127\\
0.367183183183183	3.05862438333449\\
0.369184184184184	3.05804082748818\\
0.371185185185185	3.05745474014742\\
0.373186186186186	3.05686609548796\\
0.375187187187187	3.05627486742304\\
0.377188188188188	3.05568102959995\\
0.379189189189189	3.0550845553959\\
0.38119019019019	3.05448541791462\\
0.383191191191191	3.05388358998289\\
0.385192192192192	3.05327904414625\\
0.387193193193193	3.0526717526654\\
0.389194194194194	3.05206168751268\\
0.391195195195195	3.05144882036708\\
0.393196196196196	3.05083312261153\\
0.395197197197197	3.0502145653279\\
0.397198198198198	3.04959311929321\\
0.399199199199199	3.0489687549873\\
0.4012002002002	3.04834144254138\\
0.403201201201201	3.04771115180414\\
0.405202202202202	3.04707785229069\\
0.407203203203203	3.04644151318975\\
0.409204204204204	3.04580210335915\\
0.411205205205205	3.04515959132131\\
0.413206206206206	3.04451394525855\\
0.415207207207207	3.04386513300849\\
0.417208208208208	3.04321312205923\\
0.419209209209209	3.04255787954456\\
0.42121021021021	3.04189937223907\\
0.423211211211211	3.04123756655317\\
0.425212212212212	3.04057242852807\\
0.427213213213213	3.03990392383069\\
0.429214214214214	3.03923201774844\\
0.431215215215215	3.038556675184\\
0.433216216216216	3.03787786065001\\
0.435217217217217	3.03719553826359\\
0.437218218218218	3.03650967174096\\
0.439219219219219	3.03582022439179\\
0.44122022022022	3.0351271591136\\
0.443221221221221	3.03443043838603\\
0.445222222222222	3.03373002426506\\
0.447223223223223	3.03302587837707\\
0.449224224224224	3.03231796191291\\
0.451225225225225	3.03160623562184\\
0.453226226226226	3.03089065980536\\
0.455227227227227	3.030171194311\\
0.457228228228228	3.029447798526\\
0.459229229229229	3.02872043137088\\
0.46123023023023	3.02798905129297\\
0.463231231231231	3.02725361625976\\
0.465232232232232	3.02651408375229\\
0.467233233233233	3.02577041075828\\
0.469234234234234	3.02502255376533\\
0.471235235235235	3.02427046875387\\
0.473236236236236	3.02351411119014\\
0.475237237237237	3.022753436019\\
0.477238238238238	3.02198839765662\\
0.479239239239239	3.02121894998314\\
0.48124024024024	3.02044504633518\\
0.483241241241241	3.01966663949825\\
0.485242242242242	3.01888368169903\\
0.487243243243243	3.01809612459763\\
0.489244244244244	3.01730391927961\\
0.491245245245245	3.01650701624801\\
0.493246246246246	3.01570536541521\\
0.495247247247247	3.01489891609468\\
0.497248248248248	3.01408761699264\\
0.499249249249249	3.01327141619956\\
0.50125025025025	3.01245026118163\\
0.503251251251251	3.011624098772\\
0.505252252252252	3.01079287516199\\
0.507253253253253	3.00995653589216\\
0.509254254254254	3.00911502584323\\
0.511255255255255	3.00826828922692\\
0.513256256256256	3.00741626957662\\
0.515257257257257	3.00655890973799\\
0.517258258258258	3.00569615185941\\
0.519259259259259	3.00482793738228\\
0.52126026026026	3.00395420703125\\
0.523261261261261	3.00307490080427\\
0.525262262262262	3.0021899579625\\
0.527263263263263	3.00129931702021\\
0.529264264264264	3.00040291573438\\
0.531265265265265	2.99950069109426\\
0.533266266266266	2.99859257931085\\
0.535267267267267	2.99767851580611\\
0.537268268268268	2.99675843520217\\
0.539269269269269	2.9958322713103\\
0.54127027027027	2.99489995711984\\
0.543271271271271	2.99396142478692\\
0.545272272272272	2.99301660562308\\
0.547273273273273	2.99206543008376\\
0.549274274274274	2.99110782775663\\
0.551275275275275	2.99014372734979\\
0.553276276276276	2.98917305667983\\
0.555277277277277	2.9881957426598\\
0.557278278278278	2.98721171128697\\
0.559279279279279	2.98622088763048\\
0.56128028028028	2.98522319581889\\
0.563281281281281	2.98421855902755\\
0.565282282282282	2.98320689946583\\
0.567283283283283	2.98218813836425\\
0.569284284284284	2.98116219596147\\
0.571285285285285	2.98012899149109\\
0.573286286286286	2.97908844316841\\
0.575287287287287	2.97804046817695\\
0.577288288288288	2.97698498265494\\
0.579289289289289	2.9759219016816\\
0.58129029029029	2.97485113926335\\
0.583291291291291	2.97377260831986\\
0.585292292292292	2.97268622066996\\
0.587293293293293	2.97159188701747\\
0.589294294294294	2.97048951693688\\
0.591295295295295	2.96937901885891\\
0.593296296296296	2.96826030005595\\
0.595297297297297	2.96713326662741\\
0.597298298298298	2.96599782348495\\
0.599299299299299	2.96485387433754\\
0.6013003003003	2.96370132167653\\
0.603301301301301	2.96254006676052\\
0.605302302302302	2.96137000960018\\
0.607303303303303	2.96019104894294\\
0.609304304304304	2.95900308225765\\
0.611305305305305	2.95780600571906\\
0.613306306306306	2.95659971419236\\
0.615307307307307	2.95538410121747\\
0.617308308308308	2.9541590589934\\
0.619309309309309	2.95292447836251\\
0.62131031031031	2.95168024879465\\
0.623311311311311	2.95042625837134\\
0.625312312312312	2.94916239376982\\
0.627313313313313	2.94788854024715\\
0.629314314314314	2.94660458162419\\
0.631315315315315	2.94531040026961\\
0.633316316316316	2.94400587708389\\
0.635317317317317	2.94269089148327\\
0.637318318318318	2.94136532138377\\
0.639319319319319	2.94002904318513\\
0.64132032032032	2.93868193175486\\
0.643321321321321	2.93732386041229\\
0.645322322322322	2.93595470091264\\
0.647323323323323	2.93457432343117\\
0.649324324324324	2.9331825965474\\
0.651325325325325	2.93177938722937\\
0.653326326326326	2.93036456081805\\
0.655327327327327	2.9289379810118\\
0.657328328328328	2.92749950985099\\
0.659329329329329	2.9260490077027\\
0.66133033033033	2.9245863332456\\
0.663331331331331	2.92311134345501\\
0.665332332332332	2.92162389358808\\
0.667333333333333	2.92012383716922\\
0.669334334334334	2.91861102597569\\
0.671335335335335	2.91708531002346\\
0.673336336336336	2.91554653755327\\
0.675337337337337	2.91399455501701\\
0.677338338338338	2.91242920706429\\
0.679339339339339	2.91085033652944\\
0.68134034034034	2.90925778441868\\
0.683341341341341	2.90765138989779\\
0.685342342342342	2.90603099027996\\
0.687343343343343	2.90439642101423\\
0.689344344344344	2.90274751567412\\
0.691345345345345	2.90108410594688\\
0.693346346346346	2.89940602162305\\
0.695347347347347	2.89771309058657\\
0.697348348348348	2.89600513880535\\
0.699349349349349	2.89428199032238\\
0.70135035035035	2.89254346724738\\
0.703351351351351	2.890789389749\\
0.705352352352352	2.88901957604767\\
0.707353353353353	2.88723384240899\\
0.709354354354354	2.88543200313784\\
0.711355355355355	2.88361387057311\\
0.713356356356356	2.88177925508319\\
0.715357357357357	2.87992796506209\\
0.717358358358358	2.87805980692638\\
0.719359359359359	2.87617458511292\\
0.72136036036036	2.87427210207731\\
0.723361361361361	2.87235215829327\\
0.725362362362362	2.87041455225278\\
0.727363363363363	2.86845908046716\\
0.729364364364364	2.86648553746902\\
0.731365365365365	2.86449371581518\\
0.733366366366366	2.86248340609049\\
0.735367367367367	2.86045439691266\\
0.737368368368368	2.85840647493807\\
0.739369369369369	2.85633942486862\\
0.74137037037037	2.85425302945958\\
0.743371371371371	2.85214706952857\\
0.745372372372372	2.85002132396553\\
0.747373373373373	2.84787556974386\\
0.749374374374374	2.84570958193265\\
0.751375375375375	2.84352313370996\\
0.753376376376376	2.84131599637739\\
0.755377377377377	2.83908793937565\\
0.757378378378378	2.83683873030134\\
0.759379379379379	2.83456813492493\\
0.76138038038038	2.83227591720984\\
0.763381381381381	2.8299618393327\\
0.765382382382382	2.82762566170482\\
0.767383383383383	2.82526714299473\\
0.769384384384384	2.82288604015196\\
0.771385385385385	2.82048210843191\\
0.773386386386386	2.81805510142182\\
0.775387387387387	2.81560477106793\\
0.777388388388388	2.8131308677036\\
0.779389389389389	2.81063314007859\\
0.78139039039039	2.80811133538935\\
0.783391391391391	2.80556519931019\\
0.785392392392392	2.80299447602559\\
0.787393393393393	2.80039890826322\\
0.789394394394394	2.79777823732795\\
0.791395395395395	2.7951322031366\\
0.793396396396396	2.79246054425343\\
0.795397397397397	2.78976299792632\\
0.797398398398398	2.78703930012355\\
0.799399399399399	2.78428918557108\\
0.8014004004004	2.7815123877903\\
0.803401401401401	2.7787086391361\\
0.805402402402402	2.77587767083524\\
0.807403403403404	2.77301921302479\\
0.809404404404404	2.77013299479069\\
0.811405405405405	2.76721874420619\\
0.813406406406406	2.76427618837013\\
0.815407407407407	2.76130505344484\\
0.817408408408409	2.7583050646936\\
0.819409409409409	2.75527594651755\\
0.82141041041041	2.75221742249172\\
0.823411411411412	2.74912921540025\\
0.825412412412412	2.74601104727052\\
0.827413413413413	2.74286263940597\\
0.829414414414414	2.73968371241753\\
0.831415415415415	2.73647398625347\\
0.833416416416417	2.73323318022739\\
0.835417417417417	2.72996101304425\\
0.837418418418418	2.72665720282417\\
0.839419419419419	2.72332146712381\\
0.84142042042042	2.71995352295516\\
0.843421421421422	2.71655308680145\\
0.845422422422422	2.71311987462995\\
0.847423423423423	2.70965360190148\\
0.849424424424424	2.70615398357641\\
0.851425425425425	2.70262073411674\\
0.853426426426427	2.69905356748428\\
0.855427427427428	2.6954521971344\\
0.857428428428428	2.69181633600532\\
0.859429429429429	2.68814569650249\\
0.86143043043043	2.68443999047794\\
0.863431431431432	2.68069892920428\\
0.865432432432433	2.67692222334304\\
0.867433433433433	2.67310958290711\\
0.869434434434434	2.66926071721698\\
0.871435435435435	2.66537533485061\\
0.873436436436437	2.66145314358639\\
0.875437437437438	2.65749385033923\\
0.877438438438438	2.65349716108923\\
0.879439439439439	2.64946278080275\\
0.88144044044044	2.64539041334564\\
0.883441441441442	2.64127976138815\\
0.885442442442442	2.63713052630149\\
0.887443443443444	2.63294240804548\\
0.889444444444444	2.62871510504719\\
0.891445445445445	2.6244483140701\\
0.893446446446447	2.62014173007365\\
0.895447447447447	2.61579504606273\\
0.897448448448449	2.6114079529268\\
0.899449449449449	2.60698013926848\\
0.90145045045045	2.60251129122104\\
0.903451451451452	2.59800109225475\\
0.905452452452452	2.59344922297147\\
0.907453453453454	2.58885536088732\\
0.909454454454455	2.58421918020308\\
0.911455455455455	2.57954035156184\\
0.913456456456457	2.57481854179365\\
0.915457457457457	2.57005341364673\\
0.917458458458459	2.56524462550487\\
0.91945945945946	2.56039183109054\\
0.92146046046046	2.55549467915338\\
0.923461461461462	2.5505528131436\\
0.925462462462462	2.54556587086972\\
0.927463463463464	2.54053348414029\\
0.929464464464465	2.53545527838894\\
0.931465465465466	2.53033087228237\\
0.933466466466467	2.5251598773105\\
0.935467467467467	2.51994189735832\\
0.937468468468469	2.51467652825868\\
0.93946946946947	2.50936335732536\\
0.941470470470471	2.50400196286571\\
0.943471471471471	2.49859191367187\\
0.945472472472472	2.49313276849\\
0.947473473473474	2.48762407546636\\
0.949474474474475	2.48206537156929\\
0.951475475475476	2.47645618198615\\
0.953476476476476	2.47079601949389\\
0.955477477477477	2.46508438380221\\
0.957478478478479	2.45932076086778\\
0.95947947947948	2.45350462217829\\
0.961480480480481	2.4476354240046\\
0.963481481481481	2.4417126066195\\
0.965482482482482	2.43573559348111\\
0.967483483483484	2.42970379037914\\
0.969484484484485	2.4236165845418\\
0.971485485485486	2.41747334370113\\
0.973486486486487	2.41127341511436\\
0.975487487487487	2.40501612453851\\
0.977488488488489	2.39870077515551\\
0.97948948948949	2.39232664644469\\
0.981490490490491	2.38589299299913\\
0.983491491491492	2.37939904328254\\
0.985492492492492	2.37284399832239\\
0.987493493493494	2.36622703033517\\
0.989494494494495	2.35954728127911\\
0.991495495495496	2.35280386132931\\
0.993496496496497	2.34599584726963\\
0.995497497497498	2.33912228079565\\
0.997498498498499	2.33218216672191\\
0.9994994994995	2.32517447108641\\
1.0015005005005	2.31809811914477\\
1.0035015015015	2.31095199324539\\
1.0055025025025	2.30373493057654\\
1.0075035035035	2.29644572077521\\
1.0095045045045	2.28908310338675\\
1.01150550550551	2.28164576516312\\
1.01350650650651	2.27413233718651\\
1.01550750750751	2.26654139180378\\
1.01750850850851	2.25887143935567\\
1.01950950950951	2.2511209246832\\
1.02151051051051	2.24328822339159\\
1.02351151151151	2.23537163785072\\
1.02551251251251	2.22736939290777\\
1.02751351351351	2.21927963128636\\
1.02951451451451	2.21110040864278\\
1.03151551551552	2.20282968824702\\
1.03351651651652	2.19446533525278\\
1.03551751751752	2.18600511051644\\
1.03751851851852	2.17744666392016\\
1.03951951951952	2.16878752714938\\
1.04152052052052	2.1600251058689\\
1.04352152152152	2.15115667123454\\
1.04552252252252	2.14217935067014\\
1.04752352352352	2.13309011783054\\
1.04952452452452	2.12388578166065\\
1.05152552552553	2.11456297444942\\
1.05352652652653	2.10511813876374\\
1.05552752752753	2.09554751313141\\
1.05752852852853	2.08584711632427\\
1.05952952952953	2.07601273007185\\
1.06153053053053	2.06603988000982\\
1.06353153153153	2.05592381464047\\
1.06553253253253	2.0456594820467\\
1.06753353353353	2.03524150406222\\
1.06953453453453	2.0246641475527\\
1.07153553553554	2.01392129240687\\
1.07353653653654	2.0030063957693\\
1.07553753753754	1.99191245196721\\
1.07753853853854	1.98063194748728\\
1.07953953953954	1.9691568102431\\
1.08154054054054	1.95747835223357\\
1.08354154154154	1.94558720452219\\
1.08554254254254	1.93347324325863\\
1.08754354354354	1.92112550520803\\
1.08954454454454	1.90853209093576\\
1.09154554554555	1.89568005340284\\
1.09354654654655	1.88255526923316\\
1.09554754754755	1.86914228929484\\
1.09754854854855	1.85542416445143\\
1.09954954954955	1.84138224133614\\
1.10155055055055	1.8269959217128\\
1.10355155155155	1.81224237731582\\
1.10555255255255	1.79709620987707\\
1.10755355355355	1.7815290431672\\
1.10955455455455	1.76550903004442\\
1.11155555555556	1.74900025234943\\
1.11355655655656	1.73196198448074\\
1.11555755755756	1.71434778186337\\
1.11755855855856	1.69610434214804\\
1.11955955955956	1.67717006815721\\
1.12156056056056	1.65747323476994\\
1.12356156156156	1.6369296232056\\
1.12556256256256	1.61543942951613\\
1.12756356356356	1.5928831702088\\
1.12956456456456	1.56911618242408\\
1.13156556556557	1.54396112720093\\
1.13356656656657	1.51719762096989\\
1.13556756756757	1.48854770529929\\
1.13756856856857	1.45765530205423\\
1.13956956956957	1.42405721778334\\
1.14157057057057	1.38714340593421\\
1.14357157157157	1.34610822705911\\
1.14557257257257	1.29991434872097\\
1.14757357357357	1.24736742454741\\
1.14957457457457	1.18763014397751\\
1.15157557557558	1.12180259927846\\
1.15357657657658	1.05489287820912\\
1.15557757757758	0.993768043042407\\
1.15757857857858	0.941866761318642\\
1.15957957957958	0.898730695594457\\
1.16158058058058	0.862599887391698\\
1.16358158158158	0.831820789390065\\
1.16558258258258	0.805137648799688\\
1.16758358358358	0.78164278612224\\
1.16958458458458	0.76068051117313\\
1.17158558558559	0.741769877676536\\
1.17358658658659	0.724551062811639\\
1.17558758758759	0.708749392794013\\
1.17758858858859	0.694151188259175\\
1.17958958958959	0.680587327401964\\
1.18159059059059	0.667921860017844\\
1.18359159159159	0.656043970168449\\
1.18559259259259	0.644862196826513\\
1.18759359359359	0.634300204827455\\
1.18959459459459	0.624293639455507\\
1.1915955955956	0.614787751553552\\
1.1935965965966	0.60573557941427\\
1.1955975975976	0.597096539101385\\
1.1975985985986	0.588835318588955\\
1.1995995995996	0.58092100084727\\
1.2016006006006	0.573326361539007\\
1.2036016016016	0.566027301375094\\
1.2056026026026	0.559002383398004\\
1.2076036036036	0.552232452815811\\
1.2096046046046	0.54570032236379\\
1.21160560560561	0.53939051011766\\
1.21360660660661	0.533289019618906\\
1.21560760760761	0.527383154383968\\
1.21760860860861	0.521661360544445\\
1.21960960960961	0.51611309265199\\
1.22161061061061	0.510728698673077\\
1.22361161161161	0.505499320971704\\
1.22561261261261	0.500416810683779\\
1.22761361361361	0.495473653365643\\
1.22961461461461	0.490662904179261\\
1.23161561561562	0.485978131181872\\
1.23361661661662	0.481413365531355\\
1.23561761761762	0.476963057618441\\
1.23761861861862	0.472622038297413\\
1.23961961961962	0.468385484518754\\
1.24162062062062	0.464248888776515\\
1.24362162162162	0.460208031871978\\
1.24562262262262	0.456258958569874\\
1.24762362362362	0.452397955785398\\
1.24962462462462	0.448621532991806\\
1.25162562562563	0.444926404582114\\
1.25362662662663	0.441309473954906\\
1.25562762762763	0.437767819125111\\
1.25762862862863	0.434298679687564\\
1.25962962962963	0.430899444982709\\
1.26163063063063	0.427567643333491\\
1.26363163163163	0.424300932238876\\
1.26563263263263	0.42109708942322\\
1.26763363363363	0.417954004653257\\
1.26963463463463	0.414869672244567\\
1.27163563563564	0.411842184188846\\
1.27363663663664	0.408869723840938\\
1.27563763763764	0.405950560111639\\
1.27763863863864	0.40308304211815\\
1.27963963963964	0.400265594249522\\
1.28164064064064	0.397496711608708\\
1.28364164164164	0.394774955797237\\
1.28564264264264	0.392098951011898\\
1.28764364364364	0.389467380425933\\
1.28964464464464	0.386878982830223\\
1.29164564564565	0.384332549512092\\
1.29364664664665	0.381826921351982\\
1.29564764764765	0.379360986119606\\
1.29764864864865	0.376933675953585\\
1.29964964964965	0.374543965009538\\
1.30165065065065	0.372190867263312\\
1.30365165165165	0.369873434457151\\
1.30565265265265	0.367590754177738\\
1.30765365365365	0.365341948056022\\
1.30965465465465	0.363126170079613\\
1.31165565565566	0.360942605009389\\
1.31365665665666	0.358790466892645\\
1.31565765765766	0.356668997665707\\
1.31765865865866	0.354577465839707\\
1.31965965965966	0.352515165263485\\
1.32166066066066	0.350481413958293\\
1.32366166166166	0.348475553019375\\
1.32566266266266	0.346496945579677\\
1.32766366366366	0.344544975831692\\
1.32966466466466	0.342619048103416\\
1.33166566566567	0.340718585984863\\
1.33366666666667	0.338843031501892\\
1.33566766766767	0.336991844334197\\
1.33766866866867	0.335164501074712\\
1.33966966966967	0.333360494527788\\
1.34167067067067	0.331579333043689\\
1.34367167167167	0.329820539887203\\
1.34567267267267	0.328083652638244\\
1.34767367367367	0.326368222622536\\
1.34967467467467	0.324673814370541\\
1.35167567567568	0.323000005102983\\
1.35367667667668	0.321346384241394\\
1.35567767767768	0.319712552942217\\
1.35767867867868	0.318098123653127\\
1.35967967967968	0.316502719690289\\
1.36168068068068	0.314925974835375\\
1.36368168168168	0.313367532951235\\
1.36568268268268	0.311827047615185\\
1.36768368368368	0.310304181768955\\
1.36968468468468	0.308798607384376\\
1.37168568568569	0.307310005143968\\
1.37368668668669	0.305838064135641\\
1.37568768768769	0.304382481560739\\
1.37768868868869	0.302942962454743\\
1.37968968968969	0.301519219419996\\
1.38169069069069	0.300110972369787\\
1.38369169169169	0.298717948283235\\
1.38569269269269	0.297339880970461\\
1.38769369369369	0.295976510847465\\
1.38969469469469	0.294627584720272\\
1.3916956956957	0.293292855577892\\
1.3936966966967	0.291972082393628\\
1.3956976976977	0.290665029934378\\
1.3976986986987	0.289371468577486\\
1.3996996996997	0.288091174134844\\
1.4017007007007	0.286823927683852\\
1.4037017017017	0.285569515404955\\
1.4057027027027	0.284327728425409\\
1.4077037037037	0.283098362669033\\
1.4097047047047	0.281881218711638\\
1.41170570570571	0.280676101641894\\
1.41370670670671	0.279482820927385\\
1.41570770770771	0.278301190285605\\
1.41770870870871	0.277131027559724\\
1.41970970970971	0.275972154598841\\
1.42171071071071	0.274824397142599\\
1.42371171171171	0.273687584709917\\
1.42571271271271	0.272561550491704\\
1.42771371371371	0.271446131247356\\
1.42971471471471	0.270341167204892\\
1.43171571571572	0.26924650196457\\
1.43371671671672	0.268161982405829\\
1.43571771771772	0.267087458597442\\
1.43771871871872	0.266022783710724\\
1.43971971971972	0.26496781393568\\
1.44172072072072	0.263922408399979\\
1.44372172172172	0.262886429090622\\
1.44572272272272	0.261859740778227\\
1.44772372372372	0.260842210943789\\
1.44972472472472	0.259833709707848\\
1.45172572572573	0.258834109761956\\
1.45372672672673	0.257843286302352\\
1.45572772772773	0.256861116965773\\
1.45772872872873	0.255887481767304\\
1.45972972972973	0.254922263040189\\
1.46173073073073	0.253965345377549\\
1.46373173173173	0.253016615575905\\
1.46573273273273	0.252075962580467\\
1.46773373373373	0.251143277432102\\
1.46973473473473	0.250218453215926\\
1.47173573573574	0.249301385011472\\
1.47373673673674	0.248391969844352\\
1.47573773773774	0.247490106639388\\
1.47773873873874	0.246595696175127\\
1.47973973973974	0.24570864103972\\
1.48174074074074	0.244828845588102\\
1.48374174174174	0.243956215900421\\
1.48574274274274	0.243090659741689\\
1.48774374374374	0.242232086522597\\
1.48974474474474	0.241380407261463\\
1.49174574574575	0.240535534547268\\
1.49374674674675	0.239697382503747\\
1.49574774774775	0.238865866754495\\
1.49774874874875	0.238040904389057\\
1.49974974974975	0.237222413929973\\
1.50175075075075	0.236410315300723\\
1.50375175175175	0.235604529794584\\
1.50575275275275	0.234804980044325\\
1.50775375375375	0.234011589992735\\
1.50975475475475	0.233224284863967\\
1.51175575575576	0.232442991135635\\
1.51375675675676	0.231667636511683\\
1.51575775775776	0.23089814989597\\
1.51775875875876	0.230134461366554\\
1.51975975975976	0.229376502150673\\
1.52176076076076	0.228624204600372\\
1.52376176176176	0.227877502168768\\
1.52576276276276	0.227136329386943\\
1.52776376376376	0.226400621841433\\
1.52976476476476	0.225670316152286\\
1.53176576576577	0.224945349951706\\
1.53376676676677	0.224225661863218\\
1.53576776776777	0.223511191481383\\
1.53776876876877	0.222801879352007\\
1.53976976976977	0.222097666952861\\
1.54177077077077	0.221398496674876\\
1.54377177177177	0.220704311803816\\
1.54577277277277	0.220015056502386\\
1.54777377377377	0.2193306757928\\
1.54977477477477	0.218651115539769\\
1.55177577577578	0.217976322433899\\
1.55377677677678	0.2173062439755\\
1.55577777777778	0.216640828458785\\
1.55777877877878	0.215980024956442\\
1.55977977977978	0.215323783304584\\
1.56178078078078	0.214672054088052\\
1.56378178178178	0.214024788626071\\
1.56578278278278	0.213381938958239\\
1.56778378378378	0.212743457830844\\
1.56978478478478	0.212109298683513\\
1.57178578578579	0.211479415636155\\
1.57378678678679	0.210853763476217\\
1.57578778778779	0.210232297646233\\
1.57778878878879	0.209614974231652\\
1.57978978978979	0.209001749948956\\
1.58179079079079	0.20839258213403\\
1.58379179179179	0.207787428730817\\
1.58579279279279	0.207186248280211\\
1.58779379379379	0.206588999909209\\
1.58979479479479	0.2059956433203\\
1.5917957957958	0.205406138781095\\
1.5937967967968	0.20482044711418\\
1.5957977977978	0.204238529687199\\
1.5977987987988	0.203660348403154\\
1.5997997997998	0.203085865690908\\
1.6018008008008	0.202515044495911\\
1.6038018018018	0.201947848271108\\
1.6058028028028	0.201384240968063\\
1.6078038038038	0.200824187028254\\
1.6098048048048	0.20026765137457\\
1.61180580580581	0.199714599402982\\
1.61380680680681	0.199164996974385\\
1.61580780780781	0.198618810406631\\
1.61780880880881	0.198076006466706\\
1.61980980980981	0.19753655236309\\
1.62181081081081	0.197000415738264\\
1.62381181181181	0.196467564661381\\
1.62581281281281	0.195937967621087\\
1.62781381381381	0.195411593518479\\
1.62981481481481	0.194888411660231\\
1.63181581581582	0.194368391751833\\
1.63381681681682	0.193851503890993\\
1.63581781781782	0.193337718561152\\
1.63781881881882	0.192827006625143\\
1.63981981981982	0.192319339318977\\
1.64182082082082	0.191814688245745\\
1.64382182182182	0.191313025369651\\
1.64582282282282	0.190814323010158\\
1.64782382382382	0.190318553836254\\
1.64982482482482	0.18982569086083\\
1.65182582582583	0.189335707435163\\
1.65382682682683	0.188848577243519\\
1.65582782782783	0.188364274297852\\
1.65782882882883	0.187882772932605\\
1.65982982982983	0.187404047799619\\
1.66183083083083	0.186928073863138\\
1.66383183183183	0.186454826394906\\
1.66583283283283	0.185984280969361\\
1.66783383383383	0.185516413458921\\
1.66983483483483	0.185051200029359\\
1.67183583583584	0.184588617135268\\
1.67383683683684	0.184128641515606\\
1.67583783783784	0.18367125018933\\
1.67783883883884	0.18321642045111\\
1.67983983983984	0.182764129867126\\
1.68184084084084	0.182314356270934\\
1.68384184184184	0.181867077759424\\
1.68584284284284	0.181422272688835\\
1.68784384384384	0.18097991967086\\
1.68984484484484	0.180539997568809\\
1.69184584584585	0.180102485493854\\
1.69384684684685	0.179667362801331\\
1.69584784784785	0.179234609087116\\
1.69784884884885	0.17880420418407\\
1.69984984984985	0.178376128158538\\
1.70185085085085	0.177950361306918\\
1.70385185185185	0.177526884152291\\
1.70585285285285	0.17710567744111\\
1.70785385385385	0.176686722139945\\
1.70985485485485	0.176269999432296\\
1.71185585585586	0.175855490715446\\
1.71385685685686	0.175443177597387\\
1.71585785785786	0.175033041893784\\
1.71785885885886	0.174625065625006\\
1.71985985985986	0.1742192310132\\
1.72186086086086	0.173815520479419\\
1.72386186186186	0.173413916640797\\
1.72586286286286	0.173014402307779\\
1.72786386386386	0.172616960481391\\
1.72986486486486	0.172221574350564\\
1.73186586586587	0.171828227289498\\
1.73386686686687	0.171436902855073\\
1.73586786786787	0.171047584784307\\
1.73786886886887	0.170660256991853\\
1.73986986986987	0.170274903567538\\
1.74187087087087	0.169891508773949\\
1.74387187187187	0.169510057044054\\
1.74587287287287	0.169130532978863\\
1.74787387387387	0.168752921345131\\
1.74987487487487	0.168377207073099\\
1.75187587587588	0.168003375254268\\
1.75387687687688	0.167631411139218\\
1.75587787787788	0.167261300135453\\
1.75787887887888	0.166893027805288\\
1.75987987987988	0.166526579863771\\
1.76188088088088	0.166161942176636\\
1.76388188188188	0.165799100758288\\
1.76588288288288	0.165438041769825\\
1.76788388388388	0.16507875151709\\
1.76988488488488	0.164721216448755\\
1.77188588588589	0.164365423154435\\
1.77388688688689	0.164011358362828\\
1.77588788788789	0.163659008939897\\
1.77788888888889	0.163308361887068\\
1.77988988988989	0.162959404339464\\
1.78189089089089	0.162612123564162\\
1.78389189189189	0.162266506958485\\
1.78589289289289	0.161922542048313\\
1.78789389389389	0.161580216486422\\
1.78989489489489	0.161239518050859\\
1.7918958958959	0.160900434643323\\
1.7938968968969	0.160562954287593\\
1.7958978978979	0.160227065127962\\
1.7978988988989	0.159892755427708\\
1.7998998998999	0.159560013567581\\
1.8019009009009	0.159228828044319\\
1.8039019019019	0.15889918746918\\
1.8059029029029	0.158571080566504\\
1.8079039039039	0.158244496172291\\
1.8099049049049	0.157919423232804\\
1.81190590590591	0.157595850803192\\
1.81390690690691	0.157273768046134\\
1.81590790790791	0.156953164230502\\
1.81790890890891	0.15663402873005\\
1.81990990990991	0.156316351022113\\
1.82191091091091	0.156000120686334\\
1.82391191191191	0.155685327403406\\
1.82591291291291	0.155371960953834\\
1.82791391391391	0.155060011216716\\
1.82991491491491	0.154749468168537\\
1.83191591591592	0.154440321881989\\
1.83391691691692	0.154132562524803\\
1.83591791791792	0.1538261803586\\
1.83791891891892	0.153521165737756\\
1.83991991991992	0.153217509108287\\
1.84192092092092	0.152915201006752\\
1.84392192192192	0.152614232059165\\
1.84592292292292	0.152314592979929\\
1.84792392392392	0.152016274570781\\
1.84992492492492	0.151719267719758\\
1.85192592592593	0.151423563400172\\
1.85392692692693	0.151129152669602\\
1.85592792792793	0.150836026668904\\
1.85792892892893	0.150544176621223\\
1.85992992992993	0.150253593831041\\
1.86193093093093	0.14996426968321\\
1.86393193193193	0.149676195642028\\
1.86593293293293	0.149389363250305\\
1.86793393393393	0.149103764128453\\
1.86993493493493	0.148819389973589\\
1.87193593593594	0.148536232558648\\
1.87393693693694	0.148254283731507\\
1.87593793793794	0.147973535414125\\
1.87793893893894	0.147693979601693\\
1.87993993993994	0.147415608361796\\
1.88194094094094	0.147138413833584\\
1.88394194194194	0.146862388226962\\
1.88594294294294	0.146587523821781\\
1.88794394394394	0.14631381296705\\
1.88994494494494	0.146041248080149\\
1.89194594594595	0.145769821646063\\
1.89394694694695	0.145499526216617\\
1.89594794794795	0.14523035440973\\
1.89794894894895	0.144962298908671\\
1.89994994994995	0.14469535246133\\
1.90195095095095	0.144429507879499\\
1.90395195195195	0.144164758038163\\
1.90595295295295	0.143901095874793\\
1.90795395395395	0.143638514388659\\
1.90995495495495	0.14337700664015\\
1.91195595595596	0.143116565750096\\
1.91395695695696	0.142857184899104\\
1.91595795795796	0.142598857326907\\
1.91795895895896	0.142341576331713\\
1.91995995995996	0.14208533526957\\
1.92196096096096	0.141830127553732\\
1.92396196196196	0.141575946654044\\
1.92596296296296	0.141322786096321\\
1.92796396396396	0.14107063946175\\
1.92996496496496	0.140819500386289\\
1.93196596596597	0.140569362560076\\
1.93396696696697	0.140320219726851\\
1.93596796796797	0.14007206568338\\
1.93796896896897	0.139824894278886\\
1.93996996996997	0.139578699414493\\
1.94197097097097	0.139333475042672\\
1.94397197197197	0.139089215166697\\
1.94597297297297	0.138845913840102\\
1.94797397397397	0.138603565166156\\
1.94997497497497	0.138362163297336\\
1.95197597597598	0.138121702434808\\
1.95397697697698	0.137882176827917\\
1.95597797797798	0.13764358077368\\
1.95797897897898	0.137405908616293\\
1.95997997997998	0.137169154746633\\
1.96198098098098	0.136933313601774\\
1.96398198198198	0.13669837966451\\
1.96598298298298	0.136464347462875\\
1.96798398398398	0.136231211569683\\
1.96998498498498	0.13599896660206\\
1.97198598598599	0.135767607220989\\
1.97398698698699	0.135537128130864\\
1.97598798798799	0.135307524079036\\
1.97798898898899	0.135078789855383\\
1.97998998998999	0.13485092029187\\
1.98199099099099	0.134623910262122\\
1.98399199199199	0.134397754681001\\
1.98599299299299	0.134172448504188\\
1.98799399399399	0.133947986727768\\
1.98999499499499	0.133724364387824\\
1.991995995996	0.133501576560035\\
1.993996996997	0.133279618359273\\
1.995997997998	0.133058484939216\\
1.997998998999	0.132838171491953\\
2	0.132618673247605\\
};
\addlegendentry{$\alpha\text{ = 0.003}$};

\addplot [color=red,solid]
  table[row sep=crcr]{%
0.001	3.14141302864169\\
0.003001001001001	3.14105359575541\\
0.005002002002002	3.14069415331061\\
0.007003003003003	3.14033469493343\\
0.009004004004004	3.13997521424932\\
0.011005005005005	3.13961570488272\\
0.013006006006006	3.13925616045686\\
0.015007007007007	3.1388965745934\\
0.017008008008008	3.13853694091222\\
0.019009009009009	3.13817725303111\\
0.02101001001001	3.13781750456546\\
0.023011011011011	3.13745768912806\\
0.025012012012012	3.13709780032874\\
0.027013013013013	3.13673783177414\\
0.029014014014014	3.13637777706742\\
0.031015015015015	3.13601762980796\\
0.033016016016016	3.13565738359113\\
0.035017017017017	3.13529703200795\\
0.037018018018018	3.13493656864483\\
0.039019019019019	3.13457598708334\\
0.04102002002002	3.13421528089985\\
0.043021021021021	3.13385444366529\\
0.045022022022022	3.13349346894488\\
0.047023023023023	3.13313235029783\\
0.049024024024024	3.13277108127706\\
0.051025025025025	3.13240965542893\\
0.053026026026026	3.13204806629293\\
0.055027027027027	3.13168630740143\\
0.057028028028028	3.1313243722794\\
0.059029029029029	3.13096225444408\\
0.06103003003003	3.13059994740475\\
0.063031031031031	3.13023744466242\\
0.065032032032032	3.12987473970954\\
0.067033033033033	3.12951182602976\\
0.069034034034034	3.12914869709756\\
0.071035035035035	3.12878534637806\\
0.073036036036036	3.12842176732666\\
0.075037037037037	3.12805795338882\\
0.077038038038038	3.1276938979997\\
0.079039039039039	3.12732959458393\\
0.08104004004004	3.1269650365553\\
0.083041041041041	3.12660021731648\\
0.085042042042042	3.12623513025872\\
0.0870430430430431	3.12586976876157\\
0.089044044044044	3.12550412619261\\
0.091045045045045	3.12513819590709\\
0.093046046046046	3.12477197124775\\
0.0950470470470471	3.12440544554442\\
0.0970480480480481	3.12403861211379\\
0.0990490490490491	3.12367146425912\\
0.10105005005005	3.12330399526991\\
0.103051051051051	3.12293619842163\\
0.105052052052052	3.12256806697545\\
0.107053053053053	3.12219959417789\\
0.109054054054054	3.12183077326057\\
0.111055055055055	3.12146159743989\\
0.113056056056056	3.12109205991675\\
0.115057057057057	3.12072215387624\\
0.117058058058058	3.12035187248736\\
0.119059059059059	3.1199812089027\\
0.12106006006006	3.11961015625815\\
0.123061061061061	3.11923870767261\\
0.125062062062062	3.11886685624767\\
0.127063063063063	3.11849459506732\\
0.129064064064064	3.11812191719766\\
0.131065065065065	3.11774881568659\\
0.133066066066066	3.11737528356347\\
0.135067067067067	3.11700131383889\\
0.137068068068068	3.11662689950429\\
0.139069069069069	3.11625203353172\\
0.14107007007007	3.11587670887347\\
0.143071071071071	3.11550091846182\\
0.145072072072072	3.1151246552087\\
0.147073073073073	3.11474791200538\\
0.149074074074074	3.11437068172219\\
0.151075075075075	3.11399295720817\\
0.153076076076076	3.1136147312908\\
0.155077077077077	3.11323599677564\\
0.157078078078078	3.11285674644607\\
0.159079079079079	3.11247697306294\\
0.16108008008008	3.11209666936428\\
0.163081081081081	3.11171582806494\\
0.165082082082082	3.11133444185636\\
0.167083083083083	3.11095250340614\\
0.169084084084084	3.11057000535783\\
0.171085085085085	3.11018694033055\\
0.173086086086086	3.10980330091867\\
0.175087087087087	3.10941907969153\\
0.177088088088088	3.10903426919307\\
0.179089089089089	3.10864886194156\\
0.18109009009009	3.10826285042922\\
0.183091091091091	3.10787622712195\\
0.185092092092092	3.10748898445898\\
0.187093093093093	3.10710111485253\\
0.189094094094094	3.10671261068751\\
0.191095095095095	3.1063234643212\\
0.193096096096096	3.10593366808288\\
0.195097097097097	3.10554321427354\\
0.197098098098098	3.10515209516554\\
0.199099099099099	3.10476030300227\\
0.2011001001001	3.10436782999784\\
0.203101101101101	3.10397466833671\\
0.205102102102102	3.10358081017341\\
0.207103103103103	3.10318624763215\\
0.209104104104104	3.10279097280652\\
0.211105105105105	3.10239497775915\\
0.213106106106106	3.10199825452137\\
0.215107107107107	3.10160079509287\\
0.217108108108108	3.10120259144136\\
0.219109109109109	3.10080363550223\\
0.22111011011011	3.10040391917823\\
0.223111111111111	3.1000034343391\\
0.225112112112112	3.09960217282125\\
0.227113113113113	3.0992001264274\\
0.229114114114114	3.09879728692624\\
0.231115115115115	3.09839364605212\\
0.233116116116116	3.09798919550464\\
0.235117117117117	3.09758392694837\\
0.237118118118118	3.09717783201245\\
0.239119119119119	3.09677090229029\\
0.24112012012012	3.09636312933917\\
0.243121121121121	3.09595450467994\\
0.245122122122122	3.09554501979664\\
0.247123123123123	3.09513466613617\\
0.249124124124124	3.09472343510791\\
0.251125125125125	3.0943113180834\\
0.253126126126126	3.09389830639596\\
0.255127127127127	3.09348439134035\\
0.257128128128128	3.09306956417244\\
0.259129129129129	3.0926538161088\\
0.26113013013013	3.09223713832638\\
0.263131131131131	3.09181952196216\\
0.265132132132132	3.09140095811277\\
0.267133133133133	3.09098143783417\\
0.269134134134134	3.09056095214122\\
0.271135135135135	3.09013949200742\\
0.273136136136136	3.08971704836447\\
0.275137137137137	3.08929361210194\\
0.277138138138138	3.08886917406692\\
0.279139139139139	3.08844372506364\\
0.28114014014014	3.08801725585313\\
0.283141141141141	3.08758975715284\\
0.285142142142142	3.08716121963628\\
0.287143143143143	3.08673163393268\\
0.289144144144144	3.08630099062659\\
0.291145145145145	3.08586928025755\\
0.293146146146146	3.08543649331971\\
0.295147147147147	3.08500262026148\\
0.297148148148148	3.08456765148513\\
0.299149149149149	3.0841315773465\\
0.30115015015015	3.08369438815454\\
0.303151151151151	3.08325607417102\\
0.305152152152152	3.08281662561014\\
0.307153153153153	3.08237603263817\\
0.309154154154154	3.08193428537307\\
0.311155155155155	3.08149137388413\\
0.313156156156156	3.08104728819165\\
0.315157157157157	3.08060201826648\\
0.317158158158158	3.08015555402977\\
0.319159159159159	3.07970788535252\\
0.32116016016016	3.07925900205524\\
0.323161161161161	3.07880889390761\\
0.325162162162162	3.0783575506281\\
0.327163163163163	3.07790496188358\\
0.329164164164164	3.07745111728901\\
0.331165165165165	3.07699600640703\\
0.333166166166166	3.07653961874763\\
0.335167167167167	3.07608194376778\\
0.337168168168168	3.07562297087105\\
0.339169169169169	3.07516268940728\\
0.34117017017017	3.07470108867219\\
0.343171171171171	3.07423815790706\\
0.345172172172172	3.07377388629834\\
0.347173173173173	3.07330826297728\\
0.349174174174174	3.07284127701964\\
0.351175175175175	3.07237291744524\\
0.353176176176176	3.0719031732177\\
0.355177177177177	3.07143203324402\\
0.357178178178178	3.07095948637425\\
0.359179179179179	3.07048552140114\\
0.36118018018018	3.07001012705981\\
0.363181181181181	3.06953329202735\\
0.365182182182182	3.06905500492253\\
0.367183183183183	3.06857525430543\\
0.369184184184184	3.06809402867707\\
0.371185185185185	3.06761131647913\\
0.373186186186186	3.06712710609354\\
0.375187187187187	3.06664138584219\\
0.377188188188188	3.06615414398659\\
0.379189189189189	3.0656653687275\\
0.38119019019019	3.06517504820463\\
0.383191191191191	3.0646831704963\\
0.385192192192192	3.06418972361912\\
0.387193193193193	3.06369469552764\\
0.389194194194194	3.06319807411405\\
0.391195195195195	3.06269984720784\\
0.393196196196196	3.0622000025755\\
0.395197197197197	3.06169852792019\\
0.397198198198198	3.06119541088143\\
0.399199199199199	3.0606906390348\\
0.4012002002002	3.06018419989161\\
0.403201201201201	3.05967608089862\\
0.405202202202202	3.0591662694377\\
0.407203203203203	3.05865475282559\\
0.409204204204204	3.05814151831355\\
0.411205205205205	3.05762655308711\\
0.413206206206206	3.05710984426575\\
0.415207207207207	3.05659137890263\\
0.417208208208208	3.05607114398431\\
0.419209209209209	3.05554912643046\\
0.42121021021021	3.05502531309363\\
0.423211211211211	3.05449969075891\\
0.425212212212212	3.05397224614373\\
0.427213213213213	3.05344296589757\\
0.429214214214214	3.0529118366017\\
0.431215215215215	3.05237884476896\\
0.433216216216216	3.05184397684346\\
0.435217217217217	3.0513072192004\\
0.437218218218218	3.05076855814578\\
0.439219219219219	3.05022797991621\\
0.44122022022022	3.04968547067866\\
0.443221221221221	3.04914101653023\\
0.445222222222222	3.04859460349798\\
0.447223223223223	3.04804621753865\\
0.449224224224224	3.04749584453852\\
0.451225225225225	3.04694347031317\\
0.453226226226226	3.04638908060733\\
0.455227227227227	3.04583266109464\\
0.457228228228228	3.04527419737751\\
0.459229229229229	3.04471367498692\\
0.46123023023023	3.04415107938229\\
0.463231231231231	3.04358639595128\\
0.465232232232232	3.04301961000965\\
0.467233233233233	3.04245070680113\\
0.469234234234234	3.04187967149726\\
0.471235235235235	3.04130648919725\\
0.473236236236236	3.04073114492791\\
0.475237237237237	3.04015362364345\\
0.477238238238238	3.03957391022544\\
0.479239239239239	3.03899198948268\\
0.48124024024024	3.03840784615111\\
0.483241241241241	3.03782146489371\\
0.485242242242242	3.03723283030046\\
0.487243243243243	3.03664192688825\\
0.489244244244244	3.0360487391008\\
0.491245245245245	3.03545325130865\\
0.493246246246246	3.0348554478091\\
0.495247247247247	3.03425531282615\\
0.497248248248248	3.03365283051053\\
0.499249249249249	3.03304798493966\\
0.50125025025025	3.03244076011764\\
0.503251251251251	3.03183113997525\\
0.505252252252252	3.03121910837\\
0.507253253253253	3.03060464908611\\
0.509254254254254	3.02998774583459\\
0.511255255255255	3.02936838225326\\
0.513256256256256	3.02874654190681\\
0.515257257257257	3.02812220828688\\
0.517258258258258	3.02749536481211\\
0.519259259259259	3.02686599482829\\
0.52126026026026	3.02623408160837\\
0.523261261261261	3.02559960835268\\
0.525262262262262	3.02496255818896\\
0.527263263263263	3.02432291417254\\
0.529264264264264	3.02368065928651\\
0.531265265265265	3.02303577644182\\
0.533266266266266	3.0223882484775\\
0.535267267267267	3.02173805816082\\
0.537268268268268	3.02108518818751\\
0.539269269269269	3.02042962118196\\
0.54127027027027	3.01977133969741\\
0.543271271271271	3.01911032621623\\
0.545272272272272	3.01844656315015\\
0.547273273273273	3.01778003284053\\
0.549274274274274	3.01711071755861\\
0.551275275275275	3.01643859950582\\
0.553276276276276	3.01576366081409\\
0.555277277277277	3.01508588354615\\
0.557278278278278	3.01440524969584\\
0.559279279279279	3.01372174118851\\
0.56128028028028	3.01303533988131\\
0.563281281281281	3.01234602756361\\
0.565282282282282	3.01165378595739\\
0.567283283283283	3.01095859671758\\
0.569284284284284	3.01026044143256\\
0.571285285285285	3.00955930162452\\
0.573286286286286	3.00885515874994\\
0.575287287287287	3.00814799420004\\
0.577288288288288	3.00743778930126\\
0.579289289289289	3.00672452531574\\
0.58129029029029	3.00600818344185\\
0.583291291291291	3.00528874481465\\
0.585292292292292	3.0045661905065\\
0.587293293293293	3.00384050152757\\
0.589294294294294	3.00311165882641\\
0.591295295295295	3.00237964329052\\
0.593296296296296	3.001644435747\\
0.595297297297297	3.00090601696307\\
0.597298298298298	3.0001643676468\\
0.599299299299299	2.99941946844768\\
0.6013003003003	2.99867129995732\\
0.603301301301301	2.99791984271009\\
0.605302302302302	2.99716507718387\\
0.607303303303303	2.99640698380067\\
0.609304304304304	2.99564554292744\\
0.611305305305305	2.99488073487677\\
0.613306306306306	2.9941125399076\\
0.615307307307307	2.99334093822608\\
0.617308308308308	2.99256590998629\\
0.619309309309309	2.99178743529107\\
0.62131031031031	2.9910054941928\\
0.623311311311311	2.99022006669428\\
0.625312312312312	2.98943113274953\\
0.627313313313313	2.98863867226472\\
0.629314314314314	2.98784266509895\\
0.631315315315315	2.98704309106524\\
0.633316316316316	2.98623992993135\\
0.635317317317317	2.98543316142081\\
0.637318318318318	2.98462276521372\\
0.639319319319319	2.98380872094784\\
0.64132032032032	2.98299100821947\\
0.643321321321321	2.98216960658443\\
0.645322322322322	2.98134449555913\\
0.647323323323323	2.98051565462149\\
0.649324324324324	2.97968306321201\\
0.651325325325325	2.97884670073483\\
0.653326326326326	2.97800654655869\\
0.655327327327327	2.97716258001812\\
0.657328328328328	2.9763147804144\\
0.659329329329329	2.97546312701673\\
0.66133033033033	2.97460759906329\\
0.663331331331331	2.97374817576234\\
0.665332332332332	2.97288483629342\\
0.667333333333333	2.97201755980841\\
0.669334334334334	2.9711463254327\\
0.671335335335335	2.97027111226638\\
0.673336336336336	2.96939189938538\\
0.675337337337337	2.96850866584265\\
0.677338338338338	2.96762139066938\\
0.679339339339339	2.96673005287616\\
0.68134034034034	2.96583463145422\\
0.683341341341341	2.96493510537667\\
0.685342342342342	2.96403145359965\\
0.687343343343343	2.96312365506366\\
0.689344344344344	2.96221168869474\\
0.691345345345345	2.96129553340575\\
0.693346346346346	2.96037516809759\\
0.695347347347347	2.95945057166056\\
0.697348348348348	2.95852172297548\\
0.699349349349349	2.9575886009151\\
0.70135035035035	2.95665118434534\\
0.703351351351351	2.95570945212654\\
0.705352352352352	2.95476338311478\\
0.707353353353353	2.95381295616318\\
0.709354354354354	2.95285815012318\\
0.711355355355355	2.95189894384582\\
0.713356356356356	2.95093531618312\\
0.715357357357357	2.94996724598926\\
0.717358358358358	2.94899471212198\\
0.719359359359359	2.94801769344384\\
0.72136036036036	2.94703616882355\\
0.723361361361361	2.94605011713721\\
0.725362362362362	2.94505951726967\\
0.727363363363363	2.94406434811583\\
0.729364364364364	2.94306458858188\\
0.731365365365365	2.94206021758661\\
0.733366366366366	2.94105121406273\\
0.735367367367367	2.94003755695813\\
0.737368368368368	2.9390192252371\\
0.739369369369369	2.93799619788174\\
0.74137037037037	2.93696845389306\\
0.743371371371371	2.93593597229229\\
0.745372372372372	2.93489873212221\\
0.747373373373373	2.9338567124483\\
0.749374374374374	2.93280989235996\\
0.751375375375375	2.93175825097173\\
0.753376376376376	2.93070176742462\\
0.755377377377377	2.92964042088707\\
0.757378378378378	2.92857419055638\\
0.759379379379379	2.92750305565969\\
0.76138038038038	2.92642699545522\\
0.763381381381381	2.92534598923341\\
0.765382382382382	2.92426001631799\\
0.767383383383383	2.92316905606717\\
0.769384384384384	2.92207308787465\\
0.771385385385385	2.92097209117074\\
0.773386386386386	2.91986604542337\\
0.775387387387387	2.9187549301392\\
0.777388388388388	2.91763872486452\\
0.779389389389389	2.91651740918636\\
0.78139039039039	2.9153909627334\\
0.783391391391391	2.91425936517691\\
0.785392392392392	2.91312259623169\\
0.787393393393393	2.91198063565699\\
0.789394394394394	2.91083346325737\\
0.791395395395395	2.90968105888357\\
0.793396396396396	2.90852340243329\\
0.795397397397397	2.90736047385203\\
0.797398398398398	2.90619225313386\\
0.799399399399399	2.90501872032213\\
0.8014004004004	2.90383985551024\\
0.803401401401401	2.90265563884228\\
0.805402402402402	2.90146605051369\\
0.807403403403404	2.9002710707719\\
0.809404404404404	2.8990706799169\\
0.811405405405405	2.89786485830184\\
0.813406406406406	2.89665358633346\\
0.815407407407407	2.89543684447271\\
0.817408408408409	2.89421461323513\\
0.819409409409409	2.89298687319128\\
0.82141041041041	2.89175360496714\\
0.823411411411412	2.89051478924445\\
0.825412412412412	2.88927040676102\\
0.827413413413413	2.88802043831102\\
0.829414414414414	2.88676486474521\\
0.831415415415415	2.88550366697116\\
0.833416416416417	2.88423682595333\\
0.835417417417417	2.88296432271329\\
0.837418418418418	2.88168613832976\\
0.839419419419419	2.88040225393865\\
0.84142042042042	2.87911265073297\\
0.843421421421422	2.87781730996299\\
0.845422422422422	2.8765162129359\\
0.847423423423423	2.87520934101592\\
0.849424424424424	2.87389667562391\\
0.851425425425425	2.87257819823732\\
0.853426426426427	2.87125389038983\\
0.855427427427428	2.86992373367105\\
0.857428428428428	2.86858770972622\\
0.859429429429429	2.8672458002558\\
0.86143043043043	2.86589798701494\\
0.863431431431432	2.86454425181312\\
0.865432432432433	2.8631845765135\\
0.867433433433433	2.86181894303238\\
0.869434434434434	2.86044733333865\\
0.871435435435435	2.85906972945291\\
0.873436436436437	2.857686113447\\
0.875437437437438	2.85629646744295\\
0.877438438438438	2.85490077361243\\
0.879439439439439	2.85349901417566\\
0.88144044044044	2.85209117140063\\
0.883441441441442	2.85067722760204\\
0.885442442442442	2.84925716514034\\
0.887443443443444	2.84783096642064\\
0.889444444444444	2.84639861389153\\
0.891445445445445	2.8449600900441\\
0.893446446446447	2.84351537741051\\
0.895447447447447	2.84206445856281\\
0.897448448448449	2.84060731611166\\
0.899449449449449	2.8391439327049\\
0.90145045045045	2.83767429102618\\
0.903451451451452	2.83619837379347\\
0.905452452452452	2.83471616375755\\
0.907453453453454	2.8332276437004\\
0.909454454454455	2.83173279643363\\
0.911455455455455	2.83023160479686\\
0.913456456456457	2.82872405165589\\
0.915457457457457	2.82721011990095\\
0.917458458458459	2.82568979244497\\
0.91945945945946	2.8241630522216\\
0.92146046046046	2.82262988218338\\
0.923461461461462	2.82109026529971\\
0.925462462462462	2.81954418455485\\
0.927463463463464	2.81799162294585\\
0.929464464464465	2.81643256348036\\
0.931465465465466	2.81486698917464\\
0.933466466466467	2.81329488305112\\
0.935467467467467	2.81171622813626\\
0.937468468468469	2.81013100745824\\
0.93946946946947	2.80853920404445\\
0.941470470470471	2.80694080091927\\
0.943471471471471	2.80533578110145\\
0.945472472472472	2.80372412760162\\
0.947473473473474	2.8021058234197\\
0.949474474474475	2.80048085154246\\
0.951475475475476	2.79884919494056\\
0.953476476476476	2.79721083656604\\
0.955477477477477	2.79556575934953\\
0.957478478478479	2.79391394619736\\
0.95947947947948	2.79225537998877\\
0.961480480480481	2.79059004357294\\
0.963481481481481	2.78891791976608\\
0.965482482482482	2.78723899134844\\
0.967483483483484	2.78555324106111\\
0.969484484484485	2.78386065160312\\
0.971485485485486	2.78216120562812\\
0.973486486486487	2.78045488574127\\
0.975487487487487	2.77874167449596\\
0.977488488488489	2.7770215543905\\
0.97948948948949	2.77529450786478\\
0.981490490490491	2.77356051729693\\
0.983491491491492	2.77181956499975\\
0.985492492492492	2.77007163321735\\
0.987493493493494	2.76831670412153\\
0.989494494494495	2.76655475980822\\
0.991495495495496	2.76478578229385\\
0.993496496496497	2.76300975351165\\
0.995497497497498	2.7612266553079\\
0.997498498498499	2.75943646943825\\
0.9994994994995	2.75763917756372\\
1.0015005005005	2.75583476124693\\
1.0035015015015	2.75402320194816\\
1.0055025025025	2.75220448102144\\
1.0075035035035	2.75037857971033\\
1.0095045045045	2.74854547914399\\
1.01150550550551	2.74670516033307\\
1.01350650650651	2.74485760416546\\
1.01550750750751	2.74300279140203\\
1.01750850850851	2.74114070267253\\
1.01950950950951	2.73927131847105\\
1.02151051051051	2.73739461915175\\
1.02351151151151	2.73551058492441\\
1.02551251251251	2.73361919584993\\
1.02751351351351	2.73172043183581\\
1.02951451451451	2.72981427263157\\
1.03151551551552	2.72790069782399\\
1.03351651651652	2.72597968683264\\
1.03551751751752	2.72405121890486\\
1.03751851851852	2.72211527311119\\
1.03951951951952	2.72017182834027\\
1.04152052052052	2.71822086329405\\
1.04352152152152	2.71626235648292\\
1.04552252252252	2.71429628622035\\
1.04752352352352	2.71232263061799\\
1.04952452452452	2.71034136758071\\
1.05152552552553	2.70835247480084\\
1.05352652652653	2.7063559297534\\
1.05552752752753	2.70435170969052\\
1.05752852852853	2.702339791636\\
1.05952952952953	2.70032015237997\\
1.06153053053053	2.69829276847317\\
1.06353153153153	2.69625761622144\\
1.06553253253253	2.6942146716801\\
1.06753353353353	2.692163910648\\
1.06953453453453	2.69010530866192\\
1.07153553553554	2.68803884099052\\
1.07353653653654	2.68596448262834\\
1.07553753753754	2.68388220828996\\
1.07753853853854	2.68179199240363\\
1.07953953953954	2.67969380910519\\
1.08154054054054	2.67758763223179\\
1.08354154154154	2.67547343531547\\
1.08554254254254	2.67335119157675\\
1.08754354354354	2.67122087391791\\
1.08954454454454	2.66908245491676\\
1.09154554554555	2.66693590681957\\
1.09354654654655	2.66478120153421\\
1.09554754754755	2.6626183106236\\
1.09754854854855	2.66044720529836\\
1.09954954954955	2.65826785640987\\
1.10155055055055	2.65608023444307\\
1.10355155155155	2.65388430950913\\
1.10555255255255	2.6516800513379\\
1.10755355355355	2.64946742927057\\
1.10955455455455	2.64724641225192\\
1.11155555555556	2.64501696882267\\
1.11355655655656	2.64277906711145\\
1.11555755755756	2.64053267482696\\
1.11755855855856	2.6382777592496\\
1.11955955955956	2.63601428722375\\
1.12156056056056	2.63374222514861\\
1.12356156156156	2.6314615389705\\
1.12556256256256	2.62917219417363\\
1.12756356356356	2.62687415577155\\
1.12956456456456	2.62456738829815\\
1.13156556556557	2.62225185579858\\
1.13356656656657	2.61992752182011\\
1.13556756756757	2.61759434940257\\
1.13756856856857	2.61525230106882\\
1.13956956956957	2.61290133881516\\
1.14157057057057	2.61054142410125\\
1.14357157157157	2.60817251784016\\
1.14557257257257	2.60579458038809\\
1.14757357357357	2.60340757153383\\
1.14957457457457	2.60101145048837\\
1.15157557557558	2.59860617587366\\
1.15357657657658	2.59619170571202\\
1.15557757757758	2.5937679974146\\
1.15757857857858	2.59133500777012\\
1.15957957957958	2.58889269293309\\
1.16158058058058	2.58644100841184\\
1.16358158158158	2.58397990905649\\
1.16558258258258	2.58150934904658\\
1.16758358358358	2.57902928187847\\
1.16958458458458	2.5765396603524\\
1.17158558558559	2.57404043655921\\
1.17358658658659	2.57153156186732\\
1.17558758758759	2.56901298690869\\
1.17758858858859	2.56648466156494\\
1.17958958958959	2.56394653495327\\
1.18159059059059	2.56139855541159\\
1.18359159159159	2.55884067048389\\
1.18559259259259	2.55627282690485\\
1.18759359359359	2.55369497058447\\
1.18959459459459	2.55110704659213\\
1.1915955955956	2.54850899914031\\
1.1935965965966	2.545900771568\\
1.1955975975976	2.54328230632388\\
1.1975985985986	2.54065354494872\\
1.1995995995996	2.53801442805793\\
1.2016006006006	2.53536489532315\\
1.2036016016016	2.53270488545372\\
1.2056026026026	2.53003433617776\\
1.2076036036036	2.52735318422262\\
1.2096046046046	2.52466136529506\\
1.21160560560561	2.52195881406045\\
1.21360660660661	2.51924546412252\\
1.21560760760761	2.5165212480013\\
1.21760860860861	2.51378609711142\\
1.21960960960961	2.51103994173977\\
1.22161061061061	2.50828271102222\\
1.22361161161161	2.50551433292012\\
1.22561261261261	2.5027347341959\\
1.22761361361361	2.49994384038823\\
1.22961461461461	2.4971415757869\\
1.23161561561562	2.49432786340599\\
1.23361661661662	2.49150262495751\\
1.23561761761762	2.48866578082391\\
1.23761861861862	2.48581725002922\\
1.23961961961962	2.4829569502107\\
1.24162062062062	2.48008479758832\\
1.24362162162162	2.47720070693478\\
1.24562262262262	2.47430459154356\\
1.24762362362362	2.47139636319671\\
1.24962462462462	2.46847593213174\\
1.25162562562563	2.46554320700715\\
1.25362662662663	2.46259809486737\\
1.25562762762763	2.45964050110696\\
1.25762862862863	2.45667032943294\\
1.25962962962963	2.45368748182673\\
1.26163063063063	2.45069185850488\\
1.26363163163163	2.44768335787846\\
1.26563263263263	2.44466187651135\\
1.26763363363363	2.4416273090773\\
1.26963463463463	2.43857954831569\\
1.27163563563564	2.43551848498596\\
1.27363663663664	2.43244400782071\\
1.27563763763764	2.42935600347708\\
1.27763863863864	2.42625435648729\\
1.27963963963964	2.42313894920682\\
1.28164064064064	2.42000966176157\\
1.28364164164164	2.41686637199296\\
1.28564264264264	2.41370895540174\\
1.28764364364364	2.41053728508955\\
1.28964464464464	2.40735123169867\\
1.29164564564565	2.40415066335033\\
1.29364664664665	2.40093544558015\\
1.29564764764765	2.39770544127201\\
1.29764864864865	2.39446051058976\\
1.29964964964965	2.39120051090634\\
1.30165065065065	2.38792529673056\\
1.30365165165165	2.38463471963175\\
1.30565265265265	2.38132862816114\\
1.30765365365365	2.37800686777127\\
1.30965465465465	2.37466928073207\\
1.31165565565566	2.37131570604418\\
1.31365665665666	2.3679459793492\\
1.31565765765766	2.36455993283667\\
1.31765865865866	2.36115739514773\\
1.31965965965966	2.35773819127528\\
1.32166066066066	2.35430214246061\\
1.32366166166166	2.35084906608602\\
1.32566266266266	2.34737877556332\\
1.32766366366366	2.34389108021846\\
1.32966466466466	2.34038578517192\\
1.33166566566567	2.33686269121379\\
1.33366666666667	2.33332159467492\\
1.33566766766767	2.32976228729239\\
1.33766866866867	2.32618455607003\\
1.33966966966967	2.32258818313339\\
1.34167067067067	2.31897294557887\\
1.34367167167167	2.31533861531681\\
1.34567267267267	2.31168495890803\\
1.34767367367367	2.30801173739404\\
1.34967467467467	2.30431870611976\\
1.35167567567568	2.30060561454901\\
1.35367667667668	2.29687220607264\\
1.35567767767768	2.29311821780764\\
1.35767867867868	2.28934338038841\\
1.35967967967968	2.28554741774827\\
1.36168068068068	2.28173004689227\\
1.36368168168168	2.27789097765888\\
1.36568268268268	2.27402991247174\\
1.36768368368368	2.2701465460804\\
1.36968468468468	2.26624056528809\\
1.37168568568569	2.26231164866848\\
1.37368668668669	2.25835946626784\\
1.37568768768769	2.25438367929423\\
1.37768868868869	2.2503839397912\\
1.37968968968969	2.24635989029649\\
1.38169069069069	2.24231116348291\\
1.38369169169169	2.23823738178352\\
1.38569269269269	2.23413815699631\\
1.38769369369369	2.23001308987009\\
1.38969469469469	2.22586176966898\\
1.3916956956957	2.22168377371462\\
1.3936966966967	2.21747866690489\\
1.3956976976977	2.21324600120646\\
1.3976986986987	2.20898531512146\\
1.3996996996997	2.20469613312469\\
1.4017007007007	2.20037796507072\\
1.4037017017017	2.19603030556732\\
1.4057027027027	2.1916526333152\\
1.4077037037037	2.18724441040962\\
1.4097047047047	2.18280508160226\\
1.41170570570571	2.17833407352036\\
1.41370670670671	2.17383079383957\\
1.41570770770771	2.1692946304085\\
1.41770870870871	2.16472495031899\\
1.41970970970971	2.16012109892182\\
1.42171071071071	2.15548239877886\\
1.42371171171171	2.15080814855159\\
1.42571271271271	2.14609762181654\\
1.42771371371371	2.14135006580689\\
1.42971471471471	2.13656470007002\\
1.43171571571572	2.13174071503485\\
1.43371671671672	2.12687727048381\\
1.43571771771772	2.12197349392098\\
1.43771871871872	2.11702847882381\\
1.43971971971972	2.11204128277412\\
1.44172072072072	2.10701092545115\\
1.44372172172172	2.10193638648412\\
1.44572272272272	2.09681660313833\\
1.44772372372372	2.09165046783266\\
1.44972472472472	2.08643682546291\\
1.45172572572573	2.08117447051721\\
1.45372672672673	2.07586214396272\\
1.45572772772773	2.07049852988217\\
1.45772872872873	2.06508225182748\\
1.45972972972973	2.05961186887718\\
1.46173073073073	2.05408587134549\\
1.46373173173173	2.04850267613045\\
1.46573273273273	2.04286062163503\\
1.46773373373373	2.03715796223941\\
1.46973473473473	2.03139286225725\\
1.47173573573574	2.02556338932029\\
1.47373673673674	2.01966750682888\\
1.47573773773774	2.0137030671367\\
1.47773873873874	2.00766780111007\\
1.47973973973974	2.00155930953539\\
1.48174074074074	1.99537505216374\\
1.48374174174174	1.98911233586276\\
1.48574274274274	1.9827683014576\\
1.48774374374374	1.97633990907901\\
1.48974474474474	1.96982392180633\\
1.49174574574575	1.96321688735648\\
1.49374674674675	1.95651511752583\\
1.49574774774775	1.94971466503857\\
1.49774874874875	1.94281129739013\\
1.49974974974975	1.93580046719491\\
1.50175075075075	1.92867727844978\\
1.50375175175175	1.92143644800426\\
1.50575275275275	1.91407226137787\\
1.50775375375375	1.90657852187676\\
1.50975475475475	1.89894849172455\\
1.51175575575576	1.89117482361997\\
1.51375675675676	1.88324948074738\\
1.51575775775776	1.87516364276629\\
1.51775875875876	1.86690759465425\\
1.51975975975976	1.85847059441896\\
1.52176076076076	1.84984071455316\\
1.52376176176176	1.84100465056793\\
1.52576276276276	1.83194748784368\\
1.52776376376376	1.82265241514252\\
1.52976476476476	1.81310036906554\\
1.53176576576577	1.80326958795458\\
1.53376676676677	1.79313504534855\\
1.53576776776777	1.78266772069523\\
1.53776876876877	1.77183364625738\\
1.53976976976977	1.76059264007665\\
1.54177077077077	1.74889658853565\\
1.54377177177177	1.73668706588053\\
1.54577277277277	1.7238919481556\\
1.54777377377377	1.7104204478812\\
1.54977477477477	1.69615556359598\\
1.55177577577578	1.68094208031008\\
1.55377677677678	1.66456642399558\\
1.55577777777778	1.64672038059661\\
1.55777877877878	1.62692935582227\\
1.55977977977978	1.60439056657772\\
1.56178078078078	1.57752604025496\\
1.56378178178178	1.54218992567804\\
1.56578278278278	1.45934691078994\\
1.56778378378378	1.45970763570305\\
1.56578278278278	1.4593457426138\\
1.56378178178178	1.37569518167825\\
1.56178078078078	1.33953847048076\\
1.55977977977978	1.31184492695848\\
1.55777877877878	1.2884684893784\\
1.55577777777778	1.26783102763013\\
1.55377677677678	1.24912958333869\\
1.55177577577578	1.23188938064868\\
1.54977477477477	1.21580201811527\\
1.54777377377377	1.2006537283019\\
1.54577277277277	1.18628909652687\\
1.54377177177177	1.17259091506997\\
1.54177077077077	1.15946818332556\\
1.53976976976977	1.14684855702652\\
1.53776876876877	1.1346733825863\\
1.53576776776777	1.12289431067644\\
1.53376676676677	1.11147091537059\\
1.53176576576577	1.10036897630454\\
1.52976476476476	1.08955921121303\\
1.52776376376376	1.07901632238116\\
1.52576276276276	1.06871826687257\\
1.52376176176176	1.05864568946978\\
1.52176076076076	1.04878147602652\\
1.51975975975976	1.03911039733786\\
1.51775875875876	1.0296188220255\\
1.51575775775776	1.02029448271814\\
1.51375675675676	1.01112628386679\\
1.51175575575576	1.00210414243033\\
1.50975475475475	0.993218854762093\\
1.50775375375375	0.984461984566587\\
1.50575275275275	0.975825767935874\\
1.50375175175175	0.967303032334481\\
1.50175075075075	0.958887127051661\\
1.49974974974975	0.950571863139442\\
1.49774874874875	0.942351461240821\\
1.49574774774775	0.934220506013158\\
1.49374674674675	0.926173906088372\\
1.49174574574575	0.918206858698462\\
1.48974474474474	0.910314818243663\\
1.48774374374374	0.902493468199649\\
1.48574274274274	0.894738695855927\\
1.48374174174174	0.88704656945449\\
1.48174074074074	0.879413317360458\\
1.47973973973974	0.871835308946666\\
1.47773873873874	0.864309036914685\\
1.47573773773774	0.856831100807511\\
1.47373673673674	0.849398191494929\\
1.47173573573574	0.842007076432416\\
1.46973473473473	0.834654585509449\\
1.46773373373373	0.827337597313867\\
1.46573273273273	0.820053025644541\\
1.46373173173173	0.812797806107332\\
1.46173073073073	0.805568882627261\\
1.45972972972973	0.798363193703796\\
1.45772872872873	0.791177658225493\\
1.45572772772773	0.784009160644579\\
1.45372672672673	0.776854535290766\\
1.45172572572573	0.769710549574243\\
1.44972472472472	0.762573885791675\\
1.44772372372372	0.755441121200481\\
1.44572272272272	0.748308705967109\\
1.44372172172172	0.741172938516356\\
1.44172072072072	0.734029937711049\\
1.43971971971972	0.726875611164301\\
1.43771871871872	0.71970561882312\\
1.43571771771772	0.712515330750611\\
1.43371671671672	0.705299777757044\\
1.43171571571572	0.698053593163982\\
1.42971471471471	0.690770943499307\\
1.42771371371371	0.683445445263784\\
1.42571271271271	0.676070064015653\\
1.42371171171171	0.668636990782375\\
1.42171071071071	0.661137489074985\\
1.41970970970971	0.653561703310704\\
1.41770870870871	0.645898415867975\\
1.41570770770771	0.638134734703463\\
1.41370670670671	0.630255685459638\\
1.41170570570571	0.622243669603352\\
1.4097047047047	0.614077730416876\\
1.4077037037037	0.605732536275695\\
1.4057027027027	0.597176935499494\\
1.4037017017017	0.588371839129603\\
1.4017007007007	0.579267005277197\\
1.3996996996997	0.569795937009827\\
1.3976986986987	0.559867336086481\\
1.3956976976977	0.54934976130738\\
1.3936966966967	0.538041435037946\\
1.3916956956957	0.525602652729906\\
1.38969469469469	0.511371403315017\\
1.38769369369369	0.493644099290747\\
1.38569269269269	0.45815376523676\\
1.38769369369369	0.417935464390498\\
1.38969469469469	0.404359480567653\\
1.3916956956957	0.394306227107519\\
1.3936966966967	0.38607255160976\\
1.3956976976977	0.378996891761144\\
1.3976986986987	0.372740002662957\\
1.3996996996997	0.36710058253716\\
1.4017007007007	0.361947682625312\\
1.4037017017017	0.35719050845796\\
1.4057027027027	0.35276308445547\\
1.4077037037037	0.348615706661817\\
1.4097047047047	0.344709841381341\\
1.41170570570571	0.341014910315142\\
1.41370670670671	0.337506174167761\\
1.41570770770771	0.334163288376697\\
1.41770870870871	0.330969287318226\\
1.41970970970971	0.32790985128605\\
1.42171071071071	0.324972765675815\\
1.42371171171171	0.322147514205296\\
1.42571271271271	0.31942496771534\\
1.42771371371371	0.316797142484099\\
1.42971471471471	0.314257009992506\\
1.43171571571572	0.311798345369854\\
1.43371671671672	0.309415605334188\\
1.43571771771772	0.307103828909963\\
1.43771871871872	0.304858555941246\\
1.43971971971972	0.302675759657171\\
1.44172072072072	0.30055179044068\\
1.44372172172172	0.298483328610414\\
1.44572272272272	0.296467344513759\\
1.44772372372372	0.294501064595098\\
1.44972472472472	0.292581942383754\\
1.45172572572573	0.290707633557921\\
1.45372672672673	0.288875974407579\\
1.45572772772773	0.287084963147954\\
1.45772872872873	0.285332743636116\\
1.45972972972973	0.283617591124787\\
1.46173073073073	0.28193789975054\\
1.46373173173173	0.280292171505965\\
1.46573273273273	0.278679006486614\\
1.46773373373373	0.277097094237467\\
1.46973473473473	0.275545206051802\\
1.47173573573574	0.274022188097319\\
1.47373673673674	0.272526955264075\\
1.47573773773774	0.271058485643838\\
1.47773873873874	0.269615815563436\\
1.47973973973974	0.268198035106281\\
1.48174074074074	0.26680428406428\\
1.48374174174174	0.265433748271373\\
1.48574274274274	0.26408565627519\\
1.48774374374374	0.262759276310184\\
1.48974474474474	0.261453913538976\\
1.49174574574575	0.260168907534148\\
1.49374674674675	0.258903629975005\\
1.49574774774775	0.25765748253757\\
1.49774874874875	0.256429894958464\\
1.49974974974975	0.255220323255169\\
1.50175075075075	0.254028248088183\\
1.50375175175175	0.252853173250986\\
1.50575275275275	0.251694624276109\\
1.50775375375375	0.250552147146597\\
1.50975475475475	0.249425307103366\\
1.51175575575576	0.248313687539804\\
1.51375675675676	0.247216888975986\\
1.51575775775776	0.246134528105804\\
1.51775875875876	0.245066236910535\\
1.51975975975976	0.244011661833492\\
1.52176076076076	0.242970463010626\\
1.52376176176176	0.241942313552558\\
1.52576276276276	0.24092689887396\\
1.52776376376376	0.239923916066362\\
1.52976476476476	0.238933073311304\\
1.53176576576577	0.237954089330464\\
1.53376676676677	0.236986692870035\\
1.53576776776777	0.23603062221693\\
1.53776876876877	0.235085624744207\\
1.53976976976977	0.234151456483821\\
1.54177077077077	0.233227881724645\\
1.54377177177177	0.232314672633883\\
1.54577277277277	0.231411608900419\\
1.54777377377377	0.230518477398389\\
1.54977477477477	0.229635071869738\\
1.55177577577578	0.228761192624412\\
1.55377677677678	0.227896646257046\\
1.55577777777778	0.227041245378924\\
1.55777877877878	0.226194808364486\\
1.55977977977978	0.225357159111148\\
1.56178078078078	0.224528126811777\\
1.56378178178178	0.223707545738971\\
1.56578278278278	0.222895255040374\\
1.56778378378378	0.222091098544399\\
1.56978478478478	0.221294924575671\\
1.57178578578579	0.220506585779688\\
1.57378678678679	0.219725938956029\\
1.57578778778779	0.218952844899682\\
1.57778878878879	0.218187168250063\\
1.57978978978979	0.217428777347153\\
1.58179079079079	0.216677544094451\\
1.58379179179179	0.21593334382828\\
1.58579279279279	0.215196055193232\\
1.58779379379379	0.214465560023145\\
1.58979479479479	0.213741743227653\\
1.5917957957958	0.213024492683682\\
1.5937967967968	0.21231369913183\\
1.5957977977978	0.211609256077314\\
1.5977987987988	0.210911059695204\\
1.5997997997998	0.210219008739764\\
1.6018008008008	0.209533004457694\\
1.6038018018018	0.208852950504999\\
1.6058028028028	0.208178752867413\\
1.6078038038038	0.207510319784082\\
1.6098048048048	0.206847561674443\\
1.61180580580581	0.206190391068047\\
1.61380680680681	0.205538722537305\\
1.61580780780781	0.204892472632817\\
1.61780880880881	0.20425155982142\\
1.61980980980981	0.203615904426565\\
1.62181081081081	0.202985428571114\\
1.62381181181181	0.202360056122303\\
1.62581281281281	0.201739712638871\\
1.62781381381381	0.201124325320193\\
1.62981481481481	0.200513822957311\\
1.63181581581582	0.199908135885867\\
1.63381681681682	0.199307195940747\\
1.63581781781782	0.198710936412425\\
1.63781881881882	0.198119292004893\\
1.63981981981982	0.197532198795146\\
1.64182082082082	0.196949594194117\\
1.64382182182182	0.196371416909019\\
1.64582282282282	0.195797606907018\\
1.64782382382382	0.195228105380223\\
1.64982482482482	0.194662854711837\\
1.65182582582583	0.19410179844358\\
1.65382682682683	0.193544881244146\\
1.65582782782783	0.1929920488788\\
1.65782882882883	0.192443248179989\\
1.65982982982983	0.191898427018963\\
1.66183083083083	0.191357534278305\\
1.66383183183183	0.190820519825451\\
1.66583283283283	0.190287334487\\
1.66783383383383	0.189757930023938\\
1.66983483483483	0.189232259107632\\
1.67183583583584	0.188710275296617\\
1.67383683683684	0.188191933014124\\
1.67583783783784	0.187677187526328\\
1.67783883883884	0.187165994921289\\
1.67983983983984	0.186658312088541\\
1.68184084084084	0.18615409669936\\
1.68384184184184	0.18565330718759\\
1.68584284284284	0.185155902731115\\
1.68784384384384	0.184661843233858\\
1.68984484484484	0.184171089308361\\
1.69184584584585	0.183683602258866\\
1.69384684684685	0.183199344064922\\
1.69584784784785	0.18271827736549\\
1.69784884884885	0.182240365443488\\
1.69984984984985	0.18176557221083\\
1.70185085085085	0.181293862193883\\
1.70385185185185	0.180825200519347\\
1.70585285285285	0.180359552900561\\
1.70785385385385	0.17989688562418\\
1.70985485485485	0.179437165537258\\
1.71185585585586	0.178980360034682\\
1.71385685685686	0.178526437046956\\
1.71585785785786	0.178075365028362\\
1.71785885885886	0.177627112945406\\
1.71985985985986	0.17718165026562\\
1.72186086086086	0.176738946946653\\
1.72386186186186	0.176298973425685\\
1.72586286286286	0.175861700609091\\
1.72786386386386	0.175427099862428\\
1.72986486486486	0.17499514300067\\
1.73186586586587	0.174565802278698\\
1.73386686686687	0.174139050382068\\
1.73586786786787	0.173714860418001\\
1.73786886886887	0.173293205906616\\
1.73986986986987	0.172874060772409\\
1.74187087087087	0.172457399335934\\
1.74387187187187	0.172043196305705\\
1.74587287287287	0.171631426770313\\
1.74787387387387	0.171222066190743\\
1.74987487487487	0.170815090392883\\
1.75187587587588	0.170410475560228\\
1.75387687687688	0.17000819822676\\
1.75587787787788	0.169608235270022\\
1.75787887887888	0.169210563904347\\
1.75987987987988	0.168815161674271\\
1.76188088088088	0.168422006448094\\
1.76388188188188	0.168031076411616\\
1.76588288288288	0.167642350062004\\
1.76788388388388	0.167255806201833\\
1.76988488488488	0.166871423933259\\
1.77188588588589	0.166489182652322\\
1.77388688688689	0.166109062043405\\
1.77588788788789	0.165731042073817\\
1.77788888888889	0.165355102988491\\
1.77988988988989	0.164981225304841\\
1.78189089089089	0.164609389807697\\
1.78389189189189	0.164239577544388\\
1.78589289289289	0.163871769819938\\
1.78789389389389	0.163505948192354\\
1.78989489489489	0.16314209446804\\
1.7918958958959	0.16278019069731\\
1.7938968968969	0.162420219170003\\
1.7958978978979	0.162062162411198\\
1.7978988988989	0.161706003177027\\
1.7998998998999	0.161351724450578\\
1.8019009009009	0.160999309437893\\
1.8039019019019	0.160648741564057\\
1.8059029029029	0.160300004469367\\
1.8079039039039	0.159953082005593\\
1.8099049049049	0.159607958232314\\
1.81190590590591	0.159264617413337\\
1.81390690690691	0.1589230440132\\
1.81590790790791	0.158583222693739\\
1.81790890890891	0.158245138310733\\
1.81990990990991	0.15790877591063\\
1.82191091091091	0.157574120727328\\
1.82391191191191	0.157241158179038\\
1.82591291291291	0.156909873865201\\
1.82791391391391	0.156580253563479\\
1.82991491491491	0.156252283226809\\
1.83191591591592	0.155925948980508\\
1.83391691691692	0.155601237119454\\
1.83591791791792	0.155278134105311\\
1.83791891891892	0.154956626563818\\
1.83991991991992	0.154636701282135\\
1.84192092092092	0.15431834520624\\
1.84392192192192	0.154001545438375\\
1.84592292292292	0.153686289234555\\
1.84792392392392	0.153372564002118\\
1.84992492492492	0.153060357297323\\
1.85192592592593	0.152749656823005\\
1.85392692692693	0.152440450426268\\
1.85592792792793	0.152132726096228\\
1.85792892892893	0.151826471961799\\
1.85992992992993	0.151521676289522\\
1.86193093093093	0.151218327481439\\
1.86393193193193	0.150916414073005\\
1.86593293293293	0.150615924731047\\
1.86793393393393	0.150316848251747\\
1.86993493493493	0.150019173558687\\
1.87193593593594	0.149722889700909\\
1.87393693693694	0.149427985851033\\
1.87593793793794	0.149134451303381\\
1.87793893893894	0.148842275472177\\
1.87993993993994	0.148551447889747\\
1.88194094094094	0.148261958204762\\
1.88394194194194	0.147973796180527\\
1.88594294294294	0.147686951693282\\
1.88794394394394	0.147401414730552\\
1.88994494494494	0.147117175389512\\
1.89194594594595	0.146834223875396\\
1.89394694694695	0.146552550499927\\
1.89594794794795	0.146272145679776\\
1.89794894894895	0.145992999935051\\
1.89994994994995	0.145715103887813\\
1.90195095095095	0.145438448260621\\
1.90395195195195	0.145163023875101\\
1.90595295295295	0.144888821650533\\
1.90795395395395	0.144615832602482\\
1.90995495495495	0.144344047841433\\
1.91195595595596	0.144073458571467\\
1.91395695695696	0.143804056088945\\
1.91595795795796	0.143535831781227\\
1.91795895895896	0.143268777125412\\
1.91995995995996	0.143002883687088\\
1.92196096096096	0.142738143119125\\
1.92396196196196	0.142474547160467\\
1.92596296296296	0.142212087634958\\
1.92796396396396	0.141950756450188\\
1.92996496496496	0.141690545596353\\
1.93196596596597	0.141431447145133\\
1.93396696696697	0.141173453248602\\
1.93596796796797	0.14091655613814\\
1.93796896896897	0.140660748123377\\
1.93996996996997	0.140406021591141\\
1.94197097097097	0.140152369004444\\
1.94397197197197	0.139899782901456\\
1.94597297297297	0.139648255894528\\
1.94797397397397	0.139397780669206\\
1.94997497497497	0.139148349983275\\
1.95197597597598	0.138899956665812\\
1.95397697697698	0.138652593616261\\
1.95597797797798	0.138406253803517\\
1.95797897897898	0.138160930265027\\
1.95997997997998	0.137916616105908\\
1.96198098098098	0.137673304498075\\
1.96398198198198	0.137430988679386\\
1.96598298298298	0.137189661952801\\
1.96798398398398	0.136949317685555\\
1.96998498498498	0.136709949308336\\
1.97198598598599	0.136471550314492\\
1.97398698698699	0.136234114259235\\
1.97598798798799	0.135997634758867\\
1.97798898898899	0.135762105490013\\
1.97998998998999	0.135527520188869\\
1.98199099099099	0.135293872650464\\
1.98399199199199	0.135061156727925\\
1.98599299299299	0.134829366331765\\
1.98799399399399	0.134598495429171\\
1.98999499499499	0.134368538043311\\
1.991995995996	0.134139488252647\\
1.993996996997	0.133911340190261\\
1.995997997998	0.133684088043191\\
1.997998998999	0.133457726051777\\
2	0.133232248509013\\
};
\addlegendentry{$\alpha\text{ = 0.03}$};

\end{axis}
\end{tikzpicture}%
	\caption{Frequency response of the Duffing equation with $\epsilon=1$, $\delta=0.2$, $\gamma=2.5$ for different values of non-linearity $\alpha$. With higher non-linearity, the curve becomes less symmetric and at high enough values for $\alpha$, multiple solutions are possible for a given frequency.}
	\label{fig:sphere}
\end{figure}




