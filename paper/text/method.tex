\section{Method \& Materials}
Multiple experiments were conducted with gradually increasing complexity. 

\subsection{Experimental set-up}
The set-up consists of the electrical system to drive the crystals and measure its response, and of the system to bring the coated crystals in contact with gasses. The two systems will be described consecutively in the following sections.

\subsubsection{Eletrical system}
A schematic overview of the electrical system can be found in \autoref{fig:oveele}. A function generator sweeps the desired frequencies, which are amplified before exciting the crystal. The voltage drop over a series shunt resistor is measured using a differential amplifier. This voltage drop, which is linearly proportional to to the current that flows through the quartz crystal as described by Ohm's law, is indicative of the changing impedance of the crystal as a function of frequency. Therefore, this signal can be used to show when the crystal is on resonance. 
\begin{figure}
	\centering
		\includegraphics[width=\textwidth]{figures/placeholder.png}
	\caption{Schematic overview of the electrical system of the experimental set-up. }
	\label{fig:oveele}
\end{figure}

A more detailed schematic can be found in \autoref{fig:detsch}. Preliminary tests showed that the peak voltage of about $V_{p}=10$~V is needed to see a significant anisochronic effect. The maximum current at this voltage is about $I_{p}=200$~mA. A high speed operational amplifier was used in combination with a buffer amplifier in the feedback path to decrease the output impedance and increase the output current capability of the amplifier. The slew rate of the amplifier is also an important parameter, since the signal has both high frequency and high amplitude. The needed slew rate corresponds to the maximum slope of the output signal, which for a sinusoidal signal of the following form:
\begin{equation}\label{eq:driving}
v(t) = V_{P} \sin{2\pi ft},
\end{equation}
is given by:
\begin{equation}\label{eq:slewrate}
\mbox{max}\left(\left|\frac{dv(t)}{dt}\right|\right) = \frac{\pi fV_{P}}{2}.
\end{equation}
Using crystal frequency of $f=\SI{4.606}{\mega\hertz}$ and a peak voltage of $V_{p}=10$~V, the minimum slew rate of the amplifier should therefore be \SI{425}{\volt/\micro\second}. 



\begin{figure}
	\centering
		\includegraphics[width=\textwidth]{figures/placeholder.png}
	\caption{Detailed schematic of the electrical system of the experimental set-up. }
	\label{fig:detsch}
\end{figure}

\subsection{Crystal analysis}
The goals of the first phase are to choose a resonator and find benchmark impedance-frequency data of the chosen resonator. 

Since a mechanical resonator is needed to react to the change of mass, but it is desirable to drive and analyze the system electronically, a piezoelectric resonator is used as a transformer between the electronic and mechanical signal. Piezoelectric materials exist with a Q-factor up to XXX, but they are very expensive. Ceramic resonators are cheap, but they lack stability and have low Q-factors. A good middle ground is a quartz crystal oscillator, since they have Q-factors up to $10^6$, and are mass-produced because of their use in almost any electronic device on the market. 

Quartz crystals come in a variety of fundamental frequencies, shapes and sizes. For this experiment a large crystal is preferred over a small crystal, since it eases polymer coating. The fundamental frequency is preferably small, since it loosens bandwidth requirements of all components in the circuit and reduces crosstalk and phase lag XXX. The most common crystal shapes are a planar shear resonator and a tuning fork. Since the tuning fork has an irregular shape, it is hard to get an even coating. The first tests resulted in a coating which filled up the gap XXX. 

The circuit used in the first test is 

\subsection{Measurement}
Since the crystal is driven by a voltage 