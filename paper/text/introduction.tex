\clearpage\section{Introduction}
In our pocket we carry a device which copies almost all human senses. A camera gives your phone sight, a microphone gives your phone ability to hear, and the touch screen is like our skin, allowing your phone to feel touch. The vestibular system is mimicked by a gyroscope and accelerometer combination, and thermoreceptors by thermometers. Using GPS, your phone has a sense of direction, which arguably about half of the human population doesn't even have. 

Of course our phone doesn't need the ability to discern flavors, since it is a very picky eater anyway. It only takes electrons at a small range of potentials to get its energy from. But the underlying principle of the senses of taste and smell, namely the ability to detect molecules, can have many more uses than to decide which dish is delicious or foul-tasting. Examples of uses for a nose in your phone would be to test whether you are too intoxicated to drive, or to warn whenever a dangerous amount of carbon-monoxide is in the air. Outside of mobile applications there are even more uses, such as monitoring air quality, indoors or in the city, or more specific in manufacturing plants, stables or greenhouses.

Many devices already exist which can mimic the function of the nose. Notable technologies are gas chromatography and spectroscopy, however these devices are often the size of meters and expensive[XXX]. New methods are needed to decrease the size and cost of an electronic sense of smell. A cheap electric nose which is currently popular is MQ series[XXX]. However, it is weak at discerning molecules and requires six connections per sensor, which makes it cumbersome to use. 

XXX something about how usually we try to be linear but we increase sensitivity and resolution with non-linearity.

XXX In this paper the possibility of making a cheap, small nose with a high onderscheidend vermogen XXX is researched. Something about what experiments are described in the paper